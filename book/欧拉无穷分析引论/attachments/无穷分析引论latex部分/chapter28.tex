\chapter{第十章 三阶线的基本性质}

\section{$\S 239$}

类似于二阶线, 三阶、四阶和更高阶线,其基本性质也都可以从通用方程推出. 我们 考虑三阶线的最通用方程
\[
\alpha y^{3}+\beta y^{2} x+\gamma y x^{2}+\delta x^{3}+\varepsilon y^{2}+\zeta y x+\eta x^{2}+\theta y+\omega x+\kappa=0
\]
该方程表示以 $x, y$ 为坐标, 轴及纵标倾角都任意的一切三阶线.

\section{$\S 240$}

只要 $\alpha \neq 0$, 每一横标 $x$ 就都对应一个或三个实纵标. 假定是三个, 则这三纵标间关 系, 显然可从方程得到. 令 $\alpha=1$, 方程成为
\[
y^{3}+(\beta x+\varepsilon) y^{2}+\left(\gamma x^{2}-\zeta x+\theta\right) y+\delta x^{3}+\eta x^{2}+\omega x+\kappa=0
\]
由此得, 对应于同一个横标 $x$ 的三个纵标, 和等于 $-\beta x-\varepsilon$, 两两乘积之和等于 $\gamma x^{2}+$ $\zeta x+\theta$, 三个的相乘积等于 $-\delta x^{3}-\eta x^{2}-\iota x-\kappa$. 即使有两个纵标为虚数, 这种关系也依然 成立. 但从复数的和、复数乘积的和都得不到有助于了解曲线形状的东西.

\section{$\S 241$}

设给定的三阶线是关于轴 $A Z$ 的, 如图 44 所示, $L M N, l m n$ 是依给定角画出的纵标 线, 与给定三阶线的交点都是三个. 记横标 $A P=x$, 则对应于它的纵标有三个值, $P L$, $P M,-P N$. 且. $P L+P M-P N=-\beta x-\varepsilon$. 取
\[
P O=z=\frac{P L+P M-P N}{3}
\]
则 $O$ 为中点, 即 $L O=M O+N O$. 由于 $z=-\frac{\beta x+\varepsilon}{3}$, 点 $O$ 在直线 $O Z$ 上, $O Z$ 与任一条平行 于 $L M N$ 的纵标线 $l m n$ 的交点 $O$ 都满足 $l o+m o=n o$.

这类似于二阶线直径的性质. 如果 $O, o$ 分别位于两条相平行且都与曲线有三个交 点, 又 $O, o$ 都使得一面的两个纵标之和等于另一面的第三个纵标, 则过 $O, o$ 的直线以同 样的方式分每一条平行于这两条纵标线的纵标线. 因而过 $O, o$ 的直线是三阶线的一条 直径. 


【图,待补】
%%![](https://cdn.mathpix.com/cropped/2023_02_05_1df1519cd49abfa9c645g-08.jpg?height=397&width=563&top_left_y=297&top_left_x=566)

图 44

\section{$\S 242$}

二阶线的直径相交于一点, 我们来看三阶线直径间的关系, 考虑对于同一根轴 $A P$ 倾角不同的另一种纵标线. 记横标为 $t$, 纵标为 $u$, 我们有 $y=n u, x=t-m u$, 把这两个值 代入通用方程
\[
y^{3}+\beta y^{2} x+\gamma y x^{2}+\delta x^{3}+\varepsilon y^{2}+\zeta y x+\eta x^{2}+\theta y+\omega x+\kappa=0
\]
得到方程
\[
\left.\begin{array}{l}
+n^{3} u^{3}+\beta n^{2} u^{2} t+\gamma n u t^{2}+\delta t^{3}+\varepsilon n^{2} u^{2}+\zeta n u t+\eta t^{2}+\theta n u+\iota t+\kappa \\
-\beta m n^{2} u^{3}-2 \gamma m n u^{2} t-3 \delta m u t^{2}-\zeta m n u^{2}-2 \eta m u t-\imath m u \\
+\gamma m^{2} n u^{3}+3 \delta m^{2} u^{2} t+\eta m^{2} u^{2} \\
-\delta m^{3} u^{3}
\end{array}\right\}
\]
从该方程我们看到, 横标为 $t$ 时, 记新直径上点的坐标为 $v$, 则
\[
3 v=\frac{-\beta n^{2} t+2 \gamma m n t-3 \delta m^{2} t-\varepsilon n^{2}+\zeta m n-\eta m^{2}}{n^{3}-\beta m n^{2}+\gamma m^{2} n-\delta m^{3}}
\]
\section{$\S 243$}

设图 45 上点 $O$ 为这两个直径的交点, 从点 $O$ 向轴 $A Z$ 引 平行于原纵标和新纵标的直线 $O P$ 和 $O Q$, 则 $A P=x, P O=$ $z, A Q=t, O Q=v$, 此时我们有
\[
z=n v, \quad x=t-m v
\]
因而


【图,待补】
%%![](https://cdn.mathpix.com/cropped/2023_02_05_1df1519cd49abfa9c645g-08.jpg?height=263&width=412&top_left_y=1743&top_left_x=1103)

图 45
\[
v=\frac{z}{n}, \quad t=x+\frac{m}{n} z
\]
这样,首先我们有 $3 z=-\beta x-\varepsilon$, 进而有
\[
3 v=-\frac{\beta x}{n}-\frac{\varepsilon}{n}, \quad t=x-\frac{\beta m x}{3 n}-\frac{\varepsilon m}{3 n}
\]
把这两个值代入前面的方程, 得
\[
\left.\begin{array}{l}
-\beta n^{2} x+\beta^{2} m n x-\beta \gamma m^{2} x+\frac{\beta \delta m^{3} x}{n} \\
-\varepsilon n^{2}+\beta \varepsilon m n-\gamma \varepsilon m^{2}+\frac{\delta \varepsilon m^{3}}{n} \\
+\beta n^{2} x-\frac{\beta^{2} m n x}{3}-\frac{\beta \varepsilon m n}{3}+\varepsilon n^{2} \\
-2 \gamma m n x+\frac{2 \beta \gamma m^{2} x}{3}+\frac{2 \gamma \varepsilon m^{2}}{3}-\zeta m n \\
+3 \delta m^{2} x-\frac{\beta \delta m^{3} x}{n}-\frac{\delta \varepsilon m^{3}}{n}+m^{2}
\end{array}\right\}=0
\]
也即
\[
\left.\begin{array}{c}
\frac{2}{3} \beta^{2} m n x-\frac{1}{3} \beta \gamma m^{2} x-2 \gamma m n x+3 \delta m^{2} x \\
+\frac{2}{3} \beta \varepsilon m n-\frac{1}{3} \gamma \varepsilon m^{2}-\zeta m n+\eta m^{2} \\
\oint \mathbf{2 4 4}
\end{array}\right\}=0
\]
可见直径的交点 $O$ 依赖于纵标对轴的倾角, 这倾角决定于量 $m$ 和 $n$. 因此, 如果称直 径的交点为中心, 则三阶线不一定有中心. 但可以找到有中心的三阶线. 分别令方程中含 $m n$ 和含 $m m$ 的项的和为零, 使得到的两个 $x$ 值相等, 得
\[
x=\frac{3 \zeta-2 \beta \varepsilon}{2 \beta^{2}-6 \gamma}=\frac{3 \eta-\gamma \varepsilon}{\beta \gamma-9 \delta}
\]
要使这一等式成立,则要求
\[
6 \beta^{2} \eta-2 \beta^{2} \gamma \varepsilon-18 \gamma \eta+6 \gamma^{2} \varepsilon=3 \beta \gamma \zeta-2 \beta^{2} \gamma \varepsilon-27 \delta \zeta+18 \beta \delta \varepsilon
\]
或
\[
\beta \gamma \zeta-2 \beta^{2} \eta-9 \delta \zeta+6 \gamma \eta+6 \beta \delta \varepsilon-2 \gamma^{2} \varepsilon=0
\]
由此得
\[
\eta=\frac{\beta \gamma \zeta-9 \delta \zeta+6 \beta \delta \varepsilon-2 \gamma^{2} \varepsilon}{2 \beta^{2}-6 \gamma}
\]
当 $\eta$ 取这个值时,所有的直径就相交于同一点,也即这样的三阶线有中心,其坐标为
\[
A P=\frac{3 \zeta-2 \beta \varepsilon}{2 \beta^{2}-6 \gamma}, \quad P O=\frac{-\beta \zeta+2 \gamma \varepsilon}{2 \beta^{2}-6 \gamma}
\]
\section{$\S 245$}

第一项系数 $\alpha$ 不为 1 时,如果有中心, 我们来求这中心的坐标, 此时三阶线的最通用 方程为
\[
\alpha y^{3}+\beta y^{2} x+\gamma y x^{2}+\delta x^{3}+\varepsilon y^{2}+\zeta x y+\eta x^{2}+\theta y+\omega x+\kappa=0
\]
当 

\[
\eta=\frac{\beta \gamma \zeta-9 \alpha \delta \zeta+6 \beta \delta \varepsilon-2 \gamma^2 \varepsilon}{2 \beta^2-6 \alpha \gamma}
\]
时曲线有中心,记为 $O, O$ 的坐标为
\[
A P=\frac{3 \alpha \zeta-2 \beta \varepsilon}{2 \beta^{2}-6 \alpha \gamma}, \quad P O=\frac{2 \gamma \varepsilon-\beta \zeta}{2 \beta^{2}-6 \alpha \gamma}
\]
因而,一条与曲线有三个交点的纵标线, 如果它被分点分成这样两部分, 一部分的两个纵 标值之和等于另一部分第三个纵标值, 那么通过分点与中心的直线, 将以同样的方式分 所有平行于这条纵标线的纵标线.

\section{$\S 246$}

把上面所讲用到我们列出的各类方程上去,那么,显然 $\alpha=0$ 时,第一到五类有中心, 中心在横标原点. 第六、七类没有中心, 因为系数 $\alpha$ 不能为零. 第八到十三类有中心, 都在 横标原点. 最后三类的中心在无穷远处, 因而它们的直径都相平行.

\section{$\S 247$}

对三个纵标值, 前面讨论了它们的和, 现在来看看它们的积, 至于两两之积的和, 没 有发现它有什么值得注意处. 从 $\$ 239$ 的通用方程得
\[
-P M \cdot P L \cdot P N=-\delta x^{3}-\eta x^{2}-\omega x-\kappa
\]
为了从该表达式得到结果, 我们注意, $y=0$ 时有 $\delta x^{3}+\eta x^{2}+\omega x+\kappa=0$. 该方程的根给出轴 $A Z$ 与曲线的交点, 记这交点为 $B, C, D$, 则
\[
\delta x^{3}+\eta x^{2}+\omega+\kappa=\delta(x-A B)(x-A C)(x-A D)
\]
从而
\[
P L \cdot P M \cdot P N=\delta \cdot P B \cdot P C \cdot P D
\]
如果另取一条与第一条平行的纵标线 $l m n$, 则
\[
P L \cdot P M \cdot P N: P B \cdot P C \cdot P D=p l \cdot p m \cdot p m: p B \cdot p C \cdot p D
\]
这完全类似于二阶线积的性质,四阶、五阶和更高阶线也都具有类似的性质.

\section{$\S 248$}

假定三阶线有三条渐近直线 $F B f, G D g, H C h$, 如图 46. 三阶线方程可以分解为状如 $p y+q x+r$ 的三个线性因式的时候,三阶线成为这三条渐近线, 成为复合线, 可以由一个 特殊方程给出. 该特殊方程与三阶线方程最高次项相同, 继而由于渐近线的位置由方程 的第二项决定, 所以渐近线方程与三阶线方程第二项也相同. 因此, 如果三阶线关于轴 $A P$, 横标 $A P=x$, 纵标 $P M=y$ 间方程形状为
\[
y^{3}+(\beta x+\varepsilon) y^{2}+\left(\gamma x^{2}+\zeta x+\theta\right) y+\delta x^{3}+\eta x^{2}+\omega x+\kappa=0
\]
则渐近线关于同一根轴 $A P$, 横标 $A P=x$, 纵标 $P G=z$ 间的方程形状为 

$z^3+(\beta x+\varepsilon) z^2+\left(\gamma x^2+\zeta x+B\right) Z+\delta x^3+\eta x^2+C x+D=0$

系数 $B, C, D$ 要使得该方程可分解成三个线性因式.


【图,待补】
%%![](https://cdn.mathpix.com/cropped/2023_02_05_1df1519cd49abfa9c645g-11.jpg?height=467&width=589&top_left_y=408&top_left_x=548)

图 46

\section{$\S 249$}

如果画一条纵标线 $P N$, 交曲线于 $L, M, N$, 交渐近线于 $F, G, H$, 那么, 由曲线方 程得
\[
P L+P M+P N=-\beta x-\varepsilon
\]
由渐近线方程得
\[
P F+P G+P H=-\beta x-\varepsilon
\]
由这两个等式得
\[
P L+P M+P N=P F+P G+P H
\]
或
\[
F L-G M+H N=0
\]
如果画另外一条纵标线 $p f$,那么同样会得到
\[
f n-g m+h l=0
\]
也即, 如果一条直线与曲线与渐近线的交点都是三个, 那么这根直线上从渐近线上交点 走向曲线上交点, 方向相同的两段之和等于方向不同的另一段.

\section{$\S 250$}

从而,渐近线和收敛于渐近线的分支都是三条,这样的三阶线不能位于渐近线的同 侧, 如果两部分位于渐近线的一侧, 则第三部分必定位于另一侧. 因而不会有图 47 上这 样的三阶线. 因为分别交渐近线和曲线于 $f, g, h$ 和 $l, m, n$ 的直线上, 位于渐近线同侧的 $f n, g m, h l$ 三段之和不可能为零. 位于一侧的符号为正, 位于另一侧的符号为负, 位于同 侧的符号相同, 其和不能为零. 

【图,待补】
%%![](https://cdn.mathpix.com/cropped/2023_02_05_1df1519cd49abfa9c645g-12.jpg?height=674&width=954&top_left_y=24&top_left_x=346)

图 47

\section{$\S 251$}

由以上所讲我们清楚地看出, 为什么三阶线不能同时有两条状如 $u=\frac{A}{t^{2}}$ 和一条状如 $u=\frac{A}{t}$ 的渐近直线. 这是因为状如 $u=\frac{A}{t^{2}}$ 和 $u=\frac{A}{t}$ 的双曲分支,其逼近于自己渐近线的速 度之比是无穷. 事实上, 假定图 46 上直线 $f l$ 是在无穷远处, 线段 $f n, g m, h l$ 都成了无穷 小. 又假定分支 $n x, m y$ 的形状为 $u=\frac{A}{t^{2}}$, 分支 $l z$ 的形状为 $u=\frac{A}{t}$, 那么线段 $f n, g m$ 与 $h l$ 相 比为无穷小, 因而 $g m=f n+h l$ 不可能成立.

\section{$\S 252$}

这样一来, 渐近线的条数等于阶数的高阶线, 不可能只有一条状如 $u=\frac{A}{t}$ 的渐近线, 而其余的渐近线都是 $t$ 的次数更高的, 诸如 $u=\frac{A}{t^{2}}, u=\frac{A}{t^{3}}, \cdots$ 的. 有一条状如 $u=\frac{A}{t}$ 的,则 必有另一条形状和它类似的. 同理, 如果没有状如 $u=\frac{A}{t}$ 的渐近线, 那也就不能只有一条 状如 $u=\frac{A}{t^{2}}$ 的, 要有, 则至少是两条, 这是因为对自己渐近线的逼近速度, 状如 $u=\frac{A}{t^{3}}, u=$ $\frac{A}{t^{4}}, \cdots$ 的双曲分支, 远远地快于状如 $u=\frac{A}{t^{2}}$ 的双曲分支. 这样在计算高阶线所含类别时, 就可以容易地去掉一些不可能的情形,从而少做大量的计算.

\section{$\S 253$}

假定三阶线与某直线只有两个交点, 而与所有平行于这第一条直线的直线, 都或者 只有两个交点, 或者没有交点. 任取一根轴, 取纵标 $y$ 平行于第一条直线, 则该三阶线的方程, 其形状为
\[
y^{2}+\frac{\left(\gamma x^{2}+\zeta x+\theta\right) y}{\beta x+\varepsilon}+\frac{\delta x^{3}+p x^{2}+\omega x+\kappa}{\beta x+\varepsilon}=0
\]
记横标 $A P$ 为 $x$, 则有两个纵标 $y, P M$ 与 $-P N$, 见图 48. 由方程的性质我们有
\[
P M-P N=\frac{-\gamma x^{2}-\zeta x-\theta}{\beta x+\varepsilon}
\]
记等分弦 $M N$ 的点为 $O$, 则
\[
P O=\frac{1}{2} \frac{\gamma x^{2}+\zeta x+\theta}{\beta x+\varepsilon}
\]
从而, 令 $P O=z$, 则
\[
z(\beta x+\varepsilon)=\frac{1}{2}\left(\gamma x^{2}+\zeta x+\theta\right)
\]
由此得知, 等分平行于 $M N$ 的弦的点 $O$, 全都在双曲线上, 这里要求 $\gamma x^{2}+\zeta x+\theta$ 不被 $\beta x+\varepsilon$ 整除, 否则 $O$ 在直线上.


【图,待补】
%%![](https://cdn.mathpix.com/cropped/2023_02_05_1df1519cd49abfa9c645g-13.jpg?height=275&width=500&top_left_y=1002&top_left_x=588)

图 48

\section{$\S 254$}

$\gamma x^{2}+\zeta x+\theta$ 被 $\beta x+\varepsilon$ 整除时, 曲线有直径, 也即有直线等分所有平行于 $M N$ 的弦. 这 是全体二阶线所具有的性质. $\gamma x^{2}+\zeta x+\theta$ 被 $\beta x+\varepsilon$ 整除时, 将 $x=\frac{-\varepsilon}{\beta}$ 代入, 它应为零. 因 而,如果 $\gamma_{\varepsilon}^{2}-\beta \varepsilon \zeta+\beta^{2} \theta=0$, 则三阶线有直径.

\section{$\S 255$}

根据上面所讲, 我们可以用最一般化的方法, 定出三阶线有直径的所有情形, 设通用 方程为
\[
\alpha y^{3}+\beta y^{2} x+\gamma y x^{2}+\delta x^{3}+\varepsilon y^{2}+\zeta y x+\eta x^{2}+\theta y+\omega x+\kappa=0
\]
则纵标 $y$ 有三个或一个值, 因而没有直径, 改变对原轴的倾角, 另取纵标 $u$, 使 $y=n u, x=$ $t-m u$. 代入原方程, 得 
\[
\begin{aligned}
& +\alpha n^{3} u^{3}+\beta n^{2} u^{2} t+\gamma n u t^{2}+\delta t^{3}+\varepsilon n^{2} u^{2}+\zeta n u t+\theta n u+c t+\kappa \\
& -\beta m n^{2} u^{3}-2 \gamma m n u^{2} t-3 \delta m u t^{2}-\zeta m n u^{2}-2 \eta m u t-\imath m u \\
& +\gamma m^{2} n u^{3}+3 \delta n^{2} u^{2} t+\eta m^{2} u^{2} \\
& -\delta m^{3} u^{3}
\end{aligned}
\]
在新纵标之下, 为使曲线有直径, 首先要纵标有两个值,即三次项不出现, 由此得
\[
\alpha n^{3}-\beta m n^{2}+\gamma m^{2} n-\delta m^{3}=0
\]
\section{$\S 256$}

还要 $u$ 的系数
\[
(\gamma n-3 \delta m) t^{2}+(\zeta n-2 \eta m) t+\theta n-m
\]
被 $u^{2}$ 的系数
\[
\left(\beta n^{2}-2 \gamma m n+3 \delta m^{2}\right) t+\varepsilon n^{2}-\zeta m n+\eta m^{2}
\]
除得尽, 即把
\[
t=\frac{-\varepsilon n^{2}+\zeta m n-\eta m^{2}}{\beta n^{2}-2 \gamma m n+3 \delta m^{2}}
\]
代入 $u$ 的系数得零, 由此得
\[
\begin{aligned}
\iota= & \frac{\theta n}{m}-\frac{(\zeta n-2 \eta m)\left(\varepsilon n^{2}-\zeta m n+\eta m^{2}\right)}{\left(\beta n^{2}-2 \gamma m n+3 \delta m^{2}\right) m}+ \\
& \frac{(\gamma n-3 \delta m)\left(\varepsilon n^{2}-\zeta m n+\eta m^{2}\right)^{2}}{\left(\beta n^{2}-2 \gamma m n+3 \delta m^{2}\right)^{2} m}
\end{aligned}
\]
\section{$\S 257$}
将上面所讲应用到我们的十六类上去, 得: 第一类没有直径. 第二类有一条直径, 它 等分平行于 $x$ 轴的弦. 第三类没有直径. 第四类有一条直径, 它等分平行于渐近线的弦. 第五类有三条直径,它们等分平行于每条渐近线的弦. 第六类不能有直径. 第七类恒有一 条直径, 它等分平行于一条渐近线的弦, 这条渐近线产生于因式 $x-m y$. 第八类有一条 直径, 它等分平行于轴的弦. 第九类有两条直径,一条是关于平行于轴的弦, 一条是关于 平行于另一根渐近线的弦. 第十类同于第八类. 第十一类同于第九类. 第十二类同于第八 类. 第十三类同于第九类. 第十四类有一条直径, 是关于平行于轴的弦的. 第十五、十六类 没有与曲线交于两点的弦, 因而没有直径. 直径的这些性质, 牛顿已经得到, 他的这项工 作应该在这里指出.

\section{$\S 258$}

对前面各类三阶线方程, 虽然我们都假定 $x, y$ 为直角坐标, 但变换 $x, y$ 为任何斜角 坐标, 并不改变曲线的类别. 变直角坐标为任何斜角坐标时, 一个方程给出的趋向无穷的分支的个数不变; 当然这分支的性质也不变. 双曲分支保持为双曲的, 抛物分支保持为抛 物的; 双曲分支和抛物分支的种类也不变. 可见, 任何一个属于第一类的曲线, 不管其方 程是在直角坐标还是在斜角坐标之下, 它都属于第一类, 别的类也全是这样.

\section{$\S 259$}

既然坐标角可以任取, 对前面给出的方程, 我们可以换 $y$ 为 $\nu u$, 换 $x$ 为 $t-\mu u$, 其中 $\mu^{2}+\nu^{2}=1$. 可选择坐标角, 使方程简化. 下面是化简了的斜角 $t, u$ 之间的各类方程

第一类
\[
u\left(t^{2}+n^{2} u^{2}\right)+a u^{2}+b t+c u+d=0
\]
其中 $n \neq 0, b \neq 0$.

第二类
\[
u\left(t^{2}+n^{2} u^{2}\right)+a u^{2}+c u+d=0
\]
其中 $n \neq 0$.

第三类
\[
u\left(t^{2}-n^{2} u^{2}\right)+a u^{2}+b t+c u+d=0
\]
其中 $n \neq 0, b \neq 0, \pm n b+c+\frac{a^{2}}{4 n^{2}} \neq 0$.
\[
\begin{gathered}
\text { 第四类 } \\
u\left(t^{2}-n^{2} u^{2}\right)+a u^{2}+c u+d=0
\end{gathered}
\]
其中 $n \neq 0, c+\frac{a^{2}}{4 n^{2}} \neq 0$.

第五类
\[
u\left(t^{2}-n^{2} u^{2}\right)+a u^{2}-\frac{a^{2} u}{4 n^{2}}+d=0
\]
其中 $n \neq 0$.

第六类
\[
t u^{2}+a t^{2}+b t+c u+d=0
\]
其中 $a \neq 0, c \neq 0$.

第七类
\[
t u^{2}+a t^{2}+b t+d=0
\]
其中 $a \neq 0$.

第八类
\[
t u^{2}+b^{2} t+c u+d=0
\]
其中 $b \neq 0, c \neq 0$.

第九类
\[
t u^{2}+b^{2} t+d=0
\]
其中 $b \neq 0$. 

\author{
第十类 \\ $t u^{2}-b^{2} t+c u+d=0$
}

其中 $b \neq 0, c \neq 0$.
\[
\begin{gathered}
\text { 第十一类 } \\
t u^{2}-b^{2} t+d=0
\end{gathered}
\]
其中 $b \neq 0$.
\[
\begin{gathered}
\text { 第十二类 } \\
t u^{2}+c u+d=0
\end{gathered}
\]
其中 $c \neq 0$.
\[
\begin{gathered}
\text { 第十三类 } \\
t u^{2}+d=0 \\
\text { 第十四类 } \\
u^{3}+a t^{2}+c u+d=0 \\
\text { 第十五类 } \\
u^{3}+a t u+b t+d=0
\end{gathered}
\]
其中 $a \neq 0$.
\[
\begin{gathered}
\text { 第十六类 } \\
u^{3}+a t=0
\end{gathered}
\]
