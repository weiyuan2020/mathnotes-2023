\chapter{第十三章 曲线的性质}

\section{$\S 285$}

前几章我们描述伸向无穷远分支性质时,所用方法是指出在无穷远处与曲线重合的 直线或曲线, 指出的曲线比原曲线要简单得多. 本章我们仿照前面的做法, 对有界区域中 曲线的任一部分进行讨论. 也即寻求直线或比所给曲线简单得多的曲线, 使在很小范围 内与原曲线重合. 首先, 显然曲线的切线在切点处与曲线重合, 至少有两个公共点. 也可 以有另外的曲线, 与曲线的给定部分符合得更好, 有了这样的直线和曲线就可以清楚曲 线任何一个局部的形状和性质.

\section{$\S 286$}

假定有了某曲线的 $x, y$ 间方程. 给横标一个值 $A P=p$, 见图 55 , 求出对应的 $y$ 值. 如 果这 $y$ 值不只一个, 任取一个 $P M=q, M$ 是曲线上一点, 是曲线通过的一点. 换 $x, y$ 间方 程中的 $x$ 和 $y$ 为 $p$ 和 $q$, 则方程的项相抵消, 为零. 为考察曲线上点 $M$ 所在部分的性质, 从 $M$ 引平行于 $A P$ 的直线 $M q$. 取 $M q$ 为轴, 记新横标 $M q$ 为 $t$, 新纵标 $q m$ 为 $u$, 由于点 $m$ 也在 曲线上, 延长 $m q$ 至原来的轴上点 $p$, 则换方程中的 $x$ 和 $y$ 为 $A p=p+t$ 和 $p m=q+u$, 方程 的项也抵消, 为零.


【图,待补】
%%![](https://cdn.mathpix.com/cropped/2023_02_05_b169e65bf9064e07ce6cg-09.jpg?height=361&width=624&top_left_y=1519&top_left_x=509)

图 55

\section{$\S 287$}

代换得到的结果, 其中既不含 $t$ 也不含 $u$ 的项相抵消, 只剩下含新坐标 $t$ 或 $u$ 的项, 即 结果方程为
\[
0=A t+B u+C t^{2}+D t u+E u^{2}+F t^{3}+F t^{2} u+H t u^{2}+\cdots
\]
其中 $A, B, C, D, \cdots$ 是常数, 它们由原方程的常数和 $p, q$ 构成. 我们的 $p, q$ 也为常数. 新方 程也表示所给曲线的性质, 新方程的坐标以 $M q$ 为轴, 以 $M$ 为原点.

\section{$\S 288$}

首先, 如果令 $M q=t=0$, 则 $q m=u=0$, 此时点 $m$ 与点 $M$ 重合. 其次, 我们要考察的只 是曲线上靠近点 $M$ 的很小的一部分, 因此我们把 $t$ 取得尽可能的小. 此时 $q m=u$ 的值也 将极小. 事实上我们要考察的是几乎要消失的弧 $M m$ 的性质. 显然, 取 $t, u$ 为尽可能小的 值, 则 $t^{2}, t u, u^{2} ; t^{3}, t^{2} u, t u^{2}, u^{3} ; \cdots$, 随着次数的增加, 将一步一步地更小. 因而可以舍去 $A t+B u$ 以外的项, 得 $A t+B u=0$. 这是过点 $M$ 的直线 $M_{\mu}$ 的方程. 当点 $m$ 很靠近点 $M$ 时, 这直线与曲线重合.

\section{$\S 289$}

此时直线 $M \mu$ 是曲线在点 $M$ 处的切线, 由此得到曲线上任何一点 $M$ 处切线 $\mu M T$ 的 求法,由方程 $A t+B u=0$ 得
\[
\frac{u}{t}=-\frac{A}{B}=\frac{q \mu}{M q}
\]
从而
\[
q \mu: M q=M P: P T=-A: B
\]
进而, 由 $P M=q$ 得 $P T=-\frac{B q}{A}$. 称轴上 $P T$ 这一部分为次切距, 由此我们得到求次切距的 规则.

先从方程求出对应于横标 $x=p$ 的纵标 $y=q$. 将 $x=p+t, y=q+u$ 代入方程, 只保留 代换结果中次数为 1 的项, 去掉其余的项, 得方程 $A t+B u=0$. 从该方程得 $A, B$,最后得 次切距 $P T=-\frac{B q}{A}$.

例 1 设所给曲线为抛物线, 方程为 $y^{2}=2 a x$, 这里 $A P$ 为主轴, $A$ 为顶点.

令 $A P=p, P M=q$, 则 $q^{2}=2 a p$, 从而 $q=\sqrt{2 a p}$. 将 $x=p+t, y=q+u$ 代入方程, 得
\[
q^{2}+2 q u+u^{2}=2 a p+2 a t
\]
根据规则, 只保留 $2 q u=2 a t$, 从而
\[
a t-q u=0, \quad \frac{u}{t}=\frac{a}{q}=-\frac{A}{B}
\]
由于 $q^{2}=2 a p$ 得次切距 $P T=\frac{q^{2}}{a}=2 p$, 即次切距 $P T$ 是横标 $A P$ 的两倍.

例 2 设曲线为中心在 $A$ 的椭圆,方程为
\[
y^{2}=\frac{b^{2}}{a^{2}}\left(a^{2}-x^{2}\right) \text { 或 } a^{2} y^{2}+b^{2} x^{2}=a^{2} b^{2}
\]
取 $A P=p, P M=q$, 得 $a^{2} q^{2}+b^{2} p^{2}=a^{2} b^{2}$. 将 $x=p+t, y=q+u$ 代入方程, 保留结果中次数 为 1 的项, 去掉所有其余的项, 得
\[
2 a^{2} q u+2 b^{2} p t=0
\]
从而
\[
\frac{u}{t}=-\frac{b^{2} p}{a^{2} q}=-\frac{A}{B}
\]
最后得次切距
\[
P T=-\frac{B}{A} q=-\frac{a^{2} q^{2}}{b^{2} p}=\frac{-a^{2}+p^{2}}{p}
\]
该表达式为负, 这表明点 $T$ 在另一面. 又该表达式与前面我们给出的椭圆切线定义是相 符合的.

例 3 设所给为第七类三阶线, 其方程为
\[
y^{2} x=a x^{2}+b x+c
\]
取 $A P=p, P M=q$, 得 $p q^{2}=a p^{2}+b p+c$. 将 $x=p+t, y=q+u$ 代入方程, 得
\[
(p+t)\left(q^{2}+2 q u+u^{2}\right)=a\left(p^{2}+2 p t+t^{2}\right)+b(p+t)+c
\]
去掉可忽略的项,得 $2 p q u+q^{2} t=2 a p t+b t$, 从而
\[
\frac{u}{t}=\frac{2 a p+b-q^{2}}{2 p q}=-\frac{A}{B}
\]
最后得次切距
\[
P T=-\frac{B}{A} q=\frac{2 p q^{2}}{2 a p+b-q^{2}}=\frac{2 a p^{2}+2 b p+2 c}{2 a p+b-q^{2}}=\frac{2 a p^{3}+2 b p^{2}+2 c p}{a p^{2}-c}
\]
或
\[
\begin{gathered}
P T=\frac{2 p^{2} q^{2}}{a p^{2}-c} \\
\oint 290
\end{gathered}
\]
\section{$\S 290$}

用这种方法确定了曲线在点 $M$ 处的切线, 也就知道了曲线在点 $M$ 处的方向. 视曲线 为点连续改变方向时所走过的路径, 多有助益. 画出曲线 $M m$ 的动点在点 $M$ 处的方向是 切线 $M_{\mu}$. 如果保持这个方向, 那它画出的将是直线 $M_{\mu}$, 方向改变, 画出的是曲线. 因而 要知道曲线, 就要知道它每点处切线的位置. 切线位置可用我们刚讲过的方法确定. 如果 方程有理且不含分式, 确定切线位置将没有任何困难. 我们的方程都是可化成有理且不 含分式的. 如果方程无理, 或者含有分式, 不能化为有理整式, 也仍然可以利用刚讲过的 方法, 但要做些修正, 这修正使微分学产生. 因此曲线方程不是有理整式时, 切线求法留 给微分学.

\section{$\S 291$}

用前面的方法可以确定切线 $M \mu$ 对轴, 或对平行于轴的直线 $M q$ 的倾角. 如果坐标是 直角的, 则 $\angle M q \mu$ 为直角, 此时由 $q \mu: M q=-A: B$ 得 $\angle q M \mu$ 的正切等于 $-\frac{A}{B}$; 如果坐标 是斜角的, 那么从给定 $\angle M q \mu$ 及边 $M q, q \mu$ 的比, 利用三角学知识可以求出 $\angle q M \mu$. 显然, 如果求得的方程 $A t+B u=0$ 中的 $A=0$, 则 $\angle q M \mu$ 为零. 从而切线平行于轴 $A P$. 如果 $B=$ 0 , 则切线平行于纵标 $P M$, 也即纵标 $P M$ 是曲线上点 $M$ 处的切线.

\section{$\S 292$}

求得了切线 $M T$, 过切点 $M$ 引切线的垂线 $M N$, 则 $M N$ 是曲线的法线. $M N$ 的位置在 任何情况下都易于确定, 直角坐标时尤其容易. 此时 $\triangle M q \mu$ 和 $\triangle M P N$ 相似, 从而
\[
M q: q \mu=M P: P N \text { 或 }-B: A=q: P N
\]
进而
\[
P N=-\frac{A q}{B}
\]
轴上纵标线与法线之间的这一部分,即 $P N$, 通常称它为次法距. 直角坐标时, 次法距易 于从次切距 $P T$ 求得, 此时
\[
P T: P M=P M: P N \text { 或 } P N=\frac{P M^{2}}{P T}
\]
又, $\angle A P M$ 为直角时, 切线
\[
M T=\sqrt{P T^{2}+P M^{2}}
\]
法线
\[
M N=\sqrt{P M^{2}+P N^{2}}
\]
或者,由 $P T: T M=P M: M N$, 得
\[
\begin{gathered}
M N=\frac{P M \cdot T M}{P T}=\frac{P M}{P T} \sqrt{P T^{2}+P M^{2}} \\
\S 293
\end{gathered}
\]
\section{$\S 293$}

我们看到了,方程 $A t+B u=0$ 中的 $A=0$ 时,切线平行于轴, $B=0$ 时,切线平行于纵 标线. 再一种情况是 $A, B$ 都为零, 此时考虑 $\$ 286$ 方程中次数为 2 的项. 由于 $A t+B u$ 为 零, 次数为 2 的项不能略去. 考虑方程 $0=C t^{2}+D t u+E u^{2}$, 略去了更高次项. $t, u$ 无穷小 时, 与二次项相比较, 更高次项可略去. 从该方程, 跟从通用方程一样, 显然, 令 $t=0$, 则 $u=0$, 即 $M$ 是曲线上的点,这与原来的假设是一致的.

\section{$\S 294$}

这样方程 $0=C t^{2}+D t u+E u^{2}$ 表示曲线在点 $M$ 附近的情形. 显然, 如果 $D^{2}<4 C E$, 则 只要 $t \neq 0$, 且 $u \neq 0$, 方程就是虚的. 此时点 $M$ 当然属于曲线,但它离开曲线其余部分而 单独存在, 是缩成了一点的共轭卵形线, 是前章指出过的情形. 此时谈不上切线, 切线有 相邻两点与曲线共有, 直线不能是曲线的缩成一点部分的切线. 这样, 共轭点如果存在, 可用此法得到,并把它与曲线其余的点区别开来.

\section{$\S 295$}

如果 $D^{2}>4 C E$, 则方程 $0=C t^{2}+D t u+E u^{2}$ 可分解为状如 $\alpha t+\beta u=0$ 的两个方程, 都 表示曲线的性质. 它们都给出点 $M$ 处切线的位置, 也即给出曲线上点 $M$ 处的方向. 可见, 曲线的两个分支在点 $M$ 处相交, 即点 $M$ 是二重点. 如图 56 所示, 如果取 $M q=t$, 那么 $q \mu$, $q$ 就是从方程 得到的两个 $u$ 值. 而直线 $M_{\mu}, M_{\nu}$ 就是曲线点 $M$ 处的两条切线. 因而点 $M$ 是两个分支的交点, 方向分别为 $M_{\mu}$ 和 $M_{\nu}$. 可视共轭点为二重点, 方程 $C t^{2}+D t u+$ $E u^{2}=0$ 恒指明二重点, 这跟方程 $A t+B u=0$ 恒指明单重点是一样的.


【图,待补】
%%![](https://cdn.mathpix.com/cropped/2023_02_05_b169e65bf9064e07ce6cg-13.jpg?height=357&width=369&top_left_y=918&top_left_x=663)

图 56

\section{$\S 296$}

如果 $D^{2}=4 C E$, 则切线 $M_{\mu}, M_{\nu}$ 重合, $\angle \mu M_{\nu}$ 为零. $\angle \mu M_{\nu}$ 为零表明,两个分支在点 $M$ 处不仅相交, 且方向相同, 因而相切. 此时应该认为通过点 $M$ 的直线交曲线于两点, 即 点 $M$ 依然是二重点. 这样, 只要 $\S 286$ 所得方程 中开始的两个系数 $A, B$ 为零, 我们就可 以断定曲线有二重点. 这二重点有三种类型: 或者卵形线缩为一点, 即共轭点; 或者两个 分支相交于结点; 或者曲线的两个分支相切. 二重点的这三种类型, 由方程 $0=C t^{2}+$ $D t u+E u^{2}$ 的三种情形决定.

\section{$\S 297$}

如果系数 $A, B$ 为零的同时, 系数 $C, D, E$ 也为零, 则取 $t, u$ 次数和为 3 的项,得 $F t^{3}+$ $G t^{2} u+H t u^{2}+I u^{3}=0$. 如果该方程 只有一个实线性因式, 那么这因式指明, 过点 $M$ 有唯 一的分支, 并指明该分支的方向, 即切线. 而剩下的两个线性虚因式指明点 $M$ 为消失的 卵形线. 如果三个因式都是实的,则曲线的三个分支或相交于点 $M$, 或在点 $M$ 处相切. 这 决定于三个实因式相异或相等. 无论相交还是相切, 点 $M$ 都是曲线的三重点, 并且应该认为过点 $M$ 的直线与曲线有三个交点.

\section{$\S 298$}

如果再加上系数 $F, G, H, I$ 也为零, 那么为揭示曲线在点 $M$ 处的性质, 须考察方程 中 $t, u$ 次数和为 4 的项, 此时点 $M$ 为四重点. 该四重点 $M$ 处: 或者两个共轭卵形线相合, 这时四次方程的根全为虚数; 或者曲线的两个分支与共轭点相交或相切, 这时四次方程 的根两实两虚; 或者曲线的四个分支相交, 这时四次方程的根全为实数; 如果两个, 三个 或四个根相等, 相应地, 则两条, 三条, 四条分支的相交变为相切. 类似地, 如果 $t, u$ 次数 和为 4 的项也都为零, 则应考虑次数和为 5 或次数和更高的项.

\section{$\S 299$}

对不只经过点 $M$, 且以点 $M$ 为单重, 二重, 三重, 或更多重点的, 其方程易于利用上 面所得结果求出. 记 $A P=p, P M=q$, 设 $P, Q, R, S, \cdots$ 是坐标 $x, y$ 的函数,则方程
\[
P(x-p)+Q(y-q)=0
\]
表示的是通过点 $M$ 的曲线, 这是因为只要 $P$ 不含因式 $y-q, Q$ 不含因式 $x-p$, 也即只要 用除法消不去使曲线通过点 $M$ 的因式 $x-p, y-q$, 那么我们令 $x=A P=p$, 就得到 $y=$ $P M=q$. 显然, 通过点 $M$ 的曲线都包含在方程 $P(x-p)+Q(y-q)=0$ 之中, 如果该方程 不是下面我们马上推出的多重点形状,点 $M$ 就是单重点.

\section{$\S 300$}

如果点 $M$ 为二重点, 则过它的曲线的方程含于通用形式
\[
P(x-p)^{2}+Q(x-p)(y-q)+R(y-q)^{2}=0
\]
之中. 这里假定这通用形式不能被除法破坏. 由这通用形式显然可见, 二阶线不能有二重 点. 因为要这通用形式为二阶方程, 则 $P, Q, R$ 必为常数, 而那时这将不是曲线方程, 而是 两条直线的方程. $P, Q, R$ 为状如 $\alpha x+\beta y+\gamma$ 的一次函数时, 方程表示的是以 $M$ 为二重点 的三阶线. 三阶线, 只要它不是三条直线, 它的二重点就不能多于一个. 假定它有两个二 重点, 那么过这两点的直线就交三阶线于四点, 这与三阶线的性质矛盾. 四阶线可以有两 个二重点,五阶线的二重点不能多于三个.

\section{$\S 301$}

若 $M$ 是曲线的三重点,则曲线的性质由方程
\[
P(x-p)^{3}+Q(x-p)^{2}(y-q)+R(x-p)(y-q)^{2}+S(y-q)^{3}=0
\]
表示. 如果该方程确定的是一条曲线, 那么其阶数必定大于 3 , 阶数等于 3 , 则 $P, Q, R, S$必为常数, 而那时方程有三个状如 $\alpha(x-p)+\beta(y-q)$ 的因式, 是三条直线的方程. 因而 阶数低于 4 的曲线没有三重点. 五阶线的三重点不能多于一个, 否则将有交五阶线于六 个点的直线. 六阶线有两个三重点, 这是可以的.

\section{$\S 302$}

如果方程的形状为
\[
\begin{aligned}
& P(x-p)^{4}+Q(x-p)^{3}(y-q)+R(x-p)^{2}(y-q)^{2}+ \\
& S(x-p)(y-q)^{3}+T(y-q)^{4}=0
\end{aligned}
\]
则 $M$ 为曲线的四重点. 有四重点的最简单的曲线是五阶线. 八阶线或更高阶线可以有两 个四重点. 类似地, 我们可以列出以点 $M$ 为五重或任何重点的曲线的通用方程.

\section{$\S 303$}

如果点 $M$ 是二重、三重、或任何别的重数的重点, 则点 $M$ 处必定: 或者重数那么多条 分支相交或相切; 或者相交分支的条数少于重数, 而有一个或几个共轭点. 具体的, 可用 前面讲过的方法确定. 也即, 将函数 $P, Q, R, S, \cdots$ 中的 $x$ 和 $y$ 换为 $p$ 和 $q$, 而将 $d-x$ 和 $y-q$ 换为 $t$ 和 $u$. 借助这样得到的方程, 我们就可以确定曲线在点 $M$ 处的状态及在点处 相交的各分支的切线. 

