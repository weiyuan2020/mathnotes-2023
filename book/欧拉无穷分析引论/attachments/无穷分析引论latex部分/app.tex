\part{附 录}
\chapter{第一章 物体的表面}
\section{$\S 1$}
前面讲的曲线性质和用方程表示曲线的方法, 适用范围甚广, 适用于其点都在同一 张平面上的一切曲线. 但是当曲线上的点不全在一张平面上时, 要建立这种曲线的性质, 前面讲的就不够用了. 点不在同一张平面上的曲线有双重曲率. 关于双重曲率, 非凡的几 何学家克莱罗有一篇杰出的论文. 双重曲率与本附录要讲的曲面的性质, 关系密切, 因而 不单讲,而是结合曲面一起讲.

\section{$\S 2$}

线有直线曲线之分, 面也有平面曲面之别. 我们把凸面、凹面和既凸且凹的面统称为 曲面. 例如, 球面、柱面和无底雉面都是凸面, 茶杯的内表面是凹面. 我们知道其任何三点 都在同一条直线上的线是直线. 类似地, 其任何四点都在同一张平面上的面是平面. 可 见, 一个面只要有四个点不在同一张平面上, 它就是曲面, 也即凸面或凹面.

\section{$\S 3$}

知道了曲面上各点对同一张平面的偏离程度, 也就知道了这张曲面. 曲线的性质用 曲线各点至作为轴的直线的距离表示. 类似地, 曲面的性质用曲面各点至任取的一张平 面的距离表示. 这样, 给定一张曲面, 要确定它的性质, 就应该任取一张平面, 并从曲面的 各点向所取平面引垂线. 然后, 如果能找到一个方程, 用它确定这每条垂线的长, 那么曲 面的性质就由这个方程表示. 反之, 从这个方程可以求出曲面的各点, 也即求出曲面.

\section{$\S 4$}
假定取墙面作为从曲面向它引垂线的平面, 如图 119 所示, 在墙面上取 $A B$ 作轴, 取 $A$ 作原点. 设点 $M$ 是曲面上不在墙面上的一点. 从 $M$ 向墙面引垂线, 交墙面于 $Q$. 再从 $Q$ 向直线 $A B$ 引垂线 $Q P$, 交 $A B$ 于 $P$. 这样, 点 $M$ 的位置就由 $A P, P Q, Q M$ 这三条直线的长 所完全确定. 同样地, 所给曲面上任何另外的点 $M$, 其位置也都由三个相垂直的坐标决 定. 这类似于位于平面上的曲线, 其各点位置都由相垂直的两个坐标决定.


【图,待补】
%%![](https://cdn.mathpix.com/cropped/2023_02_05_c02c95ceeae1a896f812g-19.jpg?height=282&width=524&top_left_y=460&top_left_x=552)

图 119

\section{$\S 5$}

这样, 我们有三个坐标, $A P, P Q, Q M$, 令 $A P=x, P Q=y, Q M=z$. 从任两个坐标, 比 如 $x$ 和 $y$ 的值, 我们都可以根据曲面的性质求出第三个坐标 $z$ 的值. 用这样的方法我们能 够求出曲面上的每一点. 因而曲面的性质都能用一个方程表示, 方程使坐标 $z$ 由另外两 个坐标 $x, y$ 和常数决定. 可见给定一个曲面也就给定了一个函数 $z, z$ 的变量是 $x$ 和 $y$. 反 之, 如果 $z$ 是 $x$ 和 $y$ 的一个函数, 则这方程就代表一个曲面, 并给出这曲面的性质. 事实 上, 以它们所能取的一切值替 $x$ 和 $y$, 包括正值和负值, 我们就得到取定的平面上的所有 的点 $Q$. 然后对每一点 $Q$ 利用含 $x, y$ 和 $z$ 的方程, 我们都可以求出对应的垂线 $Q M=z$ 的 长, 也即求出所有的点 $M$. 点 $M, z$ 为正时在平面 $A P Q$ 的后面, $z$ 为负时在平面 $A P Q$ 的前 面, $z$ 为零时在平面 $A P Q$ 上. $z$ 为虚数时点 $Q$ 不对应曲面上任何点 $M . z$ 有多个值时, 过 $Q$ 的垂线交曲面于多点.

\section{$\S 6$}

关于曲面的性质, 首先我们会问它是连续的(规则的), 还是不连续的(不规则的). 其 所有的点都由 $x, y, z$ 的同一个方程表示, 即 $z$ 都是 $x$ 和 $y$ 的同一个函数, 这样的曲面为连 续的. 不同的部分函数不同, 这样的曲面为不规则的. 例如, 一个曲面, 它的一部分是球 面, 另一部分是雉面、柱面或平面, 它就是不规则的. 我们只考虑规则曲面, 也即只考虑由 单个方程表示的曲面. 清楚了规则曲面, 不规则曲面就容易讨论, 因为它的每一部分都是 规则曲面.

\section{$\S 7$}

规则曲面分为代数的和超越的. 性质由坐标 $x, y, z$ 间的代数方程表示, 也即 $z$ 是 $x, y$ 的代数函数, 这时的曲面叫代数曲面. 否则, 表示曲面性质的 $x, y, z$ 间方程, 包含对数, 或 包含依赖于圆弧等量的函数, 也即 $z$ 不是 $x, y$ 的代数函数, 这时的曲面叫超越曲面. 例如 $z=\log y, z=y^{x}, z=y \sin x$ 时, 曲面就是超越的. 当然, 我们先讨论代数曲面, 然后再讨论 超越曲面.

\section{$\S 8$}

对由 $x, y$ 的函数 $z$ 给出的曲面, 现在我们从 $z$ 的取值个数进行考虑. 先考虑 $z$ 为 $x, y$ 的单值函数时的曲面. 记 $x, y$ 的单值函数或有理函数为 $P$. 如果 $z=P$, 则平面上的每一点 $Q$ 都只对应曲面上的一个点, 也即平面 $A P Q$ 的任何一根垂线都只交曲面于一点. 此时直 线 $Q M$ 的值都为实数, 每个 $Q M$ 都给出曲面的一个实点, 但函数的取值个数并不反映对 应曲面的本质特点, 因为平面 $A P Q$ 和轴的位置都任意, 而同一个曲面, 它可以对一个 $A P Q$ 是单值的, 对另一个 $A P Q$ 是多值的.

\section{$\S 9$}

设 $P$ 和 $Q$ 都是 $x, y$ 的任意单值函数. 如果方程是 $x^{2}-P z+Q=0$, 则过点 $Q$ 的每根垂 线与曲面都或者交于两点, 或者不相交, 因为 $z$ 的两个值或者都为实数, 或者都为虚数. 类似地, 如果 $P, Q, R$ 表示 $x, y$ 的单值函数, 方程为 $z^{3}-P z^{2}+Q z-R=0$, 则 $z$ 为三值函 数. $z$ 的这三个值, 也即方程的根, 可以都为实数, 可以只有一个为实数, 另外两个为虚 数. 垂线 $Q M$ 与曲面的交点个数都同于实根个数, 为三个或一个. $z$ 所在方程次数更高时, 垂线 $Q M$ 与曲面的交点个数类似. $z$ 值的个数, 即方程根的个数, 只要将 $x, y, z$ 的方程化 为有理形式, 就不难确定.

\section{$\S 10$}

曲线方程的两个坐标可以交换, 类似地, 曲面方程的三个坐标 $x, y, z$ 也可以交换. 首 先, 如果取 $A P Q$ 上垂直于 $A P$ 的 $A p$ 作轴, 则 $A p=y, p Q=x$, 这样坐标 $x, y$ 就交换了位 置. 画出长方体 $A p Q M \xi \pi q P A$. 我们看它的过点 $A$ 的三个相垂直的平面 $A P Q p, A P q \pi$ 和 $A p \xi \pi$. 从点 $M$ 向这三个平面引的垂线都可以作为坐标 $z$, 即 $z$ 有三种可能的取法. 对 $z$ 的 这每一种取法, 在它所垂直的平面上, $x, y$ 又都有两种取法. $x, y, z$ 的不同的取法共六种, $A$ 是它们共有的原点, 这六种取法及与我们取定了的那一种的对应关系如下:

对平面 $A P Q p$, 为
\[
\begin{array}{lll}
A p=x, & P Q=y, & Q M=z \\
A p=y, & P Q=x, & Q M=z
\end{array}
\]
对平面 $A P q \pi$, 为
\[
\begin{array}{lll}
A P=x, & P q=z, & q M=y \\
A \pi=z, & \pi q=x, & q M=y
\end{array}
\]
%%14p261-280
对平面 $A p \xi \pi$, 为
\[
\begin{array}{lll}
A p=y, & p \xi=z, & \xi M=x \\
A \pi=z, & \pi \xi=y, & \xi M=x
\end{array}
\]
可见对不管哪种取法, 所得点 $M$ 所在曲面的关于 $x, y, z$ 的方程, 事实上都相同. 又连 接点 $A$ 与点 $M$ 的直线 $A M=\sqrt{x^{2}+y^{2}+z^{2}}$.

\section{$\S 11$}

这样坐标 $x, y, z$ 间的一个方程就给出曲面与交于点 $A$ 的相垂直的三个平面的关系. 变量 $z$ 给出曲面上点 $M$ 至平面 $A P Q$ 的距离. 类似地, $y, x$ 分别给出点 $M$ 至平面 $A P q$ 和 $A p \xi$ 的距离. 知道了点 $M$ 至三个平面的距离, 也就知道了它的位置, 可见首先应该画出借 助 $x, y, z$ 间方程建立曲面与它们之间关系的三个平面. 如果其中之一, 例如 $A P Q$ 是水平 的, 那么另外两个就是坚直的, 分别过直线 $A P$ 和 $A p$.

\section{$\S 12$}

画出相垂直的要建立曲面与它们之间关系的三个平面之后, 下一步是从曲面上的每 一点 $M$ 向这三个平面 $A P Q, A P q, A \pi \xi$ 引垂线 $Q M, M q, M \xi$, 则 $M Q=z, M q=y, M \xi=x$. 补成长方体, 则交于点 $A$ 的三条边分别等于所引垂线, 即 $A P=x, A p=y, A \pi=z$. 知道了 这三条线就知道了点 $M$ 的位置. 点 $M$ 所在位置如 $\S 4$ 图 119 时 $x, y, z$ 都为正. 连 $M A$, 点 $M$ 在 $M A$ 的延长线上时 $x, y, z$ 都为负.

\section{$\S 13$}

如果 $x, y, z$ 间的方程中, 垂直于平面 $A P Q$ 的那个变量, 也即 $z$ 的次数都为偶数, 那 么它的值都是成对的, 每对都大小相等, 一正一负. 因而曲面被分为两部分, 位于平面 $A P Q$ 的两侧, 形状相似, 至 $A P Q$ 的距离相等. 也即该曲面所围成的立体被平面 $A P Q$ 分 成相似且相等的两部分. 我们称分平面图形为相似相等两部分的直线为直径. 类似地, 我 们称分立体为相似相等两部分的平面为直径面. 因此, 如果方程中 $z$ 的次数都为偶数, 则 平面 $A P Q$ 为直径面.

\section{$\S 14$}

同样地,曲面方程中, 如果垂直于平面 $A P q$ 的变量 $y$, 其次数都为偶数,则平面 $A P q$ 是直径面. 如果变量 $x$ 的次数都为偶数, 则平面 $A p \xi$ 为直径面. 这样, 对任何曲面, 平面 $A P Q, A P q, A p \xi$ 中随便哪一个是否为直径面, 从以 $x, y, z$ 为变量的方程都一目了然. 三 个面中, 可以两个甚至三个都是直径面. 例如, 对以 $A$ 为心, $a$ 为半径的球, 由 $A M=$ $\sqrt{x^{2}+y^{2}+z^{2}}=a$ 得 $x^{2}+y^{2}+z^{2}=a^{2}$, 三个面都分它为相似且相等的两部分, 都是直 径面.

\section{$\S 15$}

图 120 画出了建立曲面方程所用的三张平面, 它们互相垂直, 交于 $A$, 分别记为 $Q Q^{1} Q^{2} Q^{3}, T T^{1} T^{2} T^{3}, V V^{1} V^{2} V^{3}$. 想象, 这三张平面都无穷扩展, 则整个空间被限制成八 部分, 或八限. 这八限用图上字母表示为 $A X, A X^{1}, A X^{2}, A X^{3}, A X^{4}, A X^{5}, A X^{6}, A X^{7}$. 如 果令三个变量 $x, y, z$ 在限 $I$, 即 $A X$ 中都为正,则在其余各限中 $x, y, z$ 里面都有或者一 个,或者两个,或者全部三个为负,下表列出了各限中变量的符号:
\[
\begin{array}{l|l|l|l|l}
\text { 限 } A X & \text { 限 } A X^{1} & \text { 限 } A X^{2} & \text { 限 } A X^{3} \\
A P=+x & A P^{\prime}=-x & A P=+x & A P^{\prime}=-x \\
A R=+y & A R=+y & A R=+y & A R=+y \\
A S=+z & A S=+z & A S^{\prime}=-z & A S^{\prime}=-z
\end{array}
\]
限 $A X^{4} \mid$ 限 $A X^{5} \mid$ 限 $A X^{6} \mid$ 限 $A X^{7}$

$A P=+x \quad A P^{\prime}=-x \quad A P=+x \quad A P^{\prime}=-x$

$A R^{\prime}=-y \quad A R^{\prime}=-y \quad A R^{\prime}=-y \quad A R^{\prime}=-y$

$A S=+z\left|A S=+z \quad A S^{\prime}=-z\right| A S^{\prime}=-z$


【图,待补】
%%![](https://cdn.mathpix.com/cropped/2023_02_05_a694b0f97e155a8ccaf8g-02.jpg?height=562&width=549&top_left_y=1265&top_left_x=568)

图 120

\section{$\S 16$}

为简单起见, 我们用数字标记这八限. 参见图 $121, A$ 是八限的共有点, 八限由交于点 $A$ 相垂直的三个平面分隔而成. 这三个平面由交于点 $A$ 相垂直的直线 $P p, Q q, R r$ 决定. 因而八限可以用字母 $P, Q, R$ 和它们的小写标明. 直线 $A P, A Q, A R$ 构成的长方体离 $A$ 无穷扩展而成的部分为基本部分, 或限 $I$, 记为 $P Q R$. 直线 $A P, A q, A r$ 的无穷延长线构成 的长方体部分, 记为 $P q r$. 如果令 $A P=x, A Q=y, A R=z$, 则 $A p=-x, A q=-y, A r=-$ $z$. 我们标记八限如下: 


【图,待补】
%%![](https://cdn.mathpix.com/cropped/2023_02_05_a694b0f97e155a8ccaf8g-03.jpg?height=333&width=340&top_left_y=286&top_left_x=661)

图 121




【图,待补】
%%![](https://cdn.mathpix.com/cropped/2023_02_05_a694b0f97e155a8ccaf8g-03.jpg?height=1127&width=962&top_left_y=863&top_left_x=357)

\section{$\S 17$}

八限之间的邻接程度有所不同. 首先, 有这样的限, 它们的坐标符号有两个是相同 的, 只有一个不同. 这样的两个限, 其邻接点构成一个面, 称它们为面邻. 其次, 两限的坐 标符号只有一个是相同的, 有两个不同. 其邻接点构成一条直线, 称这样的两限为线邻. 最后, 三个坐标的符号都相反, 这样的两限只在点 $A$ 一个点处相邻接, 称它们为点邻. 下 面的表给出了每限与其余各限之间的邻接关系:


【图,待补】
%%![](https://cdn.mathpix.com/cropped/2023_02_05_a694b0f97e155a8ccaf8g-04.jpg?height=818&width=1037&top_left_y=424&top_left_x=305)

\section{$\S 18$}

可见, 每限有面邻线邻各三个, 点邻一个. 任何两限之间的邻接关系, 表中都一目了 然. 值得注意的是表中表示限的数字的顺序. 为了使得这顺序更明显, 我们把前面表中的 数字单提出来构成一表如下:

\begin{tabular}{|l|l|l|l|l|l|l|l|}
\hline 1 & 2 & 3 & 4 & 5 & 6 & 7 & 8 \\
\hline 2 & 1 & 5 & 6 & 3 & 4 & 8 & 7 \\
\hline 3 & 5 & 1 & 7 & 2 & 8 & 4 & 6 \\
\hline 4 & 6 & 7 & 1 & 8 & 2 & 3 & 5 \\
\hline 5 & 3 & 2 & 8 & 1 & 7 & 6 & 4 \\
\hline 6 & 4 & 8 & 2 & 7 & 1 & 5 & 3 \\
\hline 7 & 8 & 4 & 3 & 6 & 5 & 1 & 2 \\
\hline 8 & 7 & 6 & 5 & 4 & 3 & 2 & 1 \\
\hline
\end{tabular}

后面我们将看到该表的用处. 

\section{$\S 19$}

前面提过, 曲面方程中 $z$ 的次数都为偶数时, 曲面由相似相等的两部分组成, 即限 1 限 2 部分相等, 且 3 与 5 两限部分, 4 与 6 两限部分, 7 与 8 两限部分也相等. 这相等部分所 在的对应限, 恰如上节表中开头为 1 和 2 的两行两列. 如果 $y$ 的次数都为偶数, 则相等部 分所在的对应限为: 1 与 3,2 与 5,4 与 7,6 与 $8 . x$ 的次数都为偶数时, 则相等部分所在的 对应限为: 1 与 4,2 与 6,3 与 7,5 与 8 .

各坐标次数为偶数时, 相等部分所在的对应限列出如下:


\[
\begin{array}{c|c|c}
z \text { 偶 } & y \text { 偶 } & x \text { 偶 } \\
\text { 对应限为 } & \text { 对应限为 } & \text { 对应限为 } \\
\hline 1,2,3,4,5,6,7,8 & 1,2,3,4,5,6,7,8 & 1,2,3,4,5,6,7,8 \\
2,1,5,6,3,4,8,7 & 3,5,1,7,2,8,4,6 & 4,6,7,1,8,2,3,5
\end{array}
\]

\section{$\S 20$}

要曲面的线邻限 1,5 部分相等,则方程应在 $y, z$ 都取负时不变,要 $y, z$ 都取负时不 变, 则各项中 $y, z$ 的次数之和都同偶或同奇. 如果 1 与 5 两限部分相等, 则 2 与 3,4 与 8,6 与 7 两限部分也相等. 同样的,如果曲面方程各项中 $x, z$ 的次数的和都同偶或同奇, 则 1 与 6,2 与 5,3 与 8,5 与 7 两限部分相等.

下面是曲面方程各项中次数和同偶或同奇的两个变量和相等部分所在的对应限:
\[
\begin{array}{c|c|c}
y, z \text { 同 } & x, z \text { 同 } & x, y \text { 同 } \\
\text { 对应限为 } & \text { 对应限为 } & \text { 对应限为 } \\
1,2,3,4,5,6,7,8 & 1,2,3,4,5,6,7,8 & 1,2,3,4,5,6,7,8 \\
5,3,2,8,1,7,6,4 & 6,4,8,2,7,1,5,3 & 7,8,4,3,6,5,1,2
\end{array}
\]
如果曲面方程各项中, 全体三个变量 $x, y, z$ 次数的和都同偶或同奇, 则曲面的下面 每两个点邻限部分相等
\[
\begin{aligned}
& 1,2,3,4,5,6,7,8 \\
& 8,7,6,5,4,3,2,1
\end{aligned}
\]
\section{$\S 21$}

如果前面条件中的 2 个或 3 个同时被满足, 则曲面在 4 个或 8 个限中的部分相等. 如果每项中 $x, y$ 各自的次数都为偶数, 则曲面下面每四个限中的部分相等
\[
\begin{aligned}
& 1,2,3,4,5,6,7,8 \\
& 3,5,1,7,2,8,4,6
\end{aligned}
\]
\[
\begin{aligned}
& 4,6,7,1,8,2,3,5 \\
& 7,8,4,3,6,5,1,2
\end{aligned}
\]
如果每项中 $x, z$ 各自的次数都为偶数, 则曲面下面每四个限中的部分相等
\[
\begin{aligned}
& 1,2,3,4,5,6,7,8 \\
& 2,1,5,6,3,4,8,7 \\
& 4,6,7,1,8,2,3,5 \\
& 6,4,8,2,7,1,5,3
\end{aligned}
\]
如果每项中 $y, z$ 各自的次数都为偶数, 则曲面下面每四个限中的部分相等
\[
\begin{aligned}
& 1,2,3,4,5,6,7,8 \\
& 2,1,5,6,3,4,8,7 \\
& 3,5,1,7,2,8,4,6 \\
& 5,3,2,8,1,7,6,4
\end{aligned}
\]
\section{$\S 22$}

如果每项中一个变量的次数都为偶数,另两个变量次数的和, 或者都为偶数, 或者都 为奇数,则曲面四个相等部分所在限分述如下:

如果 $z$ 的次数都为偶数, $x, y$ 次数的和或者都为偶数,或者都为奇数,则曲面下面每 四个限中的部分相等
\[
\begin{aligned}
& 1,2,3,4,5,6,7,8 \\
& 2,1,5,6,3,4,8,7 \\
& 7,8,4,3,6,5,1,2 \\
& 8,7,6,5,4,3,2,1
\end{aligned}
\]
如果 $y$ 的次数都为偶数, $x, z$ 次数的和或者都为偶数,或者都为奇数,则曲面下面每 四个限中的部分相等
\[
\begin{aligned}
& 1,2,3,4,5,6,7,8 \\
& 3,5,1,7,2,8,4,6 \\
& 6,4,8,2,7,1,5,3 \\
& 8,7,6,5,4,3,2,1
\end{aligned}
\]
如果 $x$ 的次数都为偶数, $y, z$ 次数的和或者都为偶数,或者都为奇数,则曲面下面每 四个限中的部分相等
\[
\begin{aligned}
& 1,2,3,4,5,6,7,8 \\
& 4,6,7,1,8,2,3,5 \\
& 5,3,2,8,1,7,6,4 \\
& 8,7,6,5,4,3,2,1
\end{aligned}
\]
这三种情况,各项中 $x, y, z$ 次数的和,都或者同为奇数, 或者同为偶数. 

\section{$\S 23$}

曲面四个限中部分相等的情况, 还有几种.

考虑 $x$ 与 $y, y$ 与 $z$ 这两个变量对, 如果每项中各对的次数和,都或者同为偶数, 或者 同为奇数,则曲面下面每四个限中的部分相等
\[
\begin{aligned}
& 1,2,3,4,5,6,7,8 \\
& 5,3,2,8,1,7,6,4 \\
& 7,8,4,3,6,5,1,2 \\
& 6,4,8,2,7,1,5,3
\end{aligned}
\]
加上每项中 $x, z$ 的次数和也或同偶, 或同奇, 并不增加相等限的组数. 这样, 如果方程各 项中每两个变量次数之和都或同偶或同奇, 则曲面的线邻限部分都相等. 这里的变量对 共三个,应该指出, 只要有两对具有所说性质,第三对就也具有.

\section{$\S 24$}

一种四部分相似相等的条件, 再加一种它不包含的两部分相似相等的条件, 就成为 八部分都相似相等的条件, 即同时满足这样两种条件时, 曲面由八个相等部分组成. 前面 提到的性质, 这种曲面的方程全都具有. 即全体三个变量各自的次数都全为偶数, 从而每 两个和全体三个的次数之和也都全为偶数.

\section{$\S 25$}

三元方程,单个变量, 其次数的奇偶容易看出, 看三个变量次数之和的奇偶也不难, 但每两个的次数之和的奇偶, 就不那么简单. 这时我们或者将 $x=n z$, 或者将 $y=n z$, 或者

将 $x=n y$ 代入方程, 然后看所得方程中变量 $z$ (前两种代入) 或 $y$ (第三种代入) 的次数是 否全为偶数. 是, 则两个变量次数之和必或同偶, 或同奇, 因而曲面至少两部分相似相等. 

\chapter{第二章 曲面与平面的交线}

\section{$\S 26$}

线的交是点. 类似地, 面的交是线, 是直线或曲线. 从初等几何中我们已经知道, 两张 平面的交是一条直线, 球面与平面的交是圆. 曲面与平面的交对认识曲面很有帮助. 交线 是曲面的无穷个点构成的集合, 而前章讨论的变量的个别值, 只给出曲面的个别点.

\section{$\S 27$}

我们用相垂直的三个平面作曲面的参照, 因而我们先讨论曲面与这三张平面的交线 (图 121). 先看平面 $A P Q$, 它由变量 $A P=x, A Q=y$ 确定. 变量 $z$ 表示曲面到 $A P Q$ 的距 离. 显然 $z=0$ 时得到的是曲面的位于平面 $A P Q$ 上的点, 也即所得 $x, y$ 间的方程表示曲面 与平面 $A P Q$ 的交线. 类似地, 置 $y=0$, 则所得 $x, z$ 间的方程, 表示曲面与平面 $A P R$ 的交 线; 而置 $x=0$, 则所得 $y, z$ 间方程, 表示曲面与平面 $A Q R$ 的交线.

\section{$\S 28$}

前面指出过, 以 $A$ 为心, $a$ 为半径的球面, 其方程为 $x^{2}+y^{2}+z^{2}=a^{2}$. 现在就用球面作 例子来解释前节的交线. $z=0$ 时, 得方程 $x^{2}+y^{2}=a^{2}$, 它是球面与平面 $A P Q$ 的交线, 是以 $A$ 为心, $a$ 为半径的圆. 同样地, $y=0$ 时的方程 $x^{2}+z^{2}=a^{2}$, 是球面与平面 $A P R$ 的交线, 是 圆; $x=0$ 时的方程 $y^{2}+z^{2}=a^{2}$ 是球面与平面 $A Q R$ 的交线, 也是圆. 这里的结论是显然的, 过球心的平面截球面所得为最大圆,也即与球同心同半径的圆.

\section{$\S 29$}

平行于垂直三平面之一的平面与曲面的交线也不难确定. 设平面与 $A P Q$ 平行, 距离 为 $h$, 则曲面上距 $A P Q$ 为 $z=h$ 的点就都在这平行平面上. 这些点自然也就构成了曲面与 平行平面的交线. 因而置曲面方程中的 $z=h$ 就得到这条交线的方程. 事实上, 得到的是 两个直角坐标 $x, y$ 间的方程. 球面与平行于 $A P R$ 的和与平行于 $A Q R$ 的平面的交线, 都 可用同样的方法得到, 这里不再重复. 

\section{$\S 30$}

这样, 令三元曲面方程中的 $z$ 为常数 $h$, 得到的就是曲面与平行于 $A P Q$ 的平面的交 线方程. $h$ 是平行平面至 $A P Q$ 的距离, 让 $h$ 取一切可能的正值和负值,就得到曲面与一切 平行于 $A P Q$ 的平面的交线. 这交线的条数无穷, 整个曲面被它们分成了无穷多相平行的 部分,因而曲面可由这无穷多条交线决定. 这无穷多条交线都由 $x, y$ 间的同一个方程表 示 (方程中含有一个可取一切正值和一切负值的 $h$ ), 因而它们是相似的, 至少是仿射的 (参见第十八章).

\section{$\S 31$}

如果 $x, y$ 的方程不因 $h$ 的改变而改变, 则曲面与平行于 $A P Q$ 的一切平面的交线全 都相同. 不因 $h$ 而改变, 则必不含 $h$, 也即不含 $z$. 这样, 如果曲面方程不含 $z$, 则曲面与平 行于 $A P Q$ 的一切平面的交线都相同. 这时曲面方程只含 $x$ 和 $y$ 两个变量, 本身就是交线 方程. 同样的,曲面方程不含 $x$, 则曲面与平行于 $A Q R$ 的一切平面的交线都相同; 不含 $y$, 则曲面与平行于 $A P R$ 的一切平面的交线都相同.

\section{$\S 32$}

这种曲面不只易于想象,易于画出来,也易于用实际的 材料把它构造出来. 假定方程不含 $z$, 也即只含坐标 $A P=x$ 和 $A Q=P M=y$. 依据方程先在平面 $A P Q$ 上画曲线 $B M D$, 如图 122 所示. 想象一根无穷直线, 始终垂直于平面 $A P Q$, 让它沿曲线 $B M D$ 移动, 就形成不含 $z$ 方程所表示的曲面. $B M D$ 为圆时,形成的是圆柱面; 为椭圆时, 形成的是压缩圆 柱面; 如果 $B M D$ 由几段直线首尾连接而成, 则形成的是棱 柱面.


【图,待补】
%%![](https://cdn.mathpix.com/cropped/2023_02_05_a694b0f97e155a8ccaf8g-09.jpg?height=337&width=428&top_left_y=1344&top_left_x=1069)

图 122

\section{$\S 33$}

我们把这类曲面统称之为柱面, 相应地, 图形 $B M D$ 都称为底. 曲面方程只要缺少变 量 $x, y, z$ 中的一个, 它表示的就必定是柱面. 如果同时缺少 $y, z$ 两个变量, 即 $x$ 为常数, 则 $B M D$ 成为垂直于 $A D$ 的直线, 曲面成为垂直于 $A P Q$ 的平面.

\section{$\S 34$}

接下去最值得注意的是由 $x, y, z$ 的齐次方程产生的曲面. 齐次, 即变量次数的和各 项都相同, 整齐划一. 例如 $z^{2}=m x z+x^{2}+y^{2}$ 就是齐次方程. 该方程决定的曲面与平行于 同一个主平面 (两条坐标线决定的平面, 共三个) 的平面的交线都相似. 例如 $z$ 取常数 $h$, 方程成为
\[
h^{2}=m h x+x^{2}+y^{2}
\]
对 $h$ 的无穷多个值, 该方程给出的无穷多条曲线都相似, 其参数都等于 $h$ 或与 $h$ 成比例. 也即这些交线相似, 且随平行平面至 $A P Q$ 的距离的增加而增大, 因而从点 $A$ 出发过各交 线同调点(第十八章 $\S 438)$ 的线都是直线.

\section{$\S 35$}

假定我们有一个这样的 $x, y, z$ 的齐次方程. 参见图 123 , 给 $z$ 一个值 $A R=h$, 设 $T S s M m$ 是方程表示的曲面与过点 $R$ 平行于 $A P Q$ 的平面的交线,点 $R$ 使 $R V=x, V M=y$ 满足 $x, y$ 的方程. 想象一条保持通过点 $A$ 的无限伸长的直线, 使这直线沿 $T S s M m$ 移动 就可给出方程所表示的曲面. 显然, $T S s M m$ 为以 $R$ 为心的圆时, 所得为正圆雉面; $T S s M m$ 为圆, 但 $R$ 不为心时, 所得为斜圆雉面; 如果 $T S s M m$ 由几个直线段组成, 则所得 为棱雉面. 因而, 我们称该方程所表示的为雉面.


【图,待补】
%%![](https://cdn.mathpix.com/cropped/2023_02_05_a694b0f97e155a8ccaf8g-10.jpg?height=409&width=469&top_left_y=1349&top_left_x=591)

图 123

\section{$\S 36$}

上节得到, 三元齐次方程表示的曲面是圆雉面或棱雉面. 曲面与平行于平面 $A P Q$ 的 平面的交线, 彼此相似, 且各交线的参数都与交线至点 $A$ 的距离成比例. 对平面 $A P R$ 和 $A Q R$ 分别重复同样的过程, 我们得到, 曲面与平行于 $A P R$ 和 $A Q R$ 的平面的交线, 各自都 具有同样的性质. 各自彼此相似, 且对应边与至点 $A$ 的距离成比例. 后面我们证明,这里 的曲面与过点 $A$ 的任一平面平行的各平面的交线, 都彼此相似, 且其参数与至顶点 $A$ 的距离成比例.

\section{$\S 37$}

现在讲一种更一般的曲面, 设 $Z$ 是 $z$ 的某个函数, 并假定给了我们一个 $x, y, Z$ 间的 齐次函数. 记 $z=h$ 时的 $Z=H$, 得到 $x, y, H$ 间的齐次方程, 因而曲面与平行于 $A P Q$ 的各 平面的交线彼此相似, 但其参数不是与距离 $h$, 而是与 $h$ 的函数 $H$ 成比例. 由此得到, 过 这些交线同调点的线不是直线, 而是依赖于函数 $Z$ 的曲线. 由此当然得不到, 曲面与平行 于任何另外一个平面的各平面的交线彼此相似.

\section{$\S 38$}

柱面雉面是上节曲面的特例. 事实上, 如果 $Z=z$ 或 $Z=\alpha z$, 则上节方程是 $x, y, z$ 的齐 次方程, 曲面为雉面. 类似地, 如果 $Z=\alpha+\beta z$, 结果相同, 只有一点区别 : 雉面的顶点不是 $A$. 如果 $Z=\frac{b-z}{b}$, 则雉面顶点至 $A$ 的距离为 $b$. 如果令 $b=\infty$, 则曲面不再是雉面, 而是柱 面, 这时 $Z=1$. 由此得, 柱面方程是变量 $x, y$ 和常数 1 的齐次方程. 而变量 $x, y$ 的齐次方 程, 不管怎样, 只要不含 $z$, 就都可以化成 $x, y$ 和 1 的齐次方程. 因而如我们已经看到的, 凡缺少一个变量的齐次方程, 它表示的就都是柱面.

\section{$\S 39$}

与平行于 $A P Q$ 的平面的交线都相似, 这样的曲面中特别值得注意的一种是, 交线都 为圆, 且圆心都在垂直于 $A P Q$ 的直线 $A R$ 上. 这种曲面可由旋转得到, 因而叫旋转面. 这 种旋转面的通用方程为 $Z^{2}=x^{2}+y^{2}$. 给定一个 $z$ 值, 就有一个 $Z=H$, 就得到一个交线方 程 $H^{2}=x^{2}+y^{2}$, 方程表明, 交线是半径为 $H$, 圆心在 $A R$ 上的圆. $Z^{2}=z^{2}$ 时, 是直圆雉面; $Z=a^{2}$ 时, 是圆柱面; $Z=a^{2}-z^{2}$ 时,是球面. 这是旋转面的几种主要形状.

\section{$\S 40$}

参见图 124, 我们讨论这样的曲面, 它同垂直于轴 $A P$ 的平面的交线 $P T V$, 都是顶点在平行于轴 $A P$ 的直 线 $D T$ 上的三角形. 设曲线 $A V B$ 为该曲面的底部, 也即 曲面同平面 $A P Q$ 的交. 又设直线 $T D$ 到轴 $A B$ 的距离, 也 即 $A D=c$. 照例引进坐标 $A P=x, P Q=y, Q M=z$, 则 $P V$ 是 $x$ 的一个函数, 记它为 $P$. 由 $\triangle V Q M$ 与 $\triangle V P T$ 相 似, 得


【图,待补】
%%![](https://cdn.mathpix.com/cropped/2023_02_05_a694b0f97e155a8ccaf8g-11.jpg?height=321&width=481&top_left_y=1860&top_left_x=1052)

图 124 

$P: c=(P-y): z$ 或 $z=c-\frac{c y}{P}$

从而对这类曲面我们有 $\frac{c-z}{y}$ 是 $x$ 的一个函数. 这类曲面与雉面的区别在于, 这类曲面的 顶峰是一条直线 $D T$, 而雉面的顶峰是一个点. 沃利斯对于其底部 $A V B$ 为圆的这类曲面 作了详细讨论, 并称之为楔顶雉.

\section{$\S 41$}

参见图 125, 跟前节一样,曲面与垂直于 $A B$ 的平面的交 线 $P T V$ 为直角三角形, $P$ 为直角, 不同的是, 顶点 $T$ 构成一条 曲线. 底部同于前节为 $A V B$, 记三个坐标为 $A P=x, P Q=y$, $Q M=z$, 则直线 $P V$ 沿曲线 $A V B$ 移动时构成 $x$ 的函数, 记为 $P$; 同时 $P T$ 也构成 $x$ 的函数, 记为 $Q$. 同于前节从相似三角形 得 $P: Q=(P-y): z$, 从而 $z=Q-\frac{Q y}{P}$, 也即 $P z+Q y=P Q$, 或 $\frac{z}{Q}=Q y=P Q$ 为常数.


【图,待补】
%%![](https://cdn.mathpix.com/cropped/2023_02_05_a694b0f97e155a8ccaf8g-12.jpg?height=354&width=386&top_left_y=749&top_left_x=1090)

图 125

因而 $y, z$ 的方程,如果次数不高于 1 , 它描述的曲面就属我们这里的类型.

\section{$\S 42$}

我们讨论过这样的曲面, 平行于同一参照面的平面 截它而成的交线都相似. 现在我们把叙述中的相似换成 仿射进行讨论. 仿射, 即取同调横标, 则纵标彼此成比 例. 参见图 126 , 设曲面与参照面的交线分别为 $A B C$, $A C D$ 和 $A B D$, 且曲面与平行于 $A C D$ 的面都仿射. 记 $A C D$ 的底 $A C=a$, 高 $A D=b$. 取坐标 $A q=p, q m=q$, 则 $q$ 是 $p$ 的一个函数. 考虑某个平行于 $A C D$ 的面的截线 $P T V$, 记 $A P=x$, 则底 $P V$ 和高 $P T$ 都是 $x$ 的函数, 分别 记为 $P$ 和 $Q$. 记 $P Q=y, Q M=z$, 则根据仿射规律有 $a :p=P: y, b: q=Q: z$, 从而 $y=\frac{P p}{a}, z=\frac{Q q}{b}$.

【图,待补】
%%![](https://cdn.mathpix.com/cropped/2023_02_05_a694b0f97e155a8ccaf8g-12.jpg?height=364&width=483&top_left_y=1419&top_left_x=1008)

图 126 

\section{$\S 43$}

可见: 与平行于 $A C D$ 所在平面的平面的交线都与 $A C D$ 仿射, 对这样的曲面, 只要知 道了它与参照面的交线 $A B C, A C D, A B D$, 就可以确定这曲面的性质. 事实上, 由于 $P, Q$是 $x$ 的函数, 以及 $q$ 是 $p$ 的函数, 因而可以用变量 $x$ 和 $p$ 表示变量 $y$ 和 $z$. 如果要求出坐标 $x, y, z$ 间的方程, 那么由 $q$ 是 $p$ 的函数知, $p, q$ 可由方程联系. 将 $p=\frac{a y}{P}, q=\frac{b x}{Q}$ 代入 $p, q$ 间 的方程, 得 $y, z$ 间的含 $P, Q$ 的方程. 而 $P, Q$ 都是 $x$ 的函数, 这样我们就得到了 $x, y, z$ 间的 方程, 所给曲面的性质就由这个方程表示. 显然 $x=0$, 则 $P=a, Q=b$.

\section{$\S 44$}

如果曲面方程各项 $y, z$ 次数的和都相同, 则曲面与垂直于 $A P$ 轴的平面的交线由直 线组成. 事实上, 以 $x$ 替任何常数我们都得到 $y, z$ 的齐次方程, 它给出一条或几条直线. $y, z$ 指数的和各项相同, 不管同奇同偶, 都属 $\S 20$ 讲过的情形. 根据那里的表知, 此时曲 面的 1,5 限部分, 2,3 限部分, 4,8 限部分, 6, 7 限部分都相等.

\section{$\S 45$}

我们讲了多种可由直线形成的曲面, 刚讲的以及柱面和雉面都是. 诚然,柱面和雉面 与过轴 $A P$ 的平面的交线是直线. 刚讲过的这一类更一般些. 事实上, 如图 127 所示, 设 $A K M P$ 是曲面与过轴 $A P$ 的平面的交线, 记 $\angle M P V$ 为 $\varphi$, 记 $A P=x, P Q=y, Q M=z$, 则 $\tan \varphi=\frac{z}{y}$, 直线 $P M=\frac{z}{\sin \varphi}$. 要 $K M$ 为直线, 则应 $\frac{z}{\sin \varphi}=\alpha x+\beta$, 其中 $\alpha, \beta$ 是只依赖于 $\varphi$, 不 依赖于 $y, z$ 的常数. 设 $R, S$ 是类似于 $\alpha, \beta$ 的函数, 则 $x=R z+S$ 或 $x=R y+S$. 或者设 $T, S$ 分别为 $y, z$ 的一次和零次函数, 则这类曲面的通用方程为 $x=T+S$.


【图,待补】
%%![](https://cdn.mathpix.com/cropped/2023_02_05_a694b0f97e155a8ccaf8g-13.jpg?height=359&width=565&top_left_y=1431&top_left_x=534)

图 127

\section{$\S 46$}

$x, y, z$ 间方程表示的曲面, 它与过 $A P$ 轴的平面的交线易于确定. 实际上, 记交线 $A K M P$ 对平面 $A C V P$ 的倾角为 $\varphi$, 记直线 $P M=v, P M$ 为所求交线的纵标, 则 $Q M=z=$ $v \sin \varphi, P Q=y=v \cos \varphi$. 换曲面方程中的 $y, z$ 为 $v \cos \varphi, v \sin \varphi$, 得到的就是交线 $A K M P$ 的 $x, v$ 间的方程. 类似地, 可以求出曲面与过轴 $A Q$ 和过轴 $A R$ 的平面的交线 (图 121). 三 个变元 $x, y, z$ 所依赖的这三根轴 $A P, A Q, A R$ 是可以交换位置的, 因而关于任何一根轴 的结论, 也都适用于另外两根轴.

\section{$\S 47$}

取平面 $A P Q$ 作参照面,即交线所在平面都或平行于它, 或倾斜于它. 换曲面方程中 的 $z$ 为一个常数, 所得就是平行时交线的方程. 倾斜时, 交线所在平面与 $A P Q$ 的交为直 线. 这直线为 $A P$ 或 $A Q$ 时曲面与平面的交线上节刚讲. 下面我们逐步推向一般的倾斜 情形.

\section{$\S 48$}

参见图 128 , 先设倾斜截平面与 $A P Q$ 的交线 $E S$ 平行于 轴 $A P$, 倾角 $\angle Q S M=\varphi$, 距离 $A E=f$. 由 $A P=x, P Q=y$, $Q M=z$ 知, $E S=x, Q S=y+f$. 如果取 $E S$ 作截线的轴, 则横 标 $E S=x$, 纵标 $S M=v$. 由 $\angle Q S M=\varphi$ 得 $Q M=z=v \sin \varphi$, $S Q=y+f=v \cos \varphi$, 从而 $y=v \cos \varphi-f$. 将
\[
y=v \cos \varphi-f, \quad z=v \sin \varphi
\]
代入曲面的 $x, y, z$ 间方程, 得坐标 $x, v$ 间方程. 这就是我们所求的平面 $E S M$ 截曲面的截线的方程. 如果 $E S$ 垂直于轴 $A P$, 则它平行于平面 $A P Q$ 上 另一根轴, 交换变量 $x, y$, 就可以用这里的方法求出截线.


【图,待补】
%%![](https://cdn.mathpix.com/cropped/2023_02_05_a694b0f97e155a8ccaf8g-14.jpg?height=301&width=428&top_left_y=922&top_left_x=1088)

图 128 

\section{$\S 49$}

参见图 129 , 现在设交线 $E S$ 在平面 $A P Q$ 上位置任 意. 自 $E$ 引 $A P$ 的垂线和平行线 $A E$ 和 $E X T$. 记 $A E=f$, $\angle T E S=\theta$, 取变量 $A P=x, P Q=y, Q M=z$. 自 $Q$ 向 $E S$ 引 垂线 $Q S$, 再连直线 $M S$, 则 $\angle Q S M$ 为截平面对 $A P Q$ 的倾 角, 我们记它为 $\varphi$. 再设截平面上的坐标为 $E S=t$, $S M=v$. 自 $S$ 分别向 $E S$ 和 $Q P$ 的延长线引垂线 $S T$ 和 $S V$, 则
\[
\begin{gathered}
Q M=z=v \sin \varphi, \quad Q S=v \cos \varphi, \quad S V=v \cos \varphi \sin \theta \\
Q V=v \cos \varphi \cos \theta
\end{gathered}
\]

【图,待补】
%%![](https://cdn.mathpix.com/cropped/2023_02_05_a694b0f97e155a8ccaf8g-14.jpg?height=404&width=447&top_left_y=1569&top_left_x=1050)

图 129

又
\[
S T=V X=t \sin \theta, \quad E T=t \cos \theta
\]
利用以上表达式最后得
\[
A P=x=t \cos \theta+v \cos \varphi \sin \theta, \quad P Q=y=v \cos \varphi \cos \theta-t \sin \theta-f
\]
将 $x, y, z$ 换成上面的表达式,就得到所求的截线方程.

\section{$\S 50$}

这样,有了曲面方程, 我们就可以求出曲面的任何一个截线的方程. 首先, 显然其方 程是 $x, y, z$ 间代数方程的曲面, 它的截线方程全是代数的. 其次, 由于表示截线的 $t, v$ 间 方程, 是将
\[
\begin{gathered}
z=v \sin \varphi, \quad x=t \cos \theta+v \cos \varphi \sin \theta \\
y=v \cos \varphi \cos \theta-t \sin \theta-f
\end{gathered}
\]
代入曲面方程而得, 所以截线方程的次数, 比曲面方程的不会高, 但可以低, 因为代入过 程中可能有抵消.

\section{$\S 51$}

如果面方程的次数为 1 , 即其形状为
\[
\alpha x+\beta y+\gamma z=a
\]
则面的截线为直线, 后面我们证明, 此时面为平面, 初等几何中我们知道, 两张平面的交 线为直线,而方程为
\[
\alpha x^{2}+\beta y^{2}+\gamma z^{2}+\delta x y+\varepsilon x z+\xi y z+a x+b y+c z+e^{2}=0
\]
的面, 其截线必为直线或二阶线, 即截线方程的次数绝不会高于 2 .

\chapter{第三章 柱面、锥面、球面的截线}

\section{$\S 52$}

柱、雉、球都是初等几何的讨论对象. 先对这三种曲面的截线进行讨论, 大有益于稍 一般些曲面的截线的讨论.

初等几何里讲了直、斜两种圆柱. 垂直于轴的截线是圆心在同一条直线上的相等的 圆, 这样的圆柱叫直圆柱. 相等的为圆的截面不垂直于轴, 而是与轴成某个另外的角度, 这样的圆柱叫斜圆柱. 斜圆柱的另一种描述是: 垂直于轴的截线是中心在同一条直线上 的相等的椭圆, 这样的圆柱叫斜圆柱, 称中心所在的直线为圆柱的轴.

\section{$\S 53$}

参见图 130 , 给定了一个圆柱, 直的或斜的, 轴垂直于墙面. 底, 也即圆柱与墙面的交 线为 $A E B F$, 它为圆或椭圆, 关于斜圆柱的结论都可以很容易地用到直圆柱上去, 因而我 们假定底 $A B E F$ 是以 $C$ 为中心 $A B, E F$ 为共轭轴的椭圆. 记半轴 $A C=B C=a, C E=$ $C F=c$. 取定坐标 $C P=x, P Q=y, Q M=z$, 则由椭圆性质得 $a^{2} c^{2}=a^{2} y^{2}+c^{2} x^{2}$. 这也是圆 柱面的方程. 平行于 $C P Q$ 的截线都相等,因而第三个变量 $z$ 在方程中不出现.


【图,待补】
%%![](https://cdn.mathpix.com/cropped/2023_02_05_a694b0f97e155a8ccaf8g-16.jpg?height=374&width=644&top_left_y=1462&top_left_x=492)

图 130

\section{$\S 54$}

可见,一个柱面的平行于底的截线都相等, 直柱面时为圆, 斜柱面时为椭圆. 柱面与 垂直于平面 $A P Q$ 的平面的截线:相交时, 为两条相平行的直线; 相切时,合为一条直线; 不相交时, 为虚的, 垂直于 $A P Q$ 的平面的方程为 $x$, 或 $y$, 或 $x \pm \alpha y$ 等于常数. 将它们代入柱面方程, 所得即为截线方程. 对截线方程进行分析, 就得到上面的结论. 这样, 我们求出 了柱面与平行于三个柱平面中任何一个的平面的交线.

\section{$\S 55$}

为考察其余的截线, 我们先假定截平面与底平面的交线 $G T$ 平行于 $E F$, 垂直于 $A B$, 交 $A B$ 的延长线于 $G$. 记 $C G=f$, 记截平面对底平面的倾角为 $\varphi$,记截平面与柱面之轴的 交点为 $D$. 连直线 $D G$, 则 $\angle D G C=\varphi$, 因而
\[
D G=\frac{f}{\cos \varphi}, \quad C D=\frac{f \sin \varphi}{\cos \varphi}
\]
从所求截线上任一点 $M$ 引平行于 $D G$ 的直线 $M T$, 由 $T Q=f-x, \angle Q T M=\varphi$, 得
\[
T M=\frac{f-x}{\cos \varphi}, \quad Q M=\frac{(f-x) \sin \varphi}{\cos \varphi}=z
\]
引平行于 $T G$, 因而垂直于 $D G$ 的直线 $M S$, 则
\[
M S=T G=P Q=y, \quad D S=\frac{x}{\cos \varphi}
\]
\section{$\S 56$}

取直线 $D S$ 和 $S M$ 作所求截线的坐标线, 记 $D S=t, S M=u$, 则 $y=u, x=t \cos \varphi$, 由 $z=$ $\frac{(f-x) \sin \varphi}{\cos \varphi}$ 得 $z=f \tan \varphi-t \sin \varphi$. 将这里的 $x$ 和 $y$ 代入柱面方程 $a^{2} c^{2}=a^{2} y^{2}+c^{2} x^{2}$, 得 所求截线的方程
\[
a^{2} c^{2}=a^{2} u^{2}+c^{2} t^{2} \cos ^{2} \varphi
\]
该方程表明, 截线是椭圆, 中心在点 $D$, 一根主轴在直线 $D G$ 上, 另一根垂直于 $D G$. 直线 $D G$ 上的半轴 ( $u=0$ 时) 为 $\frac{a}{\cos \varphi}$. 或者画平行于 $G D$ 的直线 $B H$, 则 $B H=\frac{a}{\cos \varphi}$ 是所求的 一根半轴,另一根与它共轭的半轴为 $c=C E$.

\section{$\S 57$}

上节得到的柱面截线是半轴长为 $\frac{a}{\cos \varphi}$ 和 $c$ 的椭圆.如果底 $A E B F$ 上 $A C$ 为长半轴, 则由 $\frac{a}{\cos \varphi}$ 大于 $a$ 知, 截线椭圆的长半轴比底面椭圆的长. 如果 $c>a$, 也即, 如果 $G T$ 平行 于底面椭圆的长半轴, 则截线椭圆的两个轴可以相等, 也即截线可以是圆. 为圆时 $\frac{a}{\cos \varphi}=c$, 也即 $\varphi=\frac{a}{c} . \triangle B C H$ 中 $C$ 为直角, $\angle C B H=\varphi$, 因而
\[
\cos \varphi=\frac{B C}{B H}=\frac{a}{B H}
\]
因此 $B H=C E$ 时截线为圆. 直线 $B H$, 也即为圆的截线, 可以在底面上方, 也可以在底面 下方, 称此时的柱面为斜的, 原因就是上方或下方的这两个圆所在的平面都倾斜于 轴 $C D$.

\section{$\S 58$}

参见图 131 , 考虑截平面与底平面交线 $G T$ 的位置任意的情形,从底的中心向 $G T$ 引 垂线 $G C=f$, 设 $\angle B C G=\theta, \angle C G D=\varphi$. 引 $Q T$ 垂直于 $G T$, 则 $\angle Q T M$ 等于 $\angle C G D$. 这样我 们有
\[
D G=\frac{f}{\cos \varphi}, \quad C D=\frac{f \sin \varphi}{\cos \varphi}
\]
设 $M$ 为所求截线上一点, 自 $M$ 向底引垂线 $M Q$, 再自 $Q$ 向轴引垂线 $Q P$. 记 $C P=x, P Q=$ $y, Q M=z$, 得 $a^{2} c^{2}=a^{2} y^{2}+c^{2} x^{2}$. 向 $G T$ 引垂线 $P V$ 和 $Q T$, 则
\[
G V=x \sin \theta, \quad P V=f-x \cos \theta
\]
由 $\angle Q P W=\theta$, 得
\[
Q W=y \sin \theta, \quad P W=V T=y \cos \theta, \quad Q T=f-x \cos \theta+y \sin \theta
\]
最后连直线 $M T$, 则由 $\angle M T Q=\varphi$ 我们有
\[
T M=\frac{z}{\sin \varphi}, \quad Q T=\frac{z \cos \varphi}{\sin \varphi}
\]

【图,待补】
%%![](https://cdn.mathpix.com/cropped/2023_02_05_a694b0f97e155a8ccaf8g-18.jpg?height=488&width=440&top_left_y=1271&top_left_x=594)

图 131

\section{$\S 59$}

补成四边形 $G S M T$, 记 $D S=t, S M=G T=u$, 则
\[
u=G V+V T=x \sin \theta+y \cos \theta
\]
由
\[
Q T=f-x \cos \theta+y \sin \theta
\]
得 
 $Q T-C G=y \sin \theta-x \cos \theta$

从而
\[
D S=T M-D G=\frac{y \sin \theta-x \cos \theta}{\cos \varphi}=t
\]
由
\[
x \sin \theta+y \cos \theta=u, \quad y \sin \theta-x \cos \theta=t \cos \varphi
\]
得
\[
y=u \cos \theta+t \sin \theta \cos \varphi, \quad x=u \sin \theta-t \cos \theta \cos \varphi
\]
将 $x, y$ 的这两个表达式代入 $a^{2} c^{2}=a^{2} y^{2}+c^{2} x^{2}$, 得
\[
\begin{aligned}
a^{2} c^{2}= & a^{2} u^{2} \cos ^{2} \theta+2 a^{2} u t \sin \theta \cos \theta \cos \varphi+a^{2} t^{2} \sin ^{2} \theta \cos ^{2} \varphi+ \\
& c^{2} u^{2} \sin ^{2} \theta-2 c^{2} u t \sin \theta \cos \theta \cos \varphi+c^{2} t^{2} \cos ^{2} \theta \cos \varphi
\end{aligned}
\]
显然, 这是以 $D$ 为中心的椭圆的方程. 但只要 $a \neq c$, 即柱面不是直的, 坐标 $D S, S M$ 就不 垂直于主轴.

\section{$\S 60$}

上节求出了截线的坐标 $D S=t, M S=u$ 间的方程. 为 进一步考察, 设截线如图 132 上的 $a M e b f$ 所示. 为简单 起见,记上节所得方程为
\[
a^{2} c^{2}=\alpha u^{2}+2 \beta t u+\gamma t^{2}
\]
其中
\[
\begin{gathered}
\alpha=a^{2} \cos ^{2} \theta+c^{2} \sin ^{2} \theta \\
\beta=\left(a^{2}-c^{2}\right) \sin \theta \cos \theta \cos \varphi \\
\gamma=a^{2} \sin ^{2} \theta \cos ^{2} \varphi+c^{2} \cos ^{2} \theta \cos ^{2} \varphi
\end{gathered}
\]

【图,待补】
%%![](https://cdn.mathpix.com/cropped/2023_02_05_a694b0f97e155a8ccaf8g-19.jpg?height=356&width=467&top_left_y=1100&top_left_x=1028)

图 132

又设该截线的共轭主轴为 $a b, e f$. 向其中之一引垂线 $M p$, 记 $D p=p, M p=q$, 记 $\angle a D H=$ $\zeta$, 则
\[
u=p \sin \zeta+q \cos \zeta, \quad t=p \cos \zeta-q \sin \zeta
\]
将 $t, u$ 的这两个表达式代入简记的方程, 得
\[
\begin{aligned}
a^{2} c^{2}= & +\alpha \sin ^{2} \zeta p^{2}+2 \alpha \sin \zeta \cos \zeta p q+\alpha \cos ^{2} \zeta q^{2}+2 \beta \sin \zeta \cos \zeta+ \\
& 2 \beta\left(\cos ^{2} \zeta-\sin ^{2} \zeta\right)-2 \beta \sin \zeta \cos \zeta+\gamma \cos ^{2} \zeta-2 \gamma \sin \zeta \cos \zeta+\gamma \sin ^{2} \zeta
\end{aligned}
\]
\section{$\S 61$}

现在方程是关于成直角的直径的, 因而 $p q$ 的系数应该为零. 于是由
\[
2 \sin \zeta \cos \zeta=\sin 2 \zeta, \quad \cos ^{2} \zeta-\sin ^{2} \zeta=\cos 2 \zeta
\]
得 $(\alpha-\gamma) \sin 2 \zeta+2 \beta \cos 2 \zeta=0$, 从而 $\tan 2 \zeta=\frac{2 \beta}{\gamma-\alpha}$, 这确定了 $\angle a D H$, 因而确定了主轴的 位置, 由此得半轴 
\[
\begin{aligned}
& \text { Frinile analyoio (无穷分析与论 . Fnixaduclian } \\
& a D=\frac{a c}{\sqrt{\alpha \sin ^{2} \zeta+2 \beta \sin \zeta \cos \zeta+\gamma \cos ^{2} \zeta}} \\
& e D=\frac{a c}{\sqrt{\alpha \cos ^{2} \zeta-2 \beta \sin \zeta \cos \zeta+\gamma \cos ^{2} \zeta}}
\end{aligned}
\]
\section{$\S 62$}

将
\[
2 \beta=\frac{2(\gamma-\alpha) \sin \zeta \cos \zeta}{\cos ^{2} \zeta-\sin ^{2} \zeta}
\]
代入半轴表达式, 得
\[
\begin{aligned}
a D & =\frac{a c \sqrt{\cos ^{2} \zeta-\sin ^{2} \zeta}}{\sqrt{\gamma \cos ^{2} \zeta-\alpha \sin ^{2} \zeta}}=\frac{a c \sqrt{2 \cos 2 \zeta}}{\sqrt{(\alpha+\gamma) \cos 2 \zeta-\alpha+\gamma}} \\
e D & =\frac{a c \sqrt{\cos ^{2} \zeta-\sin ^{2} \zeta}}{\sqrt{\alpha \cos ^{2} \zeta-\gamma \sin ^{2} \zeta}}=\frac{a c \sqrt{2 \cos 2 \zeta}}{\sqrt{(\alpha+\gamma) \cos 2 \zeta+\alpha-\gamma}}
\end{aligned}
\]
从而半轴的积
\[
a D \cdot e D=\frac{2 a^{2} c^{2} \cos 2 \zeta}{\sqrt{2 \alpha \gamma\left(1+\cos ^{2} 2 \zeta\right)-\left(\alpha^{2}+\gamma^{2}\right) \sin ^{2} 2 \zeta}}
\]
但由
\[
(\gamma-\alpha) \sin 2 \zeta=2 \beta \cos 2 \zeta
\]
得
\[
\left(\alpha^{2}+\gamma^{2}\right) \sin ^{2} 2 \zeta=4 \beta^{2} \cos ^{2} 2 \zeta+2 \alpha \gamma \sin ^{2} 2 \zeta
\]
从而
\[
a D \cdot e D=\frac{2 a^{2} c^{2} \cos 2 \zeta}{\sqrt{4 \alpha \gamma \cos ^{2} 2 \zeta-4 \beta^{2} \cos ^{2} 2 \zeta}}=\frac{a^{2} c^{2}}{\sqrt{\alpha \gamma-\beta^{2}}}=\frac{a c}{\cos \varphi}
\]
\section{$\S 63$}

类似地, 由
\[
\begin{aligned}
a D^{2} & =\frac{2 a^{2} c^{2} \cos 2 \zeta}{(\alpha+\gamma) \cos 2 \zeta-\alpha+\gamma} \\
e D^{2} & =\frac{2 a^{2} c^{2} \cos 2 \zeta}{(\alpha+\gamma) \cos 2 \zeta+\alpha-\gamma}
\end{aligned}
\]
得
\[
a D^{2}+e D^{2}=\frac{4 a^{2} c^{2}(\alpha+\gamma) \cos ^{2} 2 \zeta}{4 \alpha \gamma \cos 2 \zeta-4 \beta^{2} \cos ^{2} 2 \zeta}=\frac{(\alpha+\gamma) a^{2} c^{2}}{\alpha \gamma-\beta^{2}}
\]
从而
\[
a D+e D=\frac{a c \sqrt{\alpha+\gamma+2 \sqrt{\alpha \gamma-\beta^{2}}}}{\sqrt{\alpha \gamma-\beta^{2}}}
\]
%%15p281-300
\[
a D-e D=\frac{a c \sqrt{\alpha+\gamma-2 \sqrt{\alpha \gamma-\beta^{2}}}}{\sqrt{\alpha \gamma-\beta^{2}}}
\]
这样,半轴 $a D, e D$ 是方程
\[
\left(\alpha \gamma-\beta^{2}\right) x^{4}-(\alpha+\gamma) a^{2} c^{2} x^{2}+a^{4} c^{4}=0
\]
的根, 且我们有
\[
\sqrt{\alpha \gamma-\beta^{2}}=a c \cos \varphi 
\]
\section{$\S 64$}

由 $a D \cdot e D=\frac{a c}{\cos \varphi}, \varphi$ 为截平面与底平面的夹角, 我们得到下面这个漂亮的定理.

定理 平面截柱面,截线轴的积比柱面底的轴之积, 等于 1 比截平面与底平面夹角 的余弦.

共轭轴所成平行四边形与主轴所成矩形面积相等. 由此得, 柱面截线轴所成平行四 边形与底的轴所成平行四边形,其面积的比为常数.

\section{$\S 65$}

确定柱面斜截线性质, 对此我们还有更好些的办法. 参见图 133 , 柱底面 $A E B F$ 的半 轴 $A C=B C=a, E C=C F=c$; 直线 $C D$ 垂直柱底面于点 $C$, 为柱的轴; 直线 $T H$ 为截平面与 底平面的交线, $\angle G C H=\theta ; D$ 为截平面与柱的轴的交点; $\angle C H D$ 为截平面对底平面的倾 角, 记为 $\varphi$. 这样, 令 $C G=f$, 则
\[
G H=f \sin \theta, \quad C H=f \cos \theta, \quad D H=\frac{f \cos \theta}{\cos \varphi}, \quad C D=\frac{f \cos \theta \sin \theta}{\cos \varphi}
\]
从而由 $\triangle D C G$ 的 $C$ 为直角, 得
\[
D G=\frac{f \sqrt{1+\sin ^2 \theta \sin ^2 \varphi}}{\cos \varphi}
\]

【图,待补】
%%![](https://cdn.mathpix.com/cropped/2023_02_05_11864d1515e42e275d87g-01.jpg?height=543&width=615&top_left_y=1629&top_left_x=535)

图 133 

及 $\angle D G H$ 的正弦, 余弦, 正切分别为
\[
\frac{\cos \theta}{\sqrt{1-\sin ^{2} \theta \sin ^{2} \varphi}}, \frac{\sin \theta \cos \varphi}{\sqrt{1-\sin ^{2} \theta \sin ^{2} \varphi}}, \frac{\cos \theta}{\sin \theta \cos \varphi}
\]
\section{$\S 66$}

$M Q$ 为从截线上一点 $M$ 向底面所引垂线, $Q P$ 为纵标, 令 $C P=x, P Q=y$, 则 $a^{2} c^{2}=$ $a^{2} y^{2}+c^{2} x^{2}$. 引 $Q T$ 平行于 $C G, G R$ 垂直于 $Q T$. 我们有 $G R=y, Q R=f-x$. 由 $\angle T G R=$ $\angle G C H=\theta$, 得
\[
G T=\frac{y}{\cos \theta}, \quad T R=\frac{y \sin \theta}{\cos \theta}
\]
从而
\[
Q T=f-x+\frac{y \sin \theta}{\cos \theta}
\]
由 $\triangle C D G$ 和 $\triangle Q M T$ 相似得
\[
C G: D G=Q T: M T, \quad C G:(C G-Q T)=D G: D S
\]
引 $M S$ 平行 $G T$, 则
\[
D S=\frac{(x \cos \theta-y \sin \theta) \sqrt{1-\sin ^{2} \theta \sin ^{2} \varphi}}{\cos \theta \cos \varphi}
\]
令 $D S=t, M S=u$, 得
\[
x \cos \theta-x \sin \theta=\frac{t \cos \theta \cos \varphi}{\sqrt{1-\sin ^{2} \theta \sin ^{2} \varphi}}, \quad y=u \cos \theta
\]
由此即可导出 $t, u$ 间方程, 但仍嫌复杂.

\section{$\S 67$}

参见图 133a ${ }^{\oplus}$, 换底面主轴为平行于交线 $T H$ 的直径 $E F$ 和与它共轭的直径 $A B, A B$ 的延长线交 $T H$ 于 $G$, 其余跟前节一样, 即
\[
\begin{gathered}
C G=f, \quad \angle G C H=\theta, \quad \angle C H D=\varphi \\
C A=C B=m, \quad C E=C F=n
\end{gathered}
\]
引 $Q P$ 平行于直径 $E F$, 令
\[
C P=x, \quad P Q=y
\]
则 $m^{2} n^{2}=m^{2} y^{2}+n^{2} x^{2}$, 我们有
\[
G T=M S=y, \quad D S=x \frac{D G}{C G}=\frac{x \sqrt{1-\sin ^{2} \theta \sin ^{2} \varphi}}{\cos \varphi}
\]
(1) 此图为俄译本所加. 令 $D S=t, M S=u$, 得
\[
x=\frac{t \cos \varphi}{\sqrt{1-\sin ^{2} \theta \sin ^{2} \varphi}}, \quad y=u
\]
令 $\angle C G D=\eta$, 则由 $\cos \eta=\frac{C G}{D G}$ 得 $x=t \cos \eta$, 从而对所求截线得
\[
m^{2} n^{2}=m^{2} u^{2}+n^{2} t^{2} \cos ^{2} \eta
\]
这是椭圆关于共轭直径的方程, 中心在 $D, D S$ 方向上的半直径为 $\frac{m}{\cos \eta}$, 另一个半直径为 $n$. 两直径夹角即 $\angle G S M$ 的正切和余弦分别为
\[
\frac{\cos \theta}{\sin \theta \cos \varphi} \text { 和 } \frac{\sin \theta \cos \theta}{\sqrt{1-\sin ^{2} \theta \sin ^{2} \varphi}}=\sin \theta \cos \eta
\]
这样截线的性质就很清楚了.


【图,待补】
%%![](https://cdn.mathpix.com/cropped/2023_02_05_11864d1515e42e275d87g-03.jpg?height=550&width=414&top_left_y=881&top_left_x=645)

图 $133 \mathrm{a}$

\section{$\S 68$}

前面考察了柱面截线, 现在我们转向雉面, 直的和斜的. 直雉面斜雉面的区别只在垂 直于轴的截线, 前者为圆后者为椭圆. 参见图 134 , 雉面 $O a e b f O$, 顶点在 $O$, 轴为 $O c$, 想象 $O$ 在桌面上, $O c$ 垂直于桌面, $A B, E F$ 为桌面上过点 $O$ 分别平行 $a b$ 和 $e f$ 的直线, $a b, e f$ 为 垂直于雉面轴的平面与雉面的交线的轴. $M$ 为交线 $a e b f$ 上任意一点. $M Q$ 垂直于桌面, $P Q$ 垂直于 $A B$. 令 $O P=x, P Q=y, Q M=z$, 则交线的横标 $c p=x$, 纵标 $p M=y$. 轴 $a b, e f$ 与 $O c=Q M=z$ 的比都为常数, 令 $a c=b c=m z, c e=f c=n z$, 我们有
\[
m^{2} n^{2} z^{2}=m^{2} y^{2}+n^{2} x^{2}
\]
这是雉面的三元方程.


【图,待补】
%%![](https://cdn.mathpix.com/cropped/2023_02_05_11864d1515e42e275d87g-04.jpg?height=440&width=560&top_left_y=273&top_left_x=536)

图 134

\section{$\S 69$}

置方程 $m^{2} n^{2} z^{2}=m^{2} y^{2}+n^{2} x^{2}$ 中的 $z$ 为常数, 即可清楚地看出,垂直于轴 $O c$ 的平面与 雉面的交线都为椭圆. 类似地, 雉面与垂直于 $A B$ 或 $E F$ 的平面的交线, 也易于确定. 设平 面垂直于 $A B$, 过点 $P$. 令 $O P=a$, 得截线的以 $P p=z, p M=y$ 为变量的方程 $m^{2} n^{2} z^{2}=$ $m^{2} y^{2}+n^{2} a^{2}$. 显然, 这是双曲线方程, 中心为 $P$, 横半轴为 $\frac{a}{m}$, 它与共轭的半轴为 $\frac{n a}{m}$. 同样 地, 如果令 $y$ 为常数, 得垂直于 $E F$ 的平面与雉面的交线为双曲线, 中心在直线 $E F$ 上.

\section{$\S 70$}

参见图 135, 截平面垂直于平面 $A E B F$, 但既不垂直于 $A B$, 也不垂直于 $E F$. 我们来求这时的截线方程, 也不难. 设 截平面与底平面 $A E B F$ 的交线为 $B E$, 令 $O B=a, O E=b$. 又 设 $M$ 为截线上任一点, $M Q$ 垂直于底平面, $P Q$ 为纵标. 这样 我们有 $O P=x, P Q=y, Q M=z$. 从圆雉性质得
\[
m^{2} n^{2} z^{2}=m^{2} y^{2}+n^{2} x^{2}
\]
这样我们有
\[
a: b=(a-x): y
\]
或 $y=b-\frac{b x}{a}$.


【图,待补】
%%![](https://cdn.mathpix.com/cropped/2023_02_05_11864d1515e42e275d87g-04.jpg?height=400&width=422&top_left_y=1418&top_left_x=1091)

图 135

在截平面上取坐标 $B Q=t, Q M=z$,则
\[
b: \sqrt{a^{2}+b^{2}}=y: t
\]
从而
\[
y=\frac{b t}{\sqrt{a^{2}+b^{2}}}, \quad a-x=\frac{a t}{\sqrt{a^{2}+b^{2}}}
\]
令 $\sqrt{a^{2}+b^{2}}=c$, 则 
\[
y=\frac{b t}{c}, \quad x=a-\frac{a t}{c}
\]
代入雉面方程,得 $t, z$ 间方程
\[
\begin{gathered}
m^{2} n^{2} c^{2} z^{2}=m^{2} b^{2} t^{2}+n^{2} a^{2} c^{2}-2 n^{2} a^{2} c t+n^{2} a^{2} t^{2} \\
\text { 令 } t-\frac{n^{2} a^{2} c}{m^{2} b^{2}+n^{2} a^{2}}=G Q=u \text {, 又 } B G=\frac{n^{2} a^{2} c}{m^{2} b^{2}+n^{2} a^{2}} \text {, 则 } \\
m^{2} n^{2} c^{2} z^{2}=\left(m^{2} b^{2}+n^{2} a^{2}\right) u^{2}+\frac{m^{2} n^{2} a^{2} b^{2} c^{2}}{m^{2} b^{2}+n^{2} a^{2}}
\end{gathered}
\]
\section{$\S 71$}

所得截线是双曲线, 中心在点 $G$, 横半轴
\[
G a=\frac{a b}{\sqrt{m^{2} b^{2}+n^{2} a^{2}}}
\]
其共轭半轴为 $\frac{m n a b c}{m^{2} b^{2}+n^{2} a^{2}}$. 该双曲线的渐近线交轴 $G a$ 于中心 $G$, 与 $G a$ 所成角的正切等 于 $\frac{m n c}{\sqrt{m^{2} b^{2}+n^{2} a^{2}}}$, 要求该双曲线为等轴的, 则
\[
m^{2} n^{2} a^{2}+m^{2} n^{2} b^{2}=m^{2} b^{2}+n^{2} a^{2}
\]
或
\[
\frac{b}{a}=\tan \angle O B E=\frac{n \sqrt{m^{2}-1}}{m \sqrt{1-n^{2}}}
\]
即要求双曲线为等轴的, 必则 $\frac{m^{2}-1}{1-n^{2}}$ 为正. 直圆雉时 $m=n$, 渐近线与截线轴所成角为 $\angle a O c$, 其正切等于 $n$.

\section{$\S 72$}

参见图 136 , 现在考察截平面不垂直于底,但与底平面 $A E B F$ 的交线 $B T$ 垂直于 $A B$ 的情形. 置 $O B=f$, 截平面与底平面的夹角即 $\angle O B C=\varphi$, 记截平面与雉面的轴 $O C$ 的交 点为 $C$, 则
\[
B C=\frac{f}{\cos \varphi}, \quad O C=\frac{f \sin \varphi}{\cos \varphi}
\]
$M$ 为所求截线上任一点, $M T$ 垂直于 $B T, M Q$ 垂直于底平面, $Q P$ 垂直于 $O B$. 置 $O P=x$, $P Q=y, Q M=z$, 得 $m^{2} n^{2} z^{2}=m^{2} y^{2}+n^{2} x^{2}$. 取截线的坐标为 $B T=t, T M=u$, 则由
\[
\angle Q T M=\varphi, \quad Q M=z=u \sin \varphi, \quad T Q=u \cos \varphi=f-x
\]
得
\[
y=t, \quad z=u \sin \varphi, \quad x=f-u \cos \varphi
\]
从而 
\[
m^{2} n^{2} u^{2} \sin ^{2} \varphi=m^{2} t^{2}+n^{2}(f-u \cos \varphi)^{2}
\]

【图,待补】
%%![](https://cdn.mathpix.com/cropped/2023_02_05_11864d1515e42e275d87g-06.jpg?height=392&width=338&top_left_y=333&top_left_x=645)

图 136

\section{$\S 73$}
\[
\text { 令 } \begin{aligned}
& B C=\frac{f}{\cos \varphi}=g \text {, 则 } f=g \cos \varphi \text { 从而 } \\
& x=(g-u) \cos \varphi
\end{aligned}
\]
进而得截线的方程为
\[
m^{2} n^{2} u^{2} \sin ^{2} \varphi=m^{2} t^{2}+n^{2} g^{2} \cos ^{2} \varphi-2 n^{2} g u \cos ^{2} \varphi+n^{2} u^{2} \cos ^{2} \varphi
\]
引 $M S$ 平行于 $B T$, 取
\[
B G=\frac{f}{\cos ^{2} \varphi-m^{2} \sin ^{2} \varphi}=\frac{g \cos \varphi}{\cos ^{2} \varphi-m^{2} \sin ^{2} \varphi}=\frac{g \cos \varphi}{1-\left(1+m^{2}\right) \sin ^{2} \varphi}
\]
得
\[
u-\frac{g \cos ^{2} \varphi}{\cos ^{2} \varphi-m^{2} \sin ^{2} \varphi}=S G=s
\]
取 $G S=s, S M=t$ 为坐标,得方程
\[
m^{2} t^{2}+n^{2}\left(\cos ^{2} \varphi-m^{2} \sin ^{2} \varphi\right) s^{2}-\frac{m^{2} n^{2} f^{2} \sin ^{2} \varphi}{\cos ^{2} \varphi-m^{2} \sin ^{2} \varphi}=0
\]
因而曲线是以 $G$ 为中心的圆雉曲线, 中心 $G$ 趋向无穷远时它为抛物线, $G$ 趋向无穷远时 $\tan \varphi=\frac{1}{m}$, 即直线 $B C$ 平行于雉面上直线 $O a$ 时 (图 134), 此时
\[
m^{2} t^{2}+n^{2} f^{2}-2 n^{2} f u \cos \varphi=0
\]
图 136 上取 $B G=\frac{f}{2 \cos \varphi}$ 时, 抛物线的顶点为 $G$, 它的参数为 $\frac{2 n^{2} f \cos \varphi}{m^{2}}$.

\section{$\S 74$}

由 $\cos ^{2} \varphi-m^{2} \sin ^{2} \varphi=0$ 时截线为抛物线, 显见, $\cos ^{2} \varphi>m^{2} \sin \varphi$, 也即 $\tan \varphi<\frac{1}{m}$ 时, 截线为椭圆, 此时直线 $B C$ 在上方与对面雉面上的 $O a$ 相交, 由 
\[
B G=\frac{g}{1-m^{2} \tan ^{2} \varphi}
\]
我们有 $B G>B C, G$ 是截线的中心,从而截线 $B C$ 方向上的半轴等于
\[
\frac{m f \sin \varphi}{\cos ^{2} \varphi-m^{2} \sin ^{2} \varphi}
\]
与它共轭的另一根半轴等于
\[
\frac{n f \sin \varphi}{\sqrt{\cos ^{2} \varphi-m^{2} \sin ^{2} \varphi}}
\]
参数等于
\[
\frac{n^{2}}{m} f \sin \varphi
\]
进而,如果
\[
m=n \sqrt{\cos ^{2} \varphi-m^{2} \sin ^{2} \varphi}
\]
也即
\[
m^{2}=n^{2}-n^{2}\left(1+m^{2}\right) \sin ^{2} \varphi
\]
则截线是圆. 可见, 必 $n>m$, 这类截线才可能为圆.

\section{$\S 75$}

如果 $m^{2} \sin ^{2} \varphi>\cos ^{2} \varphi$, 也即 $\tan \varphi>\frac{1}{m}$, 则直线 $B C$ 在上方与对面雉面上的 $O a$ 不相 交. 此时截线为双曲线, 其横半轴等于
\[
\frac{m f \sin \varphi}{-\cos ^{2} \varphi+m^{2} \sin ^{2} \varphi}
\]
与它共轭的半轴等于
\[
\frac{n f \sin \varphi}{\sqrt{m^{2} \sin ^{2} \varphi-\cos ^{2} \varphi}}
\]
参数等于
\[
\frac{n^{2}}{m} f \sin \varphi
\]
渐近线与轴在中心 $G$ 处之交角的正切等于
\[
\frac{n}{m} \sqrt{m^{2} \sin ^{2} \varphi-\cos ^{2} \varphi}
\]
因而,当
\[
m^{2} n^{2} \sin ^{2} \varphi-n^{2} \cos ^{2} \varphi=m^{2}=\left(m^{2}+1\right) n^{2} \sin ^{2} \varphi-n^{2}=m^{2}
\]
也即当
\[
\sin \varphi=\frac{\sqrt{m^{2}+n^{2}}}{n \sqrt{1+m^{2}}}, \quad \cos \varphi=\frac{m \sqrt{n^{2}-1}}{n \sqrt{1+m^{2}}}
\]
的时候, 该双曲线为等腰的. 可见, $n>1$ 时这类截线才可能为等腰双曲线. 

\section{$\S 76$}

直线 $A B$ 的位置任我们选取, 因而直雉面, 也即 $m=n$ 时的雉面, 其截线我们都会求. 这样, 要讨论的, 就剩下斜雉面、斜截面、截底两平面交线与 $A B$ 成斜角的情况了. 参见 图 137, 截底两平面交线为 $B R$, 记 $O B$ 为 $f, \angle O B R$ 为 $\theta$, 截底两平面所成角为 $\varphi$. 从 $O$ 引 $O R$ 垂直于 $B R$, 则 $O R=f \sin \theta, B R=f \cos \theta$. 在截平面上连直线 $R C$, 则 $\angle O R C=\varphi$, 我 们有
\[
R C=\frac{f \sin \theta}{\cos \varphi}, \quad O C=\frac{f \sin \theta \sin \varphi}{\cos \varphi}
\]

将垂直于雉面轴 $O C$ 的截线投影到底面上, 则投影主轴方向同于 $A B, E F$, 且投影主轴的 比等于 $m$ 比 $n$.


【图,待补】
%%![](https://cdn.mathpix.com/cropped/2023_02_05_11864d1515e42e275d87g-08.jpg?height=464&width=615&top_left_y=919&top_left_x=516)

图 137

\section{$\S 77$}

在截线投影上画平行于 $B R$ 的直径 $e f$, 则 $\angle B O e=\theta$. 设 $a O b$ 为 $e f$ 的共轭直径, 令半 直径 $O a=\mu, O e=\nu$, 则
\[
\begin{gathered}
\mu=\frac{\sqrt{m^{4} \sin ^{2} \theta+n^{4} \cos ^{2} \theta}}{\sqrt{m^{2} \sin ^{2} \theta+n^{2} \cos ^{2} \theta}} \\
\nu=\frac{m n}{\sqrt{m^{2} \sin ^{2} \theta+n^{2} \cos ^{2} \theta}} \\
\tan \angle B O b=\frac{n^{2} \cos \theta}{m^{2} \sin \theta}
\end{gathered}
\]
因此 $\angle B O b$ 的正弦和余弦分别为
\[
\frac{n^{2} \cos \theta}{\sqrt{m^{4} \sin ^{2} \theta+n^{4} \cos ^{2} \theta}}, \frac{m^{2} \sin \theta}{\sqrt{m^{4} \sin ^{2} \theta+n^{4} \cos ^{2} \theta}}
\]
由 $\angle O b R=\theta+\angle B O b$, 得 
\[
\begin{aligned}
& \sin \angle O b R=\frac{m^{2} \sin ^{2} \theta+n^{2} \cos ^{2} \theta}{\sqrt{m^{4} \sin ^{2} \theta+n^{4} \cos ^{2} \theta}} \\
& \cos \angle O b R=\frac{\left(m^{2}-n^{2}\right) \sin \theta \cos \theta}{\sqrt{m^{4} \sin ^{2} \theta+n^{4} \cos ^{2} \theta}}
\end{aligned}
\]
但我们有
\[
\mu=\frac{m n \sqrt{m^{4} \sin ^{2} \theta+n^{4} \cos ^{2} \theta}}{m^{2} \sin ^{2} \theta+n^{2} \cos ^{2} \theta}
\]
\section{$\S 78$}

由于 $O R=f \sin \theta$, 得
\[
\begin{gathered}
O b=\frac{O R}{\sin \angle O b R}=\frac{f \sin \theta \sqrt{m^{4} \sin ^{2} \theta+n^{4} \cos ^{2} \theta}}{m^{2} \sin ^{2} \theta+n^{2} \cos ^{2} \theta} \\
R b=\frac{\left(m^{2}-n^{2}\right) f \sin \theta \cos \theta}{m^{2} \sin ^{2} \theta+n^{2} \cos ^{2} \theta}
\end{gathered}
\]
从而由 $R$ 为直角的 $\triangle R b C$, 得 $\angle C b R$ 的正切等于
\[
\frac{m^{2} \sin ^{2} \theta+n^{2} \cos ^{2} \theta}{\left(m^{2}-n^{2}\right) \cos \theta \cos \varphi}
\]
这样, $\angle C b R$ 为已知. $M$ 为截线上任一点, $M T$ 平行于 $C b$, 交 $R T$ 于 $T ; M S$ 平行于 $R T$, 交 $C b$ 于 $S$. 记 $b T=M S=t, b S=T M=u$, 视 $t, u$ 为所求截线的斜角坐标, 且 $\angle b S M[=$ $\angle C b R]$ 的正切等于
\[
\frac{m^{2} \sin ^{2} \theta+n^{2} \cos ^{2} \theta}{\left(m^{2}-n^{2}\right) \cos \theta \cos \varphi}
\]
显然, 直雉面时, 这坐标成为直角坐标, 因为那时 $m=n$.

\section{$\S 79$}

$M$ 为截线上一点, $M Q$ 垂直于底平面 $A E B F, T Q, Q P$ 分别平行于直径 $a b, e f$. 令 $O P=x, P Q=y, Q M=z$, 则由雉面性质我们有
\[
\mu^{2} \nu^{2} z^{2}=\mu^{2} y^{2}+\nu^{2} x^{2}
\]
过点 $M$ 截取平行于底面的截线, 则其平行于 $a b$ 和 $e f$ 的半直径为 $\mu z$ 和 $\nu z . \mathrm{Rt} \triangle C O b$ 的直 角边 $O C, O b$ 已知,因而其斜边
\[
C b=\frac{f \sin \theta \sqrt{m^{4} \sin ^{2} \theta+n^{4} \cos ^{2} \theta-\left(m^{2}-n^{2}\right)^{2} \sin ^{2} \theta \cos ^{2} \theta \sin ^{2} \varphi}}{\left(m^{2} \sin ^{2} \theta+n^{2} \cos ^{2} \theta\right) \cos \varphi}
\]
由 $\triangle T Q M$ 和 $\triangle b C O$ 相似,得
\[
T M(=u): T Q(=O b-x): Q M(=z)=b c: O b: O C
\]
从而 $x=O b-\frac{O b u}{C b}, z=\frac{O C u}{C b}, y=t$, 进而 

$\mu^{2} \nu^{2} O C^{2} u^{2}=\mu^{2} C b^{2} t^{2}+\nu^{2} O b^{2}(C b-u)^{2}$ $

\section{$\S 80$}

上节方程展开得
\[
0=\mu^{2} C b^{2} t^{2}+\nu^{2}\left(O b^{2}-\mu^{2} O C^{2}\right) u^{2}-2 \nu^{2} O b^{2} C b u+\nu^{2} O b^{2} C b^{2}
\]
令 $u-\frac{O b^{2} \cdot C b}{O b^{2}-\mu^{2} O C^{2}}=s$, 或者取
\[
b G=\frac{O b^{2} \cdot C b}{O b^{2}-\mu^{2} O C^{2}}=\frac{C b}{1-\left(m^{2} \sin ^{2} \theta+n^{2} \cos ^{2} \theta\right) \tan ^{2} \varphi}
\]
并记 $G S$ 为 $s$, 则 $G$ 是雉面截线的中心, 该截线的坐标 $t, s$ 间方程为
\[
\mu^{2} \cdot C b^{2} \cdot t^{2}+\nu^{2}\left(O b^{2}-\mu^{2} O C^{2}\right) s^{2}=\frac{\mu^{2} \cdot \nu^{2} \cdot O b^{2} \cdot O C^{2} \cdot C b^{2}}{O b^{2}-\mu^{2} \cdot O C^{2}}
\]
其横半径、共轭半径和参数依次为
\[
\begin{aligned}
& \frac{\mu \cdot O b \cdot O C \cdot C b}{O b^{2}-\mu^{2} \cdot O C^{2}} \\
& \frac{\nu \cdot O b \cdot O C}{\sqrt{O b^{2}-\mu^{2} O C^{2}}} \\
& \frac{\nu^{2} \cdot O b \cdot O C}{\mu \cdot C b}
\end{aligned}
\]
显然 $\tan \varphi<\frac{1}{\sqrt{m^{2} \sin ^{2} \theta+n^{2} \cos ^{2} \theta}}$, 也即 $\tan \varphi<\frac{\nu}{m n}$ 时, 曲线是椭圆; $\tan \varphi=\frac{\nu}{m n}$ 时, 曲线 是抛物线; $\tan \varphi>\frac{\nu}{m n}$ 时, 曲线是双曲线.

\section{$\S 81$}

现在考察第三种曲面一一球面的截线. 初等几何中 我们知道,平面截球面、截线都是圆, 从曲面的方程求曲 面的截线, 这个问题, 通过通常用综合方法处理, 为更清 楚起见,我们用分析方法对它进行讨论.

参见图 138, 设 $C$ 为球面中心, 想象过 $C$ 的平面为桌 面, 它截球面所成截线为图上大圆,其半径 $A C=C B=a$, $a$ 是球面半径, 再设 $D T$ 为截平面与桌面平面的交线, 截 平面对桌面平面的倾角为 $\varphi, C D$ 垂直于 $D T$, 记 $C D=f$.


【图,待补】
%%![](https://cdn.mathpix.com/cropped/2023_02_05_11864d1515e42e275d87g-10.jpg?height=335&width=465&top_left_y=1640&top_left_x=1029)

图 138 

\section{$\S 82$}

设 $M$ 为所求截线上任一点, $M Q$ 垂直于桌面平面, $Q P$ 垂直于我们取为轴的 $C D$, 记 $C P=x, P Q=y, Q M=z$, 则由球面性质得 $x^{2}+y^{2}+z^{2}=a^{2}$. 从 $M$ 引 $M T$ 垂直于 $D T$, 连接 $Q$ 与 $T$, 则由 $Q T, M T$ 都垂直于 $D T$, 知 $\angle M T Q$ 等于截平面对底平面的倾角, 等于 $\varphi$, 因而 视 $D T, M T$ 为所求截线的坐标, 记 $D T=t, T M=u$, 则
\[
M Q=u \sin \varphi, \quad T Q=u \cos \varphi
\]
从而
\[
C P=x=f-u \cos \varphi, \quad P Q=y=t, \quad Q M=z=u \sin \varphi
\]
代入球面方程,得球面的截线方程
\[
f^{2}-2 f u \cos \varphi+u^{2}+t^{2}=a^{2}
\]
\section{$\S 83$}

显然, 这是圆的方程. 事实上, 令 $u-f \cos \varphi=s$, 得
\[
f^{2} \sin ^{2} \varphi+s^{2}+t^{2}=a^{2}
\]
这圆的半径为 $\sqrt{a^{2}-f^{2}} \sin ^{2} \varphi$. 因此, 如果引 $D c$ 平行于纵标 $T M, C c$ 垂直于 $D c$, 则由 $C D=$ $f$ 和 $\angle C D_{c}=\varphi$, 得 $D_{c}=f \cos \varphi, C c=f \sin \varphi$. 因而考虑坐标 $s, t$, 我们看到, 截线圆的圆心 为 $c$, 半径为 $\sqrt{C B^{2}-C c^{2}}$, 同初等几何告诉我们的一致. 用类似的方法我们可以考察任何 平面截任何曲面所得截线,前提是有了曲面的三元方程.

\section{$\S 84$}

为了使整个过程更清楚起见, 我们假定给了一个曲面 (图 139), 其性质由坐标 $A P=x, P Q=y, Q M=z$ 间的方程表 示, 前两个坐标在桌面上,第三个坐标 $z$ 垂直于桌面. 用任一个 平面来截这曲面. 设所用平面与桌面的交线为 $D T$, 对桌面的倾 角为 $\varphi$, 记 $A D$ 的长为 $f$, 记 $\angle A D E$ 为 $\theta$, 那么由点 $A$ 向 $D E$ 引垂 线 $A E$, 则
\[
A E=f \sin \theta, \quad D E=f \cos \theta
\]


【图,待补】
%%![](https://cdn.mathpix.com/cropped/2023_02_05_11864d1515e42e275d87g-11.jpg?height=376&width=364&top_left_y=1569&top_left_x=1149)

图 139 

然后从所求曲线上的点 $M$ 向 $D T$ 引垂线 $M T$, 连 $Q$ 与 $T$, 则
$\angle M T Q$ 等于截平面对桌面的倾角 $\varphi$. 因而如果取 $D T, T M$ 作为 所求截线的坐标,记为 $D T=t, T M=u$, 我们有
\[
Q M=u \sin \varphi, \quad T Q=u \cos \varphi
\]
\section{$\S 85$}

从 $T$ 向轴 $A D$ 引垂线 $T V$, 则由 $\angle T D V=\theta$, 得 $T V=t \sin \theta, D V=t \cos \theta$. 又由 $\angle T Q P=$ $\theta$, 得
\[
P V=u \sin \theta \cos \varphi, \quad P Q-T V=u \cos \theta \cos \varphi
\]
利用这几个表达式,我们可以把 $x, y, z$ 都用 $t, u$ 表示出来
\[
\begin{gathered}
A P=x=f+t \cos \theta-u \sin \theta \cos \varphi \\
P Q=y=t \sin \theta+u \cos \theta \cos \varphi \\
Q M=z=u \sin \varphi
\end{gathered}
\]
将这三个表达式代入以 $x, y, z$ 为变元的曲线方程, 得 $t, u$ 间也即所求截线的坐标间方程. 所求截线的性质就由该方程描述,这里的方法跟 $\S 50$ 的方法几乎完全一样. 

\chapter{第四章 坐标变换}

\section{$\S 86$}

原点位置和轴的位置, 两者都改变, 或者其中之一改变, 平面曲线的方程都随着改 变, 即一条平面曲线, 其方程的个数是无穷的. 类似地, 一张曲面, 其方程的个数更多, 比 一条平面曲线方程的个数, 要多出几层无穷. 事实上, 一张平面上两坐标的取法就有无穷 多种, 另一张平面上又有新的无穷多种. 例如, 任给一个三变量直角坐标曲面方程, 我们 都可以把它变换成描述同一曲面的另一个三变量直角坐标方程. 这种变换的种数, 是二 变量曲线情况下的几层无穷.

\section{$\S 87$}

我们先只改变原点在 $x$ 轴上的位置. 即坐标 $y, z$ 不变, 只新旧横标相差一个常数. 记 新横标为 $t$, 则 $x=t \pm a$. 将这个 $x$ 代入曲面方程, 得到以 $t, y, z$ 为变元的另一个方程. 这另 一个方程与原方程形式不同, 但描述的曲面是同一个. 让另外两个坐标也增加或减少一 个常数, 即取 $x=t \pm a, y=u \pm b, z=v \pm c$, 我们就得到描述同一曲面的以 $t, u, v$ 为变量的 方程, 这里新坐标都平行于旧坐标, 新方程比旧方程更一般, 但差别不很大.

\section{$\S 88$}

参见图 140,曲面方程的三个直角坐标构成三张相垂直的 平面. 现在我们保持坐标 $x, y$ 所在平面不动, 只在这张平面上 把轴 $A P$ 换成另外一条任取的直线 $C T . A P$ 为轴时坐标为 $A P=$ $x, P Q=y, Q M=z$; 换 $A P$ 为 $C T$ 时, 新坐标 $Q M=z$ 依旧, 但另外 两个变成了 $C T=t, T Q=u$, 这里的 $Q T$ 垂直于新轴 $C T$. 为求出 以新坐标 $t, u, z$ 为变量的方程, 从 $C$ 引 $C R$ 平行于 $A P$,引 $C B$ 垂 直于 $A P$. 令 $A B=a, B C=b, \angle R C T=\zeta$. 从 $T$ 引 $T R$ 垂直于 $C R$, 引 $T S$ 垂直于 $Q P$ 的延长线.


【图,待补】
%%![](https://cdn.mathpix.com/cropped/2023_02_05_11864d1515e42e275d87g-13.jpg?height=388&width=333&top_left_y=1606&top_left_x=1164)

图 140 

\section{$\S 89$}

这样, 在 $\triangle T C R$ 中我们有 $T R=t \sin \zeta, C R=t \cos \zeta$, 在 $\triangle Q T S$ 中 $Q=\zeta, T S=u \sin \zeta$, $Q S=u \cos \zeta$ 从而
\[
\begin{aligned}
& A P=x=C R+T S-A B=t \cos \zeta+u \sin \zeta-a \\
& Q P=Q S-T R-B C=y=u \cos \zeta-t \sin \zeta-b
\end{aligned}
\]
将所给曲面方程中的 $x$ 和 $y$ 换成这两个表达式, 我们就得到所给曲面的以新坐标 $t, u, z$ 为变量的方程. 新方程远比旧方程更具一般性, 因为它包含旧方程所不包含的三个任意 常数 $a, b, \zeta$, 新方程是 $x, y$ 所在平面不动情况下的通用方程.

\section{$\S 90$}

现在我们让前两个坐标 $x, y$ 所在平面变动. 先让它绕轴 $A P$ 转动, 方式是: 保持 $A P Q$ 上的 $A P$ 为旧新平面的交线, 并 保持 $A P$ 为新坐标的轴. 参见图 141 , 记新平面 $A P T$ 对旧平 面 $A P Q$ 的倾角即 $\angle Q P T$ 为 $\eta$, 从点 $M$ 向新平面引垂线 $M T$, 这 $M T$ 是新的第三坐标. 表示这三个新坐标为 $A P=x, P T=$ $u, T M=v$. 引 $T R$ 垂直于 $P Q, T S$ 垂直于 $Q M$, 则
\[
\begin{array}{ll}
T R=u \sin \eta, & P R=u \cos \eta \\
T S=v \sin \eta, & M S=v \cos \eta
\end{array}
\]
从而


【图,待补】
%%![](https://cdn.mathpix.com/cropped/2023_02_05_11864d1515e42e275d87g-14.jpg?height=416&width=428&top_left_y=960&top_left_x=1088)

图 141
\[
P Q=y=u \cos \eta-v \sin \eta, \quad Q M=z=v \cos \eta+u \sin \eta
\]
代入给定的方程, 得到以 $x, u, v$ 为变量的方程, 描述的曲面同于给定的方程.

\section{$\S 91$}

参见图 140 , 设新平面与 $A P Q$ 的交线任意, 为 $C T$, 倾角为 $\eta$. 取 $C T$ 作新平面上的轴. 第一步先求出平面 $A P Q$ 上以 $C T$ 为轴的坐标间方程. 令 $A B=a, B C=b, \angle T C R=\zeta$, 并记 以 $C T$ 为轴的坐标为 $C T=p, T Q=q, Q M=r$, 则根据前面的结果, 我们有
\[
x=p \cos \zeta+q \sin \zeta-a, \quad y=q \cos \zeta-p \sin \zeta-b, \quad z=r
\]
第二步, 利用上节结果, 取新坐标为 $t, u, v$, 则
\[
p=t, \quad q=u \cos \eta-v \sin \eta, \quad r=v \cos \eta+u \sin \eta
\]
代入第一步求出的 $x, y, z$ 的表达式, 得
\[
\begin{gathered}
x=t \cos \zeta+u \sin \zeta \cos \eta-v \sin \zeta \sin \eta-a \\
y=t \sin \zeta+u \cos \zeta \cos \eta-v \cos \zeta \sin \eta-b \\
z=u \sin \eta+v \cos \eta
\end{gathered}
\]
\section{$\S 92$}

现在进一步, 在坐标 $t, u$ 所在的新平面上, 我们任取一条新的直线作轴, 这样我们就 得到所给曲面的最通用方程. 参见图 140 , 依次取 $A P, P Q, Q M$ 为坐标 $t, u, v, A P$ 是新平 面与 $x, y$ 所在平面的交线. 取 $C T$ 作新轴, 对它建立最通用方程. 记 $C T=p, T Q=q$, $Q M=r$. 此外, $A B, B C$ 是定长线段, $\angle C T R=\theta$. 这样根据 $\S 89$ 我们有
\[
\begin{gathered}
t=p \cos \theta+q \sin \theta-A B \\
u=-p \sin \theta+q \cos \theta-B C \\
v=r
\end{gathered}
\]
代入前节导出的表达式,得
\[
\begin{aligned}
x= & p(\cos \zeta \cos \theta-\sin \zeta \cos \eta \sin \theta)+ \\
& q(\cos \zeta \sin \theta+\sin \zeta \cos \eta \cos \theta)-r \sin \zeta \sin \eta+f \\
y= & -p(\sin \zeta \cos \theta+\cos \zeta \cos \eta \sin \theta)- \\
& q(\sin \zeta \sin \theta-\cos \zeta \cos \eta \cos \theta)-r \cos \zeta \sin \eta+g \\
z=- & p \sin \eta \sin \theta+q \sin \eta \cos \theta+r \cos \eta+h
\end{aligned}
\]
其中 $f, g, h$ 是从开始时引进的线段经过计算得到的数.

\section{$\S 93$}

可见, 任何面的最通用方程都包含 6 个任意常数,不管这 6 个常数取什么值, 方程描 述的都是同一个面. 一个面, 即使它关于 $x, y, z$ 的方程很简单, 推出的关于 $p, q, r$ 的最通 用方程却很复杂, 因为引进的任意常数个数太多. 特别地, 当 $x, y, z$ 是高次的, 那复杂程 度当然就更高. 因而最通用方程, 用得上的情况末必会有. 虽然适当选取常数可以得到简 化方程,但是这计算太长太繁. 那么最通用方程是不是全无用处呢? 不是的,借助它我们 可以得到和证明一些重要性质.

\section{$\S 94$}

最通用方程常常很复杂,但它的次数却恒等于 $x, y, z$ 间方程的次数. 例如球面方程 $x^{2}+y^{2}+z^{2}=a^{2}$ 的次数为 2 , 它的以 $p, q, r$ 为变量的最通用方程, 次数绝对不会超过 2 . 因 而面方程的次数是面的一个实质性的特点, 它不受坐标变换的影响. 线有类似的性质, 根 据这条性质, 我们将线按方程的阶进行了分类. 仿照线, 对面我们也按方程的阶进行分 类. 称方程为一阶的面为一阶面, 方程为二阶的面为二阶面,类推.

\section{$\S 95$}

将上节所讲与前面讲的面的平面截线的有关内容相比较, 我们得出结论,截线的阶 等于产生它的曲面的阶. 事实上,假定曲面的以 $x, y, z$ 为变量的方程次数为 $n$, 又假定一 条截线的直角坐标为 $t$ 和 $u . \S 85$ 我们看到, 以 $t$ 和 $u$ 为变量的截线方程. 是将
\[
\begin{gathered}
x=f+t \cos \theta-u \sin \theta \cos \varphi \\
y=t \sin \theta+u \cos \theta \cos \varphi \\
z=u \sin \varphi
\end{gathered}
\]
代入曲面方程的结果. 显然, 结果方程的阶不能高于原方程的阶, 它们的阶相等.

\section{$\S 96$}

平面截一阶面,截线是一阶的, 是直线. 平面截二阶面,截线是二阶的, 是圆雉曲线. 圆雉面是二阶面,其方程的形状为
\[
z^{2}=\alpha x^{2}+\beta y^{2}
\]
类似地,平面截三阶面,截线是三阶的. 类推. 但是有这样的情形,截线方程可分解因式. 此时截线由两个或更多个低阶线组成. 例如, 过圆雉顶点的截线是两条直线, 但作为总体 看,它们是二阶线.

\section{$\S 97$}

这样我们就按阶把面分了类. 我们先考虑一阶面,其方程为
\[
\alpha x+\beta y+\gamma z=a
\]
由一阶面的平面截线都是直线, 知一阶面不能不是平面. 否则, 有凸或有凹, 截线中就必 定有曲线. 虽然有的非一阶面的平面截线中有直线, 但这种非一阶面的截线中必定还有 曲线. 这使我们想到, 跟直线的交恒不多于一个点的线必为直线, 在这里是跟平面的交恒 为直线的面必为平面.

\section{$\S 98$}

一阶面为平面,可从最通用方程得到清楚地证明. 事实上,从方程 $\alpha x+\beta y+\gamma z=a$, 照 $\S 92$ 步骤列出以 $p, q, r$ 为坐标的最通用的方程时, 完全可以对 6 个新的任意常数进行 选择, 使得 $p$ 和 $q$ 的系数为零, 也即使方程的形状为 $r=f$. 方程 $r=f$ 表明所求面平行于坐 标 $p, q$ 所在平面, 也即所求面为平面, 可以使方程为 $r=0$, 也即使坐标 $p, q$ 所在平面为所 给面.

\section{$\S 99$}

证明了方程 $\alpha x+\beta y+\gamma z=a$ 描述的是平面,下一步应做的是确定它关于坐标 $x, y$ 所 在平面的位置. 参见图 142 , 设 $M$ 是所求面上任一点, 三个坐标为 $A P=x, P Q=y, Q M=$ $z$. 先令 $z=0$, 得方程 $\alpha x+\beta y=a$, 这是所求面与平面 $A P Q$ 交线 的方程. 在平面 $A P Q$ 上引 $A P$ 的垂线 $A B=\frac{a}{\beta}$, 截 $A C=\frac{a}{\alpha}$, 则 $B C R$ 即为所求面与 $A P Q$ 的交线, 从而 $\angle A B C$ 的正切, 正弦, 余 弦依次为
\[
\frac{\alpha}{\beta}, \frac{\alpha}{\sqrt{\alpha^{2}+\beta^{2}}}, \frac{\beta}{\sqrt{\alpha^{2}+\beta^{2}}}
\]
延长 $Q P$ 交 $B C$ 于 $R$, 则由 $C P=x-\frac{a}{\alpha}$, 得
\[
C R=\frac{x \sqrt{\alpha^{2}+\beta^{2}}}{\beta}-\frac{a \sqrt{\alpha^{2}+\beta^{2}}}{\alpha \beta}, \quad P R=\frac{\alpha x}{\beta}-\frac{a}{\beta}
\]

【图,待补】
%%![](https://cdn.mathpix.com/cropped/2023_02_05_11864d1515e42e275d87g-17.jpg?height=395&width=390&top_left_y=272&top_left_x=1126)

图 142

\section{$\S 100$}

从 $Q$ 向 $B C$ 引垂线 $Q S$, 连直线 $M S$, 则 $\angle M S Q$ 为所求平面对平面 $A P Q$ 的倾角. 由 $P R=\frac{\alpha x-a}{\beta}$, 得 $Q R=\frac{\alpha x+\beta y-a}{\beta}=-\frac{\gamma z}{\beta}$; 由 $\angle R Q S=\angle A C B$, 得 $Q S=-\frac{\gamma z}{\sqrt{\alpha^{2}+\beta^{2}}}$. 从而 $\angle Q M S$ 的正切等于 $\frac{-\sqrt{\alpha^{2}+\beta^{2}}}{\gamma}$, 余弦等于 $\frac{\gamma}{\sqrt{\alpha^{2}+\beta^{2}+\gamma^{2}}}$. 也即所求平面对 $x, y$ 所在平面 倾角的正切等于 $\frac{\sqrt{\alpha^{2}+\beta^{2}}}{\gamma}$ 类似地, 它对 $x, z$ 所在平面倾角的正切等于 $-\frac{\sqrt{\alpha^{2}+\gamma^{2}}}{\beta}$, 对 $y$, $z$ 所在平面倾角的正切等于 $-\frac{\sqrt{\beta^{2}+\gamma^{2}}}{\alpha}$. 

\chapter{第五章 二阶面}

\section{$\S 101$}

我们按方程的次数将面分类, 方程为几次的, 就称面为几阶的.一个面, 给出了它的 代数方程, 我们立刻就可说出它是几阶的. 前面证明了一阶面都是平面. 本章我们考察二 阶面, 二阶面比二阶线类型要多, 我们尽力把各类型的二阶面讲清. 随阶数的增加, 面的 类型数猛增, 以至于无法对它们进行讨论.

\section{$\S 102$}

方程为二次的面是二阶面. 我们讨论过的柱面、雉面, 直的斜的, 以及球面都是二阶 面. 二阶面的方程全包含在下面这个通用方程中
\[
\alpha z^{2}+\beta y z+\gamma x z+\delta y^{2}+\varepsilon x y+\zeta x^{2}+\eta z+\theta y+\omega x+\kappa=0
\]
不管三个坐标怎么取,方程的形状都是这样. 不同型的二阶面由系数间的不同关系决定. 不同二阶面的个数是无穷的, 同一个二阶面可以有无穷多个不同的方程. 可见二阶面的 可能的方程是何等的多.

\section{$\S 103$}

我们把平面曲线按是否伸向无穷分成了两类. 我们也照方把同阶面分成两类, 第一 类是伸向无穷的,第二类是位于有限空间区域中的. 封面、雉面属第一类, 球面属第二类. 当然第二类面阶数必定不为奇数. 奇阶面的平面截线是奇阶的, 奇阶线一律伸向无穷, 因 而奇阶面必伸向无穷.

\section{$\S 104$}

一个曲面是伸向无穷的, 则三个坐标 $x, y, z$ 中必至少有一个是伸向无穷的. 三坐标 可互换, 面伸向无穷时, 我们假定 $z$ 伸向无穷. 为考察面的伸向无穷部分的性质, 假定 $z=$ $\infty$, 这时首先应考虑第一项 $\alpha z^{2}$, 特别是方程含否这一项. 如果方程含有这一项, 那么跟它 相比, 项 $\eta z$ 和 $\kappa$ 都可略去, 得到面的伸向无穷部分方程的形状为
\[
\alpha z^{2}+\beta y z+\gamma x z+\delta y^{2}+\varepsilon x y+\zeta x^{2}+\theta y+\omega x=0
\]
下一步是略去非无穷大项和与 $\alpha z^{2}$ 相比为无穷小的项. 

\section{$\S 105$}

假定方程具有次数为 2 的可能的各项. 事实上, 任何一个曲面, 其最通用方程都包含 最高次各项,因而方程具有次数为 2 的可能的各项, 这一假定不会减弱解的一般性. 在这 个假定之下,方程中有含 $x z$ 和 $y z$ 的项. 跟它们比,项 $\theta y$ 和 $c x$ 可略去. 方程成为
\[
\alpha z^{2}+\beta y z+\gamma x z+\delta y^{2}+\varepsilon x y+\zeta x^{2}=0
\]
由此得
\[
z=\frac{-\beta y-\gamma x \pm \sqrt{\left(\beta^{2}-4 \alpha \delta\right) y^{2}+(2 \beta \gamma-4 \alpha \varepsilon) x y+\left(\gamma^{2}-4 \alpha \zeta\right) x^{2}}}{2 \alpha}
\]
曲面的伸向无穷部分,其性质就由该方程描述.

\section{$\S 106$}

二阶面如果有伸向无穷的部分, 则这伸向无穷的部分必与下面这个方程所表示的面 的无穷部分相合
\[
\alpha z^{2}+\beta y z+\gamma x z+\delta y^{2}+\varepsilon x y+\zeta x^{2}=0
\]
这个面是通用方程所表示的面的渐近面. 该渐近面方程各项次数都是 2 , 表示的是以原 点为顶点的雉面. 原点处三变量都为零. 这样我们得到, 二阶面如果伸向无穷, 则必以雉 面为渐近面. 换句话说, 渐近雉面与伸向无穷的二阶面, 它们的无穷部分重合, 或距离为 有限.

我们用渐近直线区分曲线的伸向无穷部分. 类似地, 我们用渐近雉面区分曲面的伸 向无穷部分.

\section{$\S 107$}

渐近雉面为实, 则曲面伸向无穷, 且两个无穷部分重合, 渐近雉面为虚, 则曲面不伸 向无穷, 也即位于有限空间区域之中. 这样对二阶面是否位于某个有限区域之中的判定, 就成了对渐近面为实为虚的判定. 渐近面只在缩为一点时才为虚, 否则, 由过雉面顶点和 雉面任何另外一点的直线都在雉面上, 知雉面只要在顶点之外有一个点, 就必有伸向无 穷的部分.

\section{$\S 108$}

当方程为
\[
\alpha z^{2}+\beta y z+\gamma x z+\delta y^{2}+\varepsilon x y+\zeta x^{2}=0
\]
的渐近雉面缩为一点时, 其过顶点的截线当然也缩为一点. 从而令 $z=0$ 时, 则方程 $\delta y^{2}+ \varepsilon x y+\zeta x^{2}=0$ 应该没有 $x=0, y=0$ 以外的解. 由此得 $4 \delta \zeta>\varepsilon^{2}$. 同理, 令 $x=0$, 得 $4 \alpha \delta>\beta^{2}$; 令 $y=0$, 得 $4 \alpha \zeta>\gamma^{2}$. 这样我们得到结论: 只要 $4 \delta \zeta>\varepsilon^{2}, 4 \alpha \zeta>\gamma^{2}, 4 \alpha \delta>\beta^{2}$, 这三个条件中 有一个不满足,以
\[
\alpha z^{2}+\beta y z+\gamma x z+\delta y^{2}+\varepsilon x y+\zeta x^{2}+\eta z+\theta y+\omega x+\kappa=0
\]
为方程的二阶面就必定有伸向无穷的部分.

\section{$\S 109$}

但这三个条件,对二阶面有界并不充分, 要保证二阶面位于有界区域中, 还要加上从 渐近方程解出的 $z$ 值为虚. $z$ 值为虚的条件是: 对 $x, y$ 的非零性, 表达式
\[
\left(\beta^{2}-4 \alpha \delta\right) y^{2}+2(\beta \gamma-2 \alpha \varepsilon) x y+\left(\gamma^{2}-4 \alpha \zeta\right) x^{2}
\]
恒为负, 由于 $\beta^{2}-4 \alpha \delta$ 和 $\gamma^{2}-4 \alpha \zeta$ 为负, 该条件等价于
\[
(\beta \gamma-2 \alpha \varepsilon)^{2}<\left(\beta^{2}-4 \alpha \delta\right)\left(\gamma^{2}-4 \alpha \zeta\right)
\]
也即
\[
\alpha \varepsilon^{2}+\delta \gamma^{2}+\gamma \beta^{2}<\beta \gamma \varepsilon+4 \alpha \delta \zeta
\]
这里假定 $\alpha$ 为正, 因为我们用过它作除数. 由 $\alpha>0,4 \alpha \zeta>\gamma^{2}, 4 \alpha \delta>\beta^{2}$ 和 $4 \delta \zeta>\varepsilon^{2}$ 得 $\delta, \zeta$ 为正.

\section{$\S 110$}

这样二阶面有界四条件为: 其方程满足
\[
4 \alpha \zeta>\gamma^{2}, \quad 4 a \delta>\beta^{2}, \quad 4 \delta \zeta>\varepsilon^{2}, \quad \alpha \varepsilon^{2}+\delta \gamma^{2}+\zeta \beta^{2}>\beta \gamma \varepsilon+4 \alpha \delta \zeta^{2}
\]
我们称有界二阶面为第一类二阶面. 有界即不伸向无穷, 而是位于某个有界空间区域之 内. 球面属第一类,它的方程为
\[
z^{2}+y^{2}+x^{2}=a^{2}
\]
这里 $\alpha=1, \delta=1, \zeta=1, \beta=0, \gamma=0, \varepsilon=0$, 四条件都满足, 更一般地, $\alpha, \delta, \zeta$ 为正时方程
\[
\alpha z^{2}+\delta y^{2}+\zeta x^{2}=a^{2}
\]
表示的曲面有界,只要系数中有一个或两个不为零.

\section{$\S 111$}

对给定的二阶方程, 利用有界四条件,可直接判定它所表示的曲面有无伸向无穷的 分支. 事实上,四条件中只要有一条不满足,曲面就必定伸向无穷. 对伸向无穷的曲面应 再分类. 满足
\[
a \varepsilon^{2}+\delta \gamma^{2}+\zeta \beta^{2}>\beta \gamma \varepsilon+4 a \delta \zeta
\]
的为第一类无界面. 前面讲了, 此时曲面伸向无穷, 以雉面为渐近面, 是跟有界面完全相 反的一个子类.

%%16p301-320
\section{$\S 112$}

还有一种中间子类, 它也伸向无穷, 但不与有界完全相反, 而是在有界与完全相反之 间, 有如抛物线在椭圆和双曲线之间. 方程满足
\[
\alpha \varepsilon^{2}+\delta \gamma^{2}+\zeta \beta^{2}=\beta \gamma \varepsilon+4 a \delta \zeta
\]
和
\[
\alpha z=-\beta y-\gamma z+y \sqrt{\beta^{2}-4 \alpha \delta}+x \sqrt{\gamma^{2}-4 \alpha \zeta}
\]
时, 曲线就属这种中间情形, 此时渐近线方程
\[
\alpha z^{2}+\beta y z+\gamma x z+\delta y^{2}+\varepsilon x y+\zeta x^{2}=0
\]
有两个线性因式,或者同实,或者同虚,或者相等. 这三种不同的因式给出伸向无穷的三 种不同的子类. 这样我们总共有了五个子类, 下面对它们做更详细的考察.

\section{$\S 113$}

改变坐标与之平行的那三个轴的位置, 可以化简通用方程, 现在我们用这个方法来 化简二阶面的通用方程, 该方法不影响方程所含类型, 二阶面的通用方程为
\[
\alpha z^{2}+\beta y z+\gamma x z+\delta y^{2}+\varepsilon x y+\zeta x^{2}+\gamma z+\theta y+\omega x+\kappa=0
\]
将坐标变成原点与原来相同的 $p, q, r$, 由 $\S 92$ 知新旧坐标的关系式为
\[
\begin{aligned}
x= & p(\cos k \cos m-\sin k \sin m \cos n)+ \\
& q(\cos k \sin m+\sin k \cos m \cos n)-r \sin k \sin n \\
y= & -p(\sin k \cos m+\cos k \sin m \cos n)- \\
& q(\sin k \sin m-\cos k \cos m \cos n)-r \cos k \sin n \\
z= & -p \sin m \sin n+q \cos m \sin n+r \cos n
\end{aligned}
\]
代入原方程, 记所得方程为
\[
A p^{2}+B q^{2}+C r^{2}+D p q+E p r+F q r+G p+H q+I r+K=0
\]
\section{$\S 114$}

我们选择任意角 $K, m, n$ 使 $D, E, F$ 为零. 虽然进行这种选择的计算并不太繁, 但可 能有人会问, 算得的角一定是实数吗? 为消除这疑问, 我们分两步使 $D, E, F$ 为零. 先使 $D, E$ 为零, 这是无疑的. 再改变垂直于 $P$ 的平面上平行于坐标 $r$ 的那根轴的位置, 使 $F$ 为 零. 这一步也容易实现. 事实上,置
\[
q=t \sin i+u \cos i, \quad r=t \cos i-u \sin i
\]
则含 $q r$ 的项产生含 $t u$ 的项. 容易选择 $i$, 因而 $q r$ 的系数为零. 这样二阶面的通用方程就 成了
\[
A p^{2}+B q^{2}+C^{2}+G p+H q+I r+K=0
\]
\section{$\S 115$}

接下去, 我们用增大或减小坐标 $p, q, r$, 也即用移动坐标原点的方法, 使系数 $G, H, I$ 为零 (参见 $\S 123$ ). 这样二阶面的通用方程进一步成了
\[
A p^{2}+B q^{2}+C r^{2}+K=0
\]
从该方程我们看到, 过原点的三个主平面都分二阶面为相似相等的两部分. 因而凡二阶 面都有三个直径面, 这三个直径面相交于一点, 这个点是二阶面的中心, 中心可以在无穷 远处, 这类似于我们把圆雉曲线视为有心曲线, 其中双曲线的中心就在离顶点无穷远处.

\section{$\S 116$}

前面我们把包含所有二阶面的那个方程化成了简化形式
\[
A p^{2}+B q^{2}+C r^{2}=a^{2}
\]
现在我们考虑该方程三系数 $A, B, C$ 都为正数的情形.

此时的二阶面不只有界, 有中心, 且三个直径面垂直相交 于中心. 参见图 143 , 设 $C$ 为中心 , $C A, C B, C D$ 为相垂直的轴, 坐标 $p, q, r$ 分别与这三个轴平行, 则三个直径面为 $A B a b$, $A D a, B D b$. 这三个面都分所给面为相等的两部分.


【图,待补】
%%![](https://cdn.mathpix.com/cropped/2023_02_05_f09dd824846f2342f056g-02.jpg?height=287&width=388&top_left_y=936&top_left_x=1108)

图 143

\section{$\S 117$}

令 $r=0$, 则方程成为 $A p^{2}+B q^{2}=a^{2}$, 表示的是主截线 $A B a b$. 该截线为椭圆, 中心在 $C$, 半轴
\[
C A=C a=\frac{a}{\sqrt{A}}, \quad C B=C b=\frac{a}{\sqrt{B}}
\]
令 $q=0$, 得方程 $A p^{2}+C r^{2}=a^{2}$, 表示的是主截线 $A D a$, 是以 $C$ 为中心的椭圆, 半轴
\[
C A=C a=\frac{a}{\sqrt{A}}, \quad C D=\frac{a}{\sqrt{C}}
\]
令 $p=0$, 得第三个主截线方程 $B q^{2}+C r^{2}=a^{2}$, 它是中心为 $C$, 半轴为
\[
C B=C b=\frac{a}{\sqrt{B}}, \quad C D=\frac{a}{\sqrt{C}}
\]
的椭圆,知道了曲面的三个主截线, 或者知道了半轴
\[
C A=\frac{a}{\sqrt{A}}, \quad C B=\frac{a}{\sqrt{B}}, \quad C D=\frac{a}{\sqrt{C}}
\]
就可以确定整个曲面, 我们称这第一类二阶面为椭圆, 它的三个主截线都是椭圆. 

\section{$\S 118$}

椭圆分为三种. 第一种, 三个主半轴 $C A, C B, C D$ 相等, 此时主截线都为圆,曲面本身 为球面,方程为
\[
p^{2}+q^{2}+r^{2}=a^{2}
\]
第二种,两个主半轴相等, 设 $C D=C B$, 或者系数 $C=B$, 则截线 $B D b$ 为圆,由方程
\[
A p^{2}+B\left(q^{2}+r^{2}\right)=a^{2}
\]
知, 平行于 $B D b$ 的平面截得的截线都为圆. 称这时的椭面为伸缩球面. $A C$ 大于 $B C$ 时为 伸长球面, $A C$ 小于 $B C$ 时为缩短球面. 最后, 第三种, 系数 $A, B, C$ 都不相等, 这时的椭面 就直接称为椭面.

\section{$\S 119$}

下一类二阶面,其方程
\[
A p^{2}+B q^{2}+C r^{2}=a^{2}
\]
的系数 $A, B, C$ 都不为零, 但有一个或两个为负. 设只 有一个系数为负,即方程为
\[
A p^{2}+B q^{2}-C r^{2}=a^{2}
\]
这里 $A, B, C$ 都为正数. 这类二阶面的中心和直径面 同于系数都为正的情形. 参见图 144 , 显然该曲面的主 截线 $A B a b$ 是椭圆, 这椭圆的半轴为 $A C=\frac{a}{\sqrt{A}}, B C=$ $\frac{a}{\sqrt{B}}$. 另外两个主截线 $A q, B S$ 是中心为 $C$, 共轭半轴为$\frac{a}{\sqrt{C}}$ 的双曲线.


【图,待补】
%%![](https://cdn.mathpix.com/cropped/2023_02_05_f09dd824846f2342f056g-03.jpg?height=366&width=519&top_left_y=1129&top_left_x=995)

图 144 

\section{$\S 120$}

这种面有似一个沿双曲线向上向下伸张的漏斗, 这漏斗有渐近雉面, 渐近雉面的方 程为 $A p^{2}+B q^{2}-C r^{2}=0$, 顶点在中心 $C$, 侧面由双曲线的渐近线形成. 这渐近雉面在曲 面的内部, $A=B$ 时为正雉面, $A$ 不等于 $B$ 时为斜雉面. 雉面的轴为垂直于平面 $A B a$ 的直 线 $C D$, 垂直于 $C D$ 的平面截得的截线都为相似于椭圆 $A B a b$ 的椭圆, 垂直于平面 $A B a b$ 的 平面截得的截线都为双曲线. 称这种曲面为外切于其渐近雉面的椭圆双曲面, 这是第二 类二阶面. 

\section{$\S 121$}

这第二类二阶面也可分为三种. $a=0$ 时为第一种, 此时椭圆 $A B a b$ 缩为一点, 双曲线 成为直线, 二阶面跟它的渐近雉面重合. 这时的二阶面包含所有的雉面, 直的斜的都包 含, 这是第一种. $A=B$ 时为第二种, 这时椭圆 $A B a b$ 成圆,二阶面成旋转面, 是双曲线绕 共轭轴旋转而成, 一般情况为第三种.

\section{$\S 122$}

第三类, 我们把 $p^{2}, q^{2}, r^{2}$ 的系数中有两个为负, 也即方程为
\[
A p^{2}-B q^{2}-C r^{2}=a^{2}
\]
时的二阶面作第三类, 参见图 145 , 令 $r=0$, 得第一主轴线 $E A F$, ea $f$, 是以 $C$ 为中心的双 曲线, 横半轴等于 $\frac{a}{\sqrt{A}}$, 共轭半轴等于 $\frac{a}{\sqrt{B}}$. 令 $q=0$, 得第二主截线 $A Q, a q$, 也是双曲线, 横 半轴同于第一主截线, 共轭半轴等于 $\frac{a}{\sqrt{C}}$. 第三主截线是虚的. 最后, 这类二阶面整个位于 渐近雉面之内, 因而称之为内切于渐近雉面的双曲面. 如果 $B=C$, 则成旋转面, 由双曲线 绕横轴生成, 这是特殊的一种, $a=0$ 得到前面讨论过的雉面.


【图,待补】
%%![](https://cdn.mathpix.com/cropped/2023_02_05_f09dd824846f2342f056g-04.jpg?height=333&width=675&top_left_y=1308&top_left_x=498)

图 145

\section{$\S 123$}

再一类是系数 $A, B, C$ 中有一个为零, 设 $C=0$, 则 $\S 114$ 求得的通用方程为
\[
A p^{2}+B q^{2}+G p+H q+I r+K=0
\]
用增大或减小坐标 $p, q$ 的方法可消去方程中的 $G p, H q$, 但消不去 $I r$. 进一步利用 $I r$ 可消 去常数项 $K$, 方程成为
\[
A p^{2}+B q^{2}=a r
\]
这里又分为 $A, B$ 同为正数, 和 $A, B$ 一正一负两种情形,两种情形下曲面的中心都在轴 $C D$ 上,在无穷远处. 

\section{$\S 124$}

先讨论 $A, B$ 同为正数的情形, 此时方程为
\[
A p^{2}+B q^{2}=a r
\]
这是第四类. 参见图 146. 令 $r=0$, 得第一主截线缩为一 点. 令 $q=0, p=0$, 得第二第三主截线 $M A m, N A n$ 都是 抛物线. 对这第四类二阶面, 垂直于轴 $A D$ 的平面截得 的线都是椭圆. 过 $A D$ 的平面截得的线都是抛物线. 因 而我们称这第四类为椭圆抛物面. 应该指出, $A=B$ 时得 到的是旋转面, 叫抛物嬖雉面; $a=0$ 时方程为
\[
A p^{2}+B q^{2}=b^{2}
\]
是柱面, $A=B$ 时为正圆柱, $A$ 不等于 $B$ 时为斜圆柱.


【图,待补】
%%![](https://cdn.mathpix.com/cropped/2023_02_05_f09dd824846f2342f056g-05.jpg?height=375&width=500&top_left_y=564&top_left_x=1028)

图 146

\section{$\S 125$}

再讨论 $A, B$ 一正一负的情形, 此时方程为
\[
A p^{2}-B q^{2}=a r
\]
这是第五类. 参见图 147 , 令 $r=0$, 得第一主截线为相交于点 $A$ 的两条直线 $E e, F f$. 我们 称过 $E e, F f$ 垂直于平面 $A B C$ 的平面为平面 $E e$ 和平面 $F f$. 平行于 $A B C$ 的平面截我们的 曲面所得截线都是双曲线. 这双曲线位于平面 $E e, F f$ 之间, 并以平面 $E e, F f$ 与截平面的 交线为渐近线, 且中心在轴 $A D$ 上. 可见我们的曲面在无穷远处与平面 $E e, F f$ 重合, 也即 我们的曲面有相交的两张渐近面. 另外两张主平面 $A C D, A B d$ 上的截线都为抛物线. 因 此称我们的曲面为有两张渐近面的抛物双曲面. $a=0$ 时, 方程为 $A p^{2}-B q^{2}=b^{2}$, 曲面为 双曲柱面, 垂直于轴 $A D$ 的平面截得的截线是双曲线, 都相等. 此外, $b=0$ 时得到的是两 张渐近平面.


【图,待补】
%%![](https://cdn.mathpix.com/cropped/2023_02_05_f09dd824846f2342f056g-05.jpg?height=460&width=625&top_left_y=1680&top_left_x=530)

图 147 

\section{$\S 126$}

最后,第六类二阶面是抛物柱面,方程为
\[
A p^{2}=a q
\]
垂直于 $A D$ 轴的截线是相似相等的抛物线, 顶点都在 $A D$ 上, 轴都平行.

这样, 二阶面就被分成了六类. 二阶面全都包含在这六类之中. 又, 如果令第六类中 的 $a=0$, 则方程成为 $A p^{2}=b^{2}$, 给出相平行的两张平面, 也是一种抛物面. 这类似于二阶 曲线, 我们讲过, 相交直线是一种双曲线, 平行直线是一种抛物线.

\section{$\S 127$}

虽然这六类面是根据二阶面的简化方程划分的, 但根据二阶面的任给方程, 我们都 可以容易地判定它属于哪一类. 事实上,记给定的方程为
\[
\alpha z^{2}+\beta y z+\gamma x z+\delta y^{2}+\varepsilon x y+\zeta x^{2}+\eta z+\theta y+\omega x+\kappa=0
\]
根据它的二次部分
\[
\alpha z^{2}+\beta y z+\gamma x z+\delta y^{2}+\varepsilon x y+\zeta x^{2}
\]
即可做出判定. 如果 $4 \alpha \zeta>\gamma^{2}, 4 \alpha \delta>\beta^{2}, 4 \delta \zeta>\varepsilon^{2}$, 且 $\alpha \varepsilon^{2}+\delta \gamma^{2}+\zeta \beta^{2}<\beta \gamma \varepsilon+4 \alpha \delta \zeta$, 则曲面 有界,属于我们称之为椭面的第一类.

\section{$\S 128$}

如果上节前三个条件不都满足, 且等式 $\alpha \varepsilon^{2}+\delta \gamma^{2}+\zeta \beta^{2}=\beta \gamma \varepsilon+4 \alpha \delta \zeta$ 不成立, 则二阶面 属第二或第三类, 是有渐近雉面的双曲面. 属第二类时内切于雉面, 属第三类时外切雉 面,如果等式
\[
a \varepsilon^{2}+\delta \gamma^{2}+\zeta \beta^{2}=\beta \gamma \varepsilon+4 \alpha \delta \zeta
\]
成立, 此时表达式
\[
\alpha z^{2}+\beta y z+\gamma x z+\delta y^{2}+\varepsilon x y+\zeta x^{2}
\]
可分解成或虚或实的两个因式. 为虚时属第四类, 为实时属第五类. 如果表达式的两个因 式相等, 也即为完全平方, 则属第六类. 可见, 根据任何一个方程都容易判断二阶面的类 别. 只有一个难点, 在二、三类的区分, 它们可能合而为一.

\section{$\S 129$}

可以类似地对三阶和更高阶面进行考察和分类. 显然只需考虑通用方程的最高次部 分. 对三阶面是考察次数为 3 的部分
\[
\alpha z^{3}+\beta y z^{2}+\gamma y^{2} z+\delta x^{2} z+\varepsilon x z^{2}+\zeta x y z+\cdots
\]
首先要判明这最高次部分可否分解成线性因式. 倘不可, 则三阶线有渐近三阶雉面. 使最高次部分为零, 成为方程, 则渐近三阶雉面的性质由这方程表示. 从方程可得到多种 类型的渐近三阶雉面, 不同类型的雉面对应不同类型的三阶面, 二阶雉面有正有斜, 属于 同一类型,三阶雉面不同, 可分为很多类.

\section{$\S 130$}

上节讲了最高次部分不能分解成线性因式的情形, 现在考虑可分解为线性因式的情 形. 因式可实可虚. 最高次部分有一个实因式时, 三阶面有一张渐近平面. 置另外一个因 式为零, 则所得方程或者有解或者无解. 没有非零解时, 三阶面只有一张渐近平面; 有非 零解时, 三阶面有两个渐近面, 一个平面,一个二阶雉面. 如果表达式有三个线性因式, 则 其中必有一个为实, 由另外两个同虚或同实得新的两类三阶面. 最后, 三个线性因式都为 实数时, 其中两个相等和三个全相等又给出两类. 三阶面都伸向无穷,都无界. 

\chapter{第六章 曲面与曲面的交线}

\section{$\S 131$}

曲面与平面的交线, 前面讲了其性质的考察方法, 由于这种交线整个位于截平面上, 因而可以用截平面上的两个坐标来表示它, 并用我们掌握的平面曲线知识去分析它. 曲 面与曲面的交线, 情况不同了, 交线不在一张平面上, 不能用两个坐标表示, 须用另外的 方法导出表示交线上任意点位置的方程.

\section{$\S 132$}

不在同一张平面上的点的位置, 可用至三个相垂直平面的三个距离来表示. 这样,描 述不在同一张平面上的曲线, 要用三个变量, 并且要求任给其中一个, 即可决定另外两 个. 一个三坐标方程描述一张面, 但要借助一个三坐标方程, 做到由三变量中的一个决定 另外两个,就不够了. 要做到这一点必须有两个三坐标方程.

\section{$\S 133$}

用直角三坐标的两个方程, 来描述不在一张平面上的曲线, 是方便的. 我们记坐标为 $x, y, z$. 从三坐标的两个方程就已做到由一个变量决定另外两个. 例如, 使 $y, z$ 都是 $x$ 的 函数. 从两个方程可以消去三个变量中的任何一个, 从而得到三个二元方程. 一个含 $x$, $y$, 另一个含 $x, z$, 再一个含 $y, z$. 并且这三个方程中的任何一个都可以从另外两个推出. 例如, 从含 $x, y$ 和含 $x, z$ 的两个中消去 $x$, 就得到含 $y, z$ 的.

\section{$\S 134$}

设给定一条非平面曲线如图 148 所示, $M$ 是它的任意一点. 任取三个相垂直的轴 $A B, A C, A D$, 它们决定三张相垂直的平面 $B A C, B A D, C A D$. 从点 $M$ 向平面 $B A C$ 引垂线 $M Q$, 从 $Q$ 向轴 $A B$ 引垂线 $Q P$, 则 $A P, P Q, Q M$ 就是确定曲线性质的两个方程的那三个 坐标. 令 $A P=x, P Q=y, Q M=z$. 从给定的两个三坐标方程消去 $z$, 得到只含 $x, y$ 的方 程, 平面 $B A C$ 上点 $Q$ 的位置就由这个方程决定. 所给曲线上的每一点 $M$ 都对应一点 $Q$, 这些点 $Q$ 构成 $B A C$ 上一条曲线 $E Q F, E Q F$ 的性质就由所得 $x, y$ 间方程表示. 


【图,待补】
%%![](https://cdn.mathpix.com/cropped/2023_02_05_f09dd824846f2342f056g-09.jpg?height=330&width=580&top_left_y=278&top_left_x=555)

图 148

\section{$\S 135$}

从曲线 $G M H$ 上的每一点 $M$ 向平面 $B A C$ 引垂线 $M Q$, 称这所有的点 $Q$ 构成的曲线 $E Q F$ 为 $G M H$ 在平面 $B A C$ 上的射影. 从描述 $G M H$ 的两个三坐标方程中消去 $z$, 所得即 射影 EQF 的方程. 类似地, 消去 $y, x$, 分别得到平面 $B A D$ 和 $C A D$ 上的投影的方程,但是 仅只一条射影 $E Q F$ 决定不了曲线 $G M H$, 还要加上从每一个点 $Q$ 引的垂线 $Q M=z$. 也即 要决定曲线 $G M H$, 在以 $x, y$ 为变元的射影方程之外,还要加上以 $z, x$ 为变元, 或者以 $z$, $y$ 为变元, 或者以 $z, x, y$ 为变元的方程, 用来对每个点 $Q$ 确定垂线 $Q M=z$ 的长.

\section{$\S 136$}

$G M H$ 在平面 $B A D$ 和 $C A D$ 上的投影方程, 分别以 $z, x$ 和 $z, y$ 为变元, 而 $G M H$ 所在 曲面的方程以 $z, y, x$ 为变元. 可见曲线 $G M H$ 可由它在两个平面上的投影决定, 也可以 由它所在的一张曲面和它在一张平面上的投影决定. 后一情况下, 从射影的每一点所引 垂线 $Q M$ 与曲线 $G M H$ 所在平面的交线,构成曲线 $G M H$ 本身.

\section{$\S 137$}

有了关于非平面曲线的以上讨论, 确定两个曲面的交这件事就不难了. 事实上, 两张 平面之交为直线, 而两张曲面的交, 可以是直线, 可以是曲线. 这曲线可以是平面曲线, 可 以是非平面曲线. 不管是哪一种,这交线上的每一点都同属于两张曲面,同时满足两个方 程. 如果这两个曲面方程参考的是同一套三垂直主平面, 或者参考的是同一套三垂直轴 $A B, A C, A D$, 那么我们就用这两个方程表示交线.

\section{$\S 138$}

曲面用三坐标方程表示, 给定两个相交曲面, 就是给定坐标 $x, y, z$ 的两个方程. 这里 要两个方程的坐标轴相同. 从坐标 $x, y, z$ 的两个方程消去一个坐标, 得到一个两坐标方 程, 这个两坐标方程就是交线在剩下的这两个坐标所成平面上的投影. 也可以用这个方 法考察曲面与平面的交线. 平面的通用方程为 $\alpha z+\beta y+\gamma x=f$, 将解出的 $z=$ $\frac{f-\beta y-\gamma x}{\alpha}$ 代入曲面方程, 得到的就是交线在坐标 $x, y$ 所成平面上的投影方程, 并且 $z=\frac{f-\beta y-\gamma x}{\alpha}$ 给出射影点 $Q$ 处垂线 $Q M$ 的值, $M$ 为交线上的点.

\section{$\S 139$}

射影方程无实解, 表示两张曲面不相交. 例如射影方程为 $x^{2}+y^{2}+a^{2}=0$ 时,两张面 就不相交. 如果射影方程只给出一个点, 也即整个射影为一个点, 则交线为一个点, 即两 个面相切. 这切点也可以从方程求出. 两个面可以有无穷多个切点, 这无穷多个切点可以 构成一条线, 这线可以是直线, 可以是曲线. 例如, 平面与柱面或雉面相切, 切点构成直 线; 正雉面外切于球面, 切点构成一个圆. 射影方程有两个相等因式时, 切点所成的线可 由方程求出,此时切点所成的线是两条交线的重合.

\section{$\S 140$}

为进一步说明, 我们看看球面与平面相交的情形. 取球面中心为原点, 球面方程为
\[
z^{2}+y^{2}+x^{2}=a^{2}
\]
任何位置上的平面, 其方程都为
\[
\alpha z+\beta y+\gamma x=f
\]
将 $z=\frac{f-\beta y-\gamma x}{\alpha}$ 代入球面方程, 得到以 $x, y$ 为变元的射影方程
\[
0=f^{2}-\alpha^{2} a^{2}-2 \beta f y-2 \gamma f x+\left(\alpha^{2}+\beta^{2}\right) y^{2}+2 \beta \gamma x y+\left(\alpha^{2}+\gamma^{2}\right) x^{2}
\]
显然, 该方程有实解时, 它描述的是椭圆; 该方程的解为虚数时, 球面与平面不相交; 如果 椭圆缩为一点, 则球面与平面相切. 为具体考虑这几种情况, 我们求出
\[
y=\frac{\beta f-\beta \gamma x+\alpha \sqrt{a^{2}\left(\alpha^{2}+\beta^{2}\right)-f^{2}+2 \gamma f x-\left(\alpha^{2}+\beta^{2}+\gamma^{2}\right) x^{2}}}{\alpha^{2}+\beta^{2}}
\]
如果 $f$ 的值使根号部分不为实数, 则球面与平面既不相交也不相切.

\section{$\S 141$}

置 $f=a \sqrt{\alpha^{2}+\beta^{2}+\gamma^{2}}$, 则
\[
y=\frac{\beta f-\beta \gamma x \pm \alpha x \sqrt{-\left(\alpha^{2}+\beta^{2}+\gamma^{2}\right)} \mp \alpha \gamma a \sqrt{-1}}{\alpha^{2}+\beta^{2}}
\]
该方程只在 
\[
\begin{gathered}
\text { Finfinile analysirs 无穷分析引论 Snlvaduclion } \\
x=\frac{\gamma a}{\sqrt{\alpha^{2}+\beta^{2}+\gamma^{2}}}, \quad y=\frac{\beta a}{\sqrt{\alpha^{2}+\beta^{2}+\gamma^{2}}}
\end{gathered}
\]
时才有实解. 因此, 如果 $f=a \sqrt{\alpha^{2}+\beta^{2}+\gamma^{2}}$, 则方程 $\alpha z+\beta y+\gamma x=f$ 表示的平面与所给 球面相切于点
\[
\begin{aligned}
& x=\frac{\gamma a}{\sqrt{\alpha^{2}+\beta^{2}+\gamma^{2}}} \\
& y=\frac{\beta a}{\sqrt{\alpha^{2}+\beta^{2}+\gamma^{2}}} \\
& z=\frac{\alpha a}{\sqrt{\alpha^{2}+\beta^{2}+\gamma^{2}}}
\end{aligned}
\]
这三个值可用初等几何方法验证.

\section{$\S 142$}

由此可以推出一般规则, 来判断一个曲面与平面或另一个曲面是否相切. 从两个方 程消去一个变量, 再对所得方程的因式进行考察: 如果有两个虚线性因式, 则所给面相切 于一点. 这一点可以用置两个线性因式为零的方法得到; 如果有两个实线性因式, 且相 等, 则两个面的切点构成一条直线; 如果有两个相等的非线性因式, 也即被一个完全平方 除得尽, 则置平方根等于零, 得切点所成线的投影. 由此可见, 如果我们的方程有四个虚 因式, 则两个面相切于两点.

\section{$\S 143$}

为解释得更清楚, 我们考虑雉面和中心在雉面轴上的球面, 看它们的相切. 球面和雉 面的方程分别为 $z^{2}+y^{2}+x^{2}=a^{2}$ 和 $(f-z)^{2}=m x^{2}+n y^{2}$, 我们视 $f$ 为雉面顶点至球面中 心的距离. 消去 $y$, 得
\[
(f-z)^{2}=n a^{2}-n z^{2}+(m-n) x^{2}
\]
这是交线在坐标 $x, y$ 所定平面上的射影方程. 先考虑正雉面, 即 $m=n$ 的情形, 此时
\[
z=\frac{f \pm \sqrt{n(1+n) a^{2}-n f^{2}}}{1+n}
\]
如果 $f=a \sqrt{1+n}$, 得重根 $z=\frac{a}{\sqrt{1+n}}$, 切点构成一条线, 是一个圆, 它在过轴的平面上的 射影为一条垂直于轴的直线.

\section{$\S 144$}

再考虑斜雉面, 即 $m \neq n$ 的情形. 此时从方程看上去像是恒有交线, 但实际上常常并 无交线. 事实上, 只要 $m>n$, 我们就得到交线射影实方程. 应该指出, 射影为实并不意味 着交线一定为实. 即射影为实对交线为实并不充分, 必需再加上从射影至交线的垂线为 实. 换句话说, 曲线实射影必实, 反之, 则不然, 射影实曲线可以不实. 这是应该特别给以 注意的,不要错误地把射影实作为曲线实的充分条件.

\section{$\S 145$}

但坐标 $x, y$ 所定平面上的射影, 情况不同. 这个平面上没有不与雉面上的点对应的 点, 因而曲线在这个平面上的射影为实, 则曲线本身必实. 将 $z=\sqrt{a^{2}-x^{2}-y^{2}}$ 代入雉面 方程,得
\[
f-\sqrt{a^{2}-x^{2}-y^{2}}=\sqrt{m x^{2}+n y^{2}}
\]
或
\[
a^{2}+f^{2}-(1+m) x^{2}-(1+n) y^{2}=2 f \sqrt{a^{2}-x^{2}-y^{2}}
\]
继而

$\left.\begin{array}{l}
	& \left.\left.
	\begin{array}{l}\left(a^2-f^2\right)^2-2\left(a^2-f^2\right) \\ 
		-2\left(a^2+f^2\right)\end{array}\right\} x^2 
	\begin{array}{l}-2\left(a^2-f^2\right) \\ 
		-2\left(a^2+f^2\right) n\end{array}\right\} y^2+ \\ 
	& (1+m)^2 x^4+2(1+m)(1+n) x^2 y^2+(1+n)^2 y^4 
\end{array}
\right\}=0$

进而
\[
\begin{aligned}
& \left.\begin{array}{l}\frac{a^{2}-f^{2}+n\left(a^{2}+f^{2}\right)-(1+m)(1+n) x^{2}}{(1+n)^{2}} \\\pm \frac{2 f}{(1+n)^{2}} \sqrt{n(1+n) a^{2}-n f^{2}+(m-n)(1+n) x^{2}}\end{array}\right\}=y^{2} \\
& \left.\begin{array}{l}\frac{a^{2}-f^{2}+m\left(a^{2}+f^{2}\right)-(1+m)(1+n) y^{2}}{(1+m)^{2}} \\\pm \frac{2 f}{(1+m)^{2}} \sqrt{m(1+m) a^{2}-m f^{2}+(n-m)(1+m) y^{2}}\end{array}\right\}=x^{2}
\end{aligned}
\]
\section{$\S 146$}

要所得方程有因式, 则应 $f^{2}=(1+n) a^{2}$, 或 $f^{2}=(1+m) a^{2}$. 前一情况下我们有
\[
y^{2}=\frac{n a^{2}-(1+m) x^{2}}{1+n} \pm \frac{2 f x \sqrt{m-n}}{(1+n) \sqrt{1+n}}
\]
从而, 如果 $m<n$, 则
\[
x=0, \quad y=\pm a \sqrt{\frac{n}{1+n}}, \quad z=\frac{a}{\sqrt{1+n}}
\]
得到至雉面轴等距的两个切点, 如果 $m>n$, 则应取方程
\[
x^{2}=\frac{m a^{2}-(1+n) y^{2}}{1+m} \pm \frac{2 f y \sqrt{n-m}}{(1+m) \sqrt{1+m}}
\]
$y \neq 0$ 时,该方程没有实解,$y=0$ 时
\[
x=\pm a \sqrt{\frac{m}{1+m}}, \quad z=\frac{a}{\sqrt{1+m}}
\]
得到另外两个切点, 都位于雉面的最窄部分, 任何情况下的相切, 都可以用类似的方法进 行考察.

\section{$\S 147$}

前面讲过曲线切线的求法, 从它可以引出曲面切平面的一种更简单的求法. 参见图 149 , 设要求出其切平面的那张曲面, 由坐标 $A P=x, P Q=y, Q M=z$ 间的方程给出. 借助 这个方程, 我们来确定曲面上点 $M$ 处的切平面. 先考虑过点 $M$ 的一张平面, 求出它与所 给曲面的交线在点 $M$ 处的切线, 这条切线位于我们所要的切平面上,再考虑过点 $M$ 的另 一张平面,也求出它与所给曲面的交线在点 $M$ 处的切线,这条切线就也位于我们所要的 切平面上,这两条切线所决定的平面上. 这两条切线所决定的平面就是曲面在点 $M$ 处的 切平面.


【图,待补】
%%![](https://cdn.mathpix.com/cropped/2023_02_05_f09dd824846f2342f056g-13.jpg?height=404&width=440&top_left_y=1076&top_left_x=613)

图 149

\section{$\S 148$}

在平面 $A P Q$ 上分别引平行和垂直于轴 $A P$ 的直线 $Q S$ 和 $Q P$. 点 $M$ 与这两条直线各 决定一张平面,这两张平面都垂直于平面 $A P Q$. 记这两张平面与我们曲面的交线为 $E M$ 和 $F M$, 记这两条交线在点 $M$ 处的切线为 $M S$ 和 $M T$, 这两条切线分别交直线 $Q S$ 和 $Q T$ 于 $S$ 和 $T . Q S$ 和 $Q T$ 分别为次切线, 求得了这两条切线, 则平面 $S M T$ 就是我们曲面点 $M$ 处的切平面. 连成直线 $S T$, 则 $S T$ 为切平面与平面 $A P Q$ 的交线. 从 $Q$ 向 $S T$ 引垂线 $Q R$, 则 $Q R$ 比 $Q M$ 等于切平面与平面 $A P Q$ 所成角即 $\angle M R Q$ 的余切.

\section{$\S 149$}

令前节用切线法求得的次切线 $Q S=s, Q T=t$, 则 
\[
P T=t-y, \quad P X=s-\frac{s y}{t}
\]

从而
\[
A X=x+\frac{s y}{t}-s
\]
即直线 $S T$ 与轴 $A P$ 的交点 $X$ 已知, 又由 $\angle A X S=\angle T S Q$ 知 $\tan \angle A X S$ 等于 $\frac{t}{s}$. 这样切 平面与平面 $A P Q$ 交线的位置成为已知. 再由 $S T=\sqrt{s^{2}+t^{2}}$, 我们有
\[
Q R=\frac{s t}{\sqrt{s^{2}+t^{2}}}
\]
用 $Q M$ 除以该表达式,得
\[
\tan \angle M R Q=\frac{z \sqrt{s^{2}+t^{2}}}{s t}
\]
引 $M N$ 垂直于 $M R$, 则 $M N$ 在点 $M$ 处既垂直于切平面, 又垂直于曲面本身, 它的位置由下 面的等式决定
\[
Q N=\frac{z^{2} \sqrt{s^{2}+t^{2}}}{s t}
\]
从 $N$ 向轴 $A P$ 引垂线 $N V$, 则由 $\angle Q N V=\angle Q S T$, 我们有
\[
P V=\frac{z^{2}}{s}=Q W, \quad N W=\frac{z^{2}}{t}
\]
因此, 如果用所说的方法确定了平面 $A P Q$ 上点 $N$ 的位置, 则直线 $N M$ 垂直于所给曲面.

\section{$\S 150$}

我们讲了如何用射影求面与面的交线. 现在看看射影的阶与曲面的阶的关系. 首先, 两个一阶面, 即两个平面, 交线和交线的射影都是一阶线. 其次, 我们看到了,一阶面、二 阶面, 其交线的射影的阶数不会高于 2 . 类似地,一阶面、三阶面, 其交线的投影的阶数不 会高于 3 . 类推. 再次,两个二阶面, 其交线射影的阶数不会高于 4 . 一般地, $m$ 阶面、 $n$ 阶 面,其交线射影的阶数不会高于 $m n$.

\section{$\S 151$}

两个面都不是平面,其交线,大多不是平面曲线. 两个面, 如果从它们的方程可以产 生方程 $\alpha z+\beta y+\gamma x=f$, 则这两个面的交线为平面曲线. 例如, 从两个方程可将两个变 量, 比如 $z$ 和 $y$ 用第三个变量 $x$ 表出, 即 $z=P, y=Q, P, Q$ 是 $x$ 的函数, 这个时候, 如果有 数 $n$ 使得 $P+n Q$ 只含 $x$ 的一次幂和常数, 即 $P+n Q=m x+k$, 则交线为平面曲线, 所在平 面的方程为 $z+n y=m x+k$. 

\section{$\S 152$}

一个具体的例子, 设给定的两个曲面为柱面和椭圆双曲面, 方程分别为
\[
\begin{gathered}
z^{2}=x^{2}+y^{2} \\
z^{2}=x^{2}+2 y^{2}-2 a x-a^{2}
\end{gathered}
\]
将前一个代入后一个, 得
\[
x^{2}+2 y^{2}-2 a x-a^{2}=x^{2}+y^{2}
\]
从而
\[
y=\sqrt{2 a x+a^{2}}, \quad z=x+a
\]
最后这个方程表明, 整个交线在方程 $z=x+a$ 确定的平面上, 用这样的方法可以回答关 于曲面的很多问题. 进一步的阐述留给无穷分析本身去做, 我们的这两本书仅仅是个引 论.

