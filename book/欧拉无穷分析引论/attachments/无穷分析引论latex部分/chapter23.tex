\chapter{第五章二阶线}

\section{$\S 85$}

一阶线都为直线, 初等几何对直线的讲述已经充分, 二阶线, 在曲线中它最简单, 在 整个高等几何中又有着广泛的应用, 我们对它作些较深人的考察. 二阶线也叫圆雉曲线, 具有不少有意义的性质, 其中有古代几何学家发现的, 也有现代入补充进去的. 很多作者 接着初等几何马上就对这些性质进行讲述, 这足见其重要. 这些性质有来自方程的, 有来 自平面与圆雉截线的, 有来自其他作图方法的. 我们这里只讲来自方程而不使用其他辅 助手段的.

\section{$\S 86$}

考察二阶曲线的通用方程
\[
0=\alpha+\beta x+\gamma y+\delta x^{2}+\varepsilon x y+\zeta y^{2}
\]
我们已经证明了, 它包含关于任何轴、任何坐标角的一切二阶线, 改写它为
\[
y^{2}+\frac{(\varepsilon x+\gamma) y}{\zeta}+\frac{\delta x^{2}+\beta x+\alpha}{\zeta}=0
\]
可见, 每个横标 $x$ 都或者有两个纵标与之对应, 或者没有纵标与之对应, 决定于 $y$ 的两个 根为实为虚. 如果 $\zeta=0$, 则每一个横标 $x$ 都只有一个纵标与之对应, 另一个在无穷远处, 这并不阻碍我们的考察.

\section{$\S 87$}

两个 $y$ 值都为实数时, 纵标线 $P M N$ 交曲线于两点, 设为 $M, N$ (图 19). 两根之和
\[
P M+P N=\frac{-\varepsilon x-\gamma}{\zeta}=\frac{-\varepsilon \cdot A P-\gamma}{\zeta}
\]
这里取直线 $A E F$ 为轴, 点 $A$ 为原点, $\angle A P N$ 为可任取的坐标角, 依所取坐标角画另外一 条纵标线 $m p n$, 这里 $p m$ 为负, 我们得到
\[
p n-p m=\frac{-\varepsilon \cdot A p-\gamma}{\zeta}
\]
前式减去后式,得
\[
P M+p m+P N-p n=\frac{\varepsilon(A p-A P)}{\zeta}=\frac{\varepsilon \cdot P p}{\zeta}
\]
从点 $m, n$ 引平行于轴的直线, 记与纵标线 $P M N$ 的交点为 $\mu$ 和 $\nu$, 则
\[
M_{\mu}+N_{\nu}=\frac{\varepsilon \cdot P p}{\zeta}
\]
也即 $M_{\mu}+N_{\nu}$ 比 $P p$ 或 $m \mu$ 或 $n \nu$, 都等于 $\varepsilon$ 比 $\zeta$, 为常数. 我们指出, 不管直线 $M N, m n$ 画于 曲线何处, 只要与轴的交角等于给定角, 且 $n \nu, m \mu$ 平行于轴, 这个比式就成立.


【图,待补】
%%![](https://cdn.mathpix.com/cropped/2023_02_05_00a02f82302d074ee0d6g-12.jpg?height=304&width=584&top_left_y=561&top_left_x=531)

图 19

\section{$\S 88$}

平移纵标线 $P M N$ 使点 $M, N$ 重合, 也即使纵标 线成切线,参见图 20. 记这切线为 $K C I$, 它平行于 $M N$ 和 $m n$. 我们称两端在曲线上的线段为弦, 从点 $M, N$, $m, n$ 向切线画平行于轴的直线 $M I, N K, m i, n k$. 线段 $C K, C I$ 在点 $C$ 两侧, 应该反号, 由此得
\[
\begin{gathered}
(C I-C K): M I=\varepsilon: \zeta \\
(C i-C k): m i=\varepsilon: \zeta
\end{gathered}
\]
从而
\[
(C I-C K): M I=(C i-C k): m i
\]
或


【图,待补】
%%![](https://cdn.mathpix.com/cropped/2023_02_05_00a02f82302d074ee0d6g-12.jpg?height=460&width=445&top_left_y=1108&top_left_x=1010)

图 20
\[
M I: m i=(C I-C K):(C i-C k)
\]
\section{$\S 89$}

轴关于曲线的位置任意, 所以直线 $M I, N K, m i, n k$ 的位置也任意, 只要它们彼此平 行, 就恒有
\[
M I: m i=(C I-C K):(C i-C k)
\]
设 $L$ 平分 $M N$, 引 $M I, N K$ 使平行于 $C L$, 则 $C I=C K$, 从而 $C I-C K=0$, 且
\[
C i-C k=\frac{m i}{M I}(C I-C K)=0
\]
延长 $C L$ 至点 $l$. 由 $m i, n k$ 也平行于 $C L$, 我们有 $m l=n l$. 因而过切点的直线 $C L l$, 如果它 平分一根平行于切线的弦 $M N$, 它就平分一切平行于切线的 $m n$. 

\section{$\S 90$}

由于直线 $C L l$ 等分一切平行于切线的弦, 我们称它为二阶线或圆雉曲线的直径. 二 阶线任何一点处都有切线, 因而二阶线有无穷多条直径. 对每一条切线 $I C K$ 我们都可以 画一条平行于它的弦 $M N$, 记 $M N$ 的等分点为 $L$, 直线 $C L$ 就是我们的二阶线的直径, 它 等分平行于切线 $I K$ 的一切弦.

\section{$\S 91$}

由此,如果直线 $L l$ 等分任何两条相平行的弦 $M N$ 和 $m n$, 那它就等分一切平行于这 两条弦的弦. 二阶线上处处有切线, 对每一条切线我们都可以画平行于它的弦, 从而就有 一条直径. 由此我们得到一个新的方法, 可求出二阶线的无穷多条直径. 任意地画两条相 平行的弦 $M N$ 和 $m n$, 过其中点 $L, l$ 的直线就是一条直径, 它等分平行于这两条弦的一切 弦. 记这直径 $L l$ 与曲线的交点为 $C$, 过点 $C$ 平行于弦的直线是曲线在点 $C$ 处的切线.

\section{$\S 92$}

从方程
\[
y^{2}+\frac{\varepsilon x+y}{\zeta} y+\frac{\delta x^{2}+\beta x+\alpha}{\zeta}=0
\]
的两根之和我们得到了前面的性质, 现在我们来考察它的两根之积
\[
P M \cdot P N=\frac{\delta x^{2}+\beta x+\alpha}{\zeta}
\]
( $\S 87$ 图 19). 表达式 $\frac{\delta x^{2}+\beta x+\alpha}{\zeta}$ 或者有两个实因式, 或者没有实因式. 当曲线与轴有两 个交点 $E, F$ 时它有两个实因式. 事实上, 由交点处 $y=0$ 得 $\frac{\delta x^{2}+\beta x+\alpha}{\zeta}=0, A E, A F$ 为 根, $(x-A E)(x-A F)$ 为因式.

从而由 $x=A P$ 得
\[
\frac{\delta x^{2}+\beta x+\alpha}{\zeta}=\frac{\delta}{\zeta}(x-A E)(x-A F)=\frac{\delta}{\zeta} \cdot P E \cdot P F
\]
该式与两根之积比较, 得
\[
P M \cdot P N=\frac{\delta}{\zeta} \cdot P E \cdot P F
\]
即 $P M \cdot P N$ 比 $P E \cdot P F$ 等于 $\delta$ 比 $\zeta$, 为常数. 纵标线 $P M N$ 的位置任意, 当然它与轴所成 的 $\angle N P F$ 必须等于坐标角, 引纵标线 $m n$, 则由 $p E, m n$ 为负, 类似地得
\[
p m \cdot p n=\frac{\delta}{\zeta} p E \cdot p F
\]
\section{$\S 93$}

$E, F$ 为二阶线上任两点 (图 21), 对直线 $P E F$, 只要弦 $N M P, n p m$ 平行, 则恒有
\[
P M \cdot P N: P E \cdot P F=p m \cdot p n: p E \cdot p F
\]
因为两端的比式都等于 $\delta: \zeta$. 类似地, 如果取直线 $P M N$ 为轴(轴可任取), 那么对任意条 平行于 $P E F$ 的直线 $e q f$, 我们都有
\[
P M \cdot P N: P E \cdot P F=q M \cdot q N: q e \cdot q f=p m \cdot p n: p E \cdot p F
\]
从而
\[
q e \cdot q f: p E \cdot p F=q M \cdot q N: p m \cdot p n
\]
这样一来, 对任意两对平行弦 $e f, E F$ 和 $M N, m n$, 如果其交点为 $P, p, q, r$, 我们都有关 系式

$P M \cdot P N: P E \cdot P F=p m \cdot p n: p E \cdot p F=q M \cdot q N: q e \cdot q f=r m \cdot r n: r e \cdot r f$ 这是二阶线的第二条一般性质.


【图,待补】
%%![](https://cdn.mathpix.com/cropped/2023_02_05_00a02f82302d074ee0d6g-14.jpg?height=319&width=503&top_left_y=1066&top_left_x=574)

图 21

\section{$\S 94$}

如果曲线的两个点 $M, N$ 重合,则直线 $P M N$ 变为重合点处的切线, $P M \cdot P N$ 变为 $P M$ 或者 $P N$ 自乘. 由此得到切线的一条新性质. 设直线 $C P p$ 是二阶线点 $C$ 处的切线 (图 24), 又设 $P M N, p m n$ 是任何两条平行线, 因而与切线交角相等. 这样由求得的性 质, 我们有
\[
P C^{2}: P M \cdot P N=p C^{2}: p m \cdot p n
\]
也即, 对与切线交角相等的弦 $M N$, 我们有 $P C^{2}$ 比 $P M \cdot P N$ 为常数.

\section{$\S 95$}

由此我们得到, 如果 $C D$ 是二阶线的任何一条直径 ( 888 图 20), 它等分一切平行弦 $M N, m n$, 交曲线于点 $C, D$, 则
\[
C L \cdot L D: L M \cdot L N=C l \cdot l D: \operatorname{lm} \cdot \ln
\]
由 $L M=L N, l m=\ln$, 得
\[
L M^{2}: l m^{2}=C L \cdot L D: C l \cdot l D
\]
也即半弦平方 $L M^{2}$ 比乘积 $C L \cdot L D$ 为常数. 我们看看取直径 $C D$ 为轴, 半弦 $L M$ 为纵标线 时的二阶线方程. 设直径 $C D=a$, 横标 $C L=x$, 纵标 $L M=y$, 由 $L D=a-x$, 得 $y^{2}$ 比ax- $x^{2}$ 为常数,记这常数为 $h$ 比 $k$, 这样我们得到二阶线的方程
\[
y^{2}=\frac{h}{k}\left(a x-x^{2}\right)
\]
\section{$\S 96$}

从前面得到的二阶线的两条性质,可以推出另外 几条性质. 设二阶线的两条相平行的弦 $A B, C D$ 已给 (图 22). 连 $A C, B D$ 成四边形 $A B D C$, 过曲线上任一点 $M$ 作平行于 $A B, C D$ 的弦 $M N$, 交 $A C, B D$ 于 $P, Q$, 则 $P M$ 等于 $Q N$. 等分 $A B, C D$ 的直线等分 $M N$. 由初等几 何知该直线也等分线段 $P Q$. 这样, 由线段 $M N, P Q$ 被 同一点等分, 我们得到 $M P=N Q, M Q=N P$. 这样一 来, 借助 $A, B, C, D$ 之外的第五点 $M$, 我们可以确定第 六点 $N$, 使 $N Q=M P$.


【图,待补】
%%![](https://cdn.mathpix.com/cropped/2023_02_05_00a02f82302d074ee0d6g-15.jpg?height=419&width=503&top_left_y=801&top_left_x=1010)

图 22

\section{$\S 97$}

已知 $M Q \cdot Q N$ 比 $B Q \cdot D Q$ 为常数, 由 $Q N=M P$ 得 $M P \cdot M Q$ 比 $B Q \cdot D Q$ 为常数, 从 而任取曲线上另外一点 $c$, 过点 $c$ 引平行于 $A B, C D$ 的直线 $G c H$ 交 $A C, B D$ 于 $G, H$, 则 $c G \cdot c H$ 比 $B H \cdot D H$ 为同一常数, 即
\[
c G \cdot c H: B H \cdot D H=M P \cdot M Q: B Q \cdot D Q
\]
如果过点 $M$ 引平行于 $B D$ 的直线 $R M S$ 交 $A B, C D$ 于 $R, S$, 那么由 $B Q=M R, D Q=M S$ 也 得到 $M P \cdot M Q$ 比 $M R \cdot M S$ 为常数. 从而, 如果过任一点 $M$ 引两条直线, 一条 $M P Q$ 平行 于边 $A B, C D$, 另一条 $R M S$ 平行于边 $B D$, 记与四边形 $A B D C$ 四边的交点为 $P, Q, R, S$, 则 这四点的分布满足 $M P \cdot M Q$ 比 $M R \cdot M S$ 为常数.

\section{$\S 98$}

代替平行于 $A B$ 的弦 $C D$, 考虑直线 $D c$, 再加上弦 $A c$ 及原来的直线 $M Q, R M S$, 这两 条线过点 $M$, 分别平行于边 $A B$ 和 $B D$, 交四边形 $A B D c$ 的边于 $p, Q, R, s$, 我们得到类似 的性质. 事实上
\[
M P \cdot M Q: B Q \cdot D Q=c G \cdot c H: B H \cdot D H
\]
从而由 $R S$ 平行并等于 $B D$ 得
\[
M P \cdot M Q: M R \cdot M S=c G \cdot c H: B H \cdot D H
\]
由 $\triangle A P p$ 和 $\triangle A G c$ 相似, 得
\[
P p \cdot A P=G c \cdot A G
\]
从而由
\[
A P: A G=B Q: B H
\]
得
\[
P p \cdot B Q=G c \cdot B H
\]
由 $\triangle D S s$ 和 $\triangle c H D$ 相似, 得
\[
D S(M Q): S s=c H: D H
\]
利用 $B Q=M R$, 从这两个比式得
\[
M Q \cdot P p: M R \cdot S s=c G \cdot c H: B H \cdot D H
\]
同前面的比例式比较, 得
\[
M P \cdot M Q: M R \cdot M S=P p \cdot M Q: M R \cdot S s
\]
前面取和后项取和,得
\[
M P \cdot M Q: M R \cdot M S=M P \cdot M Q: M R \cdot M s
\]
即与点 $c, M$ 在曲线上的位置无关, 只要过点 $M$ 的直线 $M Q, R s$ 平行于弦 $A B, C D, M p$. $M Q$ 比 $M R \cdot M s$ 的值就不变.

从所得比例式得
\[
M P: M S=M p: M s
\]
即只要点 $M$ 不变,点 $c$ 变动时只是点 $p, s$ 的位置变动, $M p$ 比 $M s$ 的值不变.

\section{$\S 99$}

设 $A, B, C, D$ 为二阶线上的任意四点, 连四点成内接四边形 (图 23), 利用前面所讲, 我们可以推出圆锥曲线的一条性质. 从曲线上任何一点 $M$ 向四边形的四边引直线 $M P$, $M Q, M R, M S$, 使与四边所成的角相等, 都等于给定的角, 则引向对边的直线的长度的积 的比为常数, 即对曲线上任意一点 $M$, 只要 $P, Q, R, S$ 处的角相同, $M P \cdot M Q$ 比 $M R \cdot M S$ 就为固定的数. 为证明这一点, 我们过点 $M$ 引两条直线 $M q$ 和 $r s$, 前者平行于 $A B$, 后者平 行于 $B D$, 记这两条线与四边形各边的交点为 $p, q, r, s$, 那么由前面讲的我们知道 $M p$ $M q$ 比 $M r \cdot M s$ 为固定的数, 由于角为已给, 所以比 $M P: M p, M Q: M q, M R: M r, M S$ 都 为已给, 从而 $M P \cdot M Q$ 比 $M R \cdot M S$ 为已给.

\section{$\S 100$}

前面我们看到, 如果延长平行弦 $M N, m n$, 使其与切线 $C P p$ 相交于 $P, p$, 如图 24 , 则
\[
P M \cdot P N: C P^{2}=p m \cdot p m: C p^{2}
\]

【图,待补】
%%![](https://cdn.mathpix.com/cropped/2023_02_05_00a02f82302d074ee0d6g-17.jpg?height=354&width=850&top_left_y=273&top_left_x=413)

图 23

如果我们选择点 $L, l$ 使 $P L$ 是 $P M, P N$ 的比例中项, $p l$ 是 $p m, p m$ 的比例中项, 则
\[
P L^{2}: C P^{2}=p l^{2}: C p^{2}
\]
从而 $P L: C P=p l: C p$, 由此知 $L, l$ 在过切点 $C$ 的直线上. 因此, 如果点 $L$ 分任何一条纵 标 线 $P M N$ 成 $P L^{2}=P M \cdot P N$, 则过点 $C, L$ 的直线 $C L D$ 与任一纵标线 $p m n$ 的交点 $l$ 都分 $p m n$, 使得 $p l$ 是 $p m, p n$ 的比例中项. 或者, 如果点 $L, l$ 分纵标线 $P N, p n$, 使得
\[
P L^{2}=P M \cdot P N, \quad p l^{2}=p m \cdot p n
\]
则点 $L, l, C$ 共线, 且这条直线以同样的比分一切与 $P N, p n$ 平行的纵标线.


【图,待补】
%%![](https://cdn.mathpix.com/cropped/2023_02_05_00a02f82302d074ee0d6g-17.jpg?height=380&width=467&top_left_y=1074&top_left_x=590)

图 24

\section{$\S 101$}

前面我们讨论了二阶线直接从方程形状得到的性质. 下面我们讨论更为深人的性 质, 从二阶线的通用方程
\[
y^{2}+\frac{\varepsilon x+\gamma}{\zeta} y+\frac{\delta x^{2}+\beta x+\alpha}{\zeta}=0
\]
我们看到, 对给定的对应纵标为 $P M, P N$ 的横标 $A P=x$, 我们可以确定等分弦 $M N$ 的直 径. 事实上, 设 $I G$ 是求得的直径(图 25), 它在点 $L$ 处等分弦 $M N$, 令 $P L=z$, 由 $z=\frac{1}{2} P M+$ $\frac{1}{2} P N$ 得
\[
z=\frac{-\varepsilon x-\gamma}{2 \zeta}
\]
或 

$2\zeta z +\varepsilon x+ \gamma=0$

该方程给出直径 $I G$ 的位置.


【图,待补】
%%![](https://cdn.mathpix.com/cropped/2023_02_05_00a02f82302d074ee0d6g-18.jpg?height=395&width=824&top_left_y=401&top_left_x=402)

图 25

\section{$\S 102$}

由此我们可以算出直径 $I G$ 的长, 直径是曲线上 $M, N$ 的重合, 也即 $P M=P N$ 这样两 个点的连线, 由曲线方程知
\[
P M+P N=\frac{-\varepsilon x-\gamma}{\zeta}, \quad P M \cdot P N=\frac{\delta x^{2}+\beta x+\alpha}{\zeta}
\]
从而
\[
\begin{aligned}
(P M-P N)^{2} & =(P M+P N)^{2}-4 P M \cdot P N \\
& =\frac{\left(\varepsilon^{2}-4 \delta \zeta\right) x^{2}+2(\varepsilon \gamma-2 \beta \zeta) x+\left(\gamma^{2}-4 \alpha \zeta\right)}{\zeta^{2}}=0
\end{aligned}
\]
或
\[
x^{2}-\frac{2(2 \beta \zeta-\varepsilon \gamma)}{\varepsilon^{2}-4 \delta \zeta} x+\frac{\gamma^{2}-4 \alpha \zeta}{\varepsilon^{2}-4 \delta \zeta}=0
\]
由 $A K, A H$ 为该方程的根,我们有
\[
A K+A H=\frac{4 \beta \zeta-2 \varepsilon \gamma}{\varepsilon^{2}-4 \delta \zeta}, \quad A K \cdot A H=\frac{\gamma^{2}-4 \alpha \zeta}{\varepsilon^{2}-4 \delta \zeta}
\]
从而
\[
(A H-A K)^{2}=K H^{2}=\frac{4(2 \beta \zeta-\varepsilon \gamma)^{2}-4\left(\varepsilon^{2}-4 \delta \zeta\right)\left(\gamma^{2}-4 \alpha \zeta\right)}{\left(\varepsilon^{2}-4 \delta \zeta\right)^{2}}
\]
如果取直角坐标,则
\[
I G^{2}=\frac{\varepsilon^{2}-4 \zeta^{2}}{4 \zeta^{2}} K H^{2}
\]
\section{$\S 103$}

前面我们考虑了直角坐标情形, 现在我们来求斜角坐标方程. 从曲线上任一点 $M$ 向 轴画纵标线 $M p$, 与轴的交角为 $\angle M p H$, 记该角的正弦为 $\mu$, 余弦为 $\nu$. 令新横标 $A p=t$, 新纵标 $p M=u$, 我们有
\[
\frac{y}{u}=\mu, \quad \frac{P p}{u}=\nu
\]
从而
\[
y=\mu u, \quad x=t+\imath u
\]
将这两个值代入 $x, y$ 间方程
\[
0=\alpha+\beta x+\gamma y+\delta x^{2}+\varepsilon x y+\zeta y^{2}
\]
得
\[
\begin{aligned}
0= & \alpha+\beta t+\nu \beta u+\delta t^{2}+2 \nu \delta t u+\nu^{2} \delta u^{2}+ \\
& \mu \gamma u+\mu \varepsilon t u+\mu \nu \varepsilon u^{2}+\mu^{2} \zeta u^{2}
\end{aligned}
\]
或
\[
u^{2}+\frac{[(\mu \varepsilon+2 \nu \delta) t+\mu \gamma+\nu \beta] u+\delta t^{2}+\beta t+\alpha}{\mu^{2} \zeta+\mu \nu \varepsilon+\nu^{2} \delta}=0
\]
\section{$\S 104$}

这里每个纵标都有 $p M$ 和 $p n$ 两个值, 因此, 我们可照做过的那样, 求弦 $M n$ 的直径 $i l g$, 点 $l$ 等分 $M n$, 为直径上的点. 记 $p l=v$, 则
\[
v=\frac{p M+p n}{2}=\frac{-(\mu \varepsilon+2 \nu \delta) t-\mu \gamma-\nu \beta}{2\left(\mu^{2} \zeta+\mu \varepsilon+\nu^{2} \delta\right)}
\]
从 $l$ 向轴 $A H$ 引垂线 $l q$, 记 $\overline{A q}=p, \overline{q l}=q$, 我们有
\[
\mu=\frac{q}{\nu}, \quad \nu=\frac{p q}{v}=\frac{p-t}{v}
\]
从而

代入这两个值求得 $t, \nu$ 间方程, 得
\[
v=\frac{q}{\mu}, \quad t=p-v 0=p-\frac{\nu q}{\mu}
\]
\[
\frac{q}{\mu}=\frac{-\mu \varepsilon p-2 \nu \delta p+\nu \varepsilon q+\frac{2 \nu^{2} \delta q}{\mu}-\mu \gamma-\nu \beta}{2 \mu^{2} \zeta+2 \mu \varepsilon+2 \nu^{2} \delta}
\]
或
\[
\left(2 \mu^{2} \zeta+\mu \nu \varepsilon\right) q+\left(\mu^{2} \varepsilon+2 \mu \nu \delta\right) p+\mu^{2} \gamma+\mu \nu \beta=0
\]
或
\[
(2 \mu \zeta+\nu \varepsilon) q+(\mu \varepsilon+2 \nu \delta) p+\gamma \mu+\nu \beta=0
\]
直径 ig 的位置由该方程决定.

(1) 这里字母上方的横表示对应线段的长. $p, q$ 既表示点, 也表示线段长. 

\section{$\S 105$}

延长方程 $2 \zeta z+\varepsilon x+\gamma=0$ 决定的直径 $I G$ 交轴于点 $O$, 我们有 $A O=\frac{-\gamma}{\varepsilon}$. 从而
\[
P O=\frac{-\gamma}{\varepsilon}-x
\]
$\angle L O P$ 的正切等于
\[
\frac{z}{P O}=\frac{-\varepsilon z}{\varepsilon x+\gamma}+\frac{\varepsilon}{2 \zeta}
\]
直径 $I G$ 等分弦 $M N$ 处 $\angle M L G$ 的正切等于 $\frac{2 \zeta}{\varepsilon}$. 延长另一个直径 $i g$, 交轴于 $o$, 则
\[
A o=\frac{-\mu \gamma-\psi \beta}{\mu \varepsilon+2 \nu \delta}
\]
$\angle A o l$ 的正切等于
\[
\frac{\mu \varepsilon+2 \nu \delta}{2 \mu \zeta+\nu \varepsilon}
\]
由于 $\angle A O L$ 的正切等于 $\frac{\varepsilon}{2 \zeta}$, 两个直径交于某点 $C, \angle O C o=\angle A o l-\angle A O L$, 其正切等于
\[
\frac{4 \varkappa \zeta-\nu \varepsilon^{2}}{4 \mu \zeta^{2}+2 \nu \delta+2 \iota \zeta+\mu \varepsilon^{2}}
\]
第二个直径等分弦处的 $\angle M l o=180^{\circ}-\angle l p o-\angle A o l$, 其正切等于
\[
\frac{2 \mu^{2} \zeta+2 \mu \nu \varepsilon+2 \nu^{2} \delta}{\mu^{2} \varepsilon+2 \mu \nu \delta-2 \mu \nu \zeta-\nu^{2} \varepsilon}
\]
\section{$\S 106$}

我们来考虑两直径交点 $C$ 的位置. 从 $C$ 向轴画垂线 $C D$, 令 $A D=g, C D=h$. 先由 $C$ 在 直径 $I G$ 上,我们有
\[
2 \xi h+\varepsilon g+\gamma=0
\]
再由 $C$ 在直径 $i g$ 上我们有
\[
(2 M \zeta+\iota) h+(\mu \varepsilon+2 \nu \delta) g+\mu \gamma+\nu \beta=0
\]
用 $\mu$ 乘前一个, 从后一个减去,得
\[
\downarrow h+2 \kappa g+\nu \beta=0 \text { 或 } \varepsilon h+2 \delta g+\beta=0
\]
从而
\[
h=\frac{-\varepsilon g-\gamma}{2 \zeta}=\frac{2 \delta g-\beta}{\varepsilon}
\]
进而
\[
\left(\varepsilon^{2}-4 \delta \zeta\right) g=2 \beta \zeta-\gamma \varepsilon
\]
%%03p041-060
$g=\frac{2 \beta \zeta-\gamma \varepsilon}{\varepsilon^2-4 \delta \zeta}, \quad h=\frac{2 \gamma \delta-\beta \varepsilon}{\varepsilon^2-4 \delta \zeta}$

纵标 $p M n$ 的倾斜程度决定于 $\mu, \nu$. 我们求得的表达式中不含这两个量,这清楚地告诉我 们, 点 $C$ 的位置与纵标的倾斜程度无关.

\section{$\S 107$}

这也就是说, 所有的直径 $I G, i g$ 交于同一点 $C$. 因而这个点一旦求出来,所有的直径 都通过它, 并且通过它的弦都是直径,每条直径等分一种平行弦. 这点 $C$ 对每一个具体二 阶线唯一, 又直径都通过它, 因而通常称它为圆锥曲线的中心. 前面, 从 $x, y$ 间方程
\[
0=\alpha+\beta x+\gamma y+\delta x^{2}+\varepsilon x y+\zeta y^{2}
\]
我们得到了
\[
A D=\frac{2 \beta \zeta-\gamma \varepsilon}{\varepsilon^{2}-4 \delta \zeta}, \quad C D=\frac{2 \gamma \delta-\beta \varepsilon}{\varepsilon^{2}-4 \delta \zeta}
\]
\section{$\S 108$}

前面我们推出了
\[
A K+A H=\frac{4 \beta \zeta-2 \gamma \varepsilon}{\varepsilon^{2}-4 \delta \zeta}
\]
$I K$ 和 $G H$ 是从 $I G$ 端点向轴所引垂线. 比较 $A D$ 与 $A K+A H$ 的右端, 得
\[
A D=\frac{A K+A H}{2}
\]
即 $D$ 是线段 $K H$ 的中点,因此中心 $C$ 是 $I G$ 的中点, $I G$ 为任一直径. 这样我们得到, 所有 直径不仅相交于一点 $C$, 并且都被点 $C$ 等分.

\section{$\S 109$}

我们任取一根直径 $A I$ 作轴 (图 26), 记弦 $M N$ 与 $A I$ 所成 $\angle A P M=q$, 记该角的正弦 为 $m$, 余弦为 $n$. 取横标 $A P=x$, 纵标 $P M=y$. 由 $y$ 的两个值大小相等符号相反和为零, 知 此时二阶线通用方程的形状为
\[
y^{2}=\alpha+\beta x+\gamma x^{2}
\]
令 $y=0$,则该方程给出轴与曲线的交点 $G, I$. 因而方程
\[
x^{2}+\frac{\beta}{\gamma} x+\frac{\alpha}{\gamma}=0
\]
的根为 $x=A G, x=A I$, 从而
\[
A G+A I=\frac{-\beta}{\gamma}, \quad A G \cdot A I=\frac{\alpha}{\gamma}
\]
由中心 $C$ 为直径 $G I$ 的中点, 我们得到圆雉曲线的中心 $C$ 为 
\[
A C=\frac{A G+A I}{2}=\frac{-\beta}{2 \gamma}
\]

【图,待补】
%%![](https://cdn.mathpix.com/cropped/2023_02_05_68d01d12d0cf0d29b13dg-02.jpg?height=474&width=510&top_left_y=385&top_left_x=568)

图 26

\section{$\S 110$}

知道了圆雉曲线中心 $C$ 在轴 $A I$ 上的位置, 最好取它作横标原点. 在这样的选择下, 令 $C P=t$, 保持 $P M=y$, 则由
\[
x=A C-C P=\frac{-\beta}{2 \gamma}-t
\]
得 $t, y$ 间方程
\[
y^{2}=\alpha-\frac{\beta^{2}}{2 \gamma}+\frac{\beta^{2}}{4 \gamma}-\beta t+\beta t+\gamma t^{2}
\]
或
\[
y^{2}=\alpha-\frac{\beta^{2}}{4 \gamma}+\gamma t^{2}
\]
记常数部分为 $\alpha, t^{2}$ 的系数为 $-\beta$, 并换 $t$ 为 $x$, 我们得到直径为轴, 中心为原点时二阶线的 通用方程为 $y^{2}=\alpha-\beta x^{2}$. 令该方程中的 $y=0$, 得 $C G=C I=\sqrt{\frac{\alpha}{\beta}}$, 从而直径 $G I$ 的全长等于 $2 \sqrt{\frac{\alpha}{\beta}}$

\section{$\S 111$}

置 $x=0$, 来求过中心的弦 $E F$, 得半弦 $C E=C F=\sqrt{\alpha}$, 全弦 $E F=2 \sqrt{\alpha}$. $E F$ 过中心, 所以 它也是直径, 记它与直径 $G I$ 所成的 $\angle E C G=q$. 第二条直径 $E F$ 也等分平行于第一条直径 $G I$ 的所有弦. 在纵标 $P M$ 的对侧, 即 $I$ 侧取纵标 $a M$ 等于 $P M$. 纵标 $a M, P M$ 平行, 因而两 个点 $M$ 的连线平行于 $G I$, 且被 $E F$ 平分. 直径 $G I, E F$ 平分平行于对方的所有弦, 称这样 的两个弦为共轭弦.这两个直径, 过每一个的端点平行于另一个的直线都是切线, 即过 $E, F$ 平行于 $G I$ 的直线和过 $G, I$ 平行于 $E F$ 的直线都是切线.

\section{$\S 112$}

任意引一条斜角坐标线 $M Q$, 记 $\angle A Q M=\varphi$, 它的正弦等于 $\mu$, 余弦等于 $\nu$. 记横标 $C Q=t$, 纵标 $M Q=u$. 从 $\triangle P M Q$ 得 $\angle P M Q=\varphi-q, \sin \angle P M Q=\mu n-\mu m$ 和
\[
y: u: P Q=\mu: m:(\mu n-\nu m)
\]
从而
\[
y=\frac{\mu u}{m}, \quad P Q=\frac{(\mu n-\nu m) u}{m}
\]
进而
\[
x=t-P Q=t-\frac{(\mu m-\nu m) u}{m}
\]
把这两个值代入前面导出的方程
\[
y^{2}=\alpha-\beta x^{2} \text { 或 } y^{2}+\beta x^{2}-\alpha=0
\]
得
\[
\left[\mu^{2}+\beta(\mu m-\nu m)^{2}\right] u^{2}-2 \beta(\mu m-\nu m) m t u+\beta m^{2} t^{2}-\alpha m^{2}=0
\]
从该方程我们得到 : 对纵标 $Q M$ 和 $-Q n$ 的值, 我们有
\[
Q M-Q n=\frac{2 \beta m(\mu n-\nu m) t}{\mu^{2}+\beta(\mu m-\nu m)^{2}}
\]
设点 $p$ 等分弦 $M n$, 则直线 $C p g$ 是新的直径, 它等分平行于 $M n$ 的一切弦, 因而
\[
Q p=\frac{\beta m(\mu m-\nu m) t}{\mu^{2}+\beta(\mu m-\nu m)^{2}}
\]
\section{$\S 113$}

由此我们得到 $\angle G C g$ 的正切
\[
\frac{\mu \cdot Q p}{C Q+v \cdot Q p}=\frac{\beta m(\mu n-\nu m)}{\mu+n \beta(\mu n-\nu m)}
\]
$\angle M p g$ 的正切
\[
\frac{\mu \cdot C Q}{p Q+\nu \cdot C Q}=\frac{\mu^{2}+\beta(\mu m-\nu m)^{2}}{\mu \nu+\beta(\mu n-\nu m)(\nu n+\mu m)}
\]
$\angle M p g$ 是直径 $g i$ 等分弦 $M n$ 处的角, 进一步我们有
\[
\begin{aligned}
C p^{2} & =C Q^{2}+Q p^{2}+2 \nu \cdot C Q \cdot Q p \\
& =\frac{\mu^{4}+2 \beta \mu^{3} n(\mu n-\nu m)+\beta^{2} \mu^{2}(\mu n-\nu m)^{2}}{\left[\mu^{2}+\beta(\mu m-\nu m)^{2}\right]^{2}} t^{2}
\end{aligned}
\]
从而
\[
C p=\frac{\mu t \sqrt{\mu^{2}+2 \beta \mu n(\mu m-\nu m)+\beta^{2}(\mu n-\nu m)^{2}}}{\mu^{2}+\beta(\mu n-\nu m)^{2}}
\]
置 $C p=r, p M=s$, 则
\[
\begin{gathered}
t=\frac{\left[\mu^{2}+\beta(\mu n-\nu m)^{2}\right] r}{\mu \sqrt{\mu^{2}+2 \beta \mu n(\mu n-\nu m)+\beta^{2}(\mu n-\nu m)^{2}}} \\
u=s+Q p=s+\frac{\beta m(\mu n-\nu m) r}{\mu \sqrt{\mu^{2}+2 \beta \mu n(\mu n-\nu m)+\beta^{2}(\mu n-\nu m)^{2}}}
\end{gathered}
\]
把这两值代入 $y$ 和 $x$ 的表达式, 得
\[
\begin{gathered}
y=\frac{\mu s}{m}+\frac{\beta(\mu m-\nu m) r}{\sqrt{(\cdots)}} \\
x=-\frac{(\mu m-\nu m)_{s}}{m}+\frac{\mu r}{\sqrt{(\cdots)}}
\end{gathered}
\]
代入方程 $y^{2}+\beta x^{2}-\alpha=0$, 得
\[
\begin{gathered}
\frac{\left[\mu^{2}+\beta(\mu m-\nu m)^{2}\right] s^{2}}{m^{2}}+\frac{\beta\left[\mu^{2}+\beta(\mu m-\nu m)^{2}\right] r^{2}}{\mu^{2}+2 \beta \mu n(\mu n-\nu m)+\beta^{2}(\mu n-\nu m)^{2}}-\alpha=0 \\
\oint \mathbf{1} \mathbf{4}
\end{gathered}
\]
\section{$\S 114$}

记半直径 $C G$ 为 $f$, 其共轭半直径 $C E=C F$ 为 $g$, 我们有
\[
f=\sqrt{\frac{\alpha}{\beta}}, \quad g=\sqrt{\alpha}
\]
或
\[
\alpha=g^{2}, \quad \beta=\frac{g^{2}}{f^{2}}
\]
从而
\[
y^{2}+\frac{g^{2} x^{2}}{f^{2}}=g^{2}
\]
记 $\angle G C g=p$, 则
\[
\tan p=\frac{\beta m(\mu m-\nu m)}{\mu+n \beta(\mu n-\nu m)}
\]
$\angle G C E=q$, 如果令 $\angle E C e=\pi$, 则 $\angle A Q M=\varphi=q+\pi$, 于是
\[
\mu=\sin (q+\pi), \quad \nu=\cos (q+\pi), \quad m=\sin q, \quad n=\cos q
\]
从而
\[
\begin{gathered}
\tan p=\frac{\beta \sin q \sin \pi}{\sin (q+\pi)+\beta \cos q \sin \pi}=\frac{\beta \tan q \tan \pi}{\tan q+\tan \pi+\beta \tan \pi} \\
\sin p=\frac{\beta \sin q \sin \pi}{\sqrt{\mu^{2}+2 \beta \mu n(\mu n-\nu m)+\beta^{2}(\mu n-\nu m)^{2}}}
\end{gathered}
\]
且
\[
\mu^{2}+\beta(\mu m-\nu m)^{2}=\sin ^{2}(q+\pi)+\beta \sin ^{2} \pi
\]
利用这些值我们得到 $r, s$ 间方程 

$\frac{\left[\sin ^{2}(q+\pi)+\beta \sin ^{2} \pi\right] s^{2}}{\sin ^{2} q}+\frac{\beta\left[\sin ^{2}(q+\pi)+\beta \sin ^{2} \pi\right] r^{2}}{\beta^{2} \sin ^{2} q \sin ^{2} \pi} \cdot \sin ^{2} p-\alpha=0$

但
\[
\beta=\frac{\tan p \sin (q+\pi)}{(\sin q-\cos q \tan p) \sin \pi}=\frac{\tan p(\tan q+\tan \pi)}{\tan \pi(\tan q-\tan p)}=\frac{g^{2}}{f^{2}}=\frac{\cot \pi \tan q+1}{\cot p \tan q-1}
\]
从而
\[
\tan q=\frac{f^{2}+g^{2}}{g^{2} \cot p-f^{2} \cot \pi}
\]
由此可以得到多个推论
\[
\frac{g^{2}}{f^{2}}=\frac{\sin p \sin (q+\pi)}{\sin \pi \sin (q-p)}
\]
是其中之一.

\section{$\S 115$}

令半直径 $C g=a$, 令其共轭半直径 $C e=b$, 那么由前面求出的方程, 得
\[
\begin{aligned}
a & =\frac{\sin q \sin \pi \sqrt{\alpha \beta}}{\sin p \sqrt{\sin ^{2}(q+\pi)+\beta \sin ^{2} \pi}} \\
& =\frac{g^{2} \sin q \sin \pi}{\sin p \sqrt{f^{2} \sin ^{2}(q+\pi)+g^{2} \sin ^{2} \pi}} \\
b & =\frac{f g \sin q}{\sqrt{f^{2} \sin ^{2}(q+\pi)+g^{2} \sin ^{2} \pi}}
\end{aligned}
\]
从而
\[
a: b=g \sin \pi: f \sin p
\]
再者,由
\[
\begin{aligned}
\sin ^{2}(q+\pi)+\frac{g^{2}}{f^{2}} \sin ^{2} \pi & =\frac{\sin (q+\pi)}{\sin (q-p)}[\sin (q-p) \sin (q+\pi)+\sin p \sin \pi] \\
& =\frac{\sin q \sin (q+\pi) \sin (q+\pi-p)}{\sin (q-p)}
\end{aligned}
\]
得
\[
a=\frac{g^{2} \sin \pi}{f \sin p} \sqrt{\frac{\sin q \sin (q-p)}{\sin (q+\pi) \sin (q+\pi-p)}}
\]
由
\[
\frac{g^{2}}{f^{2}}=\frac{\sin p \sin (q+\pi)}{\sin \pi \sin (q-p)}
\]
得
\[
\begin{aligned}
& a=f \sqrt{\frac{\sin q \sin (q+\pi)}{\sin (q-p) \sin (q+\pi-p)}} \\
& b=g \sqrt{\frac{\sin q \sin (q-p)}{\sin (q+\pi) \sin (q+\pi-p)}}
\end{aligned}
\]
从而
\[
a: b=f \sin (q+\pi) \cdot g \sin (q-p), \quad a b=\frac{f g \sin q}{\sin (q+\pi-p)}
\]
\section{$\S 116$}

如果有两对共轭直径 $G I, E F$ 和 $g i, e f$, 则
\[
C g: C e=C E \sin \angle E C e: C G \sin \angle G C g
\]
从而
\[
\sin \angle G C g: \sin \angle E C e=C E \cdot C e: C G \cdot C g
\]
画出弦 $E e, G g$, 则 $\triangle C G g$ 和 $\triangle C E e$ 相等, 我们有
\[
C g: C e=C G \sin \angle G C e: C E \sin \angle g C E
\]
或
\[
C e \cdot C G \sin \angle G C e=C E \cdot C g \sin \angle g C E
\]
从而,画出弦 $G e, g E$, 则 $\triangle G C e$ 和 $\triangle g C E$ 面积相等, 它们的对顶 $\triangle I C f$ 和 $\triangle i C F$ 的面积 也相等.上节末的方程
\[
a b \sin (q+\pi-p)=f g \sin q
\]
给出
\[
\mathrm{Cg} \cdot \mathrm{Ce} \sin \angle g C e=C G \cdot C E \sin \angle G C E
\]
即弦 $E G, e g$ 与中心 $C$ 所成 $\triangle G C E$ 和 $\triangle g C e$ 面积相等. 它们的对顶 $\triangle I C F$ 和 $\triangle i C f$ 的面 积自然也相等, 由此得, 以共轭两直径为对角线的这些平行四边形相等.

\section{$\S 117$}

这样,我们有三对面积相等的三角形:

I. $\triangle F C f, \triangle I C i ;$

II. $\triangle f C I, \triangle F C i$;

III. $\triangle F C I, \triangle f C i$.

由此得四边形 $E f C I, i I C f$ 相等. 从这两个四边形中去掉同一个 $\triangle f C I$, 得 $\triangle F I f$ 和 $\triangle I f i$ 相等. 由于这两个三角形共底 $f I$, 我们得到弦 $F i, f I$ 平行, 将相等的 $\triangle F I i$ 和 $\triangle$ ifF 加到相 等的 $\triangle F C I$ 和 $\triangle f C i$ 上去,得四边形 FCIi,iCfF 相等.

\section{$\S 118$}

利用这一结果, 我们可以求出任何二阶曲线上任何一点 $M$ 处的切线 $M T$ (图 27). 取 直径 $G I$ 作轴, $E C$ 是与它共轭的半径. 从 $M$ 引平行于直线 $C E$ 的弦 $M N$ 交轴于 $P$, 则 $P N=$ $P M$. 连半直径 $C M$, 我们来求 $C M$ 的共轭半直径 $C K$, 过点 $M$ 平行于这 $C K$ 的直线就是我 们所要的切线 $M T$. 设 $\angle G C E=q, \angle G C M=p, \angle E C K=\pi$. 前面我们推出了
\[
\begin{gathered}
\frac{E C^{2}}{G C^{2}}=\frac{\sin p \sin (q+\pi)}{\sin \pi \sin (q-p)} \\
M C=C G \sqrt{\frac{\sin q \sin (q+\pi)}{\sin (q-p) \sin (q+\pi-p)}}
\end{gathered}
\]
从 $\triangle C M P$ 得
\[
\begin{gathered}
M C^{2}=C P^{2}+M P^{2}+2 P M \cdot C P \cos q \\
M P: M C=\sin p: \sin q \\
M P: C P=\sin p: \sin (q-p)
\end{gathered}
\]

【图,待补】
%%![](https://cdn.mathpix.com/cropped/2023_02_05_68d01d12d0cf0d29b13dg-07.jpg?height=460&width=273&top_left_y=282&top_left_x=1221)

图 27

从角已知的 $\triangle C M T$ 得
\[
C M: C T: M T=\sin (q+\pi): \sin (q+\pi-p): \sin p
\]
消去角,得
\[
M C=C G \sqrt{\frac{M C \cdot C M}{C P \cdot C T}}
\]
或 $C G^{2}=C P \cdot C T$. 从而 $C P: C G=C G: C T$, 由此即可确定切线的位置. 事实上, 对该比式 应用分比定理 1 得 $C P: P G=C G: T G$, 应用合比定理, 并利用 $C G=C I$, 得 $C P :$ $I P=C G: T I$

\section{$\S 119$}

由
\[
\frac{C E^{2}}{C G^{2}}=\frac{\sin p \sin (q+\pi)}{\sin \pi \sin (q-p)}, \quad \frac{C K^{2}}{C M^{2}}=\frac{\sin p \sin (q-p)}{\sin \pi \sin (q+\pi)}
\]
以及
\[
\begin{aligned}
& \frac{C M^{2}}{C G^{2}}=\frac{\sin q \sin (q+\pi)}{\sin (q-p) \sin (q+\pi-p)} \\
& \frac{C K^{2}}{C E^{2}}=\frac{\sin q \sin (q-p)}{\sin (q+\pi) \sin (q+\pi-p)}
\end{aligned}
\]
得
\[
\frac{C E^{2}+C G^{2}}{C G^{2}}=\frac{\sin p \sin (q+\pi)+\sin \pi \sin (q-p)}{\sin \pi \sin (q-p)}
\]
和
\[
\frac{C K^{2}+C M^{2}}{C M^{2}}=\frac{\sin p \sin (q-p)+\sin \pi \sin (q+\pi)}{\sin \pi \sin (q+\pi)}
\]
利用

(1) 这里我们用了现在的术语 一译者. 

$\sin A \sin B=\frac{1}{2} \cos (A-B)-\frac{1}{2} \cos (A+B)$

和
\[
\frac{1}{2} \cos A-\frac{1}{2} \cos B=\sin \frac{A+B}{2} \sin \frac{B-A}{2}
\]
得
\[
\begin{aligned}
& \sin p \sin (q+\pi)+\sin \pi \sin (q-p) \\
= & \frac{1}{2} \cos (q+\pi-p)-\frac{1}{2} \cos (q+\pi+p)+ \\
& \frac{1}{2} \cos (q-\pi-p)-\frac{1}{2} \cos (q+\pi-p) \\
= & \frac{1}{2} \cos (q-\pi-p)-\frac{1}{2} \cos (q+\pi+p)=\sin q \sin (p+\pi)
\end{aligned}
\]
和
\[
\begin{aligned}
& \sin p \sin (q-p)+\sin \pi \sin (q+\pi) \\
= & \frac{1}{2} \cos (q-2 p)-\frac{1}{2} \cos q+\frac{1}{2} \cos q-\frac{1}{2} \cos (q+2 \pi) \\
= & \frac{1}{2} \cos (q-2 p)-\frac{1}{2} \cos (q+2 \pi)=\sin (q+\pi-p) \sin (p+\pi)
\end{aligned}
\]
代入我们得到的表达式,得
\[
\frac{C E^{2}+C G^{2}}{C G^{2}}=\frac{\sin q \sin (p+\pi)}{\sin \pi \sin (q-p)}
\]
和
\[
\frac{C K^{2}+C M^{2}}{C M^{2}}=\frac{\sin (q+\pi-p) \sin (p+\pi)}{\sin \pi \sin (q+\pi)}
\]
从而
\[
\frac{C E^{2}+C G^{2}}{C K^{2}+C M^{2}}=\frac{C G^{2}}{C M^{2}} \frac{\sin q \sin (q+\pi)}{\sin (q-p) \sin (q+\pi-p)}=\frac{C G^{2}}{C M^{2}} \frac{C M^{2}}{C G^{2}}
\]
我们得到
\[
C E^{2}+C G^{2}=C K^{2}+C M^{2}
\]
即二阶曲线的共轭半直径的平方和为常数.

\section{$\S 120$}

这样,有了一对共轭半直径 $C G, C E$, 我们就可以求出任一半直径 $C M$ 的共轭半直径 $C K$, 且
\[
C K=\sqrt{C E^{2}+C G^{2}-C M^{2}}
\]
从而根据 $\S 92$ 推出的圆雉曲线的性质, 我们有
\[
T G \cdot T I: T M^{2}=C G \cdot C I: C K^{2}
\]
\[
=C G^{2}: C K^{2}=C G^{2}:\left(C E^{2}+C G^{2}-C M^{2}\right)
\]
从而
\[
T M=\frac{1}{C G} \sqrt{T G \cdot T I\left(C E^{2}+C G^{2}-C M^{2}\right)}
\]
同样地, 如果画弦 $M N$, 再画切线 $N T$, 则切线 $M T, N T$ 交轴 $T I$ 于同一点, 都为 $T$, 因为对 这两个交点我们都有 $C P: C G=C G: C T$. 如果画直线 $C N$, 则
\[
T N=\frac{1}{C G} \sqrt{T G \cdot T I\left(C E^{2}+C G^{2}-C N^{2}\right)}
\]
从而
\[
T M^{2}: T N^{2}=\left(C E^{2}+C G^{2}-C M^{2}\right):\left(C E^{2}+C G^{2}-C N^{2}\right)
\]
由点 $P$ 等分 $M N$ 得
\[
\begin{aligned}
& \sin \angle C T M: \sin \angle C T N=T N: T M \\
= & \sqrt{C E^{2}+C G^{2}-C N^{2}}: \sqrt{C E^{2}+C G^{2}-C M^{2}}
\end{aligned}
\]
\section{$\S 121$}

作直径端点 $A, B$ 处切线 $A K, B L$ (图 28), 分别交任 一切线 $M T$ 于 $K, L$. 设 $E C F$ 为 $A B$ 的共轭直径,纵标 $M P$ 平行于 $E C F$, 也平行于切线 $A K, B L$. 由切线性质, 我 们有
\[
C P: C A=C A: C T
\]
再利用 $C B=C A$, 得
\[
\begin{aligned}
& C P: A P=C A: A T \\
& C P: B P=C A: B T
\end{aligned}
\]
从而
\[
C P: C A=C A: C T=A P: A T=B P: B T
\]
进而


【图,待补】
%%![](https://cdn.mathpix.com/cropped/2023_02_05_68d01d12d0cf0d29b13dg-09.jpg?height=531&width=402&top_left_y=1113&top_left_x=1072)

图 28
\[
A T: B T=A P: B P
\]
但 $A T: B T=A K: B L$, 因而 $A K: B L=A P: B P$. 我们有
\[
A T=\frac{C A \cdot A P}{C P}, \quad B T=\frac{C A \cdot B P}{C P}
\]
和
\[
P T=\frac{C A \cdot A P}{C P}+A P=\frac{A P \cdot B P}{C P}
\]
从而
\[
A T: P T=C A: B P=A K: P M
\]
类似地, 我们得到
\[
B T: P T=C A: A P=B L: P M
\]
从而
\[
A K=\frac{C A \cdot P M}{B P}, \quad B L=\frac{C A \cdot P M}{A P}
\]
和
\[
A K \cdot B L=\frac{C A^{2} \cdot P M^{2}}{A P \cdot B P}
\]
但 $A P \cdot B P: P M^{2}=A C^{2}: C E^{2}$, 我们得到一条重要性质
\[
A K \cdot B L=C E^{2}
\]
继而得到
\[
\begin{gathered}
A K=C E \sqrt{\frac{A P}{B P}}, \quad B L=C E \sqrt{\frac{B P}{A P}} \\
A P: B P=A K^{2}: C E^{2}=C E^{2}: B L^{2}=K M: M L
\end{gathered}
\]
以及
\[
A K: B L=K M: L M
\]
\section{$\S 122$}

我们得到了这样的结论, 当从曲线上任一点 $M$ 引切线交另外两条平行切线 $A K, B L$ 于点 $K, L$ 时, 则平行于切线 $A K, B L$ 的半直径 $C E$ 是 $A K, B L$ 的比例中项, 即 $C E^{2}=$ $A K \cdot B L$. 换任意点 $M$ 为 $m$, 换切线 $K M L$ 为 $k m l$, 则 $C E^{2}=A k \cdot B l$, 从而
\[
A K: A k=B l: B L
\]
由此得
\[
A k: K k=B l: L l
\]
如果切线 $K L, k l$ 交于点 $o$, 则
\[
A K: B l=A k: B L=K k: L l=k o: l o=K_{o}: L o
\]
牛顿在他的《原理》中, 利用圆雉曲线的这些基本性质,解决了不少重要问题.

\section{$\S 123$}

由于 $A K: B L=K o: L_{0}$, 如果延长切线 $L B$ 至 $I$, 使 $B I=A K$, 那么跟切线 $L K$ 上点 $K$ 是平行于 $B L$ 的切线 $A K$ 与 $L K$ 的交点一样, $I$ 是 $K L$ 对面与 $K L$ 平行的切线与 $L B$ 的交 点, 因而直线 $I K$ 过中心 $C$, 并被 $C$ 等分. 如果照刚说的这样延长任何两条切线 $B L$ 和 $M L$ 到 $I$ 和 $K$, 又如果第三条切线 $l m o$ 交它们于 $l, o$, 则 $B I: B l=K o: L o$, 应用合比得 $I B: I l=$ $K o: K L$. 从而对不管哪一点处的第三条切线 $l m o$, 我们都有 $I B \cdot K L=I l \cdot K o$, 如果画第 四条切线 $\lambda \mu \omega$, 交 $I L, K L$ 于 $\lambda$ 和 $\omega$, 则同样地我们有
\[
I B \cdot K L=I \lambda \cdot K \omega
\]
因而 $I l \cdot K o=I \lambda \cdot K \omega$ 或 $I l: I \lambda=K \omega: K o$. 连直线 $I \omega$ 和 $\lambda o$, 并以相同的比划分它们, 不管 这个比是怎样的, 过这两个分点的直线, 都以这个比分 $I K$. 特别地, 如果平分 $l \omega, \lambda o$, 则过 分点的直线也平分 $I K$, 即过中心 $C$.

\section{$\S 124$}

参见图 $29, I l: I \lambda=K \omega: K o$ 时, 直线 $n m H$ 以分直线 $l \omega, \lambda o$ 的比分直线 $K I$. 这可用几 何方法证明. 设直线 $m n$ 分 $l \omega, \lambda o$ 的比都为 $m: n$, 即 $\lambda m: m o=l n: n \omega=m: n$. 延长 $m n$ 交切 线 $I L, K L$ 于 $Q, R$, 则
\[
\sin Q: \sin R=\frac{l n}{Q l}: \frac{n \omega}{R \omega}=\frac{\lambda m}{Q \lambda}: \frac{m o}{R o}=\frac{m}{Q l}: \frac{n}{R \omega}
\]
从而 $Q l: R \omega=Q \lambda: R o$, 应用分比, 得
\[
i \lambda: \omega_{\omega}=Q \lambda: R o=Q l: R \omega
\]
由 $l \lambda: \alpha \omega=I \lambda: K o$, 得
\[
Q I: R K=l \lambda: \omega \omega, \quad \sin Q: \sin R=\frac{m}{l \lambda}: \frac{n}{o \omega}
\]
但我们有
\[
\sin Q: \sin K=\frac{H I}{Q I}: \frac{H K}{K R}=\frac{H I}{l \lambda}: \frac{H K}{o \omega}
\]
从而


【图,待补】
%%![](https://cdn.mathpix.com/cropped/2023_02_05_68d01d12d0cf0d29b13dg-11.jpg?height=325&width=737&top_left_y=1178&top_left_x=472)

图 29

\section{$\S 125$}

给定夹角 $\angle G C E=q$ 的共轭半直径 $C G, C E$, 我们就可以求出夹角 $\angle M C K$ 为直角的 另外两个共轭半直径 $C M, C K$ (图 27). 设 $\angle G C M=p$, 如果令 $\angle E C K=\pi$, 则 $q+\pi-p=$ $90^{\circ}$, 因而
\[
\sin \pi=\cos (q-p), \quad \sin (q+\pi)=\cos p
\]
由此根据 §119 得
\[
\begin{aligned}
\frac{C E^{2}}{C G^{2}} & =\frac{\sin p \cos p}{\sin (q-p) \cos (q-p)}=\frac{\sin 2 p}{\sin 2(q-p)} \\
& =\frac{\sin 2 p}{\sin 2 q \cos 2 p-\cos 2 q \sin 2 p}
\end{aligned}
\]
从而 

$\frac{C G^{2}}{C E^{2}}=\sin 2 q \cot 2 p-\cos 2 q$

进而
\[
\cot 2 \angle G C M=\cot 2 q+\frac{C G^{2}}{C E^{2} \sin 2 q}
\]
该方程恒可解,由
\[
\frac{C M^{2}}{C G^{2}}=\frac{\sin q \cos p}{\sin (q-p)}, \quad \frac{C G^{2}}{C M^{2}}=1-\frac{\tan p}{\tan q}
\]
得
\[
\tan p=\tan q-\frac{C G^{2}}{C M^{2}} \tan q
\]
由
\[
C M^{2}+C K^{2}=C G^{2}+C E^{2}, \quad C K \cdot C M=C G \cdot C E \sin q
\]
得
\[
\begin{aligned}
& C M+C K=\sqrt{C G^{2}+2 C G \cdot C E \sin q+C E^{2}} \\
& C M-C K=\sqrt{C G^{2}-2 C G \cdot C E \sin q+C E^{2}}
\end{aligned}
\]
由此即可确定互相垂直的共轭直径.

\section{$\S 126$}

设 $C A, C E$ 是圆雉曲线的相垂直的共轭半直径 (图 30). 这种 相交于中心 $C$ 成直角的直径叫主直径. 设横标 $C P=x$, 纵标 $P M=$ $y$, 如我们看到了的, 则 $y^{2}=\alpha-\beta x^{2}$, 记主半直径 $A C=a, C E=b$, 则 $\alpha=b^{2}, \beta=\frac{b^{2}}{a^{2}}$, 从而
\[
y^{2}=b^{2}-\frac{b^{2} x^{2}}{a^{2}}
\]

【图,待补】
%%![](https://cdn.mathpix.com/cropped/2023_02_05_68d01d12d0cf0d29b13dg-12.jpg?height=323&width=350&top_left_y=1337&top_left_x=1163)

图 30

该方程不因 $x, y$ 为正或为负而改变,可见曲线被直径 $A C, C E$ 分 成四个相似相等的部分, 即象限 $A C E$ 部分相似相等于象限 $A C F$ 部分, 直径 $E F$ 对侧的两部分与这两部分相似相等.

\section{$\S 127$}

如果从我们取作横标原点的中心 $C$ 引直线 $C M$, 则它等于
\[
\sqrt{x^{2}+y^{2}}=\sqrt{b^{2}-\frac{b^{2} x^{2}}{a^{2}}+x^{2}}
\]
可见, $b=a$, 即 $C E=a$ 时, $C M=\sqrt{b^{2}}=b=a$. 也即, 此时从中心 $C$ 引到曲线的直线都相等, 而至中心距离相等, 这样的曲线为圆. 可见, 共轭主直径相等的圆雉曲线是圆. 取圆的半径 $C A=a$, 令 $C P=x, P M=y$, 在直角坐标下圆的方程为 $y^{2}=a^{2}-x^{2}$.

\section{$\S 128$}

如果 $b$ 不等于 $a$, 那么直线 $C M$ 就不能用 $x$ 有理表示, 但是轴上存在这样的点 $D$, 从 $D$ 引到曲线的直线可以用 $x$ 有理表示, 我们来求这个点 $D$, 设 $C D=f$, 则 $D P=f-x$, 从而
\[
\begin{aligned}
D M^{2} & =f^{2}-2 f x+x^{2}+b^{2}-\frac{b^{2} x^{2}}{a^{2}} \\
& =b^{2}+f^{2}-2 f x+\frac{\left(a^{2}-b^{2}\right) x^{2}}{a^{2}}
\end{aligned}
\]
如果 $f^{2}=\frac{\left(a^{2}-b^{2}\right)\left(a^{2}+f^{2}\right)}{a^{2}}$, 即 $0=a^{2}-b^{2}-f^{2}$, 从而 $f=\pm \sqrt{a^{2}-b^{2}}$, 则 $D M^{2}$ 的表达式 为完全平方. 此时 $A C$ 轴上有两个点 $D$, 至中心距离 $C D$ 都为 $\sqrt{a^{2}-b^{2}}$. 此时我们有
\[
D M^{2}=a^{2}-2 x \sqrt{a^{2}-b^{2}}+\frac{\left(a^{2}-b^{2}\right) x^{2}}{a^{2}}
\]
从而
\[
D M=a-\frac{x \sqrt{a^{2}-b^{2}}}{a}=A C-\frac{C D \cdot C P}{A C}
\]
取 $C P=0$, 得 $D M=D E=a=A C$, 取横标 $C P=C D$, 即 $x=\sqrt{a^{2}-b^{2}}$, 则直线 $D M$ 变成 纵标 $D G$, 得
\[
D G=\frac{b^{2}}{a}=\frac{C E^{2}}{A C}
\]
即 $D G$ 是 $A C, C E$ 的比例第三项.

\section{$\S 129$}

主直径上具有这种特殊性质的点 $D$, 还具有另外很多重要性质. 这点 $D$ 应该受到特 别注意, 给它取了个专门的名字, 叫圆雉曲线的焦点或脐点, 并称 $D$ 所在直径 $a$ 为主横 轴, 称另一直径 $b$ 为共轭轴. 两个焦点处的直角纵标 $D G$ 都叫半参数, 过点 $D$ 的弦叫全参 数, 全参数的长是半参数 $D G$ 的两倍, 也称全参数为 latus rectum. 共轭半轴 $C E$ 是半参数 $D G$ 和横半轴 $A C$ 的比例中项. 主横轴的端点, 也即主横轴与曲线的交点叫顶点, $A$ 是一个 顶点, 顶点处切线垂直于主横轴 $A C$.

\section{$\S 130$}

置半参数 $D G=c$, 顶点到焦点的距离 $A D=d$, 则
\[
C D=a-d=\sqrt{a^{2}-b^{2}}, \quad D G=\frac{b^{2}}{a}=c
\]
从而
\[
b^{2}=a c, \quad a-d=\sqrt{a^{2}-a c}
\]
进而
\[
a c=2 a d-d^{2}, \quad a=\frac{d^{2}}{2 d-c}, \quad b=d \sqrt{\frac{c}{2 d-c}}
\]
即焦点到顶点的距离 $A D=d$ 和半参数 $D G=c$ 完全决定圆雉曲线, 令 $C P=x$, 则
\[
D M=a-\frac{(a-d) x}{a}=\frac{d^{2}}{2 d-c}-\frac{(c-d) x}{d}
\]
令 $D P=t$, 则
\[
x=C D-t=\frac{(c-d) d}{2 d-c}-t
\]
从而
\[
D M=c+\frac{(c-d) t}{d}
\]
记 $\angle A D M$ 为 $v$, 则
\[
\frac{t}{D M}=-\cos v
\]
从而
\[
\begin{gathered}
d \cdot D M=c d+(d-c) D M \cos v \\
D M=\frac{c d}{d-(d-c) \cos v}, \quad \cos v=\frac{d(D M-D G)}{(d-c) D M}
\end{gathered}
\]
