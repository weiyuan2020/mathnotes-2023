\chapter{第十六章 拆数为和}

\section{$\S 297$}

给定表达式
\[
\left(1+x^{\alpha} z\right)\left(1+x^{\beta} z\right)\left(1+x^{\gamma} z\right)\left(1+x^{\delta} z\right)\left(1+x^{\varepsilon} z\right) \cdots
\]
我们来考察它的展开式. 记展开式为
\[
1+P z+Q z^{2}+R z^{3}+S z^{4}+\cdots
\]
显然
\[
\begin{aligned}
P= & \text { 原有幂的和 } x^{\alpha}+x^{\beta}+x^{\gamma}+x^{\delta}+x^{\varepsilon}+\cdots \\
Q= & x \text { 的每两个幂之积的和 }= \\
& x \text { 的以 } \alpha, \beta, \gamma, \delta, \varepsilon, \zeta, \eta, \cdots \text { 相异两成员之和为指数的一切幂之和 } \\
R= & x \text { 的以不同三成员之和为指数的一切幂之和 } \\
S= & x \text { 的以 } \alpha, \beta, \gamma, \delta, \varepsilon \cdots \text { 中不同四成员之和为指数的一切幂之和 }
\end{aligned}
\]
类推.

\section{$\S 298$}

$P, Q, R, S, \cdots$ 各表达式里, $x$ 的每一个幂,其指数都由 $\alpha, \beta, \gamma, \delta, \cdots$ 构成,其系数都等 于这构成方式的种数. 例如, $Q$ 里的 $N x^{n}$ 表示 $\alpha, \beta, \gamma, \delta, \cdots$ 中两个之和等于 $n$ 的共有 $N$ 组. 一般地, 展开式中的 $N x^{n} z^{m}$ 表示: $\alpha, \beta, \gamma, \delta, \varepsilon, \zeta, \cdots$ 中, 每组 $m$ 个, 和等于 $n$ 的组, 共有 $N$ 个.

\section{$\S 299$}

从乘积
\[
\left(1+x^{\alpha} z\right)\left(1+x^{\beta} z\right)\left(1+x^{\gamma} z\right)\left(1+x^{\delta} z\right) \cdots
\]
的展开式可直接说出: $\alpha, \beta, \gamma, \delta, \varepsilon, \zeta, \cdots$ 中 $m$ 个一组, 和等于 $n$ 的组, 共有多少个. 只需找 到含 $x^{n} z^{m}$ 的项,它的系数就是我们所要的组数. 

\section{$\S 300$}

为进一步说明,我们考虑无穷乘积
\[
(1+x z)\left(1+x^{2} z\right)\left(1+x^{3} z\right)\left(1+x^{4} z\right)\left(1+x^{5} z\right) \cdots
\]
其展开式为
\[
\begin{aligned}
& 1+z\left(x+x^{2}+x^{3}+x^{4}+x^{5}+x^{6}+x^{7}+x^{8}+x^{9}+\cdots\right)+ \\
& z^{2}\left(x^{3}+x^{4}+2 x^{5}+2 x^{6}+3 x^{7}+3 x^{8}+4 x^{9}+4 x^{10}+5 x^{11}+\cdots\right)+ \\
& z^{3}\left(x^{6}+x^{7}+2 x^{8}+3 x^{9}+4 x^{10}+5 x^{11}+7 x^{12}+8 x^{13}+10 x^{14}+\cdots\right)+ \\
& z^{4}\left(x^{10}+x^{11}+2 x^{12}+3 x^{13}+5 x^{14}+6 x^{15}+9 x^{16}+11 x^{17}+15 x^{18}+\cdots\right)+ \\
& z^{5}\left(x^{15}+x^{16}+2 x^{17}+3 x^{18}+5 x^{19}+7 x^{20}+10 x^{21}+13 x^{22}+18 x^{23}+\cdots\right)+ \\
& z^{6}\left(x^{21}+x^{22}+2 x^{23}+3 x^{24}+5 x^{25}+7 x^{26}+11 x^{27}+14 x^{28}+20 x^{29}+\cdots\right)+ \\
& z^{7}\left(x^{28}+x^{29}+2 x^{30}+3 x^{31}+5 x^{32}+7 x^{33}+11 x^{34}+15 x^{35}+21 x^{36}+\cdots\right)+ \\
& z^{8}\left(x^{36}+x^{37}+2 x^{38}+3 x^{39}+5 x^{40}+7 x^{41}+11 x^{42}+15 x^{43}+22 x^{44}+\cdots\right)+
\end{aligned}
\]
从这个级数我们立即就可以说出 $1,2,3,4,5,6,7,8, \cdots$ 中 $m$ 个一组, 和等于 $n$ 的组有多少 个. 例如 $m$ 为 $7, n$ 为 35 时, 我们找到含 $z^{7}$ 和 $x^{35}$ 的项, 系数 15 就是 $1,2,3,4,5,6,7,8, \cdots$ 中 7 个一组,和等于 35 的组的组数.

\section{$\S 301$}

令上节中 $z=1$, 乘积变为
\[
(1+x)\left(1+x^{2}\right)\left(1+x^{3}\right)\left(1+x^{4}\right)\left(1+x^{5}\right)\left(1+x^{6}\right) \cdots
\]
展开,整理,得
\[
1+x+x^{2}+2 x^{3}+2 x^{4}+3 x^{5}+4 x^{6}+5 x^{7}+6 x^{8}+\cdots
\]
则项 $N x^{n}$ 的指数 $n$ 和系数 $N$ 告诉我们 $1,2,3,4,5,6,7,8, \cdots$ 中和等于 $n$ 的组有 $N$ 个. 例如 $6 x^{8}$ 告诉我们 $1,2,3,4,5,6,7, \cdots$ 中和等于 8 的组有 6 个. 我们指出, 它们是
\[
\begin{gathered}
8=8 \\
8=7+1 \\
8=6+2 \\
8=5+3 \\
8=5+2+1 \\
8=4+3+1
\end{gathered}
\]
请注意,这里 8 本身也是一组,即一组可以只是一个数. 

\section{$\S 302$}

将一个给定的数,拆成不同数的和,这拆法有多少种,我们已经知道了.如果把前面的乘积改作分母, 就可以去掉这句话中“不同” 两字. 考虑表达式
\[
\frac{1}{\left(1-x^{\alpha} z\right)\left(1-x^{\beta} z\right)\left(1-x^{\gamma} z\right)\left(1-x^{\delta} z\right)\left(1-x^{\varepsilon} z\right) \cdots}
\]
进行除法,记所得无穷级数为
\[
1+P z+Q z^{2}+R z^{3}+S z^{4}+\cdots
\]
则
\[
\begin{gathered}
P=x \text { 的以序列 } \alpha, \beta, \gamma, \delta, \varepsilon, \zeta, \cdots \text { 各成员为指数之幂的和 } \\
Q=x \text { 的以 } P \text { 中序列两成员 (可以相同) 的和为指数之幂的和 } \\
R=x \text { 的以 } P \text { 中序列三成员 (可以相同) 的和为指数之幂的和 } \\
S=x \text { 的以 } P \text { 中序列四成员 (可以相同) 的和为指数之幂的和 }
\end{gathered}
\]
类推.

\section{$\S 303$}

写出无穷级数,归并同类项之后, 我们就可以说出将 $n$ 拆成序列 $\alpha, \beta, \gamma, \delta, \varepsilon, \cdots$ 中 $m$ 个成员 (可相同) 之和的拆法共有多少种. 只要从级数中找出项 $N x^{n} z^{m}$, 这系数 $N$ 就是我 们所要的种数. 我们看到这一问题与前一问题解决方法类似.

\section{$\S 304$}

考虑一个重要的特殊情形:给定的表达式为
\[
\frac{1}{(1-x z)\left(1-x^{2} z\right)\left(1-x^{3} z\right)\left(1-x^{4} z\right)\left(1-x^{5} z\right) \cdots}
\]
完成除法,得
\[
\begin{aligned}
& 1+z\left(x+x^{2}+x^{3}+x^{4}+x^{5}+x^{6}+x^{7}+x^{8}+x^{9}+\cdots\right)+ \\
& z^{2}\left(x^{2}+x^{3}+2 x^{4}+2 x^{5}+3 x^{6}+3 x^{7}+4 x^{8}+4 x^{9}+5 x^{10}+\cdots\right)+ \\
& z^{3}\left(x^{3}+x^{4}+2 x^{5}+3 x^{6}+4 x^{7}+5 x^{8}+7 x^{9}+8 x^{10}+10^{11}+\cdots\right)+ \\
& z^{4}\left(x^{4}+x^{5}+2 x^{6}+3 x^{7}+5 x^{8}+6 x^{9}+9 x^{10}+11 x^{11}+15 x^{12}+\cdots\right)+ \\
& z^{5}\left(x^{5}+x^{6}+2 x^{7}+3 x^{8}+5 x^{9}+7 x^{10}+10 x^{11}+13 x^{12}+18 x^{13}+\cdots\right)+ \\
& z^{6}\left(x^{6}+x^{7}+2 x^{8}+3 x^{9}+5 x^{10}+7 x^{11}+11 x^{12}+14 x^{13}+20 x^{14}+\cdots\right)+ \\
& z^{7}\left(x^{7}+x^{8}+2 x^{9}+3 x^{10}+5 x^{11}+7 x^{12}+11 x^{13}+15 x^{14}+21 x^{15}+\cdots\right)+ \\
& z^{8}\left(x^{8}+x^{9}+2 x^{10}+3 x^{11}+5 x^{12}+7 x^{13}+11 x^{14}+15 x^{15}+22 x^{16}+\cdots\right)+
\end{aligned}
\]
有了这个展开式, 我们就可以说出, 将 $n$ 拆成序列
\[
1,2,3,4,5,6,7, \cdots
\]
中 $m$ 个数的和, 拆法有多少种. 例如, 将 13 拆成该序列中 5 个数的和. 我们从表达式中找 到 $x^{13} z^{5}$ 所在的项, 它的系数 18 就是我们所要的种数. 也即, 将 13 拆成 5 个正整数的和, 拆法有 18 种.

\section{$\S 305$}

如果 $z=1$, 则上节分式成
\[
\frac{1}{(1-x)\left(1-x^{2}\right)\left(1-x^{3}\right)\left(1-x^{4}\right)\left(1-x^{5}\right)\left(1-x^{6}\right) \cdots}
\]
其展开式,整理后为
\[
1+x+2 x^{2}+3 x^{3}+5 x^{4}+7 x^{5}+11 x^{6}+15 x^{7}+22 x^{8}+\cdots
\]
式中每一项, 系数都是指数能够拆成整数和的种数. 和中整数可以相等, 也可以不等. 例 如项 $11 x^{6}$ 表示: 将 6 拆成整数和,拆法有 11 种. 我们指出, 它们是
\[
\begin{gathered}
6=6 \\
6=5+1 \\
6=4+2 \\
6=4+1+1 \\
6=3+3 \\
6=3+2+1 \\
6=3+1+1+1 \\
6=2+2+2 \\
6=2+2+1+1 \\
6=2+1+1+1+1 \\
6=1+1+1+1+1+1
\end{gathered}
\]
我们再指出, 6 是 $1,2,3,4,5,6, \cdots$ 的成员, 它也是等于 6 的和中的一种.

\section{$\S 306$}

上面的讨论告诉我们从展开式所能得到的结果. 下面我探讨展开式的求法. 先求拆 成不同数之和时所用的展开式. 为此,我们将
\[
Z=(1+x z)\left(1+x^{2} z\right)\left(1+x^{3} z\right)\left(1+x^{4} z\right)\left(1+x^{5} z\right) \cdots
\]
的展开式按 $z$ 的升幂排列为
\[
Z=1+P z+Q z^{2}+R z^{3}+S z^{4}+T z^{5}+\cdots
\]
我们要讨论的是: $x$ 的函数 $P, Q, R, S, T, \cdots$ 的求法, 有了它们,也就有了展开式.

\section{$\S 307$}

换 $z$ 为 $x z$, 得
\[
\left(1+x^{2} z\right)\left(1+x^{3} z\right)\left(1+x^{4} z\right)\left(1+x^{5} z\right) \cdots=\frac{Z}{1+x z}
\]
也即, 将 $z$ 换为 $x z$ 时, 乘积的值由 $Z$ 变成 $\frac{Z}{1+x z}$. 由
\[
Z=1+P z+Q z^{2}+R z^{3}+S z^{4}+\cdots
\]
得
\[
\frac{Z}{1+x z}=1+P x z+Q x^{2} z^{2}+R x^{3} z^{3}+S x^{4} z^{4}+\cdots
\]
两边乘 $1+x z$, 得
\[
\begin{aligned}
Z= & 1+P x z+Q x^{2} z^{2}+R x^{3} z^{3}+S x^{4} z^{4}+\cdots+ \\
& x z+P x^{2} z^{2}+Q x^{3} z^{3}+R x^{4} z^{4}+\cdots
\end{aligned}
\]
两 $Z$ 比较,得
\[
P=\frac{x}{1-x}, Q=\frac{P x^{2}}{1-x^{2}}, R=\frac{Q x^{3}}{1-x^{3}}, S=\frac{R x^{4}}{1-x^{4}}, \cdots
\]
由前向后依次代入,我们得到
\[
\begin{gathered}
P=\frac{x}{1-x} \\
Q=\frac{x^{3}}{(1-x)\left(1-x^{2}\right)} \\
S=\frac{x^{6}}{(1-x)\left(1-x^{2}\right)\left(1-x^{3}\right)} \\
T=\frac{x^{10}}{(1-x)\left(1-x^{2}\right)\left(1-x^{3}\right)\left(1-x^{4}\right)\left(1-x^{5}\right)} \\
\vdots
\end{gathered}
\]
\section{$\S 308$}

展开所得分数函数为级数, 揷一句, 这级数都是递推的, 从展成的级数我们就能说 出,一数拆成若干个数之和的种数. 例如, 第一个表达式
\[
P=\frac{x}{1-x}
\]
展成的是几何级数
\[
x+x^{2}+x^{3}+x^{4}+x^{5}+x^{6}+x^{7}+\cdots
\]
它告诉我们每个数拆成单个数之和的种数都为 1 . 事实上, 整数都在这里出现, 但只一 次.

\section{$\S 309$}

第二个表达式

%%12p221-240
\[
\frac{x^3}{(1-x)\left(1-x^{2}\right)}
\]
给出的级数为
\[
x^{3}+x^{4}+2 x^{5}+2 x^{6}+3 x^{7}+3 x^{8}+4 x^{9}+4 x^{10}+\cdots
\]
该级数的每一项,系数都是指数可以拆成两个不同数之和的种数. 例如, $4 x^{9}$ 告诉我们,, 9 拆成两个不同数的和, 拆法有 4 种, 用 $x^{3}$ 除我们的级数, 得到的是由分式
\[
\frac{1}{(1-x)\left(1-x^{2}\right)}
\]
产生的级数
\[
1+x+2 x^{2}+2 x^{3}+3 x^{4}+3 x^{5}+4 x^{6}+4 x^{7}+\cdots
\]
记它的通项为 $N x^{n}$, 那么从这个级数的产生我们知道, 系数 $N$ 是指数 $n$ 可以拆成数 1 与 2 之和的种数. 由前一个级数的通项为 $N x^{n+3}$, 我们得到下面的定理:

数 $n$ 拆成数 1 与 2 之和的种数, 等于数 $n+3$ 拆成两个不同数之和的种数.

\section{$\S 310$}

第三个表达式
\[
\frac{x^{6}}{(1-x)\left(1-x^{2}\right)\left(1-x^{3}\right)}
\]
展成的级数为
\[
x^{6}+x^{7}+2 x^{8}+3 x^{9}+4 x^{10}+5 x^{11}+7 x^{12}+8 x^{13}+\cdots
\]
它的每一项, 系数都是指数能够拆成三个不同数之和的种数. 将分式
\[
\frac{1}{(1-x)\left(1-x^{2}\right)\left(1-x^{3}\right)}
\]
展成级数, 得
\[
1+x+2 x^{2}+3 x^{3}+4 x^{4}+5 x^{5}+7 x^{6}+8 x^{7}+\cdots
\]
记它的通项为 $N x^{n}$, 则系数 $N$ 是指数 $n$ 拆成数 $1,2,3$ 之和的种数. 由前一个级数之通项为 $N x^{n+6}$ 我们得到下面的定理:

数 $n$ 拆成数 $1,2,3$ 之和的种数, 等于数 $n+3$ 拆成三个不同数之和的种数.

\section{$\S 311$}

第四个表达式
\[
\frac{x^{10}}{(1-x)\left(1-x^{2}\right)\left(1-x^{3}\right)\left(1-x^{4}\right)}
\]
展成的级数为
\[
x^{10}+x^{11}+2 x^{12}+3 x^{13}+5 x^{14}+6 x^{15}+9 x^{16}+\cdots
\]
它的每一项, 系数都是指数能够拆成 4 个不同数之和的种数. 表达式 
\[
\frac{1}{(1-x)\left(1-x^{2}\right)\left(1-x^{3}\right)\left(1-x^{4}\right)}
\]
展成的级数为
\[
1+x+2 x^{2}+3 x^{3}+5 x^{4}+6 x^{5}+9 x^{6}+11 x^{7}+\cdots
\]
等于前一个级数除上 $x^{10}$. 记这个级数的通项为 $N x^{n}$, 则系数 $N$ 是指数 $n$ 能够拆成数 1,2 , 3,4 之和的种数. 由前一个级数的通项为 $N x^{n+10}$, 我们得到下面的定理:

数 $n$ 拆成 $1,2,3,4$ 之和的种数, 等于数 $n+10$ 拆成 4 个不同数之和的种数.

\section{$\S 312$}

一般地,如果表达式
\[
\frac{1}{(1-x)\left(1-x^{2}\right)\left(1-x^{3}\right)\left(1-x^{4}\right)\left(1-x^{5}\right) \cdots\left(1-x^{m}\right)}
\]
展成的级数, 其通项为 $N x^{n}$, 则系数 $N$ 是指数 $n$ 能够拆成数 $1,2,3,4, \cdots, m$ 之和的种数. 另 一方面,如果表达式

\[
\frac{x^{\frac{m(m+1)}{2}}}{(1-x)\left(1-x^2\right)\left(1-x^3\right) \cdots\left(1-x^m\right)}
\]
展成的级数, 其通项为 $N x^{n+\frac{m(m+1)}{2}}$, 则系数 $N$ 是指数 $n+\frac{m(m+1)}{2}$ 拆成 $m$ 个不同数之和的 种数. 从而我们得到下面的定理:

数 $n$ 拆成数 $1,2,3,4, \cdots, m$ 之和的种数, 等于数 $n+\frac{m(m+1)}{2}$ 拆成 $m$ 个不同数之和 的种数.

\section{$\S 313$}

前面我们讨论了将一个数拆成若干份的种数, 份与份不相等. 现在我们讨论份与份 可以相等的情形. 这种拆法的种数, 从表达式
\[
Z=\frac{1}{(1-x z)\left(1-x^{2} z\right)\left(1-x^{3} z\right)\left(1-x^{4} z\right)\left(1-x^{5} z\right) \cdots}
\]
得到. 记该式展成的级数为
\[
Z=1+P z+Q z^{2}+R z^{3}+S z^{4}+T z^{5}+\cdots
\]
将表达式中的 $z$ 换成 $x z$, 得
\[
\frac{1}{\left(1-x^{2} z\right)\left(1-x^{3} z\right)\left(1-x^{4} z\right)\left(1-x^{5} z\right) \cdots}=(1-x z) Z
\]
对级数作同样的替换, 得
\[
(1-x z) Z=1+P x z+Q x^{2} z^{2}+R x^{3} z^{3}+S x^{4} z^{4}+\cdots
\]
乘原级数以 $1-x z$, 得 
\[
\begin{aligned}
& (1-x z) Z=1+P z+Q z^{2}+R z^{3}+S z^{4}+\cdots- \\
& x z-P x z^{2}-Q x z^{3}-R x z^{4}-\cdots
\end{aligned}
\]
将这两个表达式相比较,我们得到
\[
P=\frac{x}{1-x}, Q=\frac{P x}{1-x^{2}}, R=\frac{Q x}{1-x^{3}}, S=\frac{R x}{1-x^{4}}, \cdots
\]
由左向右逐个代入, 我们得到
\[
\begin{gathered}
P=\frac{x}{1-x} \\
Q=\frac{x^{2}}{(1-x)\left(1-x^{2}\right)} \\
R=\frac{x^{3}}{(1-x)\left(1-x^{2}\right)\left(1-x^{3}\right)} \\
S=\frac{x^{4}}{(1-x)\left(1-x^{2}\right)\left(1-x^{3}\right)\left(1-x^{4}\right)}
\end{gathered}
\]
\section{$\S 314$}

$P, Q, R, S, \cdots$ 的前后两种表达式, 不同的只是分子的指数, 前高于后. 因此后面表达 式的展开级数及其系数的含义都与前面完全类似. 由此我们得到与前面类似地下列定 理:

数 $n$ 拆成数 1,2 之和的种数, 等于数 $n+2$ 拆成两个数之和的种数.

数 $n$ 拆成数 $1,2,3$ 之和的种数, 等于数 $n+3$ 拆成三个数之和的种数.

数 $n$ 拆成数 $1,2,3,4$ 之和的种数, 等于数 $n+4$ 拆成四个数之和的种数.

一般地, 数 $n$ 拆成 $1,2,3, \cdots, m$ 之和的种数, 等于数 $n+m$ 拆成 $m$ 个数之和的种数.

\section{$\S 315$}

拆数 $n$ 成 $m$ 个相异或不一定相异数之和的种数,这两个问题都可用拆成数 $1,2,3$, $4, \cdots, m$ 之和的种数, 根据从前面定理推出来的下面两个定理来回答:

数 $n$ 拆成 $m$ 个相异数之和的种数, 等于数 $n-\frac{m(m+1)}{2}$ 拆成数 $1,2,3,4, \cdots, m$ 之和 的种数.

数 $n$ 拆成不一定相异的 $m$ 个数之和的种数, 等于数 $n-m$ 拆成 $1,2,3,4, \cdots, m$ 之和的 种数.

从这两个定理进而得到下面两个定理:

$n$ 拆成 $m$ 个相异数之和的种数, 等于 $n-\frac{m(m+1)}{2}$ 拆成 $m$ 个不一定相异数之和的种数.

$n$ 拆成 $m$ 个不一定相异的数之和的种数, 等于 $n+\frac{m(m+1)}{2}$ 拆成 $m$ 个相异数之和的 种数.

\section{$\S 316$}

我们可以利用构成递推级的办法, 得到数 $n$ 拆成数 $1,2,3, \cdots, m$ 之和的种数, 作法是 将分式
\[
\frac{1}{(1-x)\left(1-x^{2}\right)\left(1-x^{3}\right) \cdots\left(1-x^{m}\right)}
\]
展成递推级数到项 $N x^{n}$. 这系数 $N$ 就是数 $n$ 拆成数 $1,2,3,4, \cdots, m$ 之和的种数. 但是当 $m$ 和 $n$ 比较大时, 这个求 $N$ 的方法不好用. 这时分母给出的递推尺度, 其项数多, 求级数的 高次项很麻烦.

\section{$\S 317$}

我们先弄清几种简单情形,由此出发再去考虑更复杂的情形, 这样事情会容易一些. 记分式
\[
\frac{1}{(1-x)\left(1-x^{2}\right)\left(1-x^{3}\right) \cdots\left(1-x^{m}\right)}
\]
产生的级数的通项为 $N x^{n}$. 记分式
\[
\frac{x^{m}}{(1-x)\left(1-x^{2}\right)\left(1-x^{3}\right) \cdots\left(1-x^{m}\right)}
\]
产生的级数的通项为 $M x^{n}$. 这里的数 $M$ 是数 $n-m$ 拆成数 $1,2,3, \cdots, m$ 之和的种数. 从前 一个分式减去后一个,得
\[
\frac{1}{(1-x)\left(1-x^{2}\right)\left(1-x^{3}\right) \cdots\left(1-x^{m-1}\right)}
\]
显然, 它产生的级数, 其通项为 $(N-M) x^{n}$. 这里的 $N-M$ 是 $n$ 拆成 $1,2,3, \cdots, m-1$ 之和 的种数.

\section{$\S 318$}

由此我们得到下面的规则:

记 $n$ 拆成 $1,2,3, \cdots, m-1$ 之和的种数为 $L$;

记 $n-m$ 拆成 $1,2,3, \cdots, m$ 之和的种数为 $M$;

记 $n$ 拆成 $1,2,3, \cdots, m$ 之和的种数为 $N$.

在这样的记号之下我们有 
\[
L=N-M
\]
从而
\[
N=L+M
\]
这样, 知道了 $n$ 拆成 $1,2,3, \cdots, m-1$ 之和的种数, 和 $n-m$ 拆成 $1,2,3, \cdots, m$ 之和的种数, 进行加法就可得到 $n$ 拆成 $1,2,3, \cdots, n$ 之和的种数. 借助这条规则, 可以从比较简单情况 下的结果, 推出更复杂情况下的结果. 附表就是用这种方法计算出来的.

附表

【图,待补】
%%![](https://cdn.mathpix.com/cropped/2023_02_04_9193a04af89085c44386g-05.jpg?height=5003&width=4016&top_left_y=2075&top_left_x=654)

【图,待补】
%%![](https://cdn.mathpix.com/cropped/2023_02_04_9193a04af89085c44386g-06.jpg?height=5250&width=3964&top_left_y=942&top_left_x=538)

例如, 查 50 拆成 7 个相异数之和的种数. 从最左坚列中查到 $50-\frac{7 \cdot 8}{2}=22$, 从最上 横行中查到 VII,22 所属横行与 VII 所属坚列交点处的 522 就是答案.

再例如, 查 50 拆成 7 个不一定相异数之和的种数. 从最坚列中查到 $50-7=43$, 从最 上横行中查到 VII, 43 所属横行与 VII 所属坚列交点处的 8946 就是答案. 

\section{$\S 319$}

附表的每一竖列都是一个级数的系数. 虽然这里的级数是递推的, 但其系数却与自 然数、三角形数、四面体数等有着密切的关系. 对这种关系我们做些说明. 分式
\[
\frac{1}{(1-x)\left(1-x^{2}\right)}
\]
产生的级数为
\[
1+x+2 x^{2}+2 x^{3}+3 x^{4}+3 x^{5}+\cdots
\]
从而
\[
\frac{x}{(1-x)\left(1-x^{2}\right)}
\]
产生的级数为
\[
x+x^{2}+2 x^{3}+2 x^{4}+3 x^{5}+3 x^{6}+\cdots
\]
这两个级数相加, 得级数
\[
1+2 x+3 x^{2}+4 x^{3}+5 x^{4}+6 x^{5}+7 x^{6}+\cdots
\]
实际上它是分式
\[
\frac{1+x}{(1-x)\left(1-x^{2}\right)}=\frac{1}{(1-x)^{2}}
\]
产生的级数. 最后这个级数的系数就是自然数. 令第一个级数中的 $x=1$, 得到的就是表中 列 II 所成的级数. 取它的每项与前一项相加, 用和作新级数的项. 这新级数的项为自然 数
\[
\begin{gathered}
1+1+2+2+3+3+4+4+5+5+6+6+\cdots \\
1+2+3+4+5+6+7+8+9+10+11+12+\cdots
\end{gathered}
\]
反之, 从以自然数为项的级数, 也可以得到以表中列 II 为项的级数. 方法是: 从前者的对 应项减去后者的前一项, 用结果作后者的项.

\section{$\S 320$}

表中列 III 所成级数由分式
\[
\frac{1}{(1-x)\left(1-x^{2}\right)\left(1-x^{3}\right)}
\]
产生. 但从
\[
\frac{1}{(1-x)^{3}}=\frac{(1+x)\left(1+x+x^{2}\right)}{(1-x)\left(1-x^{2}\right)\left(1-x^{3}\right)}
\]
我们看到, 先让列 III 所成级数每项与前两项相加, 再让得到的级数每项与前一项相加, 最后得到的就是三角形数所成级数,下面列的就是这三个级数
\[
1+1+2+3+4+5+7+8+10+12+14+16+19+\cdots
\]
\[
\begin{aligned}
& 1+2+4+6+9+12+16+20+25+30+36+42+49+\cdots \\
& 1+3+6+10+15+21+28+36+45+55+66+78+91+\cdots
\end{aligned}
\]
反之,如何从三角形数所成级数得到列 III 所成级数,这也是明显的.

\section{$\S 321$}

类似地,列 IV 所成级数,由分式
\[
\frac{1}{(1-x)\left(1-x^{2}\right)\left(1-x^{3}\right)\left(1-x^{4}\right)}
\]
产生,且
\[
\frac{(1+x)\left(1+x+x^{2}\right)\left(1+x+x^{2}+x^{3}\right)}{(1-x)\left(1-x^{2}\right)\left(1-x^{3}\right)\left(1-x^{4}\right)}=\frac{1}{(1-x)^{4}}
\]
这里, 对列 IV 所成级数, 使每项与前三项相加, 得第二个级数, 使第二个级数每项与前两 项相加, 得第三个级数, 最后使第三个级数每项与前一项相加, 得到的就是四面体数所成 级数. 下面是逐次得到的结果
\[
\begin{gathered}
1+1+2+3+5+6+9+11+15+18+23+27+\cdots \\
1+2+4+7+11+16+23+31+41+53+67+83+\cdots \\
1+3+7+13+22+34+50+70++95+125+161+203+\cdots \\
1+4+10+20+35+56+84+120+165+220+286+264+\cdots
\end{gathered}
\]
类似地, 从列 $\mathrm{V}$ 所成级数推出二阶四面体数所成级数, 从列 VI所成级数推出三阶四面体 数所成级数.

\section{$\S 322$}

反之, 从自然数,三角形数等也可算出表中各列, 下面是计算列 II, III, IV, V 的中间 和最后结果
\[
\begin{gathered}
1+2+3+4+5+6+7+8+9+10+\cdots \\
1+1+2+2+3+3+4+4+5+5+\cdots \text { 列 } I I \\
1+3+6+10+15+21+28+36+45+55+\cdots \\
1+2+4+6+9+12+16+20+25+30+\cdots \\
1+1+2+3+4+5+7+8+10+12+\cdots \text { 列 III } \\
1+4+10+20+35+56+84+120+165+220+\cdots \\
1+3+7+13+22+34+50+70+95+125+\cdots \\
1+2+4+7+11+16+23+31+41+53+\cdots \\
1+1+2+3+5+6+9+11+15+18+\cdots \text { 列 } \mathrm{IV} \\
1+5+15+35+70+126+210+330+495+715+\cdots \\
1+4+11+24+46+80+130+200+295+420+\cdots
\end{gathered}
\]
\[
\begin{aligned}
& 1+3+7+14+25+41+64+95+136+189+\cdots \\
& 1+2+4+7+12+18+27+38+53+71+\cdots \\
& 1+1+2+3+5+7+10+13+18+23+\cdots \text { 列 } \mathrm{V}
\end{aligned}
\]
这是四组级数,其第一行的项,依次是自然数、三角形数、四面体数、二阶四面体数. 第二行的项都等于第一行的对应项减去第二行的前一项. 第三行的项都等于第二行的对 应项减去第三行前两项的和. 类推下去, 从前一行的对应项减去本行前三项的和, 第四项 的和等以得到本行, 直至得到我们所求的开头几项为 $1+1+2+\cdots$ 的级数, 即表中的各 歹列.

\section{$\S 323$}

表中各列开头几项都相同, 并且越向右相同的项越多, 可见当列无穷时, 各列会完全 相同,那时的级数是由分式
\[
\frac{1}{(1-x)\left(1-x^{2}\right)\left(1-x^{3}\right)\left(1-x^{4}\right)\left(1-x^{5}\right)\left(1-x^{6}\right)\left(1-x^{7}\right) \cdots}
\]
产生的. 这个级数是递推的. 为得到递推尺度, 展开分母,得
\[
1-x-x^{2}+x^{5}+x^{7}-x^{12}-x^{15}+x^{22}+x^{26}-x^{35}-x^{40}+x^{51}+\cdots
\]
仔细观察我们发现,指数为 $\frac{3 n^{2} \pm n}{2}, n$ 为奇数的项为负, $n$ 为偶数的项为正.

\section{$\S 324$}

递推尺度为
\[
+1,+1,0,0,-1,0,-1,0,0,0,0,+1,0,0,+1,0,0, \cdots
\]
因而分式
\[
\frac{1}{(1-x)\left(1-x^{2}\right)\left(1-x^{3}\right)\left(1-x^{4}\right)\left(1-x^{5}\right)\left(1-x^{6}\right)\left(1-x^{7}\right) \cdots}
\]
产生的递推级数为
\[
\begin{aligned}
& 1+x+2 x^{2}+3 x^{3}+5 x^{4}+7 x^{5}+11 x^{6}+15 x^{7}+22 x^{8}+30 x^{9}+ \\
& 42 x^{10}+56 x^{11}+77 x^{12}+101 x^{13}+135 x^{14}+176 x^{15}+231 x^{16}+297 x^{17}+ \\
& 385 x^{18}+490 x^{19}+627 x^{20}+792 x^{21}+1002 x^{22}+1250 x^{23}+1570 x^{24}+\cdots
\end{aligned}
\]
这个级数的每一项, 系数都等于指数拆为整数之和的种数. 例如, 7 拆成整数之和的种数 是 15 , 具体拆法为
\[
\begin{gathered}
7=7 \\
7=6+1 \\
7=5+2 \\
7=5+1+1
\end{gathered}
\]
\[
\begin{gathered}
7=4+3 \\
7=4+2+1 \\
7=4+1+1+1 \\
7=3+3+1 \\
7=3+2+2 \\
7=3+2+1+1 \\
7=3+1+1+1+1 \\
7=2+2+2+1 \\
7=2+2+1+1+1 \\
7=2+1+1+1+1+1 \\
7=1+1+1+1+1+1+1
\end{gathered}
\]
\section{$\S 325$}

乘积
\[
(1+x)\left(1+x^{2}\right)\left(1+x^{3}\right)\left(1+x^{4}\right)\left(1+x^{5}\right)\left(1+x^{6}\right) \cdots
\]
展开, 得级数
\[
1+x+x^{2}+2 x^{3}+2 x^{4}+3 x^{5}+4 x^{6}+5 x^{7}+6 x^{8}+8 x^{9}+10 x^{10}+\cdots
\]
这里系数等于指数拆成相异数之和的种数. 例如, 9 拆成相异数之和的种数是 8 , 这 8 种拆 法是
\[
\begin{gathered}
9=9 \\
9=8+1 \\
9=7+2 \\
9=6+3 \\
9=6+2+1 \\
9=5+4 \\
9=5+3+1 \\
9=4+3+2
\end{gathered}
\]
\section{$\S 326$}

为对这两个表达式进行比较,记
\[
\begin{aligned}
& P=(1-x)\left(1-x^{2}\right)\left(1-x^{3}\right)\left(1-x^{4}\right)\left(1-x^{5}\right)\left(1-x^{6}\right) \cdots \\
& Q=(1+x)\left(1+x^{2}\right)\left(1+x^{3}\right)\left(1+x^{4}\right)\left(1+x^{5}\right)\left(1+x^{6}\right) \cdots
\end{aligned}
\]
从而
\[
P Q=\left(1-x^{2}\right)\left(1-x^{4}\right)\left(1-x^{6}\right)\left(1-x^{8}\right)\left(1-x^{10}\right)\left(1-x^{12}\right) \cdots
\]
$P Q$ 的因式都含于 $P$ 中, 用 $P Q$ 除 $P$, 得 
\[
\frac{1}{Q}=(1-x)\left(1-x^{3}\right)\left(1-x^{5}\right)\left(1-x^{7}\right)\left(1-x^{9}\right) \cdots
\]
从而
\[
Q=\frac{1}{(1-x)\left(1-x^{3}\right)\left(1-x^{5}\right)\left(1-x^{7}\right)\left(1-x^{9}\right) \cdots}
\]
将该分式展成无穷级数, 则所得级数的每一项,系数都等于指数拆成奇数和的种数. $Q$ 是 我们上节考察过了的,因此我们得到定理.

一数拆成相异整数之和的种数, 等于拆成奇数之和的种数, 奇数可以相等.

\section{$\S 327$}

我们已经看到了
\[
P=1-x-x^{2}+x^{5}+x^{7}-x^{12}-x^{15}+x^{22}+x^{26}-x^{35}-x^{40}+\cdots
\]
将 $x$ 换成 $x^{2}$, 得
\[
P Q=1-x^{2}-x^{4}+x^{10}+x^{14}-x^{24}-x^{30}+x^{44}+x^{52}-\cdots
\]
用 $P$ 除 $P Q$, 得
\[
Q=\frac{1-x^{2}-x^{4}+x^{10}+x^{14}-x^{24}-x^{30}+\cdots}{1-x-x^{2}+x^{5}+x^{7}-x^{12}-x^{15}+x^{22}+x^{26}-\cdots}
\]
可见 $Q$ 也可展成递推级数,而且这个级数可以由 $\frac{1}{P}$ 乘上
\[
1-x^{2}-x^{4}+x^{10}+x^{14}-x^{24}-\cdots
\]
得到. 从 $\S 324$ 我们知道
\[
\frac{1}{P}=1+x+2 x^{2}+3 x^{3}+5 x^{4}+7 x^{5}+11 x^{6}+15 x^{7}+22 x^{8}+30 x^{9}+\cdots
\]
乘它以
\[
1-x^{2}-x^{4}+x^{10}+x^{14}-\cdots
\]
得
\[
\begin{gathered}
1+x+2 x^{2}+3 x^{3}+5 x^{4}+7 x^{5}+11 x^{6}+15 x^{7}+22 x^{8}+30 x^{9}+\cdots \\
-x^{2}-x^{3}-2 x^{4}-3 x^{5}-5 x^{6}-7 x^{7}-11 x^{8}-15 x^{9}-\cdots \\
-x^{4}-x^{5}-2 x^{6}-3 x^{7}-5 x^{8}-7 x^{9}-\cdots
\end{gathered}
\]
或
\[
1+x+x^{2}+2 x^{3}+2 x^{4}+3 x^{5}+4 x^{6}+5 x^{7}+6 x^{8}+8 x^{9}+\cdots=Q
\]
从一个数拆成可以相同的数之和的种数, 可以推出拆成相异数之和的种数, 进而又 可推出拆成奇数之和的种数.

\section{$\S 328$}

还有几种情形应该注意, 它们也对了解数的性质有帮助. 考虑表达式 
\[(1+x)(1+x^2)(1+x^4)(1+x^8)(1+x^16)(1+x^22)\cdots\]
其中每个指数都是前一个的两倍. 展开式为级数
\[
1+x+x^{2}+x^{3}+x^{4}+x^{5}+x^{6}+x^{7}+x^{8}+\cdots
\]
可能会问. 这个级数是按几何级数一直延续下去吗? 我们来回答这个问题. 记
\[
P=(1+x)\left(1+x^{2}\right)\left(1+x^{4}\right)\left(1+x^{8}\right)\left(1+x^{16}\right)\left(1+x^{32}\right) \cdots
\]
设其展开式为
\[
P=1+\alpha x+\beta x^{2}+\gamma x^{3}+\delta x^{4}+\varepsilon x^{5}+\zeta x^{6}+\eta x^{7}+\theta x^{8}+\cdots
\]
将 $x$ 换为 $x^{2}$, 得乘积
\[
\left(1+x^{2}\right)\left(1+x^{4}\right)\left(1+x^{8}\right)\left(1+x^{16}\right)\left(1+x^{32}\right) \cdots=\frac{P}{1+x}
\]
对展开式做同样的代换, 得
\[
\frac{P}{1+x}=1+\alpha x^{2}+\beta x^{4}+\gamma x^{6}+\delta x^{8}+\varepsilon x^{10}+\zeta x^{12}+\cdots
\]
两边乘 $1+x$, 得
\[
P=1+x+\alpha x^{2}+\alpha x^{3}+\beta x^{4}+\beta x^{5}+\gamma x^{6}+\gamma x^{7}+\delta x^{8}+\delta x^{9}+\cdots
\]
两个 $P$ 相比较,得
\[
\alpha=1, \beta=\alpha, \gamma=\alpha, \delta=\beta, \varepsilon=\beta, \zeta=\gamma, \eta=\gamma, \cdots
\]
结果是系数都为 1 , 即展开式确实是按几何级数一直延续下去, 是几何级数
\[
1+x+x^{2}+x^{3}+x^{4}+x^{5}+x^{6}+x^{7}+\cdots
\]
\section{$\S 329$}

$x$ 的各次幂都在上节几何级数中出现, 并且只出现一次. 从这几何级数等于乘积
\[
(1+x)\left(1+x^{2}\right)\left(1+x^{4}\right)\left(1+x^{8}\right)\left(1+x^{16}\right)\left(1+x^{32}\right) \cdots
\]
我们得到:每个数都可表示成以 2 为公比的几何级数
\[
1,2,4,8,16,32, \cdots
\]
的项的和,并且表示法是唯一的.

这一性质也可从天平的使用中得到证实. 设砝码的克数为 $1,2,4,8,16,32, \cdots$, 用这 样的砝码我们可以称重量为任何克数的物体. 注意, 不足 1 克的重量, 这里不计. 用重量 为 $1,2,4,8,16,32,64,128,256,512$ 克的这 10 个砝码, 我们可以称 1 克到 1024 克的所有 重量. 如果再加上一个 1024 克的砝码, 我们就可以称 1 克到 2048 克的所有重量.

\section{$\S 330$}

用更少的砝码, 即用以 3 为公比的几何级数
\[
1,3,9,27,81, \cdots
\]
的项做砝码的克数, 也可以称任何重量. 这里不足一克的重量也不计. 但这里有一点与前 节不同. 前节物体、砝码分置两个托盘, 这里放物体的托盘中, 有时也要放砝码. 这是因为用以 3 为公比的几何级数的不同项形成所有的数时, 需要加减法并用. 例如
\[
\begin{gathered}
1=1 \\
2=3-1 \\
3=3 \\
4=3+1 \\
5=9-3-1 \\
6=9-3 \\
7=9-3+1 \\
8=9-1 \\
9=9 \\
10=9+1 \\
11=9+3-1 \\
12=9+3 \\
\vdots 
\end{gathered}
\]
\section{$\S 331$}

为证明上节结论, 我们考虑无穷乘积
\[
\left(x^{-1}+1+x\right)\left(x^{-3}+1+x^{3}\right)\left(x^{-9}+1+x^{9}\right)\left(x^{-27}+1+x^{27}\right) \cdots=P
\]
其展开式中的指数, 全都由数 $1,3,9,27,81, \cdots$ 经过加减两种运算形成. 我们问: 是不是 每一个数都在展开式的指数中出现. 为回答这个问题, 我们令
\[
P=\cdots+c x^{-3}+b x^{-2}+a x^{-1}+1+a x+\beta x^{2}+\gamma x^{3}+\delta x^{4}+\varepsilon x^{5}+\cdots
\]
替 $x$ 以 $x^{3}$,得
\[
\frac{P}{x^{-1}+1+x^{1}}=\cdots+b x^{-6}+a x^{-3}+1+a x^{3}+\beta x^{6}+\gamma x^{9}+\cdots
\]
由此得
\[
P=\cdots+a x^{-4}+a x^{-3}+a x^{-2}+x^{-1}+1+x+a x^{2}+a x^{3}+\alpha x^{4}+\beta x^{5}+\beta x^{6}+\beta x^{7}+\cdots
\]
两 $P$ 相较,得
\[
\begin{gathered}
\alpha=1, \beta=\alpha, \gamma=\alpha, \delta=\alpha, \varepsilon=\beta, \zeta=\beta, \cdots \\
a=1, b=a, c=a, d=a, e=b, \cdots
\end{gathered}
\]
这样一来,我们看到
\[
\begin{aligned}
P= & 1+x+x^{2}+x^{3}+x^{4}+x^{5}+x^{6}+x^{7}+\cdots+ \\
& x^{-1}+x^{-2}+x^{-3}+x^{-4}+x^{-5}+x^{-6}+x^{-7}+\cdots
\end{aligned}
\]
我们看到, 正的、负的每一个数都在展开式的指数中出现, 也即, 每一个数都能够由公比 为 3 的几何级数的项通过加减两种运算得到, 并且得到的方式只有一种. 

