\chapter{第十二章 分解分数函数为实部分分式}

\section{$\S 199$}

第二章讲了, 分数函数可分解成其分母线性因式个数, 那么多个部分分式, 每一个因 式都是一个部分分式的分母. 自然, 线性因式是虚的, 由它作分母的部分分式也是虚的. 实分数函数分解成的虚部分分式是很少有用处的. 但我们讲过, 作为分数函数分母的整 函数, 不管它含有多少个虚线性因式, 我们都可以把它们表示成实二次因式. 这样, 在允 许部分分式的分母为实二次因式的条件之下, 我们可以把任何一个分数函数都分解为实 部分分式.

\section{$\S 200$}

记给定的分数函数为 $\frac{M}{N}, N$ 的实线性因式所对应的部分分式, 求法我们讲过了. 对于 $N$ 的虚线性因式, 我们改为考虑因式
\[
p^{2}-2 p q z \cos \varphi+q^{2} z^{2}
\]
这时的分数函数, 其形状为
\[
\frac{A+B z+C z^{2}+D z^{3}+E z^{4}+\cdots}{\left(p^{2}-2 p q z \cos \varphi+q^{2} z^{2}\right)\left(\alpha+\beta z+\gamma z^{2}+\delta z^{3}+\cdots\right)}
\]
我们求的以 $p^{2}-2 p q z \cos \varphi+q^{2} z^{2}$ 为分母的部分分式应该为
\[
\frac{\mathfrak{U}+\mathfrak{a} z}{p^{2}-2 p q z \cos \varphi+q^{2} z^{2}}
\]
该分式的分母是二次的, 因而分子的次数不能大于 1 , 否则该分式含有一个整函数, 应该 分出去.

\section{$\S 201$}

为简单起见, 令分子
\[
A+B z+C z^{2}+\cdots=M
\]
令分母的第二因式 
\[
\alpha+\beta z+\gamma z^{2}+\cdots=Z
\]
记因式 $Z$ 产生的部分分式为 $\frac{Y}{Z}$,则
\[
Y=\frac{M-\mathscr{U} Z-a Z z}{p^{2}-2 p q z \cos \varphi+q^{2} z^{2}}
\]
$Y$ 应该是 $z$ 的整函数, 因而必定 $M-\mathfrak{U} Z-a Z z$ 被 $p^{2}-2 p q z \cos \varphi+q^{2} z^{2}$ 整除. 从而, 当
\[
p^{2}-2 p q z \cos \varphi+q^{2} z^{2}=0
\]
也即
\[
z=\frac{p}{q}(\cos \varphi+\sqrt{-1} \sin \varphi)
\]
或
\[
z=\frac{p}{q}(\cos \varphi-\sqrt{-1} \sin \varphi)
\]
时, $M-9 \mathfrak{Q} Z-a Z z$ 为零. 置 $\frac{p}{q}=f$, 则
\[
z^{n}=f^{n}(\cos n \varphi \pm \sqrt{-1} \sin n \varphi)
\]
将 $z^{n}$ 的这两个表达式代入,得到决定 $\mathfrak{A}$ 和 $a$ 的两个方程.

\section{$\S 202$}

方程 $M=\mathfrak{A} Z+\mathfrak{a} Z z$ 经过两种代入成为方程
\[
\begin{aligned}
& \left.\begin{array}{rl}A+B f \cos \varphi+C f^{2} \cos 2 \varphi+D f^{3} \cos 3 \varphi+\cdots \pm \\& \left(B f \sin \varphi+C f^{2} \sin 2 \varphi+D f^{3} \sin 3 \varphi+\cdots\right) \sqrt{-1}\end{array}\right\}= \\
& \left\{\begin{array}{l}\mathfrak{A}\left(\alpha+\beta f \cos \varphi+\gamma f^{2} \cos 2 \varphi+\delta f^{3} \cos 3 \varphi+\cdots\right) \pm \\\mathfrak{A}\left(\beta f \sin \varphi+\gamma f^{2} \sin 2 \varphi+\delta f^{3} \sin 3 \varphi+\cdots\right) \sqrt{-1}+ \\\mathfrak{a}\left(\alpha f \cos \varphi+\beta f^{2} \cos 2 \varphi+\gamma f^{3} \cos 3 \varphi+\cdots\right) \pm \\a\left(\alpha f \sin \varphi+\beta f^{2} \sin 2 \varphi+\gamma f^{3} \sin 3 \varphi+\cdots\right) \sqrt{-1}\end{array}\right.
\end{aligned}
\]
为便于计算, 令
\[
\begin{aligned}
& A+ \beta f \cos \varphi+C f^{2} \cos 2 \varphi+D f^{3} \cos 3 \varphi+\cdots=\mathfrak{B} \\
& \beta f \sin \varphi+C f^{2} \sin 2 \varphi+D f^{3} \sin 3 \varphi+\cdots=\mathfrak{p} \\
& \alpha+\beta f \cos \varphi+\gamma f^{2} \cos 2 \varphi+\delta f^{3} \cos 3 \varphi+\cdots=\Im \\
& \beta f \sin \varphi+\gamma f^{2} \sin 2 \varphi+\delta f^{3} \sin 3 \varphi+\cdots=\mathfrak{q} \\
& \alpha f \cos \varphi+\beta f^{2} \cos 2 \varphi+\gamma f^{3} \cos 3 \varphi+\cdots=\mathfrak{R} \\
& a f \sin \varphi+\beta f^{2} \sin 2 \varphi+\gamma f^{3} \sin 3 \varphi+\cdots=\mathfrak{r}
\end{aligned}
\]
在新的记号之下, 我们的方程成为
\[
\mathfrak{B}+\mathfrak{p} \sqrt{-1}=\mathfrak{U} \mathfrak{D}+\mathfrak{U} \mathfrak{q} \sqrt{-1}+\mathfrak{a} \mathfrak{R} \pm \mathfrak{a r} \sqrt{-1}
\]
%%08p141-160

\section{$\S 203$}

由于双重符号, 我们得到方程组
\[
\begin{gathered}
\mathfrak{B}=\mathfrak{A} \cong+\mathfrak{a} \mathfrak{R} \\
\mathfrak{p}=\mathfrak{U} \mathfrak{q}+\mathfrak{a} \mathfrak{r}
\end{gathered}
\]
解为
\[
\begin{aligned}
& \mathfrak{U}=\frac{\mathfrak{B r}-\mathfrak{p} \mathfrak{R}}{\mathfrak{D r}-\mathfrak{q} \mathfrak{R}} \\
& \mathfrak{a}=\frac{\mathfrak{B} \mathfrak{q}-\mathfrak{p} \cong}{\mathfrak{q} \mathfrak{R}-\mathfrak{D r}}
\end{aligned}
\]
这样一来, 我们就得到了分数函数
\[
\frac{M}{\left(p^{2}-2 p q z \cos \varphi+q^{2} z^{2}\right) Z}
\]
的部分分式
\[
\frac{\mathfrak{U}+a z}{p^{2}-2 p q z \cos \varphi+q^{2} z^{2}}
\]
的求法.

记 $f=\frac{p}{q}$, 则:

代换 $z^{n}=f^{n} \cos n \varphi$,使 $M=\mathfrak{B}$;

代换 $z^{n}=f^{n} \sin n \varphi$,使 $M=\mathfrak{p}$;

代换 $z^{n}=f^{n} \cos n \varphi$,使 $Z=\cong$;

代换 $z^{n}=f^{n} \sin n \varphi$, 使 $Z=\mathfrak{q}$;

代换 $z^{n}=f^{n} \cos n \varphi$,使 $z Z=\Re$;

代换 $z^{n}=f^{n} \sin n \varphi$,使 $z Z=\mathfrak{r}$.

有了 $\mathfrak{B}, \mathfrak{Q}, \mathfrak{R}, \mathfrak{p}, \mathfrak{q}, \mathfrak{r}$,也就有了

\[
\mathfrak{A}=\frac{\mathfrak{Br}-\mathfrak{pR}}{\mathfrak{Dr}-\mathfrak{qR}}, a=\frac{\mathfrak{pD}-\mathfrak{B q}}{\mathfrak{D r}-\mathfrak{qR}}
\]

例 1 设给定的分数函数为
\[
\frac{z^{2}}{\left(1-z+z^{2}\right)\left(1+z^{4}\right)}
\]
我们先求对应于因式 $1-z+z^{2}$ 的部分分式
\[
\frac{\mathfrak{U}+\mathfrak{a} z}{1-z+z^{2}}
\]
与通用表达式
\[
p^{2}-2 p q z \cos \varphi+q^{2} z^{2}
\]
相比较,这里 
\[
p=1, q=1, \cos \varphi=\frac{1}{2}
\]
从而
\[
\varphi=60^{\circ}=\frac{\pi}{3}
\]
由 $M=z^{2}, Z=1+z^{4}, f=1$, 我们得到
\[
\begin{gathered}
\mathfrak{B}=\cos \frac{2 \pi}{3}=-\frac{1}{2}, \mathfrak{p}=\frac{\sqrt{3}}{2} \\
\mathfrak{D}=1+\cos \frac{4 \pi}{3}=\frac{1}{2}, \mathfrak{q}=-\frac{\sqrt{3}}{2} \\
\mathfrak{R}=\cos \frac{\pi}{3}+\cos \frac{5 \pi}{3}=1, \mathfrak{x}=0
\end{gathered}
\]
由此得
\[
\mathfrak{A}=-1, a=0
\]
从而所求部分分式为
\[
\frac{-1}{1-z+z^{2}}
\]
所给函数等于求得的这个分式与下面这个分式的和
\[
\frac{1+z+z^{2}}{1+z^{4}}
\]
该分式分母 $1+z^{4}$ 的因式为
\[
1+z \sqrt{2}+z^{2} \text { 和 } 1-z \sqrt{2}+z^{2}
\]
求对应于这两个分母的部分分式时, $\varphi$ 相同, 都为 $\frac{\pi}{4}$, 一个的 $f=-1$, 另一个的 $f=+1$.

例 2 我们来求这两个部分分式, 也即求分数函数
\[
\frac{1+z+z^{2}}{\left(1+z \sqrt{2}+z^{2}\right)\left(1-z \sqrt{2}+z^{2}\right)}
\]
的部分分式. 这里
\[
M=1+z+z^{2}
\]
对于第一个因式我们有
\[
f=-1, \varphi=\frac{\pi}{4}, Z=1-z \sqrt{2}+z^{2}
\]
从而
\[
\begin{gathered}
\mathfrak{B}=1-\cos \frac{\pi}{4}+\cos \frac{2 \pi}{4}=\frac{\sqrt{2}-1}{\sqrt{2}} \\
\mathfrak{p}=-\sin \frac{\pi}{4}+\sin \frac{2 \pi}{4}=\frac{\sqrt{2}-1}{\sqrt{2}} \\
₫=1+\sqrt{2} \cos \frac{\pi}{4}+\cos \frac{2 \pi}{4}=2
\end{gathered}
\]
\[
\begin{aligned}
& \text { Infinite analysio (无忩分析论). Fntroduction } \\
& q=+\sqrt{2} \sin \frac{\pi}{4}+\sin \frac{2 \pi}{4}=2 \\
& \mathfrak{R}=-\cos \frac{\pi}{4}-\sqrt{2} \cos \frac{2 \pi}{4}-\cos \frac{3 \pi}{4}=0 \\
& \mathfrak{r}=-\sin \frac{\pi}{4}-\sqrt{2} \sin \frac{2 \pi}{4}-\sin \frac{3 \pi}{4}=-2 \sqrt{2}
\end{aligned}
\]
继而
\[
\$ r-q \mathfrak{R}=-4 \sqrt{2}
\]
和
\[
\mathfrak{A}=\frac{\sqrt{2}-1}{2 \sqrt{2}}, \mathfrak{a}=0
\]
这样我们得到对应于因式 $1+z \sqrt{2}+z^{2}$ 的部分分式为
\[
\frac{(\sqrt{2}-1): 2 \sqrt{2}}{1+z \sqrt{2}+z^{2}}
\]
类似地,我们得到另一个部分分式为
\[
\frac{(\sqrt{2}+1): 2 \sqrt{2}}{1-z \sqrt{2}+z^{2}}
\]
现在我们看到,例 1 所给函数
\[
\frac{z^{2}}{\left(1-z+z^{2}\right)\left(1+z^{4}\right)}
\]
分解成了
\[
\frac{-1}{1-z+z^{2}}+\frac{(\sqrt{2}-1): 2 \sqrt{2}}{1+z \sqrt{2}+z^{2}}+\frac{(\sqrt{2}+1): 2 \sqrt{2}}{1-z \sqrt{2}+z^{2}}
\]
例 3 分解函数
\[
\frac{1+2 z+z^{2}}{\left(1-\frac{8}{5} z+z^{2}\right)\left(1+2 z+3 z^{2}\right)}
\]
为部分分式. 因式 $1-\frac{8}{5} z+z^{2}$ 产生的部分分式为
\[
\frac{\mathfrak{A}+a z}{1-\frac{8}{5}+z^{2}}
\]
这里
\[
p=1, q=1, \cos \varphi=\frac{4}{5}, f=1
\]
又
\[
M=1+2 z+z^{2}, Z=1+2 z+3 z^{2}
\]
由于这里角 $\varphi$ 与直角的比末知, 所以需计算 $\varphi$ 和其倍数的正弦和余弦. 由 $\cos \varphi$ 得到所需 

结果
\[
\begin{gathered}
\cos \varphi=\frac{4}{5}, \sin \varphi=\frac{3}{5} \\
\cos 2 \varphi=\frac{7}{25}, \sin 2 \varphi=\frac{24}{25} \\
\cos 3 \varphi=-\frac{44}{125}, \sin 3 \varphi=-\frac{117}{125}
\end{gathered}
\]
从而
\[
\begin{gathered}
\mathfrak{B}=1+2 \cdot \frac{4}{5}+\frac{7}{25}=\frac{72}{25} \\
\mathfrak{p}=2 \cdot \frac{3}{5}+\frac{24}{25}=\frac{54}{25} \\
\mathfrak{Q}=1+2 \cdot \frac{4}{5}+3 \frac{7}{25}=\frac{86}{25} \\
\mathfrak{q}=2 \cdot \frac{3}{5}+3 \cdot \frac{24}{25}=\frac{102}{25} \\
\mathfrak{R}=\frac{4}{5}+2 \cdot \frac{7}{25}-3 \cdot \frac{44}{125}=\frac{38}{125} \\
\mathfrak{r}=\frac{3}{5}+2 \cdot \frac{24}{25}+3 \cdot \frac{117}{125}=\frac{666}{125}
\end{gathered}
\]
由此得

\[
\mathfrak{D}r-\mathfrak{qR}=\frac{53400}{25 \cdot 125}=\frac{2136}{125}
\]

从而
\[
\mathfrak{A}=\frac{1836}{2136}=\frac{153}{178}, \mathfrak{a}=-\frac{540}{2136}=-\frac{45}{178}
\]
这样我们得到由 $1-\frac{8}{5} z+z^{2}$ 产生的部分分式为
\[
\frac{9(17-5 z): 178}{1-\frac{8}{5} z+z^{2}}
\]
对应于另一个因式的部分分式, 求法类似.

我们有
\[
p=1, q=-\sqrt{3}, \cos \varphi=\frac{1}{\sqrt{3}}
\]
和
\[
f=-\frac{1}{\sqrt{3}}, M=1+2 z+z^{2}, Z=1-\frac{8}{5} z+z^{2}
\]
由 $\cos \varphi=\frac{1}{\sqrt{3}}$ 得所需结果 
\[
\begin{gathered}
\cos \varphi=\frac{1}{\sqrt{3}}, \sin \varphi=\frac{\sqrt{2}}{\sqrt{3}} \\
\cos 2 \varphi=-\frac{1}{3}, \sin 2 \varphi=\frac{2 \sqrt{2}}{3} \\
\cos 3 \varphi=-\frac{5}{3 \sqrt{3}}, \sin 3 \varphi=\frac{\sqrt{2}}{3 \sqrt{3}}
\end{gathered}
\]
利用这些结果, 我们得到
\[
\begin{gathered}
\mathfrak{B}=1-\frac{2}{\sqrt{3}} \cdot \frac{1}{\sqrt{3}}+\frac{1}{3} \cdot\left(-\frac{1}{3}\right)=\frac{2}{9} \\
\mathfrak{p}=-\frac{2}{\sqrt{3}} \cdot \frac{\sqrt{2}}{\sqrt{3}}+\frac{1}{3} \cdot \frac{2 \sqrt{2}}{3}=-\frac{4 \sqrt{2}}{3} \\
\mathfrak{Q}=1+\frac{8}{5 \sqrt{3}} \cdot \frac{1}{\sqrt{3}}+\frac{1}{3} \cdot\left(-\frac{1}{3}\right)=\frac{64}{45} \\
\mathfrak{q}=\frac{8}{5 \sqrt{3}} \cdot \frac{\sqrt{2}}{\sqrt{3}}+\frac{1}{3} \cdot \frac{2 \sqrt{2}}{3}=\frac{34 \sqrt{2}}{45} \\
\mathfrak{R}=-\frac{1}{\sqrt{3}} \cdot \frac{1}{\sqrt{3}}-\frac{8}{5 \cdot 3}\left(-\frac{1}{3}\right)-\frac{1}{3 \sqrt{3}} \cdot\left(-\frac{5}{3 \sqrt{3}}\right)=\frac{4}{135} \\
\mathfrak{x}=-\frac{1}{\sqrt{3}} \cdot \frac{\sqrt{2}}{\sqrt{3}}-\frac{8}{5 \cdot 3} \cdot \frac{2 \sqrt{2}}{3}-\frac{1}{3 \cdot 3 \sqrt{3}} \cdot \frac{\sqrt{2}}{3 \cdot \sqrt{3}}=-\frac{98 \sqrt{2}}{135}
\end{gathered}
\]
从而
\[
\supseteq \mathfrak{x}-\mathfrak{R}=-\frac{712 \sqrt{2}}{675}
\]
继而
\[
\mathfrak{A}=\frac{100}{712}=\frac{25}{178}, \mathfrak{a}=\frac{540}{712}=\frac{135}{178}
\]
最后我们得到函数
\[
\frac{1+2 z+z^{2}}{\left(1-\frac{8}{5} z+z^{2}\right)\left(1+2 z+3 z^{2}\right)}
\]
的部分分式表示为
\[
\frac{9(17-5 z): 178}{1-\frac{8}{5} z+z^{2}}+\frac{5(5+27 z): 178}{1+2 z+3 z^{2}}
\]
\section{$\S 204$}

$\Re$ 和 $\mathfrak{r}$ 的值可由 $\supseteq$ 和 $\mathfrak{q}$ 决定. 事实上,由
\[
\varsubsetneqq=\alpha+\beta f \cos \varphi+\gamma f^{2} \cos 2 \varphi+\delta f^{3} \cos 3 \varphi+\cdots
\]
$\mathfrak{q}=\beta f \sin \varphi+\gamma f^{2} \sin 2 \varphi+\delta f^{3} \sin 3 \varphi+\cdots$

得
\[
₫ \cos \varphi-q \sin \varphi=\alpha \cos \varphi+\beta f \cos 2 \varphi+\gamma f^{2} \cos 3 \varphi+\cdots
\]
从而
\[
\mathfrak{R}=f(₫ \cos \varphi-\mathfrak{q} \sin \varphi)
\]
类似地
\[
Ð \sin \varphi+q \cos \varphi=\alpha \sin \varphi+\beta f \sin 2 \varphi+\gamma f^{2} \sin 3 \varphi+\cdots
\]
从而
\[
\mathfrak{x}=f(D \sin \varphi+\mathfrak{q} \cos \varphi)
\]
进一步, 得到
\[
\begin{aligned}
& \varsubsetneqq \mathfrak{x}-\mathfrak{q} \mathfrak{R}=\left(D^{2}+\mathfrak{q}^{2}\right) f \sin \varphi
\end{aligned}
\]
从而
\[
\begin{gathered}
\mathfrak{A}=\frac{\mathfrak{B} \mathscr{D}+\mathfrak{p q}}{\mathfrak{D}^{2}+\mathfrak{q}^{2}}+\frac{\mathfrak{B} \mathscr{D}-\mathfrak{p q}}{\mathfrak{D}^{2}+\mathfrak{q}^{2}} \cdot \frac{\cos \varphi}{\sin \varphi} \\
\alpha=-\frac{\mathfrak{B} \mathfrak{q}-\mathfrak{p} \rightrightarrows}{\mathfrak{D}^{2}+\mathfrak{q}^{2}} \cdot \frac{1}{f \sin \varphi}
\end{gathered}
\]
这样,因式 $p^{2}-2 p q z \cos \varphi+q^{2} z^{2}$ 产生的部分分式为

\[
\frac{(\mathfrak{BD}+\mathfrak{p q}) f \sin \varphi+(\mathfrak{BD}-\mathfrak{p q})(f \cos \varphi-z)}{\left(p^2-2 p q z \cos \varphi+q^2 z^2\right)\left(\mathfrak{Q}^2+\mathfrak{q}^2\right) f \sin \varphi}
\]

或者换 $f$ 为 $\frac{p}{q}$, 将它写成
\[
\frac{(\mathfrak{B} \cap+\mathfrak{p} \mathfrak{q}) p \sin \varphi+(\mathfrak{B} \mathfrak{q}-\mathfrak{p} \rightrightarrows)(p \cos \varphi-q z)}{\left(p^{2}-2 p q z \cos \varphi+q^{2} z^{2}\right)\left(\mathfrak{D}^{2}+\mathfrak{q}^{2}\right) p \sin \varphi}
\]
\section{$\S 205$}

上节我们推出了分数函数
\[
\frac{M}{\left(p^{2}-2 p q z \cos \varphi+q^{2} z^{2}\right) Z}
\]
分母的因式 $p^{2}-2 p q z \cos \varphi+q^{2} z^{2}$ 所产生的部分分式. 所得表达式中的 $\mathfrak{B}, \mathfrak{p}, \mathfrak{D}, \mathfrak{q}$ 可以从 $M, Z$ 求出. 方法是对 $M, Z$ 做代换. 做代换 $z^{n}=\frac{p^{n}}{q^{n}} \cos n \varphi$, 得
\[
M=\mathfrak{B}, Z=\mathfrak{S}
\]
做代换 $z^{n}=\frac{p^{n}}{q^{n}}$, 得
\[
M=\mathfrak{p}, Z=\$
\]
注意, 代换之前应将 $M$ 和 $Z$ 展开, 即要使得它们的形状为
\[
\begin{aligned}
M & =A+B z+C z^{2}+D z^{3}+E z^{4}+\cdots \\
Z & =\alpha+\beta z+\gamma z^{2}+\delta z^{3}+\varepsilon z^{4}+\cdots
\end{aligned}
\]
这样我们有
\[
\begin{gathered}
\mathfrak{B}=A+B \frac{p}{q} \cos \varphi+C \frac{p^{2}}{q^{2}} \cos 2 \varphi+D \frac{p^{3}}{q^{3}} \cos 3 \varphi+\cdots \\
\mathfrak{p}=B \frac{p}{q} \sin \varphi+C \frac{p^{2}}{q^{2}} \sin 2 \varphi+D \frac{p^{3}}{q^{3}} \sin 3 \varphi+\cdots \\
\mathfrak{D}=\alpha+\beta \frac{p}{q} \cos \varphi+\gamma \frac{p^{2}}{q^{2}} \cos 2 \varphi+\delta \frac{p^{3}}{q^{3}} \cos 3 \varphi+\cdots \\
\mathfrak{q}=\beta \frac{p}{q} \sin \varphi+\gamma \frac{p^{2}}{q^{2}} \sin 2 \varphi+\delta \frac{p^{3}}{q^{3}} \sin 3 \varphi+\cdots
\end{gathered}
\]

\section{$\S 206$}

如果 $p^{2}-2 p q z \cos \varphi+q^{2} z^{2}$. 是 $Z$ 的因式,那么代换
\[
z^{n}=f^{n}(\cos n \varphi \pm \sqrt{-1} \sin n \varphi)
\]
使 $Z$ 为零,因而在这个代换之下, 从方程
\[
\boldsymbol{M}=\mathfrak{U} z+\mathfrak{a} Z z
\]
得不到任何东西. 所以前面讲的方法不能使用. 当 $\left(p^{2}-2 p q z \cos \varphi+q^{2} z^{2}\right)^{2}$ 或更高次幂为 函数 $\frac{M}{N}$ 分母的因式时, 我们求另外形状的部分分式. 先讨论
\[
N=\left(p^{2}-2 p q z \cos \varphi+q^{2} z^{2}\right)^{2} Z
\]
的情形. 此时我们设 $\left(p^{2}-2 p q z \cos \varphi+q^{2} z^{2}\right)^{2}$ 产生的部分分式为
\[
\frac{\mathfrak{A}+\mathfrak{a} z}{\left(p^{2}-2 p q z \cos \varphi+q^{2} z^{2}\right)^{2}}+\frac{\mathfrak{B}+\mathfrak{b} z}{p^{2}-2 p q z \cos \varphi+q^{2} z^{2}}
\]
$\mathfrak{A}, \mathfrak{a}, \mathfrak{B}, \mathfrak{b}$ 待定.

\section{$\S 207$}

依前节所设, 表达式
\[
\frac{M-(\mathfrak{A}+a z) Z-(\mathfrak{B}+\mathfrak{b} z) Z\left(p^{2}-2 p q z \cos \varphi+q^{2} z^{2}\right)}{\left(p^{2}-2 p q z \cos \varphi+q^{2} z^{2}\right)^{2}}
\]
应该是整函数,也即分子应被分母除得尽. 首先, 同于前面,表达式
\[
M-(\mathfrak{A}+a z) Z
\]
应被 $p^{2}-2 p q z \cos \varphi+q^{2} z^{2}$ 除得尽, 因而照用前面的方法即可得到 $\mathfrak{U}$ 和 $a$.
\[
\text { 将 } z^{n}=\frac{p^{n}}{q^{n}} \cos n \varphi \text { 代入 } M \text { 和 } Z \text {, 得 }
\]
\[
M=\mathfrak{B}, Z=\mathfrak{N}
\]
将 $z^{n}=\frac{p^{n}}{q^{n}} \sin n \varphi$ 代入 $M$ 和 $Z$, 得
\[
M=\mathfrak{p}, Z=n
\]
有了这几个值, 利用前面的规则, 我们得到
\[
\begin{gathered}
\mathfrak{U}=\frac{\mathfrak{B} \mathfrak{N}+\mathfrak{p} \mathfrak{n}}{\mathfrak{R}^{2}+\mathfrak{n}^{2}}+\frac{\mathfrak{B} \mathfrak{n}-\mathfrak{p} \mathfrak{R}}{\mathfrak{R}^{2}+\mathfrak{n}^{2}} \cdot \frac{\cos \varphi}{\sin \varphi} \\
\mathfrak{a}=-\frac{\mathfrak{B} \mathfrak{n}-\mathfrak{p} \mathfrak{R}}{\mathfrak{R}^{2}+\mathfrak{n}^{2}} \cdot \frac{\mathfrak{q}}{p \sin \varphi}
\end{gathered}
\]
\section{$\S 208$}

求得了 $\mathfrak{U}$ 和 $\mathfrak{a}$,那么
\[
\frac{M-(\mathfrak{U}+\mathfrak{a}) Z}{p^{2}-2 p q z \cos \varphi+q^{2} z^{2}}
\]
是一个整函数, 记它为 $P$, 则跟前面一样的表达式
\[
P-(\mathfrak{B}+\mathfrak{b} z) Z
\]
一样地被 $p^{2}-2 p q z \cos \varphi+q^{2} z^{2}$ 整除, 从而将 $P$ 中的 $z^{n}$ 换为 $\frac{p^{n}}{q^{n}} \cos n \varphi$ 和 $\frac{p^{n}}{q^{n}} \sin n \varphi$, 记所得 为 $\mathfrak{R}$ 和 $r$, 得
\[
\begin{gathered}
\mathfrak{B}=\frac{\mathfrak{R} \mathfrak{R}+\mathfrak{r} \mathfrak{n}}{\mathfrak{R}^{2}+\mathfrak{n}^{2}}+\frac{\mathfrak{B} \mathfrak{n}-\mathfrak{r} \mathfrak{R}}{\mathfrak{R}^{2}+\mathfrak{n}^{2}} \cdot \frac{\cos \varphi}{\sin \varphi} \\
\mathfrak{b}=-\frac{\mathfrak{R} \mathfrak{n} \mathfrak{R}}{\mathfrak{R}^{2}+\mathfrak{n}^{2}} \cdot \frac{q}{p \sin \varphi}
\end{gathered}
\]
\section{$\S 209$}

对应于
\[
\left(p^{2}-2 p q z \cos \varphi+q^{2} z^{2}\right)^{k}
\]
的部分分式如何求, 从以上所讲, 我们已经可以得出结论, 记
\[
N=\left(p^{2}-2 p q z \cos \varphi+q^{2} z^{2}\right)^{k} Z
\]
也即我们要将
\[
\frac{M}{\left(p^{2}-2 p q z \cos \varphi+q^{2} z^{2}\right)^{k} Z}
\]
分解成部分分式, 假定因式 $\left(p^{2}-2 p q z \cos \varphi+q^{2} z^{2}\right)^{k}$ 产生的部分分式为
\[
\begin{aligned}
& \frac{\mathfrak{U}+\mathfrak{a} z}{\left(p^{2}-2 p q z \cos \varphi+q^{2} z^{2}\right)^{k}}+\frac{\mathfrak{B}+\mathfrak{b} z}{\left(p^{2}-2 p q z \cos \varphi+q^{2} z^{2}\right)^{k-1}}+ \\
& \frac{\mathfrak{S}+c z}{\left(p^{2}-2 p q z \cos \varphi+q^{2} z^{2}\right)^{k-2}}+\frac{\mathfrak{D}+\mathfrak{b} z}{\left(p^{2}-2 p q z \cos \varphi+q^{2} z^{2}\right)^{k-3}}+\cdots
\end{aligned}
\]
那么记 
\[
\begin{aligned}
& z^{n}=\frac{p^{n}}{q^{n}} \cos n \varphi \text { 时, } M=\mathfrak{M}, Z=\mathfrak{N} \\
& z^{n}=\frac{p^{n}}{q^{n}} \sin n \varphi \text { 时, } M=m, Z=n
\end{aligned}
\]
则
\[
\begin{gathered}
\mathfrak{U}=\frac{\mathfrak{M} \mathfrak{R}+\mathfrak{M} \mathfrak{R}}{\mathfrak{R}^{2}+\mathfrak{n}^{2}}+\frac{\mathfrak{M} \mathfrak{n}-\mathfrak{M} \mathfrak{N}}{\mathfrak{R}^{2}+\mathfrak{n}^{2}} \cdot \frac{\cos \varphi}{\sin \varphi} \\
\mathfrak{a}=-\frac{\mathfrak{M} \mathfrak{n}-\mathfrak{n} \mathfrak{N}}{\mathfrak{N}^{2}+\mathfrak{n}^{2}} \cdot \frac{q}{p \sin \varphi}
\end{gathered}
\]
首先, 令
\[
\frac{M-(\mathfrak{Q}+a z) Z}{p^{2}-2 p q z \cos \varphi+q^{2} z^{2}}=P
\]
记
\[
\begin{aligned}
z^{n} & =\frac{p^{n}}{q^{n}} \cos n \varphi \text { 时, } P=\mathfrak{B} \\
z^{n} & =\frac{p^{n}}{q^{n}} \sin n \varphi \text { 时, } P=\mathfrak{p}
\end{aligned}
\]
则
\[
\begin{aligned}
& \mathfrak{B}=\frac{\mathfrak{B} \mathfrak{M}+\mathfrak{n} \mathfrak{N}}{\mathfrak{N}^{2}+\mathfrak{n}^{2}}+\frac{\mathfrak{B} \mathfrak{n}-\mathfrak{p} \mathfrak{N}}{\mathfrak{N}^{2}+\mathfrak{n}^{2}} \cdot \frac{\cos \varphi}{\sin \varphi} \\
& b=-\frac{\mathfrak{B} \mathfrak{n}-\mathfrak{p} \mathfrak{R}}{\mathfrak{N}^{2}+\mathfrak{n}^{2}} \cdot \frac{q}{p \sin \varphi}
\end{aligned}
\]
其次, 令
\[
\frac{P-(\mathfrak{B}+\mathfrak{b} z) Z}{p^{2}-2 p q z \cos \varphi+q^{2} z^{2}}=Q
\]
记
\[
\begin{aligned}
& z^{n}=\frac{p^{n}}{q^{n}} \cos n \varphi \text { 时, } Q=\Im \\
& z^{n}=\frac{p^{n}}{q^{n}} \sin n \varphi \text { 时, } Q=\mathfrak{q}
\end{aligned}
\]
则

\[
\mathcal{c}=-\frac{\mathfrak{Dn}-\mathfrak{qR}}{\mathfrak{R}^2+\mathfrak{n}^2}\cdot\frac{q}{p\sin\varphi}
\]
\[

再次, 令
\[
\frac{Q-(\mathfrak{C}+\mathfrak{c} z)}{p^{2}-2 p q z \cos \varphi+q^{2} z^{2}}=R
\]
记


\[
\begin{aligned}
z^{n} & =\frac{p^{n}}{q^{n}} \cos n \varphi \text { 时, } R=\mathfrak{R} \\
z^{n} & =\frac{p^{n}}{q^{n}} \sin n \varphi \text { 时, } R=\mathfrak{x}
\end{aligned}
\]
则
\[
\begin{aligned}
\mathfrak{S}= & \frac{\mathfrak{R} \mathfrak{N}+\mathfrak{R} \mathfrak{n}}{\mathfrak{R}^{2}+\mathfrak{n}^{2}}+\frac{\mathfrak{R} \mathfrak{n}-\mathfrak{r} \mathfrak{R}}{\mathfrak{N}^{2}+\mathfrak{n}^{2}} \cdot \frac{\cos \varphi}{\sin \varphi} \\
& \mathfrak{b}=-\frac{\mathfrak{R} \mathfrak{n}-\mathfrak{N}}{\mathfrak{N}^{2}+\mathfrak{n}^{2}} \cdot \frac{q}{p \sin \varphi}
\end{aligned}
\]
继续下去, 直至以 $p^{2}-2 p q z \cos \varphi+q^{2} z^{2}$ 为分母的那个部分分式的分子被确定.

例 4 考虑分数函数
\[
\frac{z-z^{3}}{\left(1+z^{2}\right)^{4}\left(1+z^{4}\right)}
\]
记分母的因式 $\left(1+z^{2}\right)^{4}$ 产生的部分分式为
\[
\frac{\mathfrak{A}+\mathfrak{a} z}{\left(1+z^{2}\right)^{4}}+\frac{\mathfrak{B}+\mathfrak{b} z}{\left(1+z^{2}\right)^{3}}+\frac{\mathfrak{S}+\mathfrak{c} z}{\left(1+z^{2}\right)^{2}}+\frac{\mathfrak{D}+\mathfrak{D} z}{1+z^{2}}
\]
与一般形式比较, 得
\[
p=1, q=1, \cos \varphi=0, \varphi=\frac{\pi}{2}
\]
又
\[
M=z-z^{3}, Z=1+z^{4}
\]
从而
\[
\mathfrak{M}=0, \mathfrak{M}=2, \mathfrak{N}=2, \mathfrak{n}=0, \sin \varphi=1
\]
这样我们得到
\[
\mathfrak{U}=-\frac{4}{4} \cdot 0=0, \mathfrak{a}=1
\]
即
\[
\mathfrak{P}+\propto z=z
\]
从而
\[
P=\frac{z-z^{3}-z-z^{5}}{1+z^{2}}=-z^{3}
\]
继而
\[
\mathfrak{P}=0, \mathfrak{p}=1
\]
进而
\[
\mathfrak{B}=0, \mathfrak{b}=\frac{1}{2}
\]
这样, 我们有
\[
\mathfrak{B}+\mathfrak{b} z=\frac{1}{2} z \text { 和 } Q=\frac{-z^{3}-\frac{1}{2} z-\frac{1}{2} z^{5}}{1+z^{2}}=-\frac{1}{2} z-\frac{1}{2} z^{3}
\]
从而
\[
®=0, \mathfrak{q}=0
\]
继而
\[
\mathfrak{S}=, \mathfrak{C}=0
\]
由此得
\[
R=-\frac{\frac{1}{2} z+\frac{1}{2} z^{3}}{1+z^{2}}=-\frac{1}{2} z
\]
从而
\[
\Re=0, r=-\frac{1}{2}
\]
进而
\[
\text { D }=0, \mathfrak{b}=-\frac{1}{4}
\]
得所求部分分式为
\[
\frac{z}{\left(1+z^{2}\right)^{4}}+\frac{z}{2\left(1+z^{2}\right)^{3}}-\frac{z}{4\left(1+z^{3}\right)}
\]
剩下的那个分式的分子为
\[
S=\frac{R-(\mathfrak{D}+\mathfrak{b} z)}{1+z^{2}}=-\frac{1}{4} z+\frac{1}{4} z^{3}
\]
分式为
\[
\frac{-z+z^{3}}{4\left(1+z^{4}\right)}
\]
\section{$\S 210$}

这个方法在得到因式 $\left(p^{2}-2 p q z \cos \varphi+q^{2} z^{2}\right)^{k}$ 产生的部分分式的同时, 还得到了以 $Z$ 为分母的那个分式 (两部分加起来等于题给的函数). 在求
\[
\frac{M}{\left(p^{2}-2 p q \cos \varphi+q^{2} z^{2}\right)^{K} Z}
\]
的由因式 $\left(p^{2}-2 p q \cos \varphi+q^{2} z^{2}\right)^{k}$ 产生的各个部分分式的过程中形成了序列 $P, Q, R, S$, $T, \cdots$. 这序列的最后一项, 就是剩下那个分式的分子, 分母为 $Z$. 如果 $K=1, \frac{P}{Z}$ 就是剩下 那个分式; 如果 $K=2, \frac{Q}{Z}$ 就是剩下那个公式; 如果 $K=3, \frac{R}{Z}$ 就是剩下那个分式, 类推. 得 到了以 $Z$ 为分母的那个分式, 我们要做的是再将它分解成部分分式. 

