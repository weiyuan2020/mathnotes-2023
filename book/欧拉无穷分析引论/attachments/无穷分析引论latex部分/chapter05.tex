\chapter{第五章 多元函数}

\section{$\S 77$}

到现在为止, 我们考察过的都是一个变量的函数. 只要这一个变量确定了, 函数就完 全确定. 现在我们要考察两个或更多个变量的函数. 变量是彼此独立的,一个变量确定 了, 其他变量仍保持为变量. 比如记这种变量为 $x, y, z$, 则 $x, y, z$ 的每一个都可以取任何确 定的值, 彼此之间完全独立. 我们把 $z$ 换成任何一个确定的值, $x, y$ 依旧如 $z$ 代换之前, 是 不确定的. 我们也称函数为因变量, 称函数的变量为自变量. 自变量彼此独立, 自变量都 取了确定的值, 因变量, 也即函数的值才确定.

\section{$\S 78$}

变量 $x, y, z$ 以任何方式所构成的表达式是 $x, y, z$ 的一个二元或更多元函数.

表达式 $x^{3}+x y z+a z^{2}$ 是三个变元 $x, y, z$ 的函数,一个变元, 比如 $z$ 确定了, 也即把 $z$ 换 成了常数, 这个表达式仍然是变量 $x$ 和 $y$ 的函数. 进一步, $y, z$ 都确定了, 那么它仍然还是 $x$ 的函数. 多元函数, 只要有一个变量没确定, 它就仍然是个函数. 任何一个变量的确定方 式都有无穷多种, 所以一个二元函数, 当一个变量用无穷多种方式中的一种确定下来之 后, 另一个变量依然还有无穷多种确定方式. 也即它有无穷多个无穷种确定方式. 对三元 函数就再多一层无穷. 类推下去,每多一个变元就多一层无穷.

\section{$\S 79$}

跟一元函数一样, 多元函数也首先分为代数函数和超越函数.

只包含代数运算的表达式叫代数函数, 含有超越运算的表达式叫超越函数. 这超越 运算涉及的变量, 可以是全体, 可以是几个, 也可以只是一个, 至少一个. 表达式 $z^{2}+$ $y \log z$ 就是 $y$ 和 $z$ 的超越函数, 因为它包含 $\log z$. 如果指定 $z$ 为常数, 它就成了 $y$ 的代数函 数,不再是超越函数了. 对超越函数暂不做进一步的划分. 

\section{$\S 80$}

代数函数分为有理函数和无理函数. 有理函数又分为整函数和分数函数.

跟第一章的区分方法一样, 变量完全不受无理性影响的代数函数叫有理函数. 分母 中不含变量的有理函数叫整函数; 分母中含有变量的有理函数叫分数函数. 两个变量 $x, y$ 的整函数, 其一般形状为
\[
\alpha+\beta y+\gamma z+\delta y^{2}+\varepsilon y z+\zeta z^{2}+\eta y^{3}+\vartheta y^{2} z+\iota y z^{2}+\chi z^{3}+\cdots
\]
二元分数函数的一般形状为 $\frac{P}{Q}, P, Q$ 都是二元整函数. 最后无理函数可以是显式的, 也可 以是隐式的. 显式的, 是借助根号完全解出来的, 隐式的由解不出的方程给出. 方程
\[
V^{5}=\left(a y z+z^{3}\right) V^{2}+\left(y^{4}+z^{4}\right) V+y^{5}+2 a y z^{3}+z^{5}
\]
给出的 $V$ 就是 $y$ 和 $z$ 的隐式无理函数.

\section{$\S 81$}

多元函数也可以是多值的.

有理函数是单值的, 因为变量都确定了的时候, 它只取一个值. 设 $P, Q, R, S, \cdots$ 都是 变量 $x, y, z$ 的单值函数,则满足方程
\[
V^{2}-P V+Q=0
\]
的 $V$ 就是 $x, y, z$ 的二值函数, 因为不管 $x, y, z$ 取什么值, 函数 $V$ 都有两个确定的值. 如果
\[
V^{3}-P V^{2}+Q V-R=0
\]
则 $V$ 是三值函数,如果
\[
V^{4}-P V^{3}+Q V^{2}-R V+S=0
\]
则 $V$ 是四值函数. 类似地我们可以确定更多值的函数.

\section{$\S 82$}

一个变元 $z$ 的函数, 我们令它为零, 得到一个或几个 $z$ 值. 类似地, 两个变元 $y, z$ 的函 数, 我们令它为零, 每个变元就都由另一个决定. 这样就使得本来是独立的两个变元, 每 一个都成了另一个的函数. 同样地, 令三个变元 $x, y, z$ 的函数为零, 可以使每一个变元都 由另外两个决定, 也即使得每一个变元都是另外两个变元的函数. 我们不令函数等于零, 而令它等于某个常数, 甚至等于另外某个函数, 那么由得到的这个方程, 不管它含有几个 变量, 每一个变量就都可以由其余的变量确定, 因而都是其余变量的函数. 变量相同, 方 程不同, 确定的函数不同. 

\section{$\S 83$}

与一元函数不同的一点是, 多元函数可分为齐次函数和非齐次函数.

各项次数都相同的函数叫齐次函数, 相应地, 各项次数不都相同的函数叫非齐次函 数. 单个变元的次数为 1 ;一个变元的乘方或两个不同变元的乘积, 它们的次数都是 2 ; 三 个变元, 不管相同与否, 其乘积的次数都是 3 ; 类推. 常数没有次数. 我们看几个具体的例 子. $\alpha y, \beta z$ 的次数都是 $1 ; \alpha y^{2}, \beta y z, \gamma z^{2}$ 的次数都是 $2 ; \alpha y^{3}, \beta y^{2} z, \gamma y z^{2}, \delta z^{3}$ 的次数都是 3 ; $\alpha y^{4}, \beta y^{3} z, \gamma y^{2} z^{2}, \delta y z^{3}, \varepsilon z^{4}$ 的次数都是 4 .

\section{$\S 84$}

我们先看齐次整函数, 只看二元的, 多元类似.

各项次数都相同的整函数叫齐次整函数. 一到四次的齐次整函数的一般形状依次为
\[
\begin{gathered}
\alpha y+\beta z \\
\alpha y^{2}+\beta y z+\gamma z^{2} \\
\alpha y^{3}+\beta y^{2} z+\gamma y z^{2}+\delta z^{3} \\
\alpha y^{4}+\beta y^{3} z+\gamma y^{2} z^{2}+\delta y z^{3}+\varepsilon z^{4}
\end{gathered}
\]
更高次的类推. 我们把常数的次数看作是零.

\section{$\S 85$}

分数函数是齐次的,指分子分母都是齐次的.

分式 $\frac{a y^{2}+b z^{2}}{\alpha y+\beta z}$ 是 $y, z$ 的齐次分数函数. 分数函数的次数等于分子的次数减去分母的 次数. 我们的这个例子的次数是 1 . 分式 $\frac{y^{5}+z^{5}}{y^{2}+z^{2}}$ 的次数是 3. 如果分子分母的次数相同, 则 分数函数的次数为零. 例如 $\frac{y^{3}+z^{3}}{y^{2} z}, \frac{y}{z}, \frac{\alpha z^{2}}{y^{2}}, \frac{\beta y^{3}}{z^{3}}$ 都是零次的. 如果分母的次数大于分子 的, 那么分数函数的次数是负的. 例如: $\frac{y}{z^{2}}$ 是 - 1 次的; $\frac{y+z}{y^{4}+z^{4}}$ 是 $-3$ 次的; $\frac{1}{y^{5}+a y z^{4}}$ 是 - 5 次的, 这里分子的次数是零. 次数相同的齐次函数相加相减, 结果仍为齐次函数, 次数不变. 例如, 表达式
\[
\alpha y+\frac{\beta z^{2}}{y}+\frac{\gamma y^{4}-\delta z^{4}}{y^{2} z+y z^{2}}
\]
是一次的,而
\[
\alpha+\frac{\beta y}{z}+\frac{\gamma z^{2}}{y^{2}}+\frac{y^{2}+z^{2}}{y^{2}-z^{2}}
\]
是零次的.

\section{$\S 86$}

齐次函数的概念也可以用到无理函数上去. 如果 $P$ 是 $n$ 次齐次函数, 则 $\sqrt{P}$ 的次数为 $\frac{1}{2} n, \sqrt[3]{P}$ 的次数为 $\frac{1}{3} n$. 一般地, $P^{\frac{\mu}{\nu}}$ 的次数为 $\frac{\mu}{\nu} n$. 函数
\[
\sqrt{y^{2}+z^{2}}, \sqrt[3]{y^{9}+z^{9}},\left(y z+z^{2}\right)^{\frac{3}{4}}, \frac{y^{2}+z^{2}}{\sqrt{y^{4}+z^{4}}}
\]
的次数依次为 $1,3, \frac{3}{2}$ 和 0 , 而表达式
\[
\frac{1}{y}+\frac{y \sqrt{y^{2}+z^{2}}}{z^{3}}-\frac{y}{\sqrt[3]{y^{6}-z^{6}}}+\frac{y \sqrt{z}}{z^{2} \sqrt{y}+\sqrt{y^{5}+z^{5}}}
\]
的次数为 $-1$.

\section{$\S 87$}

上节所讲也可用于判断隐式无理函数是否为齐次的. 设 $V$ 由方程
\[
V^{3}+P V^{2}+Q V+R=0
\]
给出, 其中, $P, Q, R$ 都是 $x, y$ 的函数. 首先必须 $P, Q, R$ 全是齐次函数, $V$ 才可能是齐次函 数. 其次, 如果 $V$ 是 $n$ 次函数, 则 $V^{2}$ 的次数是 $2 n, V^{3}$ 的次数是 $3 n$. 再次, 由方程中每项的次 数应该相同, 知 $V$ 的次数为 $n$ 时必须 $P$ 的次数为 $n, Q$ 的为 $2 n, R$ 的为 $3 n$. 反之, 如果 $P, Q$, $R$ 分别是 $n, 2 n$ 和 $3 n$ 次齐次函数, 我们可以断定 $V$ 的次数为 $n$. 例如
\[
V^{5}+\left(y^{4}+z^{4}\right) V^{3}+a y^{8} V-z^{10}=0
\]
所决定的 $V$ 是 $y, z$ 的二次齐次函数.

\section{$\S 88$}

代换 $y=u z$ 变 $y, z$ 的 $n$ 次齐次函数为 $z^{n}$ 与 $u$ 的一个函数的乘积.

代换 $y=u z$ 给每一项增加了 $y$ 的次数那么多个 $z$. 由于每项中 $y, z$ 次数的和为 $n$, 所以 变换后每项中 $z$ 的总次数为 $n$, 即每项都含 $z^{n}$. 因而函数 $V$ 被 $z^{n}$ 除得尽, 商为单个变元 $u$ 的 函数.

这对整函数尤其清楚, 例如
\[
V=\alpha y^{3}+\beta y^{2} z+\gamma y z^{2}+\delta z^{3}
\]
时, 置 $y=u z$, 则
\[
V=z^{3}\left(\alpha u^{3}+\beta u^{2}+\gamma u+\delta\right)
\]
在分数函数情况下,这也是明显的,例如对 $-1$ 次函数 

$V=\frac{\alpha y+\beta z}{y^2+z^2}$

置 $y=u z$, 得
\[
V=z^{-1} \cdot \frac{\alpha u+\beta}{u^{2}+1}
\]
即使对无理函数, 这一规则也照样适用. 例如, 对 $-\frac{3}{2}$ 次函数
\[
V=\frac{y+\sqrt{y^{2}+z^{2}}}{z \sqrt{y^{3}+z^{3}}}
\]
令 $y=u z$, 得
\[
V=z^{-\frac{3}{2}} \cdot \frac{u+\sqrt{u^{2}+1}}{\sqrt{u^{3}+1}}
\]
经过这样的变换, 二元函数变成了一元函数, $z$ 的幂成了对 $u$ 的函数无影响的因式.

\section{$\S 89$}

代换 $y=u z$ 变两个变元 $y, z$ 的零次齐次函数 $V$ 为一元函数.

此时 $V$ 化为 $z$ 的零次幂 $z^{0}=1$ 与 $u$ 的函数的乘积,即 $z$ 在变换后的表达式中消失. 例 如令 $y=u z$,则
\[
V=\frac{y+z}{y-z}
\]
变为
\[
V=\frac{u+1}{u-1}
\]
无理函数
\[
V=\frac{y-\sqrt{y^{2}+z^{2}}}{z}
\]
变为
\[
V=u-\sqrt{u^{2}+1}
\]
\section{$\S 90$}

两个变元 $y, z$ 的齐次整函数, 可以表示成其次数那么多个, 状如 $\alpha y+\beta z$ 的因式的乘 和分

由于函数是齐次的, 所以代换 $y=u z$ 变它为 $z^{n}$ 与 $u$ 的一个整函数的乘积. $u$ 的这个整 函数可以分解成 $\alpha u+\beta$ 状的线性因式的乘积. 由这线性因式的个数与 $z^{n}$ 所含 $z$ 的个数相 等, 可以乘这每个因式以 $z$, 再利用 $u z=y$, 这样对每个因式我们都得到 $\alpha u z+\beta z=\alpha y+\beta z$. 

注意,线性因式可以是实的,也可以是虚的,即 $\alpha, \beta$ 可实可虚.

由此知,二次齐次函数
\[
a y^{2}+b y z+c z^{2}
\]
% 有两个 $\alpha+\beta z$ 状的因式; 函数
有两个 $\alpha y+\beta z$ 状的因式; 函数
\[
a y^{3}+b y^{2} z+c y z^{2}+d z^{3}
\]
% 有三个 $\alpha y+\beta y$ 状的因式. 更高次的齐次整函数类似.
有三个 $\alpha y+\beta z$ 状的因式. 更高次的齐次整函数类似.

\section{$\S 91$}

由上节知,一、二、三次齐次整函数依次可表成
\[
\begin{gathered}
\alpha y+\beta z \\
(\alpha y+\beta z)(\gamma y+\delta z) \\
(\alpha y+\beta z)(\gamma y+\delta z)(\varepsilon y+\zeta z)
\end{gathered}
\]
一般地, 凡二元齐次整函数都可表示成, 其次数那么多个, 状如 $\alpha y+\beta z$ 的线性因式的乘 积. 这线性因式, 可以照一元整函数那样, 用解方程的方法求得. 但三元和更多元的函数, 则不具备这种性质.
\[
a y^{2}+b y z+c y x+d x z+e x^{2}+f z^{2}
\]
分解不成
\[
(\alpha y+\beta z+\gamma x)(\delta y+\varepsilon z+\zeta x)
\]
更多元的函数,一般地,也分解不成线性因式的积.

\section{$\S 92$}

各项次数不都相同的函数叫非齐次函数. 非齐次函数可以按各项次数的种数分类. 我们把含有两种次数的函数, 也即两个不同次数的齐次函数之和称为二齐函数. 例如
\[
y^{5}+2 y^{3} z^{2}+y^{2}+z^{2}
\]
就是一个二齐函数, 它含有 5 和 2 两种次数, 是一个五次与一个二次齐次函数之和. 类似 地, 我们把含有三种次数的函数, 也即三个不同次数的齐次函数之和称为三齐函数.
\[
y^{6}+y^{2} z^{2}+z^{4}+y-z
\]
就是一个三齐函数.

分数函数和无理函数, 例如
\[
\frac{y^{3}+a y z}{b y+z^{2}}, \frac{a+\sqrt{y^{2}+z^{2}}}{y^{2}-b z}
\]
它们拆不成齐次函数的和, 因而不能用各项次数的种数分类.

\section{$\S 93$}

用适当的变量代换可以把有的非齐次函数化成齐次函数. 我们给不出可以做到这一 

点的比较普遍一点的条件, 只限于举几个例子. 例如函数
\[
y^{5}+z^{2} y+y^{3} z+\frac{z^{3}}{y}
\]
代换 $z=x^{2}$ 化它为
\[
y^{5}+x^{4} y+y^{3} x^{2}+\frac{x^{6}}{y}
\]
是 $x, y$ 的五次齐次函数. 又例如函数
\[
y+y^{2} x+y^{3} x^{2}+y^{5} x^{4}+\frac{a}{x}
\]
代换 $x=\frac{1}{z}$ 化它为
\[
y+\frac{y^{2}}{z}+\frac{y^{3}}{z^{2}}+\frac{y^{5}}{z^{4}}+a z
\]
是 $y, z$ 的一次齐次函数. 例子还可举出一些,但代换都不这么简单, 要复杂得多.

\section{$\S 94$}

再一种方法, 是按项的最高次数对整函数进行分类. 例如整函数
\[
x^{2}+y^{2}+z^{2}+a y-a^{2}
\]
项的最高次数为 2 , 是二次整函数; 整函数
\[
y^{4}+y z^{3}-a y^{2} z+a b y z-a^{2} y^{2}+b^{4}
\]
是四次整函数. 这是很重要的一种分类方法,讨论曲线时通常都用它.

\section{$\$ 95$}

整函数还可分为可约和不可约两种. 可以表示成两个或更多个函数之积的整函数称 为可约的. 例如
\[
y^{4}-z^{4}+2 a z^{3}-2 b y z^{2}-a^{2} z^{2}+2 a b z y-b^{2} y^{2}
\]
可分解成
\[
\left(y^{2}+z^{2}-a z+b y\right)\left(y^{2}-z^{2}+a z-b y\right)
\]
是可约整函数. 前面讲了, 二元齐次整函数都是其次数那么多个状如 $\alpha y+\beta z$ 的因式之 积, 因而它们都是可约的. 不能表示成有理因式之积的整函数, 称为不可约的. 易知
\[
y^{2}+z^{2}-a^{2}
\]
没有有理因式, 是不可约的. 从除法的角度看, 有除式的为可约的, 无除式的为不可约的. 

