\chapter{第十二章 曲线的形状}

\section{$\S 272$}

前几章我们根据方程讨论了曲线伸向无穷时的形状. 但是要做到根据方程确定曲线 在有界区域内的形状, 这常常是极困难的. 因为对每个有限横标, 这都要我们从方程求出 对应于它的全体纵标, 并分清为实为虚, 而对高阶方程, 在很多情况下, 这都是分析还做 不到的. 给定横标一个确定的值, 纵标就成了方程的末知数, 因而次数决定着方程求解的 困难程度. 轴和纵标倾角两者的选取, 都可以化简方程, 再有, 横标纵标可以调换, 因而可 以取次数低的作纵标, 这三点都可以使方程的求解变得容易一些.

\section{$\S 273$}

例如, 为讨论第一类三阶线的形状, 我们利用 $\S 259$ 的最简方程, 此时取次数为 2 的 $t$ 为纵标, 取 $u$ 为横标, 这样我们得到方程
\[
y^{2}=\frac{2 b y+a x^{2}+c x+d-n^{2} x^{2}}{x}
\]
它的解为
\[
y=\frac{b \pm \sqrt{b^{2}+d x+c x^{2}+a x^{3}-n^{2} x^{4}}}{x}
\]
其中 $b \neq 0, n \neq 0$.

\section{$\S 274$}

考虑函数 $b^{2}+d x+c x^{2}+a x^{3}-n^{2} x^{4}$, 使它为正的 $x$ 值对应两个 $y$ 值; 使它为零的 $x$ 值 对应一个 $y$ 值, 也即此时纵标 $y$ 的两个值相等; 使它为负的 $x$ 值不对应 $y$ 值, 我们的这函 数从正变到负, 必须经过零 (此时两个 $y$ 值相等). 因而应该对 $b^{2}+d x+c x^{2}+a x^{3}-n^{2} x^{4}$ 为零的情形特别加以注意. 该函数至少在一正一负两点处为零, 在这两点所成区间之外 该函数的值为负, 纵标值为虚数. 

\section{$\S 275$}

假定表达式 $b^2+d x+c x^2+a x^3-n^2 x^4$ 只有两个实因式, 也即只在两处为零, 记这两点为图 49 上的 $P$ 和 $S$, 在这两点处纵标都只有一个值, 在区间 $P S$ 内纵标有两个值, 都是 实的. 在 $P S$ 之外纵标值都是虚的, 也即整个曲线位于纵标 $K k$ 和 $N n$ 之间, 原点 $A$ 处纵标 是曲线的渐近线, 而且曲线与这条纵标相交于一点, 令 $x=0$, 得
\[
\sqrt{b^{2}+d x+c x^{2}+a x^{3}-n^{2} x^{4}}=b+\frac{d x}{2 b}
\]
从而
\[
y=\frac{b \pm\left(b+\frac{d x}{2 b}\right)}{x}
\]
也即或者 $y=\infty$, 或者 $y=-\frac{d}{2 b}$. 因而曲线的形状如图 50 所示.


【图,待补】
%%![](https://cdn.mathpix.com/cropped/2023_02_05_b169e65bf9064e07ce6cg-05.jpg?height=400&width=371&top_left_y=839&top_left_x=418)

图 49


【图,待补】
%%![](https://cdn.mathpix.com/cropped/2023_02_05_b169e65bf9064e07ce6cg-05.jpg?height=435&width=312&top_left_y=826&top_left_x=933)

图 50

\section{$\S 276$}

假定表达式 $b^{2}+d x+c x^{2}+a x^{3}-n^{2} x^{4}$ 有四个相 异的线性因式, 也即它在四个点处为零, 记这四个点 为 $P, Q, R, S$, 则在这四点处纵标都只有一个值. 这样 在图 51 上,对应于轴 $X P$ 部分的纵标为虚, $P Q$ 部分的 为实, $Q R$ 部分的又为虚, $R S$ 部分又为实,最后从 $S$ 向 $Y$ 又为虚, 即曲线由相分离的两部分组成, 一部分在 直线 $K k, L l$ 之间, 另一部分在直线 $M m, N n$ 之间. 原 点 $A$ 处纵标值为实, 因而 $A$ 必定位于 $P Q$ 或 $R S$ 内. 也 即, 在本情况下, 曲线的形状如图 51. 它由相分离的两 部分组成, 另一部分为卵形线, 称它为共轭卵形线.


【图,待补】
%%![](https://cdn.mathpix.com/cropped/2023_02_05_b169e65bf9064e07ce6cg-05.jpg?height=608&width=505&top_left_y=1477&top_left_x=990)

图 51

\section{$\S 277$}

如果两个根相等, 则或 $P$ 与 $Q$, 或 $Q$ 与 $R$, 或 $R$ 与$S$ 重合. 如果 $P, Q$ 重合,那么由 $A$ 在 $P, Q$ 之间, 知两根都为 $x$, 由 $b \neq 0$ 知这不可能; 如果 $R, S$ 重合, 则共轭卵形线成无穷小, 化为共轭点. 如果 $Q, R$ 重合, 则卵形线与另一部分相 接, 得到纽结曲线, 如图 52 所示. 三个根相等, 即点 $Q, R, S$ 重合, 则纽结变为一个尖点, 如 图 53 所示. 这样我们得到第一类的五种不同形状的曲线, 同于牛顿所划分的种数.


【图,待补】
%%![](https://cdn.mathpix.com/cropped/2023_02_05_b169e65bf9064e07ce6cg-06.jpg?height=556&width=391&top_left_y=512&top_left_x=379)

图 52


【图,待补】
%%![](https://cdn.mathpix.com/cropped/2023_02_05_b169e65bf9064e07ce6cg-06.jpg?height=553&width=331&top_left_y=513&top_left_x=933)

图 53

\section{$\S 278$}

牛顿用类似的方法对其他各类曲线也都进行了进一步划分, 因为所有方程的两个坐 标中,都至少有一个,其次数不高于 2 , 当一个坐标的次数为 1 时,曲线的形状很容易确 定. 此时方程的形状为 $y=P, P$ 是坐标 $x$ 的某个有理函数. 如果 $P$ 是分数函数, 则分母在 一处或几处为零时, $y$ 为无穷,曲线有渐近线.

\section{$\S 279$}

设 $y=\frac{P}{Q}$, 则方程 $Q=0$ 的实根给出上述无穷纵标, 对该方程的任何一个根, 比如 $x=$ $f$, 如果取横标 $x=f$, 则 $Q=0, y$ 就成为无穷. 显然, 如果 $x>f$ 时 $y$ 为正, 则 $x<f$ 时 $y$ 为 负,纵标线是 $u=\frac{A}{t}$ 状的渐近线, 单根都是这样. $Q$ 有二重因式, 设为 $(x-f)^{2}$ 时, 那么 $x>f$ 和 $x<f$ 时纵标都是正的, $x=f$ 时得状如 $u^{2}=\frac{A}{t}$ 的渐近线. 如果分母 $Q$ 有三重因 式, 设为 $(x-f)^{3}$, 则横标从小于 $f$ 变到大于 $f$ 时, 纵标变号, 同于第一种情形.

\section{$\S 280$}

接下去状如 $y^{2}=\frac{2 P y-R}{Q}$ 的方程就容易讨论了, 这里 $P, Q, R$ 都是横标 $x$ 的整函数. 对横标的任何一个值都或者有两个纵标值, 或者没有纵标值. $P^{2}>Q R$ 时有两个纵标值, $P^{2}<Q R$ 时没有纵标值. $P^{2}=Q R$ 给出实纵标与虚纵标或者实纵标与零纵标的分界, 此时 $y=\frac{P}{Q}$, 也即纵标线只在一点处与曲线相接或相切, 这样为确定曲线的形状, 就应该考察 方程 $P^{2}-Q R=0$. 它的实根给出纵标与曲线在一点处相切的点. 我们在轴上标出这些点. 如果根都是单根, 则这些点将横标分成若干部分, 不同部分所对应的纵标虚实交替. 这 样,曲线由相分离的几部分组成, 部分数等于交替次数. 共轭卵形线即源于此.

\section{$\S 281$}

如果方程 $P^{2}-Q R=0$ 的两个根相等,那么前面说的轴上的点中的两个重合. 这时纵 标为虚或为实的一个区间消失. 为虚时曲线成为纽结曲线, 如图 52 所示, 为实时, 共轭卵 形线变为共轭点. 如果有三个相等的根, 那么纽结成无穷小, 变为尖点, 如图 53 所示. 如 果方程有四个相等的根, 那么或者分离的两个卵形线聚为一点, 或者尖点与纽结点合一, 或者两个尖点反向连在一起. 如果有五个相等的根, 那几乎得不到新的形状. 这时与尖点 重合的不是一个纽结点, 而是两个. 甚至有更多个相等的根, 也不构成新的形状.

\section{$\S 282$}

结点或曲线两个分支的交点也叫二重点, 因为应该视直线在这种点处与曲线的交点 为两个. 如果还有另外一个分支通过这种点, 则它为曲线的三重点. 两个二重点相合, 当 然构成四重点. 由此可以推知任何重数的点的产生和性质. 聚为一点的卵形线 (亦称共轭 点)是二重点,共轭点连上曲线其余部分所成尖点也是二重点.

\section{$\S 283$}

如果用横标 $x$ 表示纵标 $y$ 的方程是三次或更高次的, 那么这 $y$ 就是 $x$ 的多值函数,每 个 $x$ 值所对应的 $y$ 值的个数, 都或者等于 $y$ 的次数, 或者比 $y$ 的次数少 $2,4,6$ 等. 纵标必定 两个同时变虚, 变虚之前这两个值相等. 由虚变实的方式有多种, 但它们都或者是已经讲 过的情形, 或者是讲过情形的组合. 对有正有负的很多横标值, 知道了它们对应的全体纵 标, 那么借助知道的这些点, 就可以画出曲线, 确定其形状.

\section{$\S 284$}

我们用一个例子来解释前节所讲, 这方程次数高, 但纵标 $y$ 可用二次方根表示, 设
\[
2 y=\pm \sqrt{6 x-x^{2}} \pm \sqrt{6 x+x^{2}} \pm \sqrt{36-x^{2}}
\]
从该方程, 每个横标对应 8 个纵标. 显然, $x<0$ 和 $x>6$ 时 $y$ 为虚数. 由此知整个曲线在$x=0$ 和 $x=6$ 之间. 依次令 $x$ 等于 $0,1,2,3,4,5,6$, 得

\begin{tabular}{c|c|c|c|c|c|c|c}
\hline & $x=0$ & $x=1$ & $x=2$ & $x=3$ & $x=4$ & $x=5$ & $x=6$ \\
\hline$\sqrt{6 x-x^{2}}$ & $0.000$ & $2.236$ & $2.828$ & $3.000$ & $2.828$ & $2.236$ & $0.000$ \\
$\sqrt{6 x+x^{2}}$ & $0.000$ & $2.646$ & $4.000$ & $5.196$ & $6.325$ & $7.416$ & $8.485$ \\
$\sqrt{36-x^{2}}$ & $6.000$ & $5.916$ & $5.657$ & $5.196$ & $4.472$ & $3.317$ & $0.000$ \\
\hline 和 & $6.000$ & $10.798$ & $12.485$ & $13.392$ & $13.625$ & $12.969$ & $8.485$ \\
\hline & & & & & & & \\
\hline 符号 & & & & & & & \\
\hline$+++$ & $3.000$ & $5.399$ & $6.242$ & $6.696$ & $6.812$ & $6.484$ & $4.242$ \\
$-++$ & $3.000$ & $3.163$ & $3.414$ & $3.696$ & $3.984$ & $4.248$ & $4.242$ \\
$+--$ & $3.000$ & $2.754$ & $2.242$ & $1.500$ & $0.487$ & $0.932$ & $-4.242$ \\
$++-$ & $-3.000$ & $-0.517$ & $0.586$ & $1.500$ & $2.341$ & $3.167$ & $4.242$ \\
\hline
\end{tabular}

没列出的四种符号组合, 它们对应的 $y$ 值是列出值的反号 数. 每个横标对应 8 个纵标, 图 54 上是画出的曲线, 由 $A F B E c a g b c D A$ 和 $a f b E C A G B C D a c$ 两部分交织而成, 有两个 尖点 $A$ 和 $a$, 有四个二重点, 也即有四个分支相交的点, 它们是 $D, E, C$ 和 $c$.


【图,待补】
%%![](https://cdn.mathpix.com/cropped/2023_02_05_b169e65bf9064e07ce6cg-08.jpg?height=630&width=347&top_left_y=1035&top_left_x=1129)

图 54 

