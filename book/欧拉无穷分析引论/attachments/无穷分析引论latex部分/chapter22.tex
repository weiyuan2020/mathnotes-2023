\chapter{第四章各阶线的基本性质}

\section{$\S 66$}

对任何阶数的线, 它与直线的相交, 也即与直线交点的个数, 都是我们首先要考察 的. 一阶线, 也即直线, 与另一条直线的交点只有一个, 但曲线与直线的交点可多于一个. 自然我们会问, 每阶曲线的这种交点各是几个, 这个问题的解决可以帮助了解各阶线的 性质. 我们将看到, 各阶线与直线交点的个数是: 二阶线不多于两个, 三阶线不多于三个, 类推.

\section{$\S 67$}

前面我们讲过轴与任何曲线交点个数的求法. 给定横标 $x$ 纵标 $y$ 间的一个方程, 由 于轴与曲线交点处纵标 $y$ 为零, 置方程中的 $y=0$, 得到只含 $x$ 的方程, 这个方程的根就是 轴与曲线交点的横标. 例如, 令我们前面求得的圆的方程 $y^{2}=2 a x-x^{2}$ 中的 $y=0$, 得 $0=$ $2 a x-x^{2}$, 根为 $x=0$ 和 $x=2 a$, 即轴 $R S$ 与圆先交于原点 $A$, 再交于点 $B, A B=2 a$. 对其他 曲线, 求法类似, 也即令方程中的 $y=0$, 得到 $x$ 的方程, 这个方程的根就是曲线与轴的 交点.

\section{$\S 68$}

对于曲线的通用方程, 每条直线都可以作轴, 取一条直线作轴. 令通用方程中的纵标 $y=0$, 得到只含 $x$ 的方程, 这个方程的根就是曲线与轴, 也就是与我们所取的直线的交 点, 根的个数就是这曲线与直线交点的个数. 可见, 交点个数决定于方程中 $x$ 的最高次 幂, 不能超过 $x$ 的最高次幂的次数, 如果根都是实数, 则交点个数等于最高幂的次数; 如 果有虚根,则交点个数相应的减少.

\section{$\S 69$}

对每阶线我们都给了最通用方程, 对它应用所说的方法, 就得到各阶线与直线交点 的个数. 先取一阶线, 即直线的通用方程
\[
0=\alpha+\beta x+\gamma y
\]
令 $y=0$, 得 $0=\alpha+\beta x$, 它的根不多于一个, 从而一条直线与另一条直线最多只能有一个交点. 如果 $\beta=0$, 则不可能等式 $0=\alpha$ 告诉我们, 此时直线与轴不相交, 事实上 $\beta=0$ 时方程 为 $0=\alpha+\gamma y$, 此时直线与轴平行.

\section{$\S 70$}

令二阶线通用方程
\[
0=\alpha+\beta x+\gamma y+\delta x^{2}+\varepsilon x y+\zeta y^{2}
\]
中的 $y=0$, 得
\[
0=\alpha+\beta x+\delta x^{2}
\]
它或者有两个实根, 或者没有实根, 或者有一个实根, 这是在 $\delta=0$ 时. 因而二阶线与直线 交点的个数或为 2 , 或为 1 , 或者为 0 . 也可简单地说成: 二阶线与直线交点的个数不多 于 2 .

\section{$\S 71$}

置三阶线通用方程中的 $y=0$, 得方程
\[
0=\alpha+\beta x+\gamma x^{2}+\delta x^{3}
\]
该方程根的个数不多于 3 , 因而三阶线与直线交点的个数不多于 3 . 但这交点个数可以少 于 3 : 如果 $\delta=0$, 且 $0=\alpha+\beta x+\gamma x^{2}$ 的两个根都为实数, 则交点个数为 2 ; 如果所得三次方 程有两个虚根, 或 $\delta=0, \gamma=0$, 则交点个数为 1 ; 最后, 如果 $\delta=0$, 且另外两个根为虚数, 或 者 $\beta, \gamma, \delta$ 都为零, 而 $\alpha$ 不为零, 则交点个数为零.

\section{$\S 72$}

类似地, 我们可以得到, 四阶线与直线交点的个数不多于 4 , 且这个性质可类推到各 阶线, 即 $n$ 阶线与直线交点的个数不多于 $n$. 当然, 也跟我们对二阶线和三阶线所指出的 一样, 这交点个数可以少于 $n$, 甚至可以为零. 这样我们得到: 线与另一直线交点的个数, 不多于它的阶数.

\section{$\S 73$}

可见, 从曲线与直线交点的个数, 得不到曲线的阶数. 交点个数为 $n$ 时, 曲线的阶数 可以不是 $n$, 可以比 $n$ 高, 甚至可以不是代数曲线, 而是超越曲线, 但可以断定, 与直线交 点个数为 $n$ 的曲线, 其阶数绝对不会小于 $n$. 如果曲线与直线交点个数为 4 , 则可断定, 这 曲线不是二阶的, 也不是三阶的, 但断不定它是四阶的, 还是更高的哪一阶的, 甚至断不 定它是不是超越曲线. 

\section{$\S 74$}

各阶线的通用方程中都含有若干个任意常数, 取任意常数为确定的值, 曲线就完全 确定, 并且可以对给定的轴把它画出来, 而包含在这通用方程中的其他曲线就一概被排 除. 例如, 虽然一阶方程 $0=\alpha+\beta x+\gamma y$ 只含直线, 但由于常数 $\alpha, \beta, \gamma$ 不同值的组数无穷, 因而这直线关于轴的位置的种数也无穷. 但是, 只赋予这三个常数以确定的值, 直线的位 置就完全确定, 另外的任何直线就都排除在外.

\section{$\S 75$}

方程 $0=\alpha+\beta x+\gamma y$ 有三个可任意决定的常数(简称为三个任意定), 但由方程的性 质知, 给出两个系数对第三个的比, 方程就完全确定, 也即我们的这个方程, 实质上只有 两个任意定. 例如, 用 $\alpha$ 确定 $\beta, \gamma$, 使 $\beta=-\alpha, \gamma=2 \alpha$, 那么约去方程 $0=\alpha-\alpha x+2 \alpha y$ 中的 $\alpha$, 我们就得到完全确定的方程. 同样的道理, 含六个任意常数的二阶线通用方程有五个任 意定; 三阶线通用方程有九个任意定;一般地, $n$ 阶线通用方程有 $\frac{(n+1)(n+2)}{2}-1$ 个任 意定.

\section{$\S 76$}

让曲线通过一个给定点, 可确定下一个任意定, 设给定某阶线的通用方程, 我们让曲 线通过给定点 $B$ 来确定一个任意定 (图 17). 任意取定一条轴和轴上原点 $A$,自点 $B$ 向轴 引垂线 $B b$. 曲线过 $B$, 则方程中横标 $x$ 为 $A b$ 时, 纵标 $y$ 必为 $B b$. 换通用方程中的 $x, y$ 为 $A b, B b$, 从结果等式中确定出 $\alpha, \beta, \gamma, \delta, \cdots$ 中的一个, 把确定出的这一个换人原通用方程, 则任意定减少一个, 但所含曲线都过点 $B$.


【图,待补】
%%![](https://cdn.mathpix.com/cropped/2023_02_05_00a02f82302d074ee0d6g-07.jpg?height=378&width=481&top_left_y=1666&top_left_x=607)

图 17 

\section{$\S 77$}

如果还要求曲线通过点 $C$, 那我们从点 $C$ 向轴引垂线 $C c$, 将 $x=A c, y=C c$ 代入所含 曲线都过点 $B$ 的方程, 从得到的等式又可确定 $\alpha, \beta, \gamma, \cdots$ 中的一个. 可见, 从曲线应通过 的三个点 $B, C, D$ 可确定三个常数, 从四个点 $B, C, D, E$ 可确定四个常数. 如果点数等于 通用方程的任意定数,那么曲线就完全而唯一地确定.

\section{$\S 78$}

一阶线, 即直线的通用方程只有两个任意定, 所以给定两个点要直线通过, 直线就完 全确定. 事实上, 过两点只有一条直线, 这是我们从欧几里得的 《几何原本》中已经知道 了的,如果只给一个点,则方程不完全确定, 通过这个点的直线有无穷多条.

\section{$\S 79$}

二阶线的通用方程有五个任意定, 如果给定五个点, 使曲线必须通过, 那么二阶线就 完全确定, 因而通过五个点只能引一条二阶线. 如果只给四个点, 或更少, 那么方程就不 完全确定, 就有无穷多条二阶线通过给定的这些点, 二阶线与直线不能有三个交点, 因此 给定的五个点中, 如果有三个在一条直线上, 我们求出的就不能是连续曲线, 而是由两条 直线组成的复合线, 即二阶通用方程此时包含的是两条直线.

\section{$\S 80$}

三阶线的通用方程含九个任意定, 所以对任意取定的九个点都有一条三阶线通过, 而且只有一条. 如果点数少于 9 , 则有无穷多条三阶线通过它们. 同样地,每 14 个给定点 都有唯一的四阶线通过, 每 20 个给定点都有唯一的五阶线通过. 一般地,每
\[
\frac{(n+1)(n+2)}{2}-1=\frac{n(n+3)}{2}
\]
个给定点都有唯一的一条 $n$ 阶线通过,每少于这么多个点都有无穷多条 $n$ 阶线通过.

\section{$\S 81$}

通过不多于 $\frac{n(n+3)}{2}$ 个点有一条或无穷多条 $n$ 阶线. 点数等于 $\frac{n(n+3)}{2}$ 时有一条, 少于时有无穷多条. 不管给定的点如何分布, 解都是存在的, 因为确定系数 $\alpha, \beta, \gamma, \delta, \cdots$ 的 方程都是线性的, 没有二次的, 也没有更高次的. 也因此, 求得的 $\alpha, \beta, \gamma, \cdots$ 不会是虚的, 也不会是多值的, 即求得的通过给定点的线都是实线, 而且只要点数等于通用方程的任 意定个数, 线就是唯一的.

\section{$\S 82$}

轴可任取, 这可以使系数的确定变得容易些, 先通过一个给定点, 并且就取这个给定 点作原点, 这时 $x=0, y=0$ 代入通用方程
\[
0=\alpha+\beta x+\gamma y+\delta x^{2}+\varepsilon x y+\zeta y^{2}+p x^{3}+\cdots
\]
立即得到 $\alpha=0$. 再通过另外的给定点, 这样就减少了用点的位置来确定的量的个数. 最后 可取斜角坐标, 使过原点的纵标线通过一个给定点. 不管坐标是直角的还是斜角的, 我们 都可以从方程了解曲线, 根据方程画出曲线.

\section{$\S 83$}

我们来求通过图 18 上五个给定点 $A, B, C, D, E$ 的二阶曲线. 取过点 $A, B$ 的直线为 轴, 取点 $A$ 为原点, 连接点 $A, C$, 取纵标对轴的倾角为 $\angle C A B$. 从点 $D, E$ 向轴引平行于 $A C$ 的纵标线 $D d$ 和 $E e$, 令 $A B=a, A C=b, A d=c, D d=d, A e=e, e E=f$, 对二阶线的通用 方程
\[
0=\alpha+\beta x+\gamma y+\delta x^{2}+\varepsilon x y+\zeta y^{2}
\]
显然

\begin{tabular}{c|c}
\hline 令 & 则 \\
\hline$x=0$ & $y=0$ \\
\hline$x=0$ & $y=b$ \\
\hline$x=a$ & $y=0$ \\
\hline$x=c$ & $y=d$ \\
\hline$x=e$ & $y=f$ \\
\hline
\end{tabular}

由此我们得到 5 个方程:
\[
\begin{aligned}
& \text { I. } 0=\alpha . \\
& \text { II . } 0=\alpha+\gamma b+\gamma b^{2} . \\
& \text { II. } 0=\alpha+\beta a+\delta a^{2} . \\
& \text { IV . } 0=\alpha+\beta c+\gamma d+\delta c^{2}+\varepsilon c d+\zeta d^{2} . \\
& \text { V . } 0=\alpha+\beta e+\gamma f+\delta e^{2}+\varepsilon e f+\zeta f^{2} .
\end{aligned}
\]
从而 $\alpha=0, \gamma=-\zeta b, \beta=-\delta a$. 将这三个值代入 $\mathbb{N}, V$, 得
\[
\begin{aligned}
& 0=-\delta a c-\zeta b d+\delta c^{2}+\varepsilon c d+\zeta d^{2} \\
& 0=-\delta a e-\zeta b f+\delta e^{2}+\varepsilon e f+\zeta f^{2}
\end{aligned}
\]
用 $e f$ 乘前一个,用 $c d$ 乘后一个, 然后相减消去 $\varepsilon$, 得
\[
0=-\text {-acef }-\zeta b d e f+\delta c^{2} e f+\zeta d^{2} e f+\delta a c d e+\zeta b c d f-\delta c d e^{2}-\zeta c d f^{2}
\]
或
\[
\frac{\delta}{\zeta}=\frac{b d e f-b c d f-d^{2} e f+c d f^{2}}{a c d e-a c e f-c d e^{2}+c^{2} e f}
\]
由此得
\[
\begin{aligned}
& \delta=d f(b e-b c-d e+c f) \\
& \zeta=c e(a d-a f-b e+c f)
\end{aligned}
\]
我们求出了所有的系数.


【图,待补】
%%![](https://cdn.mathpix.com/cropped/2023_02_05_00a02f82302d074ee0d6g-10.jpg?height=376&width=546&top_left_y=645&top_left_x=565)

图 18

\section{$\S 84$}

用上面的方法,对取定的轴和坐标角,求出通用方程
\[
0=\alpha+\beta x+\gamma y+\delta x^{2}+\cdots
\]
的全体系数, 我们就得到了通过包括给定点在内的无穷多个点的一条曲线的方程. 如果 给定点的个数少于通用方程任意定的个数, 则不足的点可任取, 都取定之后, 通过给定点 的方程就完全决定. 为了画出决定了的这条曲线, 我们取正负整数横标值 $0,1,2,3,4,5$, $6, \cdots$ 和 $-1,-2,-3,-4, \cdots$, 再根据完全决定了的方程算出对应于这每一个横标的纵 标, 这样我们就得到了曲线的很多相当靠近的点, 这就可以大致地画出曲线. 

