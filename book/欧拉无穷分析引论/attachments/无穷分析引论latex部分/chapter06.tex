\chapter{第六章 指数和对数}

\section{$\S 96$}

超越函数属积分学范畴, 但有几种却可以,也应该放在积分学前面讲, 它们用得最 多, 也为进一步学习所必需. 先考虑指数函数, 即指数是变数, 这样的幂. 显然, 这种幂不 是代数函数, 代数函数中指数是常数. 指数函数, 可以指数为变数, 底为常数, 如 $a^{z}$; 也可 以底和指数都是变数, 如 $y^{z}$; 指数本身也可以是指数函数, 如 $a^{a^{2}}, a^{y^{2}}, y^{a^{2}}, x^{y^{2}}$. 我们只讲指 数为变数底为常数一种, 即 $a^{z}$. 明白了这一种, 别的也就清楚了.

\section{$\S 97$}

考虑指数函数 $a^{z}, a$ 为常数, 指数 $z$ 为变数, 因而可以是任何确定的数. 先让 $z$ 依次取 正整数, 得 $a^{1}, a^{2}, a^{3}, a^{4}, a^{5}, a^{6}, \cdots ;$ 再让 $z$ 依次取负整数 $-1,-2,-3, \cdots$, 得 $\frac{1}{a}, \frac{1}{a^{2}}, \frac{1}{a^{3}}$, $\frac{1}{a^{4}}, \cdots ; z=0$, 得 $a^{0}=1 ;$ 让 $z$ 为分数,例如 $\frac{1}{2}, \frac{1}{3}, \frac{2}{3}, \frac{1}{4}, \frac{3}{4}, \cdots$, 得
\[
\sqrt{a}, \sqrt[3]{a}, \sqrt[3]{a^{2}}, \sqrt[4]{a}, \sqrt[4]{a^{3}}, \cdots
\]
这是方根, 方根可以不止一个值, 但我们视 $a^{z}$ 为单值函数, 因而只考虑其正的实数值. 例 如 $a^{\frac{5}{2}}$, 它既等于 $-a^{2} \sqrt{a}$, 又等于 $+a^{2} \sqrt{a}$, 但我们只取后者, 即认为它在 $a^{2}$ 与 $a^{3}$ 之间. $z$ 为 无理数时, 考虑方式类似, 但理解起来要困难些. 例如 $a^{\sqrt{ }}$ 的值在 $a^{2}$ 与 $a^{3}$ 之间. 对 $z$ 的虚数 值我们不考虑.

\section{$\S 98$}

指数函数 $a^{z}$ 的值依赖于常数 $a . a=1$, 对任何的 $z$ 值都有 $a^{z}=1 . a>1: a^{z}$ 的值随 $z$ 的 增大而增大, 且 $z=\infty$ 时, $a^{z}$ 也趋向无穷; $z=0$, 则 $a^{z}=1 ; z<0$ 时, $a^{z}$ 的值小于 1 , 且 $z=-\infty$ 时, $a^{z}=0.0<a<1$ : 对大于零的 $z, a^{z}$ 的值随 $z$ 的增大而减小; 对小于零的 $z, a^{z}$ 的值随 $z$ 的增大而增大. $a<1$, 则 $\frac{1}{a}>1$. 记 $\frac{1}{a}=b$, 则 $a^{z}=b^{-z}$. 因而 $a<1$ 的情形可由 $a>1$ 的情形推出.

\section{$\$ 99$}

$a=0$, 则 $a^{z}$ 的值是跳跃式的: $z$ 为正数, 即 $z$ 大于零时恒有 $a^{z}=0 ; z=0$ 时 $a^{0}=1 ; z$ 为 负数时, $a^{z}$ 为无穷大, 例如 $z=-3$, 则 $a^{z}=0^{-3}=\frac{1}{0^{3}}=\frac{1}{0}$, 是无穷大. 也即 $a=0$, 则 $a^{z}$ 的值 从 0 跳到 1 , 再从 1 跳到无穷大.

$a$ 取负值, 则 $a^{z}$ 的跳跃更频, 例如 $a$ 取 $-2: z$ 依次取整数时, $a^{z}$ 的值正负交替. 此时
\[
a^{-4}, a^{-3}, a^{-2}, a^{-1}, a^{0}, a^{1}, a^{2}, a^{3}, a^{4}, \cdots
\]
的值为
\[
+\frac{1}{16},-\frac{1}{8},+\frac{1}{4},-\frac{1}{2}, 1,-2,+4,-8,+16, \cdots
\]
$z$ 取分数值时, $z^{z}=(-2)^{z}$ 时实时虚, 例如 $a^{\frac{1}{2}}=\sqrt{-2}$ 是虚数,而
\[
a^{\frac{1}{3}}=\sqrt[3]{-2}=-\sqrt[3]{2}
\]
是实数; 如果指数 $z$ 取无理数, 则 $a^{z}$ 可能为实数可能为虚数, 何时为实何时为虚, 事先不 能确定.

\section{$\S 100$}

上节我们看到, $a$ 为负值时, $a^{z}$ 的值或正或负, 或实或虚, 不定. 又 $a$ 在 0,1 之间的情 形可化为 $a>1$ 的情形. 所以我们取 $a$ 为正数, 为大于 1 的数. 令 $y=a^{z}$, 让 $z$ 取 $-\infty$ 到 $+\infty$ 的所有实数, 则 $y$ 取 0 到 $+\infty$ 的所有正实数. $z=+\infty$, 则 $y=+\infty ; z=0$, 则 $y=1 ; z=-\infty$, 则 $y=0$. 反之, 对 $y$ 的任何一个正值, 都有 $z$ 的一个实数值与之对应, 使得 $a^{z}=y$. 对 $y$ 的负 值,则没有实的 $z$ 值与之对应.

\section{$\$ 101$}

这样, 记 $y=a^{z}$, 则 $y$ 是 $z$ 的函数, 即每一个 $z$ 值都确定一个 $y$ 值. 从指数的性质我们可 以看到 $y$ 对 $z$ 的依赖方式. 例如, $y^{2}=a^{2 z}, y^{3}=a^{3 z}$, 一般地 $y^{n}=a^{n z}$. 由此得 $\sqrt{y}=a^{\frac{1}{2^{z}}}, \sqrt[3]{y}=a^{\frac{1}{3 z}}$, $\frac{1}{y}=a^{-z}, \frac{1}{y^{2}}=a^{-2 z}, \frac{1}{\sqrt{y}}=a^{-\frac{1}{2} z}$, 等等. 进一步,若 $v=a^{x}$, 则
\[
v y=a^{x+z}, \frac{v}{y}=a^{x-z}
\]
根据上述性质, 每给定一个 $z$ 值都可确定一个 $y$ 值.

取 $a=10$, 那么 $z$ 为整数时, 我们能直接写出 $y$, 例如
\[
10^{1}=10,10^{2}=100,10^{3}=1000,10^{4}=10000,10^{0}=1
\]
\[
10^{-1}=\frac{1}{10}=0.1,10^{-2}=\frac{1}{100}=0.01,10^{-3}=\frac{1}{1000}=0.001
\]
$z$ 取分数值时, 可用求根的方法得到 $y$ 值. 例如 $10^{\frac{1}{2}}=\sqrt{10}=3.162277$ 等.

\section{$\S 102$}

前面我们看到, 给定了数 $a$, 那么对每一个 $z$ 值, 我们都能求得一个 $y$ 值, 使得 $y=a^{z}$. 现在倒过来, 对每一个 $y$ 值, 要我们求出一个 $z$ 值, 使得 $a^{z}=y$, 也即把 $z$ 看成是 $y$ 的函数. 此时称 $z$ 为 $y$ 的对数. 讨论对数时, 我们都假定 $a$ 是一个固定的数, 称它为底, 有了底, 我 们称等式 $a^{z}=y$ 中幂 $a^{z}$ 的指数 $z$ 为 $y$ 的对数. 通常记 $y$ 的对数为 $\log y$. 如果
\[
a^{z}=y \text {, 则 } z=\log y
\]
由此我们知道, 对数的底虽然由我们指定, 但是它应该大于 1 . 还有, 只有正数的对数是 实数.

\section{$\$ 103$}

1 的对数为 0 , 即不管取什么数为底, 我们都有 $\log 1=0$, 这是因为在决定 $z=\log y$ 的 方程 $a^{z}=y$ 中, $y=1$ 时恒有 $z=0$.

大于 1 的数的对数为正. 例如
\[
\log a=1, \log a^{2}=2, \log a^{3}=3, \log a^{4}=4, \cdots
\]
可以推出底是什么, 是其对数为 1 的那个数. 小于 1 的正数的对数为负. 例如
\[
\log \frac{1}{a}=-1, \log \frac{1}{a^{2}}=-2, \log \frac{1}{a^{3}}=-3, \cdots
\]
负数的对数, 不是实数, 而是虚数. 这我们前面指出过.

\section{$\S 104$}

类似地, 如果 $\log y=z$, 则 $\log y^{2}=2 z, \log y^{3}=3 z$, 一般地 $\log y^{n}=n z$, 将 $z=\log y$ 代入, 得
\[
\log y^{n}=n \log y
\]
即 $y$ 的幂的对数等于指数乘上 $y$ 的对数. 例如
\[
\log \sqrt{y}=\frac{1}{2} \log y, \log \frac{1}{\sqrt{y}}=\log y^{-\frac{1}{2}}=-\frac{1}{2} \log y
\]
等. 可见知道了一个数的对数, 我们就可以求出它任何一个幂的对数. 如果知道了两个数 的对数. 例如
\[
\log y=z, \log v=x
\]
那么由 $a^{z}=y$ 和 $a^{x}=v$, 我们得到 

\[
\log (v y)=x+z=\log v+\log y
\]

即两数积的对数等于两数对数的和. 类似地, 我们有
\[
\log \frac{y}{v}=z-x=\log y-\log x
\]
即商的对数等于分子的对数减去分母的对数. 根据已知的几个数的对数, 利用上述规则, 可以求出很多数的对数.

\section{$\S 105$}

从前面讲的我们看到, 一个数, 如果不是底的幂, 它的对数就不能是有理数. 也即, 如 果 $b$ 不是底 $a$ 的幂, 则 $b$ 的对数就不能是有理数. 在 $b$ 不是底 $a$ 的幂时, $b$ 的对数也不能是 无理数. 假定 $\log b=\sqrt{n}$, 即 $a^{\sqrt{n}}=b$, 但在 $a$ 和 $b$ 都是有理数时, 这是不可能的. 我们首先要 知道的是有理数和整数的对数, 分数和无理数的对数可从有理数和整数的对数得到. 有 理数和无理数之外的是超越数, 因而不是底的幂的数, 其对数是超越数. 即对数是超越 量.

\section{$\S 106$}

当对数是超越数时, 我们只能用小数近似地表示它. 这小数的位数取得越多, 近似程 度就越好. 下面是一种用计算平方根来求对数近似值的方法. 我们知道, 若
\[
\log y=z, \log v=x \text {, 则 } \log \sqrt{v y}=\frac{x+z}{2}
\]
现在我们来求数 $b$ 的对数的近似值. 假定 $b$ 在 $a^{2}$ 和 $a^{3}$ 之间, 这两个数的对数分别为 2 和 3. 我们先求出 $a^{2}, a^{3}$ 的几何平均数 $a^{\frac{5}{2}}$ 或 $a^{2} \sqrt{a}$. 这时 $b$ 必定或者在 $a^{2}$ 与 $a^{2} \sqrt{a}$ 之间, 或者在 $a^{2} \sqrt{a}$ 与 $a^{3}$ 之间. 在哪两个数之间, 我们就再求哪两个数的几何平均数. 这样, 我们 就把 $b$ 所在的区间进一步缩小. 重复下去, $b$ 所在的区间就越来越小. 最终就可以得到 $b$ 的 具有我们所要的那么多位小数的近似值. 由于区间端点的对数都计算出来了, 所以 $b$ 的 对数也就有了.

例 1 设对数的底 $a=10$, 即取常用对数, 我们来求 5 的对数的近似值. 5 在 1 与 10 之 间, 1 和 10 的对数分别为 0 和 1 . 下面我们逐次地求平方根, 直至达到数 5 .
\[
\begin{gathered}
A=1.000000, \log A=0.0000000 \\
B=10.000000, \log B=1.0000000, C=\sqrt{A B} \\
C=3.162277, \log C=0.5000000, D=\sqrt{B C} \\
D=5.623413, \log D=0.7500000, E=\sqrt{C D} \\
E=4.216964, \log E=0.6250000, F=\sqrt{D E} \\
F=4.869674, \log F=0.6875000, G=\sqrt{D F}
\end{gathered}
\]
%%04p061-080
\[
\begin{aligned}
& G=5.232991, \log G=0.7187500, H=\sqrt{F G} \\
& H=5.048065, \log H=0.7031250, J=\sqrt{F H} \\
& J=4.958069, \log J=0.6953125, K=\sqrt{H J} \\
& K=5.002865, \log K=0.6992187, L=\sqrt{J K} \\
& L= 4.980416, \log L=0.6972656, M=\sqrt{K L} \\
& M= 4.991627, \log M=0.6982421, N=\sqrt{K M} \\
& N= 4.997242, \log N=0.6987304, O=\sqrt{K N} \\
& O= 5.000052, \log O=0.6989745, P=\sqrt{N O} \\
& P= 4.998647, \log P=0.6988525, Q=\sqrt{O P} \\
& Q= 4.999350, \log Q=0.6989135, R=\sqrt{O Q} \\
& R= 4.999701, \log R=0.6989440, S=\sqrt{O R} \\
& S= 4.999876, \log S=0.6989592, T=\sqrt{O S} \\
& T= 4.999963, \log T=0.6989668, V=\sqrt{O T} \\
& V= 5.000008, \log V=0.6989707, W=\sqrt{T V} \\
& W= 4.999984, \log W=0.6989687, X=\sqrt{V W} \\
& X= 4.999997, \log X=0.6989697, Y=\sqrt{V X} \\
& Y= 5.000003, \log Y=0.6989702, Z=\sqrt{X Y} \\
& Z=5.000000, \log Z=0.6989700
\end{aligned}
\]
最后, 这几何平均收敛于 $Z=5.000000$. 从而得到底为 10 时, 5 的对数为
\[
0. 6989700
\]
因而近似地有
\[
10^{\frac{69897}{100000}}=5
\]
Briggs 和 Vlasc 制造数学史上第一本对数表时,用的就是这种方法. 当然后来人们找到了 几种更简捷的方法.

\section{$\S 107$}

一种底决定一种对数, 有多少种底, 就有多少种对数. 底的个数无穷, 因而对数的种 数也无穷. 但每两种对数的比都为常数. 设两种对数的底为 $a$ 和 $b$, 又设数 $n$ 以 $a$ 为底的对 数为 $p$, 以 $b$ 为底的对数为 $q$, 即 $a^{p}=n, b^{q}=n$. 从而 $a^{p}=b^{q}$, 进而 $a=b^{p}$. 即两种对数的比 $\frac{q}{p}$ 是一个与 $n$ 无关的常数. 这样, 有了一个数的一种对数, 利用这个常数, 即可算出它的 另一种对数. 这是一条“黄金” 规则. 知道了所有数的一种对数,利用这条“黄金” 规则, 就可以算出所有数的另一种对数. 例如有了以 10 为底的对数, 我们就可以算出以任何别的数为底的对数. 设这别的数为 2 , 记数 $n$ 以 10 为底的对数为 $p$, 以 2 为底的对数为 $q$. 由 以 10 为底 $\log 2=0.3010300$, 以 2 为底, $\log 2=1$, 得
\[
0. 3010300: 1=p: q
\]
从而
\[
q=\frac{p}{0.3010300}=3.3219277 \cdot p
\]
也即常用对数乘上 $3.3219277$, 就是以 2 为底的对数.

\section{$\S 108$}

相异两数, 其对数的比, 是一个与底无关的常数.

设数 $M$ 和 $N$ 以 $a$ 为底的对数分别为 $m$ 和 $n$. 即 $M=a^{m}, N=a^{n}$, 则 $a^{m n}=M^{n}=N^{m}$, 从 而 $M=N^{\frac{m}{n}}$. 这个等式中不含 $a$, 即比 $\frac{m}{n}$ 与底 $a$ 无关. 如果对另一个底 $b$, 数 $M$ 和 $N$ 的对数 分别为 $\mu$ 和 $\nu$, 经由同样的推导, 得 $M=N^{\frac{\mu}{\nu}}$. 由 $N^{\frac{m}{n}}=N^{\frac{\mu}{\nu}}$ 得 $\frac{m}{n}=\frac{\mu}{\nu}$ 或 $m: n=\mu: \nu$. 我们 已经看到, 同一个数的不同幂的任何一种对数的比, 都等于指数的比. 例如 $\gamma^{m}$ 和 $y^{n}$ 的任 何一种对数的比都等于 $m: n$.

\section{$\S 109$}

造对数表, 可以用前面讲的计算对数的方法, 当然也可以用别的更方便的方法. 但由 于合数的对数等于其因数的对数之和, 所以要造对数表, 只需算出质数的对数, 有了质数 的对数, 合数的对数用简单的加法即可得到. 例如有了 3 和 5 的对数, 则
\[
\log 15=\log 3+\log 5, \log 45=2 \log 3+\log 5
\]
前面我们已经算出了以 10 为底时
\[
\log 5=0.6989700
\]
且 $\log 10=1$. 从而由
\[
\log \frac{10}{5}=\log 2=\log 10-\log 5
\]
得
\[
\log 2=1-0.6989700=0.3010300
\]
有了 2 和 5 的对数, 那么所有只以 2 , 只以 5 , 或只以 2 和 5 为因数的合数, 诸如 $4,8,16,32$, $64, \cdots$ 和 $20,40,80,25,50, \cdots$, 它们的对数就都可以用加法得到.

\section{$\S 110$}

用对数表可以从数查对数, 也可以从对数查数. 两相结合使得对数表在数值计算中大显身手. 假定给了六个数 $c, d, e, f, g, h$, 要我们计算的不是这六个数的积, 而是
\[
\frac{c^{2} d \sqrt{e}}{f \sqrt[3]{g h}}
\]
这样一个复杂表达式的值. 这个表达式的对数为
\[
2 \log c+\log d+\frac{1}{2} \log e-\log f-\frac{1}{3} \log g-\frac{1}{3} \log h
\]
算出来的这个对数, 它所对应的数就是我们所要的数. 对数把求幂求根两种运算转换成 了乘法和除法. 因而求复杂的幂和根时, 对数表显示其特别的作用.

例 2 我们计算 $2^{\frac{7}{12}}$ 的值. 它的对数为 $\frac{7}{12} \log 2$, 乘 2 的对数 $0.3010300$ 以 $\frac{7}{12}$, 即 $\frac{1}{12}+$ $\frac{1}{2}$, 我们得到 $\log 2^{\frac{7}{12}}=0.1756008$. 对应于这个对数的数为 $1.498307$. 这就是 $2^{\frac{7}{12}}$ 的近似 值.

例 3 某地现有人口 100000 , 年增长率为 $\frac{1}{30}$. 求百年后该地人口数.

为简便计, 记现有人口数为 $n$, 即
\[
n=100000
\]
一年后人口数为
\[
\left(1+\frac{1}{30}\right) n=\frac{31}{30} n
\]
两年后人口数为 $\left(\frac{31}{30}\right)^{2} n$, 三年后人口数为 $\left(\frac{31}{30}\right)^{3} n$, 百年后人口数为
\[
\left(\frac{31}{30}\right)^{100} n=\left(\frac{31}{30}\right)^{100} 100000
\]
百年后人口数的对数为
\[
100 \cdot \log \frac{31}{30}+\log 100000
\]
由
\[
\log \frac{31}{30}=\log 31-\log 30=0.014240439
\]
得
\[
100 \log \frac{31}{30}=1.4240439
\]
加上 $\log 100000=5$, 得所求人口数的对数为
\[
\text { 6. } 4240439
\]
对应的人口数为
\[
2654874
\]
即百年后人口数是现有人口数的 26 倍半稍多.

例 4 一场洪水使得某地只剩下了 6 个人, 人们希望该地 200 年后人口数为 1000000 . 问人口的年增长率应该是多少?

设年增长率为 $\frac{1}{x}$, 则 200 年后人口数为
\[
\left(\frac{1+x}{x}\right)^{200} \cdot 6=1000000
\]
由此得
\[
\frac{1+x}{x}=\left(\frac{1000000}{6}\right)^{\frac{1}{200}}
\]
取对数,得
\[
\log \frac{1+x}{x}=\frac{1}{200} \log \frac{1000000}{6}=\frac{1}{200} \cdot 5.2218487=0.0261092
\]
查对数表, 得
\[
\frac{1+x}{x}=\frac{1061963}{1000000}
\]
从而
\[
1000000=61963 x
\]
最后得 $x$ 的近似值为 16 . 即年增长率为 $\frac{1}{16}$, 就能达到人们预期的目标. 现在我们假定保持 这个增长速度 400 年, 那时人口数将为
\[
1000000 \cdot \frac{1000000}{6}=166666666666
\]
这么多人,整个地球恐怕都负担不了.

例 5 百年人口增加一倍, 问这年增长率是多少?

假定年增长率为 $\frac{1}{x}$, 原有人口数为 $n$. 那么, 百年之后人口数为
\[
\left(\frac{1+x}{x}\right)^{100} n=2 n
\]
从而
\[
\frac{1+x}{x}=2^{\frac{1}{100}}
\]
取对数, 得
\[
\log \frac{1+x}{x}=\frac{1}{100} \log 2=0.0030103
\]
查表,得
\[
\frac{1+x}{x}=\frac{10069555}{10000000}
\]
从而
\[
x=\frac{10000000}{69555}
\]
其近似值为 144. 即年增长 $\frac{1}{144}$, 百年就可加倍.

\section{$\S 111$}

对数的最重要的应用, 是解末知量含于指数的方程. 例如求满足方程
\[
a^{x}=b
\]
的 $x$ 值, 我们就必须应用对数. 对 $a^{x}=b$ 两边取对数,得
\[
\log a^{x}=x \log a=\log b
\]
从而
\[
x=\frac{\log b}{\log a}
\]
我们指出,用随便以什么数为底的对数都可以,求得的比不因底而不同.

例 6 知人口年增长率为 $\frac{1}{100}$, 求人口增长到 10 倍所需的时间.

记所需年数为 $x$, 记原有人口数为 $n$, 则 $x$ 年后人口数为 $\left(\frac{101}{100}\right)^{x} \cdot n$, 等于 $10 n$. 从而
\[
\left(\frac{101}{100}\right)^{x}=10
\]
取对数,得
\[
x \log \frac{101}{100}=\log 10
\]
由此得
\[
x=\frac{\log 10}{\log 101-\log 100}=\frac{10000000}{43214}=231
\]
即年增长率为 $\frac{1}{100}$ 时, 231 年人口就增长到十倍, 462 年就增长到百倍, 693 年就增长到千 倍.

例 7 某人以 $5 \%$ 的年利借款 400000 弗罗林(1), 商定了每年归还 25000 弗罗林. 问 还清这笔债要多少年?

记借款数 400000 弗罗林为 $a$, 记每年还款数 25000 弗罗林为 $b$, 那么满一年欠款数 为
\[
\frac{105}{100} a-b
\]
满二年欠款数为
\[
\left(\frac{105}{100}\right)^{2} a-\left(\frac{105}{100}\right) b-b
\]
(1)一种货币, 起源于佛罗伦萨.一—一编者注. 

满三年欠赦数为
\[
\left(\frac{105}{100}\right)^{3} a-\left(\frac{105}{100}\right)^{2} b-\frac{105}{100} b-b
\]
为简便计, 令 $n=\frac{105}{100}$, 那么满 $x$ 年欠款数为
\[
n^{x} a-n^{x-1} b-n^{x-2} b-n^{x-3} b-\cdots-b=n^{x} a-b\left(1+n+n^{2}+\cdots+n^{x-1}\right)
\]
由几何级数的性质知
\[
1+n+n^{2}+\cdots+n^{x-1}=\frac{n^{x}-1}{n-1}
\]
从而, 满 $x$ 年债务人欠款数为
\[
n^{x} a-\frac{n^{x} b-b}{n-1}
\]
欠款还清时欠款数为零, 由此我们得到方程
\[
n^{x} a=\frac{n^{x} b-b}{n-1} \text { 或 }(n-1) n^{x} a=n^{x} b-b
\]
从而
\[
(b-n a+a) n^{x}=b \text { 或 } n^{x}=\frac{b}{b-(n-1) a}
\]
取对数,解出 $x$, 得
\[
x=\frac{\log b-\log (b-(n-1) a)}{\log n}
\]
由
\[
a=400000, b=25000, n=\frac{105}{100}
\]
得
\[
(n-1) a=20000, b-(n-1) a=5000
\]
从而还清这笔债务的年数为
\[
x=\frac{\log 25000-\log 5000}{\log \frac{105}{100}}=\frac{\log 5}{\log \frac{21}{20}}=\frac{6989700}{211893}
\]
即 33 年不到一点. 满 33 年时, 还款数本利和比借款数本利和多, 多出来的数为
\[
\frac{\left(n^{33}-1\right) b}{n-1}-n^{33} a=\frac{\left(\frac{21}{20}\right)^{33} 5000-25000}{\frac{1}{20}}=100000\left(\frac{21}{20}\right)^{33}-500000
\]
由
\[
\log \frac{21}{20}=0.0211892991
\]
得
\[
\log \left(\frac{21}{20}\right)^{33}=0.69924687, \log 100000\left(\frac{21}{20}\right)^{33}=5.6992469
\]
这个数是 $500318.8$ 的对数. 即满 33 年时债权人应退给债务人 $318.8$ 弗罗林.

\section{$\S 112$}

我们来看常用对数. 常用对数以 10 为底, 我们通常使用十进制数, 因而常用对数比 别种对数特别地有用. 10 的幂的对数是整数, 不是 10 的幂的数的对数含有小数. 例如 1 与 10 之间的数的对数在 0 和 1 之间, 10 与 100 之间的数的对数在 1 和 2 之间, 等等. 因此 对数都由整数和小数两部分构成. 我们称整数部分为首数, 称小数部分为尾数. 一个数的 对数的首数等于它的整数部分的位数减 1 . 例如五位数 78509 的对数的首数为 4 . 反之, 从一个数的对数, 我们也立刻可以说出它的位数. 例如, 对数为 $7.5804631$ 的数是 8 位 数.

\section{$\S 113$}

两个数, 如果其对数的尾数相同, 首数不同, 则这两数的比为 10 的幂. 也即这两个数 的数字相同. 例如, 对数为 $4.9130187$ 和 $6.9130187$ 的数分别为 81850 和 8185000 . 对 数为 $3.9130187$ 和 $0.9130187$ 的数, 分别为 8185 和8. 185. 即尾数给出数的数字, 首数 告诉我们数的整数部分的位数. 例如对数 $2.7603429$, 尾数给出数的数字为 5758945 , 首数 2 告诉我们整数部分为 3 位, 即数为 $575.8945$. 如果首数为 0 , 则告诉我们整数部分 为 1 位, 即数为 $5.758945$. 如果首数为负数, 例如 $-1$, 则数为 $0.5758945$. 首数为 $-2$, 则 数为 $0.05758945$. 首数
\[
-1,-2,-3
\]
常常记为 $9,8,7$, 即从记的数减去 10 为实际的首数. 以上所讲, 对数表的说明中有更详细 的解释.

例 8 数列 $2,4,16,256, \cdots$, 每一项都是前一项的平方求它的第 25 项.

该数列可记为
\[
2,2^{2}, 2^{4}, 2^{8}, \cdots
\]
可见这数列的指数成几何级数,第 25 项的指数为
\[
2^{24}=16777216
\]
因而所求的项为
\[
2^{16777216}
\]
它的对数为
\[
16777216 \log 2
\]
由
\[
\log 2=0.301029995663981195
\]
得所求项的对数为
\[
5050445.25973367
\]
首数告诉我们所求项整数部分的位数为
\[
5050446
\]
从对数表中查得尾数 259733675932 对应的数为 181858 . 它后面还有 5050440 位. 从位 数更多的对数表中可多查到几位. 事实上, 该数的前 11 位为 18185852986 . 

