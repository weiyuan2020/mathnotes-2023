\chapter{第十九章 曲线的交点}

\section{$\S 457$}

曲线与直线的交点, 前几章我们讨论过不只一次了, 还证明了曲线与直线的交点个 数, 二阶线不能多于 2 , 三阶线不能多于 3 , 四阶线不能多于 4 , 类推. 本章我们讨论两条曲 线的交点个数. 我们还是从直线开始, 先讨论任一直线与已给曲线的交点, 并拿这一讨论 作为曲线交点讨论的准备. 曲线交点的讨论广泛用于列高阶方程, 列高阶方程是下一章 的内容.

\section{$\S 458$}

设 $A M m$ 为任给的一条曲线, 其性质用直角坐标 $A P=x, P M=y$ 的方程表示. 现在任 画一条直线 $B M m$ (图 94). 我们来确定这直线与曲线 $A M m$ 的交点个数, 并求出交点. 为 此, 首先我们应求出这直线关于曲线所用坐标的方程, 也即关于以 $A P$ 为轴, $A$ 为原点的 直角坐标 $x, y$ 的方程. 直线方程的形状为 $\alpha x+\beta y=\gamma$. 令 $x=0$ 得 $y=A D=\frac{\gamma}{\beta}$, 令 $y=0$ 得 $x=-A B=\frac{\gamma}{\alpha}$, 由此得该直线与轴的交点 $B$ 处, 交角的正切等于 $\frac{A D}{A B}=-\frac{\alpha}{\beta}$. 这样我们就 把曲线和直线用相同的坐标 $x$ 和 $y$ 表示了出来.


【图,待补】
%%![](https://cdn.mathpix.com/cropped/2023_02_05_a0e82ac431ec1349400cg-01.jpg?height=335&width=524&top_left_y=1549&top_left_x=566)

图 94

\section{$\S 459$}

对横标 $x$ 的同一个值, 从两个方程得到的 $y$ 值, 差别越小, 表明在该 $x$ 处两条线离得 越近. 如果得到的 $y$ 值相同, 则表明曲线与直线在该 $x$ 处重合, 即得到了一个交点. 可见 交点是其横标纵标同时满足这两个方程的点, 求交点就是求同时满足这两个方程的横标 $x$ 和纵标 $y$. 如果从两个方程中消去 $y$, 我们就得到一个变元 $x$ 的方程. 它的解就是横标 $A P$ 和 $A p$. 从这两个横标得到纵标 $P M$ 和 $p m$, 也就得到了交点 $M$ 和 $m$.

\section{$\S 460$}

直线 $B M m$ 的方程为 $\alpha x+\beta y=\gamma$, 解出 $y$, 得 $y=\frac{\gamma-\alpha x}{\beta}$. 将这个 $y$ 代入曲线方程, 得不 含 $y$ 只含 $x$ 的方程. 这新方程的实根为交点的横标, 交点个数等于实根个数. 表达式 $y=$ $\frac{\gamma-\alpha x}{\beta}$ 中 $x$ 的次数为 1 , 因而将它代入以 $x, y$ 为变量的曲线方程时, 不会增加方程的次 数, 即新方程中 $x$ 的次数等于或小于原方程的次数, 代入时 $x$ 的最高次幂可以消失.

\section{$\S 461$}

有了交点的横标 $A P$ 和 $A p$, 利用它即可求出交点 $M$ 和 $m$. 画出点 $P$ 和 $p$ 处的纵标 线, 这两条纵标线与直线 $B M m$ 的交点即所求交点 $M$ 和 $m$. 当然也可以通过求纵标线与 曲线 $A M m$ 的交点得到所求交点 $M$ 和 $m$. 但纵标线与曲线的交点可以不只一个, 因而还 要进一步判断哪一个是交点. 直线 $B M m$ 则不存在这个问题, 它与纵标线的交点只有一 个. 如果两个 $x$ 值相等, 则交点 $M$ 和 $m$ 合为一点. 此时直线 $B M m$ 或者为曲线的切线, 或 者交曲线于二重点.

\section{$\S 462$}

消去了 $y$ 得到的那个新的 $x$ 的方程, 如果它没有实根, 则直线 BMm 与曲线既不相交 也不相切. 新方程的实根 (可以不只一个) 给出所有的交点. 这每一个实根 (横标) 只对应 直线 $B M m$ 的一个实纵标, 这实纵标也是曲线的对应纵标. 因而得到的必定是交点. 应该 指出, 对于两条曲线, 新方程的实根不一定都给出交点. 待后面考察了两条曲线的交点, 我们就会明白这原因.

\section{$\S 463$}
参见图 95, 任给两条相交曲线 $M E m$ 和 $M F m$. 为求交点, 先列出它们的直角坐标 $x$ 和 $y$ 的方程, 这里轴为 $A B$, 原点为 $A$. 交点处, 两条曲线的横标纵标都相同. 从两个方程 消去 $y$, 得到单个变元 $x$ 的新方程. 不管有多少个交点 $M, m, m$, 它们的横标都由新方程 给出. 也即交点 $M, m, m, \cdots$ 的横标是新方程的根. 


【图,待补】
%%![](https://cdn.mathpix.com/cropped/2023_02_05_a0e82ac431ec1349400cg-03.jpg?height=392&width=471&top_left_y=290&top_left_x=607)

图 95

\section{$\S 464$}

有了交点的横标 $A P, A p$ 等, 还不是那么容易地就可以求出交点. 两条曲线的 $y$ 都是 $x$ 的多值函数时,对同一个横标 $A P$, 每条曲线都有多个纵标与之对应. 因而得从两组纵 标中选出相等的,纵标的个数越多,挑选就越麻烦,但这麻烦可以避开. 方法是利用从两 个方程消去 $y$ 时那个用 $x$ 表示 $y$ 的表达式. 对求得的每一个 $x$ 值,从这个表达式都可求出 从点 $P$ 到交点的纵标值, 而不涉及两条曲线中任何一条的性质, 也不涉及它们的共有 性质.

\section{$\S 465$}

设一条曲线为抛物线,方程为
\[
y^{2}-2 x y+x^{2}-2 a x=0
\]
另一条曲线为圆,方程为
\[
y^{2}+x^{2}-c^{2}=0
\]
为消去 $y$,从第二个方程减去第一个,得
\[
2 x y+2 a x-c^{2}=0
\]
从而
\[
y=\frac{c^{2}-2 a x}{2 x}
\]
由该式知, 对得到的每一个 $x$ 值, 相应地都能得到一个实 $y$ 值. 将求得的 $y$ 的这个表达式 代入第二个方程,得
\[
c^{4}-4 a c^{2} x+4\left(a^{2}-c^{2}\right) x^{2}+4 x^{4}=0
\]
该方程的每一个实根都确实地给出一个交点, 令 $c=2 a$, 得
\[
4 a^{4}-4 a^{3} x-3 a^{2} x^{2}+x^{4}=0
\]
该方程的一个根为 $x=2 a$. 消去这个根所对应的因式, 得
\[
x^{3}+2 a x^{2}+a^{2} x-2 a^{3}=0
\]
该方程也有一个实根, 对应于第一个和第二个实根的纵标都可从方程 $y=\frac{2 a^{2}-a x}{x}$ 求 出. 对应于第一个实根, 即 $x=2 a$ 的纵标为 $y=0$. 也即交点在轴上.

\section{$\S 466$}

由此清楚地看出, 在消去 $y$ 的过程中, 只要从 $x$ 和 $y$ 的两个方程,能将 $y$ 用 $x$ 有理表 出, 那么消去了 $y$ 的方程的每一个实根 $x$, 就都对应一个真实交点. 如果在消去 $y$ 的过程 中做不到将 $y$ 用 $x$ 有理地表出, 那么消去了 $y$ 的方程的有的实根可能不对应真实的交点. 此时可以得到这样的 $x$ 值, 从哪根曲线上都求不出对应于它的实纵标. 这种情况发生时, 不要怀疑为计算上有错. 对应于这种横标, 两条曲线的纵标都为虚数. 两个虚数跟两个实 数一样, 也可以相等或不等, 即使两个虚纵标相等, 也得不到真实的交点.

\section{$\S 467$}

为了解释得更清楚, 如图 96 所示, 在同一条轴 $B A E$ 上, 先画出以 $2 a$ 为参数的抛物线 $E M$, 再在这抛 物线 $E M$ 的外面画一个半径为 $c$ 的圆. 这两条曲线完 全不相交,记它们间的距离 $A E=d$. 取 $A$ 为原点,自 $A$ 向 $E$ 为正, 向 $B$ 为负. 此时抛物线的方程为 $y^{2}=2 a x-$ $2 a b$, 圆的方程为 $y^{2}=-2 c x-x^{2}$. 为了求出交点, 消去 $y$, 得


【图,待补】
%%![](https://cdn.mathpix.com/cropped/2023_02_05_a0e82ac431ec1349400cg-04.jpg?height=302&width=522&top_left_y=1091&top_left_x=974)

图 96
\[
x^{2}+2(a+c) x-2 a b=0
\]
从它得到 $x$ 的两个实值
\[
x=-a-c \pm \sqrt{(a+c)^{2}+2 a b}
\]
一正一负. 虽然如此, 但这两条曲线却完全地没有交点. 这里的抛物线和圆对应于这两个 横标和纵标都是虚数,但相等. 将求得的 $x$ 值代入, 得
\[
y=\sqrt{-2 a^{2}-2 a c-2 a b \pm 2 a \sqrt{a^{2}+2 a c+c^{2}+2 a b}}
\]
这个表达式当然是虚数.

\section{$\S 468$}

上节的例子表明,两条曲线可以具有虚交点. 它不是真实的交点, 但可以按计算真实 交点的算法把它求出来. 因此交点个数可以少于消去了 $y$ 的新方程的根的个数, 甚至新 方程有两个或更多个实根, 而曲线完全没有交点. 反之则不然, 每一个交点必对应于新方 程的一个实根 $x$. 也即新方程的实根个数绝对地不会少于交点个数, 但可以多于交点个 数. 一个实根是否对应于真实的交点, 由对应 $y$ 值的实虚决定. $y$ 实则交点真实, $y$ 虚则交 点虚.

\section{$\S 469$}

这样交点数少于消去了 $y$ 的新方程实根 $x$ 的个数, 只发生于两种情况之下.一种情 况是, 两个曲线方程中 $y$ 的次数都全为偶数,因而主轴同时是两条曲线的直径. 再一种情 况是, 从两个方程消去 $y^{2}$ 时, 同时也消去了 $y$, 即 $y$ 不能用 $x$ 有理表示. 例如, 一个方程为
\[
y^{2}-x y=a^{2}
\]
另一个方程为
\[
y^{4}-2 x y^{3}+x^{3} y=b^{2} x^{2}
\]
从第一个方程得
\[
\left(y^{2}-x y\right)^{2}=a^{4} \text { 或 } y^{4}-2 x y^{3}=a^{4}-x^{2} y^{2}
\]
代入第二个方程,得
\[
a^{4}-x^{2} y^{2}+x^{3} y=b^{2} x^{2} \text { 或 } y^{2}-x y=\frac{a^{4}-b^{2} x^{2}}{x^{2}}=a^{2}
\]
从而
\[
x^{2}=\frac{a^{4}}{a^{2}+b^{2}}
\]
进而
\[
x=\frac{\pm a^{2}}{\sqrt{a^{2}+b^{2}}}
\]
这里看上去像是有两个交点, 但这两个交点是否为真实交点, 由从方程 $y^{2}-x y=a^{2}$ 得到 的 $y$ 值决定. 我们有
\[
y^{2}=\frac{\pm a^{2} y}{\sqrt{a^{2}+b^{2}}}+a^{2}
\]
它的根都为实数, 因而有四个交点, 也即横标
\[
x=\frac{\pm a^{2}}{\sqrt{a^{2}+b^{2}}}
\]
中的每一个都对应两个真实的交点.

\section{$\S 470$}

轴同时为两条曲线的直径, 或者消去 $y$ 的高次幂时 $y$ 的一次幂也被消去. 只要不是 这两种情况, 我们就都可以把 $y$ 表示成 $x$ 的有理函数. 从而消去了 $y$ 的新方程的每一个实 根就都对应一个真实的交点. 当然也就无需特别的注意. 如我们看到的, 当一条线为直线 时和纵标为横标的单值函数时, 就都不需特别注意. 这两种情况下横标都不对应虚纵标, 从而新方程的实根都给出真实交点. 在多数情况下, 虽然两个方程都含有 $y$ 的高于一次 的幂, 但在消去 $y$ 时都可以得到 $y$ 的有理函数,也即 $y$ 用 $x$ 表示的单值函数.

\section{$\S 471$}

我们举过的抛物线和圆的那个例子, 对应于求得的横标, 两条曲线的纵标都是虚数. 当然交点也就都是虚的. 对应于求得的横标, 即使一条曲线的纵标都是实数, 交点也可以 是虚的. 例如,三阶线
\[
y^{3}-3 a y^{2}+2 a^{2} y-6 a x^{2}=0
\]
对所有的横标, 其纵标都为实数, 且 $x<\frac{a}{3} \sqrt[4]{\frac{1}{3}}$ 时纵标有三个值. 考虑该三阶线与抛物 线 $y^{2}-2 a x=0$ 的交点. 横标为负时, 这抛物线的纵标为虚数, 即横标为负时它与所给三 阶线不相交.

\section{$\S 472$}

现在我们消去 $y$. 由抛物线方程得 $y^{2}=2 a x$, 代入三阶线方程, 得
\[
2 a x y-6 a^{2} x+2 a^{2} y-6 a x^{2}=0
\]
从而
\[
y=\frac{6 a^{2} x+6 a x^{2}}{2 a^{2}+2 a x}=3 x
\]
即所得方程被 $y-3 x$ 除得尽. 做除法得消去了 $y$ 的方程 $2 a^{2}+2 a x=0$, 由此得 $x=-a$. $x=-a$ 时抛物线的纵标为虚数. 将 $x=-a$ 代入三阶线方程, 得
\[
y^{3}-3 a y^{2}+2 a^{2} y-6 a^{3}=0
\]
由该方程得实纵标 $y=3 a$, 另外两个纵标由方程 $y^{2}+2 a^{2}=0$ 给出, 是虚数. 这两个虚纵标 等于抛物线的对应虚纵标, 得到两个虚交点. 但由因式 $y-3 x=0$ 得到两个真实交点. 将 $y=3 x$ 代入抛物线方程, 得 $9 x^{2}-2 a x=0$. 从而, 一个真实交点为原点, $x=0, y=0$. 另一个 为 $x=\frac{2 a}{9}, y=3 x=\frac{2 a}{3}$.

\section{$\S 473$}

这样,虽然消去 $y$ 时得到的方程
\[
2 a x y-6 a^{2} x+2 a^{2} y-6 a x^{2}=0
\]
只含 $y$ 的一次幂, 可以将 $y$ 表示为 $x$ 的有理函数, 但我们得到了虚交点. 前面我们是把 $y$ 由 $x$ 有理表示作为没有虚交点的判别准则的. 矛盾吗? 不. 实际上, 如果消去 $y$ 时得到的 方程没有因式, 也就会有虚交点. 去掉因式得到的方程就不再含 $y, y$ 也就不能用 $x$ 有理 表出. 当消去过程中得到的方程能分解成因式时, 我们就应该分别对每个因式进行考虑. 可以一个因式有虚交点, 而另一个没有.

\section{$\S 474$}

进行了上面的讨论之后,下面我们讲两条给定曲线交点的具体求法. 因为这求法涉 及坐标 $y$ 的消去, $x$ 的次数不影响消去过程,所以我们只考虑两个方程中 $y$ 的次数. 设 $P$, $Q, R, S, T, \cdots$ 和 $p, q, r, s, t, \cdots$ 都是 $x$ 的有理函数. 首先假定我们求其交点的那两条曲线 的方程为
\[
\begin{gathered}
I \\
P+Q y=0 \\
\Pi \\
p+q y=0
\end{gathered}
\]
分别以 $p$ 和 $P$ 乘 I, II ,相减,再除以 $y$,得
\[
p Q-P q=0
\]
该方程只含变量 $x$ 和常数. 它的实根为交点的横标, 对应的纵标可从
\[
y=-\frac{p}{q}=-\frac{P}{Q}
\]
求出. 因而, 如果两条曲线的纵标 $y$ 都能表示成 $x$ 的有理, 也即单值函数,则没有虚交点.

\section{$\S 475$}

设一条曲线的 $y$,跟前面一样,可表示成 $x$ 的单值函数,而另一条的 $y$ 可表示成 $x$ 的 二值函数, 即
\[
\begin{gathered}
\mathrm{I} \\
P+Q y=0 \\
\text { II } \\
p+q y+r y^{2}=0
\end{gathered}
\]
分别以 $p$ 和 $P$ 乘 I, II, 相减,再除以 $y$,得
\[
p Q-P q-\operatorname{Pr} y=0 \text { 或 }(P q-p Q)+\operatorname{Pr} y=0
\]
分别以 $\operatorname{Pr}$ 和 $Q$ 乘 I, III, 相减得不含 $y$ 的方程
\[
P^{2} r-P Q q+p Q^{2}=0
\]
该方程的每一个根都给出对应交点的横标, 对应的纵标为
\[
y=-\frac{P}{Q}=\frac{p Q-P q}{P r}
\]
都是实数,交点是真实的.

\section{$\S 476$}

保持一条曲线的纵标为 $x$ 的单值函数, 设另一曲线的纵标由一个三次方程表示, 也 即设它为 $x$ 的三值函数, 这两个方程为
\[
\begin{gathered}
\mathrm{I} \\
P+Q y=0 \\
\text { II } \\
p+q y+r y^{2}+s y^{3}=0
\end{gathered}
\]
分别以 $p$ 和 $P$ 乘 I, II ,相减,再除以 $y$,得
\[
\begin{gathered}
\text { III } \\
(P q-p Q)+P r y+P s y^{2}=0
\end{gathered}
\]
将从 I 求出的 $y=-\frac{P}{Q}$ 代入,去分母,得
\[
P Q^{2} q-p Q^{3}-P^{2} Q r+P^{3} s=0
\]
或
\[
Q^{3} p-P Q^{2} q+P^{2} Q r-P^{3} s=0
\]
将 $y=-\frac{P}{Q}$ 代入 II 也得到这一结果. 求出这最后方程的所有实根, 并依 $y=-\frac{P}{Q}$ 求出对 应的实纵标, 得到实根个数那么多真实交点.

\section{$\S 477$}

类似地, 保持 I 的纵标为 $x$ 的单值函数, 换 II 为 $y$ 的四次或更高次方程, $y$ 也是容易 消去的,设方程为
\[
\begin{gathered}
\mathrm{I} \\
P+Q y=0 \\
\text { II } \\
p+q y+r y^{2}+s y^{3}+t y^{4}=0
\end{gathered}
\]
从 I 得 $y=-\frac{P}{Q}$, 代入 II 得只含 $x$ 和已知量的方程
\[
Q^{4} p-P Q^{3} q+P^{2} Q^{2} r-P^{3} Q s+P^{4} t=0
\]
从该方程的每一个实根都得到一个实纵标 $y=-\frac{P}{Q}$. 因而实根给出同样数目的真实 交点.

\section{$\S 478$}

设两条曲线的方程都是 $y$ 的纯二次方程, 即不含 $y$ 的一次项, 其形状为 
\[
\begin{gathered}
I \\
P+R y^{2}=0 \\
\text { II } \\
p+r y^{2}=0
\end{gathered}
\]
消去 $y^{2}$, 得
\[
\operatorname{Pr}-R p=0
\]
该方程的实根中使 $-\frac{P}{R}$ 和 $-\frac{p}{r}$ 都为正数,给出真实交点. 对每一个这样的实根, $y^{2}=$ $-\frac{P}{R}=-\frac{p}{r}$ 给出 $y$ 的一正一负两个实值. 因而 $\operatorname{Pr}-R p=0$ 的每一个这样的根对应两个 交点,这两个交点至轴的距离相等,因为轴是这两条曲线的直径. $\operatorname{Pr}-R p=0$ 的根中使 $-\frac{P}{R}=-\frac{p}{r}$ 为负数, 对应的 $y$ 为虚数,因而给出的交点为虚的.

\section{$\S 479$}

设 $y$ 的两个二次方程都具有含 $y$ 的项,即其形状为
\[
\begin{gathered}
\text { I } \\
P+Q y+R y^{2}=0 \\
\text { II } \\
p+q y+r y^{2}=0
\end{gathered}
\]
为消去 $y$, 分别以 $p$ 和 $P$ 乘 I 和 II, 相减, 再除以 $y$, 得

III
\[
(P q-Q p)+(P r-R p) y=0
\]
再分别以 $r$ 和 $R$ 乘 I 和 II, 相减,得
\[
\begin{gathered}
\mathrm{IV} \\
(P r-R p)+(Q r-R q) y=0
\end{gathered}
\]
由 III 和 IV 得
\[
y=\frac{Q p-P q}{P r-R p}=\frac{R p-P r}{Q r-R q}
\]
从而
\[
(Q p-P q)(Q r-R q)+(P r-R p)^{2}=0
\]
展开, 得
\[
P^{2} r^{2}-2 P R p r+R^{2} p^{2}+Q^{2} p r-P Q q r-Q R p q+P R q^{2}=0
\]
该方程的每个实根都对应一个真实交点,因为对 $x$ 的每个实值都可以从 III 和 IV 得到 $y$ 的一个实值. 但是当 III 和 IV 有因式,除以因式得到不含 $y$ 的方程时,求出这个不含 $y$ 的 方程的根, 再求出这个根对应的 $y$ 值, 如果这个 $y$ 值为虚数, 那么我们就得到虚交点. 

\section{$\S 480$}

设两条曲线的 $y$ 分别是 $x$ 的二值和三值函数,即方程为
\[
\begin{gathered}
\mathrm{I} \\
P+Q y+R y^{2}=0 \\
\text { II } \\
p+q y+r y^{2}+s y^{3}=0
\end{gathered}
\]
分别以 $p$ 和 $P$ 乘 $\mathrm{I}$ 和 II, 相减,再除以 $y$, 得
\[
(P q-Q p)+(P r-R p) y+P s y^{2}=0
\]
依次记 $P q-Q p, P r-R p, P s$ 为 $p, q, r$, 则 III 成为上节的 II, 因而我们有
\[
y=\frac{P Q q-Q^{2} p-P^{2} r+P R p}{P^{2} s-P R q+Q R p}
\]
和
\[
y=\frac{P R q-Q R p-P^{2} s}{P Q s-P R r+R^{2} p}
\]
从而
\[
0=\left(P R q-Q R p-P^{2} s\right)^{2}+\left(P Q s-P R r+R^{2} p\right)\left(P Q q-Q^{2} p-P^{2} r+P R p\right)
\]
展开,得
\[
\left.\begin{array}{l}
P^{4} s^{2}-2 P^{3} R q s+3 P^{2} Q R p s-P Q R^{2} p q+Q^{2} R^{2} p^{2} \\
-P^{3} Q r s+P^{2} R^{2} q^{2}-P Q^{3} p s-Q^{2} R^{2} p^{2} \\
+P^{3} R r^{2}+P^{2} Q^{2} q s+P Q^{2} R p r \\
-P^{2} Q R q r+P R^{3} p^{2} \\
-2 P^{2} R^{2} p r
\end{array}\right\}
\]
删去 $Q^{2} R^{2} p^{2}-Q^{2} R^{2} p^{2}=0$, 以公因式 $P$ 除其余各项,得
\[
\begin{aligned}
& P^{3} s^{2}-2 P^{2} R q s-P^{2} Q r s+P^{2} R r^{2}+3 P Q R p s+P R^{2} q^{2}+P Q^{2} q s- \\
& P Q R q r-2 P R^{2} p r-Q R^{2} p q-Q^{3} p s+Q^{2} R p r+R^{3} p^{2}=0
\end{aligned}
\]
该方程的实根对应的 $y$ 值也为实数时, 得到的就是真实交点.

\section{$\S 481$}

现在讨论曲线方程的次数都为 3 的情形, 即
\[
\begin{gathered}
\mathrm{I} \\
P+Q y+R y^{2}+S y^{3}=0 \\
\text { II } \\
p+q y+r y^{2}+s y^{3}=0
\end{gathered}
\]
先分别以 $p$ 和 $P$ 乘 $\mathrm{I}$ 和 II, 相减, 得 
\[
\begin{aligned}
& \text { Infinite analysies 无穷分析与论 } \text { Intraductian } \\
& \qquad(P q-Q p)+(P r-R p) y+(P s-S p) y^{2}=0
\end{aligned}
\]
再分别以 $s$ 和 $S$ 乘 I 和 II, 相减,得
\[
(S p-P s)+(S q-Q s) y+(S r-R s) y^{2}=0
\]
将这里的 III 和 IV 与 $\$ 479$ 的两个方程相比较,得
\[
\begin{array}{l|l}
P=P p-Q p & p=S p-P s \\
Q=P r-R p & q=S q-Q s \\
R=P s-S p & r=S r-R s
\end{array}
\]
将这些值代入 $\$ 479$ 的最后一个方程,得
\[
\begin{aligned}
& (P q-Q p)^{2}(S r-R s)^{2}-2(P q-Q p)(P s-S p)(S p-P s)(S r-R s)+ \\
& (P s-S p)^{2}(S p-P s)^{2}+(P r-R p)^{2}(S p-P s)(S r-R s)- \\
& (P q-Q p)(P r-R p)(S q-Q s)(S r-R s)- \\
& (P r-R p)(P s-S p)(S p-P s)(S q-Q s)+ \\
& (P q-Q p)(P s-S p)(S q-Q s)^{2}=0
\end{aligned}
\]
该方程共七项,第一第五两项之外, 各项都有因式 $S p-P s$.一、五两项的和可分解为两个 因式,一个为 $(P q-Q p)(S r-R s)$, 另一个为
\[
(P q-Q p)(S r-R s)-(P r-R p)(S q-Q s)
\]
展开这另一个因式,得
\[
P Q r s+R S p q-P R q s-Q S p r
\]
它等于 $(S p-P s)(R q-Q r)$, 也即一、五两项的和为
\[
(P q-Q p)(S r-R s)(S p-P s)(R q-Q r)
\]
也含有因式 $S p-P s$. 约去这个因式,则本节的结果方程成为
\[
\begin{aligned}
0= & (P q-Q p)(S r-R s)(R q-Q r)+2(P q-Q p)(S p- \\
& P s)(S r-R s)+(S p-P s)^{3}+(P r-R p)^{2}(S r-R s)+ \\
& (P r+R p)(S p-P s)(S q-Q s)-(P q-Q p)(S q-Q s)^{2}
\end{aligned}
\]
展开成为
\[
\begin{aligned}
& S^{3} p^{3}-3 P S^{2} p^{2} s+P^{2} S r^{3}+2 P R^{2} p r s-P^{2} R r^{2} s+P^{2} Q r s^{2}+ \\
& P R S q^{2} r-P^{3} s^{3}+3 P^{2} S p s^{2}-R^{3} p^{2} s-2 P R S p r^{2}+R^{2} S p^{2} r- \\
& R S^{2} p^{2} q-Q^{2} R p r s-P R^{2} q^{2} s-P Q S q r^{2}+P Q R q r s+ \\
& 3 P S^{2} p q r-3 P^{2} S q r s+P Q S p r s+Q^{2} S p r^{2}+Q R^{2} p q s- \\
& Q R S p q r-3 P Q R p s^{2}+3 Q R S p^{2} s-P R S p q s+2 P^{2} R q s^{2}+ \\
& 2 P Q S q^{2} s-P S^{2} q^{3}-P Q^{2} q s^{2}-2 Q S^{2} p^{2} r-2 Q^{2} S p q s+ \\
& Q^{3} p s^{2}+Q S^{2} p q^{2}=0
\end{aligned}
\]
\section{$\S 482$}

为了把从两个高次方程中消去 $y$ 这一方法解释得更清楚, 我们假定两个方程的次数都为 4 , 即
\[
\begin{gathered}
\mathrm{I} \\
P+Q y+R y^{2}+S y^{3}+T y^{4}=0 \\
\text { II } \\
p+q y+r y^{2}+s y^{3}+t y^{4}=0
\end{gathered}
\]
先分别以 $p$ 和 $P$ 乘 $\mathrm{I}$ 和 II, 相减, 得
\[
(P q-Q p)+(P r-R p) y+(P s-S p) y^{2}+(P t-T p) y^{3}=0
\]
再分别以 $t$ 和 $T$ 乘 I 和 II, 相减,得
\[
(P t-T p)+(Q t-T q)+(R t-T r) y^{2}+(S t-T s) y^{3}=0
\]
为简便起见, 令
\[
\begin{array}{l|l|l}
P q-Q p=A & P t-T p=a & S q-Q s=\alpha \\
P r-R p=B & Q t-T q=b & R q-Q r=\beta \\
P s-S p=C & R t-T r=c & \\
P t-T p=D & S t-T s=d &
\end{array}
\]
这里应该指出, 不仅有 $a=D$, 并且还有
\[
\begin{aligned}
& A d-C b=(P t-T p)(S q-Q s)=D \alpha \\
& A c-B c=(P t-T p)(R q-Q r)=D \beta
\end{aligned}
\]
换成简单表示, 则 III 和 IV 成为
\[
\begin{gathered}
\text { III } \\
A+B y+C y^{2}+D y^{3}=0 \\
\text { IV } \\
a+b y+c y^{2}+d y^{3}=0
\end{gathered}
\]
先分别以 $d$ 和 $D$ 乘 III 和 IV, 相减,得
\[
(A d-D a)+(B d-D b) y+(C d-D c) y^{2}=0
\]
再分别以 $a$ 和 $A$ 乘 III 和 $\mathrm{IV}$, 相减, 得
\[
(A b-B a)+(A c-C a) y+(A d-D a) y^{2}=0
\]
也为简便起见,再令
\[
\begin{aligned}
& A b-B a=E \mid A d-D a=e \\
& A c-C a=F \quad B d-D b=f \quad C b-B c=\xi \\
& A d-D a=G|C d-D c=g|
\end{aligned}
\]
这里 $G=e$ 且 $E g-F f=G \xi$, 即 $G$ 是 $E g-F f$ 的因式, 换成简单表示, 则 $V$, VI 成为
\[
E+F y+G y^{2}=0
\]
\[
\begin{gathered}
\text { VI } \\
e+f y+g y^{2}=0
\end{gathered}
\]
对这两个方程进行前面做过的运算,得
\[
\begin{aligned}
& \mathrm{VII} \\
&(E f-F e)+(E g-G e) y=0 \\
& \mathrm{VII} \\
&(E g-G e)+(F g-G f) y=0
\end{aligned}
\]
最后, 为简单起见, 令
\[
\begin{array}{cc}
E f-F e & =H, \quad E g-G e=h \\
E g-G e & =I, \quad F g-G f=i
\end{array}
\]
这里 $I=h, \mathrm{VII}$ 和 VIII 成为
\[
\begin{gathered}
\mathrm{VI}^{\prime} \\
H+I y=0 \\
\mathrm{VII^{ \prime }} \\
h+i y=0
\end{gathered}
\]
由这两个方程得不含 $y$ 的方程
\[
H i-I h=0
\]
向后, 逐次将简记符号复原, 最终我们将得到只含 $P, Q, R, \cdots, p, q, r, \cdots$ 的方程, 含字母 $E, F, G, e, f, g$ 的方程被 $G=e$ 除得尽, 而含字母 $A, B, C, D, a, b, c, d$ 的方程被 $D^{2}=a^{2}$ 除 得尽. 结果方程的每一项只含 8 个字母, 大小写各 4 个. 一般地, 不管两个方程中 $y$ 的次数 为几, 用这个方法都可消去 $y$, 得到只含一个变量 $x$ 的方程.

\section{$\S 483$}

从两个方程中消去一个末知数,虽然讲过的这种方法有着足够的通用性, 我们还是 要再讲一种方法, 它不要求那么多层的代入. 假定两个方程的次数都任意
\[
\begin{gathered}
\text { I } \\
P y^{m}+Q y^{m-1}+R y^{m-2}+S y^{m-3}+\cdots=0 \\
\text { II } \\
p y^{n}+q y^{n-1}+r y^{n-2}+s y^{n-3}+\cdots=0
\end{gathered}
\]
我们要从这两个方程推出一个不含 $y$ 的方程. 为此将下式乘以 II
\[
P y^{k-n}+A y^{k-n-1}+B y^{k-n-2}+C y^{k-n-3}+\cdots
\]
其中待定字母 $A, B, C, \cdots$ 的个数为 $k-n$. 然后使两个乘积相等, 让 $y$ 的同次幂的系数相 等, 从而相抵消. 最后剩下的不含 $y$ 的项就给出我们所要的方程. 两个乘积的次数最高项 相同, 都为 $P p y^{k}$, 相抵消. 式中剩下应该对应相等的系数, 每边 $k-1$ 个, 共给出 $k-1$ 个用 来确定待定字母的方程. 待定字母的个数为 $2 k-m-n$, 它应该等于 $k-1$, 因而我们令 $k=m+n+1$. 

\section{$\S 484$}

分别以系数待定表达式乘 I 和 II
\[
p y^{n-1}+a y^{n-2}+b y^{n-3}+c y^{n-4}+\cdots
\]
和
\[
P y^{m-1}+A y^{m-2}+B y^{m-3}+C y^{m-4}+\cdots
\]
使两乘积中 $y$ 的同次幂的系数相等,得
\[
\begin{aligned}
P p & =P p \\
P a+Q p & =p A+q P \\
P b+Q a+R p & =p B+q A+r P \\
P c+Q b+R a+S p & =p C+q B+r A+s P
\end{aligned}
\]
等. 连同 $P p=P p$, 这组等式的个数为 $m+n$. 从这组等式确定出待定字母 $A, B, C, \cdots, a$, $b, c, \cdots$, 结果方程将如我们所要求的, 只含 $P, Q, R, \cdots, p, q, r, \cdots$.

\section{$\S 485$}

引进新的待定量 $\alpha, \beta, \gamma, \cdots$, 可以使待定字母的确定大为简化. 我们用例子对此作 说明.

设给定的两个方程为
\[
\begin{gathered}
\mathrm{I} \\
P y^{2}+Q y+R=0 \\
\text { II } \\
p y^{3}+q y^{2}+r y+s=0
\end{gathered}
\]
分别以 $p y^{2}+a y+b$ 和 $P y+A$ 乘 I 和 II, 得
\[
\begin{gathered}
P p=P p \\
P a+Q p=p A+q P=\alpha \\
P b+Q a+R p=q A+r P=\beta \\
Q b+R a=r A+s P \\
R b=s A
\end{gathered}
\]
去掉恒等的第一个方程, 由第二个方程得
\[
\begin{aligned}
& a=\frac{\alpha-Q p}{P} \\
& A=\frac{\alpha-q P}{p}
\end{aligned}
\]
由第三个方程得
\[
b=\frac{\beta}{P}-\frac{Q a}{P}-\frac{R p}{P}=\frac{\beta}{P}-\frac{\alpha Q}{P^{2}}+\frac{Q^{2} p}{P^{2}}-\frac{R p}{P}
\]
和
\[
\beta=\frac{\alpha q}{p}-\frac{q^{2} P}{p}+r P
\]
将 $\beta$ 的这个值代入 $b$ 的表达式, 得
\[
b=\frac{\alpha q}{P p}-\frac{q^{2}}{p}+r-\frac{\alpha Q}{P^{2}}+\frac{Q^{2} p}{P^{2}}-\frac{R p}{P}
\]
或
\[
b=\frac{\alpha(P q-Q p)}{P^{2} p}+\frac{Q^{2} p^{2}-P^{2} q^{2}}{R^{2} p}+\frac{P r-R p}{P}
\]
将 $b$ 的这个表达式代入第四个方程, 得
\[
\begin{gathered}
\frac{\alpha Q(P q-Q p)}{P^{2} p}-\frac{Q(P q-Q p)(Q p+P q)}{P^{2} p}+ \\
\frac{Q(P r-R p)}{P}+\frac{\alpha R}{P}-\frac{R Q p}{P}=\frac{r(\alpha-q P)}{p}+s P
\end{gathered}
\]
乘以 $P^{2} p$, 得
\[
\begin{aligned}
& \alpha Q(P q-Q p)+\alpha P(R p-P r)-Q(P q-Q p)(P q+Q p)+ \\
& P Q p(P r-2 R p)+P^{3} q r-P^{3} p s=0
\end{aligned}
\]
从而
\[
\alpha=\frac{P^{2} Q q^{2}-Q^{3} p^{2}-P^{2} Q p r+2 P Q R p^{2}-P^{3} q r+P^{3} p s}{P Q q-Q^{2} p+P R p-P^{2} r}
\]
第五个方程给出
\[
\frac{\alpha R(P q-Q p)}{P^{2} p}-\frac{R\left(P^{2} q^{2}-Q^{2} p^{2}\right)}{P^{2} p}+\frac{R(P r-R p)}{P}=\frac{\alpha s}{p}-\frac{P q s}{p}
\]
从而
\[
\alpha=\frac{P^{2} R q^{2}-Q^{2} R p^{2}-P^{2} R p r+P R^{2} p^{2}-P^{3} q_{s}}{P R q-Q R p-P^{2} s}
\]
$\alpha$ 的这两个表达式给出我们所要的方程, 形状同于 $\$ 480$ 所得. 

