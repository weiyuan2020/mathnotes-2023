\chapter{第六章二阶线分类}

\section{$\S 131$}

前\所讲性质, 不是只属于某些二阶线的, 而是属于所有二阶线的, 虽然二阶线具有 那么多共有性质, 但它们的形状却可以极为不同. 因此, 我们根据形状对二阶线进行分 类,并对各类所独有的特性进行考察.

\section{$\S 132$}

只需变动轴和原点的位置, 就可使二阶线通用方程的形状成为
\[
y^{2}=\alpha+\beta x+\gamma x^{2}
\]
$x, y$ 是直角坐标. 由此我们看到, 对每一个横标 $x$, 纵标 $y$ 都有一正一负两个值, 也即横标 $x$ 所在的轴分曲线为相同的两部分, 我们称这样的轴为曲线的正交直径. 凡二阶线都有 正交直径, 我们取它为横标轴.

\section{$\S 133$}

前节方程含 $\alpha, \beta, \gamma$ 三个常数,这三个常数取值种数无穷, 相应地也就有无穷多条曲 线, 形状可以或多或少有所不同. 改变原点在轴上的位置, 也即横标 $x$ 加上或减去一个 数, 这样从方程 $y^{2}=\alpha+\beta x+\gamma x^{2}$ 得到的这无穷多条曲线, 它们的大小和形状都是相同 的,不同的方程所表示的曲线,可以形状完全一样,只是大小不同.例如半径不同的圆,种 数也无穷. 可见 $\alpha, \beta, \gamma$ 的改变, 可以不改变二阶线的形状,即不改变二阶线的类.

\section{$\S 134$}

系数 $\gamma$ 的正负对方程
\[
y^{2}=\alpha+\beta x+\gamma x^{2}
\]
所定曲线影响最大. 先看 $\gamma$ 为正的情形: $x=+\infty$,则 $\gamma x^{2}$ 比 $\alpha+\beta x$ 大得多,因而表达式 $\alpha+\beta x+\gamma x^{2}$ 为正. 从而纵标 $y$ 有两个无穷大值, 一正一负; $x=-\infty$, 表达式 $\alpha+\beta x+\gamma x^{2}$ 的值也为正无穷大. 可见, $\gamma$ 为正时二阶线有四条伸向无穷的分支, $x=+\infty$ 和 $x=-\infty$ 时 各两条. 我们把有四条伸向无穷的分支的二阶线归为一类, 叫双曲线. 

\section{$\S 135$}

再看 $\gamma$ 为负的情形: $x=+\infty$ 和 $x=-\infty$ 时, 表达式 $\alpha+\beta x+\gamma x^{2}$ 都为负, 从而纵标都 为虚数. 可见 $\gamma$ 为负时曲线的纵标横标都不能为无穷, 即曲线没有伸向无穷的分支,整个 曲线位于一个确定的有限范围之内. 我们把这样的二阶线,也即 $\gamma$ 为负时方程
\[
y^{2}=\alpha+\beta x+\gamma x^{2}
\]
所定曲线归为一类, 叫椭圆.

\section{$\S 136$}

$\gamma$ 为正为负二阶线的性质有截然的不同. 根据 $\gamma$ 为正为负我们决定了两类二阶线. 现 在我们看 $\gamma=0$ 的情形, 0 在正负之间, 这时的曲线性质也在双曲线与椭圆之间. 我们称 $\gamma=0$ 时的二阶线为抛物线. 此时方程为 $y^{2}=\alpha+\beta x . \beta$ 为正为负均可, $\beta$ 为负时取 $x$ 为负就 成了 $\beta$ 为正 $x$ 为正的情形. 取 $\beta$ 为正, 则 $x$ 趋向无穷时纵标 $y$ 也趋向无穷, 且一正一负,即 此时抛物线有趋向无穷的两个分支, 且只有两个分支, $\beta$ 为正时, $x=-\infty$, 则 $y$ 为虚数.

\section{$\S 137$}

这样我们有三类二阶线: 椭圆,抛物线和双曲线, 彼此分得清清楚楚,完全不会混淆. 主要的判别标准是无穷分支的条数. 椭圆位于一个有限的范围之内, 没有无穷分支, 抛物 线有两条无穷分支, 双曲线有四条. 上一章我们考虑的是圆雉曲线共有的性质, 本章我们 考虑每类圆雉曲线所特有的性质.

\section{$\S 138$}

先考虑椭圆 (图 31 ), 方程为
\[
y^{2}=\alpha+\beta x-\gamma x^{2}
\]
横标在正交直径上, 坐标原点可移动, 移动长度为 $\frac{\beta}{2 \gamma}$ 的一段, 则方程变为
\[
y^{2}=\alpha-\gamma x^{2}
\]
此时横标以图形中心为起点. 记中心为 $C$, 正交直径为 $A B$, 则横标 $C P=x$, 纵标 $P M=y$. 取 $x=\pm \sqrt{\frac{\alpha}{\gamma}}$, 则 $y=0 ; x$ 大于 $+\sqrt{\frac{\alpha}{\gamma}}$ 或小于 $-\sqrt{\frac{\alpha}{\gamma}}$, 则 $y$ 为虚数,即曲线在这两个数所 限定的范围之内. 因而我们有 $C A=C B=\sqrt{\frac{\alpha}{\gamma}}$. 取 $x=0$, 得 $C D=C E=\sqrt{\alpha}$. 记半直径或半轴 $C A=C B=a$, 记共轭半轴 $C D=C E=b$, 则 $\alpha=b^{2}, \gamma=\frac{b^{2}}{a^{2}}$. 从而椭圆方程成为 FInfinile anallyisis (无穷分析引论) Silraduclion


【图,待补】
%%![](https://cdn.mathpix.com/cropped/2023_02_05_68d01d12d0cf0d29b13dg-17.jpg?height=441&width=683&top_left_y=275&top_left_x=489)

图 31

\section{$\S 139$}

共轭半轴 $a, b$ 相等时, 椭圆变为圆, 此时方程变为 $y^{2}=a^{2}-x^{2}$ 或 $y^{2}+x^{2}=a^{2}, C M=$ $\sqrt{x^{2}+y^{2}}=a$, 即曲线上的点到中心 $C$ 的距离都相等. 这是圆的性质. 如果 $a, b$ 不相等, 即 $A B$ 大于 $D E$ 或 $D E$ 大于 $A B$, 那么曲线的形状就成了扁长的. 共轭轴 $A B, D E$ 可以换位, 我们取长轴作 $A B$, 从而 $a$ 大于 $b$, 且中心至椭圆焦点 $F, G$ 的距离相等, $C F=C G=$ $\sqrt{a^{2}-b^{2}}$, 椭圆半参数, 即焦点处的纵标为 $\frac{b^{2}}{a}$.

\section{$\S 140$}

从椭圆的任一点 $M$ 向焦点引直线 $F M, G M$, 如我们所指出过的, 有
\[
\begin{gathered}
F M=A C-\frac{C F \cdot C P}{A C}=a-\frac{x \sqrt{a^{2}-b^{2}}}{2} \\
G M=\frac{a+x \sqrt{a^{2}-b^{2}}}{a}, \quad F M+G M=2 a
\end{gathered}
\]
即椭圆上任一点 $M$ 至焦点距离 $F M, G M$ 之和都等于长轴 $A B=2 a$ 为常数. 这正是一个有 名的椭圆画法的依据.

\section{$\S 141$}

作点 $M$ 处的切线 $T M t$, 交轴于 $T$ 和 $t$, 则根据前面所讲, 有
\[
C P: C A=C A: C T
\]
或 $C T=\frac{a^{2}}{x}$. 类似地, 有 $C t=\frac{b^{2}}{y}$, 从而
\[
T P=\frac{a^{2}}{x}-x, \quad T F=\frac{a^{2}}{x}-\sqrt{a^{2}-b^{2}}, \quad T A=\frac{a^{2}}{x}-a
\]
由
\[
T P=\frac{a^{2}-x^{2}}{x}=\frac{a^{2} y^{2}}{b^{2} x}, \quad T M=\frac{y \sqrt{b^{4} x^{2}+a^{4} y^{2}}}{b^{2} x}
\]
得
\[
\begin{gathered}
\tan \angle C T M=\frac{b^{2} x}{a^{2} y}, \quad \sin \angle C T M=\frac{b^{2} x}{\sqrt{b^{4} x^{2}+a^{4} y^{2}}} \\
\cos \angle C T M=\frac{a^{2} y}{\sqrt{b^{4} x^{2}+a^{4} y^{2}}}
\end{gathered}
\]
因而过点 $A$ 作轴的垂线 $A V$ ,它也是曲线的切线, 则利用 $a y=b \sqrt{a^{2}-x^{2}}$, 我们有
\[
A V=\frac{a(a-x)}{x} \frac{b^{2} x}{a^{2} y}=\frac{b^{2}(a-x)}{a y}=b \sqrt{\frac{a-x}{a+x}}
\]
\section{$\S 142$}

由
\[
F T=\frac{a^{2}-x \sqrt{a^{2}-b^{2}}}{x}, \quad F M=\frac{a^{2}-x \sqrt{a^{2}-b^{2}}}{a}
\]
得 $F T: F M=a: x$. 类似地, 由
\[
G T=\frac{a^{2}+x \sqrt{a^{2}-b^{2}}}{x}, \quad G M=\frac{a^{2}+x \sqrt{a^{2}-b^{2}}}{a}
\]
得 $G T: G M=a: x$, 从而 $F T: F M=G T: G M$, 再由
\[
\begin{aligned}
& F T: F M=\sin \angle F M T: \sin \angle C T M \\
& G T: G M=\sin \angle G M t: \sin \angle C T M
\end{aligned}
\]
得 $\sin \angle F M T=\sin \angle G M t$, 从而 $\angle F M T=\angle G M t$, 即从两个焦点引向曲线任一点 $M$ 的 直线, 同 $M$ 处切线的夹角相等, 这是焦点的最重要的性质.

\section{$\S 143$}

由 $G T: G M=a: x, C T=\frac{a^{2}}{x}$ 得 $C T: C A=a: x$, 从而 $G T: G M=C T: C A$. 因此, 自 $C$ 引 $C S$ 平行于 $G M$, 交切线于 $S$, 则 $C S=C A=a$. 同样地, 从 $C$ 向切线引平行于 $F M$ 的直线, 它也等于 $C A=a$. 知 $T M=\frac{y}{b^{2} x} \sqrt{b^{4} x^{2}+a^{4} y^{2}}$, 将 $a^{2} y^{2}=a^{2} b^{2}-b^{2} x^{2}$ 代入, 得
\[
T M=\frac{y}{b x} \sqrt{a^{4}-x^{2}\left(a^{2}-b^{2}\right)}
\]
将 $F T \cdot G T=\frac{a^{4}-x^{2}\left(a^{2}-b^{2}\right)}{x^{2}}$ 代入, 得
\[
T M=\frac{y}{b} \sqrt{F T \cdot G T}
\]
由 $T G: T C=T M: T S$, 得
\[
T S=\frac{T M \cdot C T}{T G}
\]
从而
\[
T S=\frac{y \cdot C T}{b} \sqrt{\frac{F T}{G T}}=\frac{y \cdot C T \cdot F T}{b \sqrt{F T \cdot G T}}=\frac{y^{2} \cdot C T \cdot F T}{b^{2} \cdot T M}
\]
又我们有 $P T=\frac{a^{2} y^{2}}{b^{2} x}=\frac{C T \cdot y^{2}}{b^{2}}$ 和 $T S=\frac{P T \cdot F T}{T M}$, 因而
\[
T M: P T=F T: T S
\]
可见 $\triangle T M P$ 和 $\triangle T F S$ 相似. 因而从焦点 $F$ 引向切线的直线 $F S$ 垂直于切线. 类似地, 我 们也得到 $S V=\frac{A F \cdot M V}{G M}$.

\section{$\S 144$}

因而从焦点 $F$ 向切线引垂线 $F S$, 则点 $S$ 与中心 $C$ 的连线 $C S$ 等于长半轴 $A C=a$. 由于 $T M: y=T F: F S$, 得到
\[
F S=\frac{y \cdot T F}{T M}=\frac{b \cdot T F}{\sqrt{F T \cdot G T}}=b \sqrt{\frac{F T}{G T}}
\]
从而
\[
G T: F T=G M: F M=C D^{2}: F S^{2}
\]
从另一个焦点向切线引的垂线等于 $b \sqrt{\frac{G T}{F T}}$. 因而短半轴 $C D=b$ 是这两条垂线的比例中 项. 现在我们从中心 $C$ 向切线引垂线 $C Q$, 则 $T F: F S=C T: C Q$, 因而
\[
C Q=\frac{b \cdot C T}{\sqrt{F T \cdot G T}}=\frac{b x \cdot C T}{a \sqrt{F M \cdot G M}}=\frac{a b}{\sqrt{F M \cdot C M}}
\]
如果引平行于切线的直线 $F X$, 则
\[
C Q-F S=\frac{b \cdot C F}{\sqrt{F T \cdot G T}}=C X
\]
由此得
\[
C Q-C X=\frac{b \cdot T F}{\sqrt{F T \cdot G T}}, \quad C Q+C X=\frac{b \cdot T G}{\sqrt{F T \cdot G T}}
\]
从而
\[
C Q^{2}-C X^{2}=b^{2}, \quad C X=\sqrt{C Q^{2}-b^{2}}
\]
我们得到, 如果短轴已给, 那么在垂线 $C Q$ 上可以找到一点 $X$, 使得 $X$ 与焦点 $F$ 的连 线垂直于 $C Q$. 

\section{$\S 145$}

前面讲了焦点性质, 现在我们来看任何两条共轭直径. 设 $C M$ 是半直径, 过中心平行 于切线 $T M$ 的直线 $C K$ 就是 $C M$ 的共轭直径. 记 $C M=p, C K=q, \angle M C K=\angle C M T=s$. 前 面我们看到了 $p^{2}+q^{2}=a^{2}+b^{2}, p q \sin s=a b$, 我们有
\[
\begin{gathered}
p^{2}=x^{2}+y^{2}=b^{2}+\frac{\left(a^{2}-b^{2}\right) x^{2}}{a^{2}} \\
q^{2}=a^{2}+b^{2}-p^{2}=a^{2}-\frac{\left(a^{2}-b^{2}\right) x^{2}}{a^{2}}=F M \cdot G M
\end{gathered}
\]
还有 $p^{2}=F K \cdot G K$. 再由 $C Q=\frac{a b}{\sqrt{F M \cdot G M}}$, 得
\[
\sin \angle C M Q=\sin s=\frac{a b}{p \sqrt{F M \cdot G M}}
\]
我们得到
\[
T M: T P=\frac{y}{b} \sqrt{F T \cdot G T}: \frac{a^{2} y^{2}}{b^{2} x}=\sqrt{F M \cdot G M}: \frac{a y}{b}=C K: C R
\]
从而
\[
C R=\frac{a y}{b}, \quad K R=\frac{b x}{a}
\]
继而
\[
C R \cdot K R=C P \cdot P M
\]
我们有
\[
\sin \angle F M S=\frac{b}{\sqrt{G M \cdot F M}}=\frac{b}{q}
\]
由
\[
x=C P=\frac{a \sqrt{p^{2}-b^{2}}}{\sqrt{a^{2}-b^{2}}}, \quad y=\frac{b \sqrt{a^{2}-p^{2}}}{\sqrt{a^{2}-b^{2}}}=P M
\]
及
\[
C R=\frac{a \sqrt{a^{2}-p^{2}}}{\sqrt{a^{2}-b^{2}}}, \quad K R=\frac{b \sqrt{p^{2}-b^{2}}}{\sqrt{a^{2}-b^{2}}}
\]
得
\[
\tan \angle A C M=\frac{y}{x}, \quad \tan 2 \angle A C M=\frac{2 y x}{x^{2}-y^{2}}=\frac{2 a b \sqrt{\left(a^{2}-p^{2}\right)\left(p^{2}-b^{2}\right)}}{\left(a^{2}+b^{2}\right) p^{2}-2 a^{2} b^{2}}
\]
将
\[
\begin{gathered}
a b=p q \cdot \sin s, \quad a^{2}+b^{2}=p^{2}+q^{2} \\
\sqrt{\left(a^{2}-p^{2}\right)\left(p^{2}-b^{2}\right)}=-p q \cos s
\end{gathered}
\]
代入, 由 $\cos s$ 为负, 得

%%04p061-080
$\tan 2 \angle A C M=\frac{-q^{2} \sin 2 s}{p^{2}+q^{2} \cos 2 s}$

最后我们有 $C K^{2}=M T \cdot M t$. 从以上结果得到
\[
M V=q \sqrt{\frac{A P}{B P}}, \quad A V=b \sqrt{\frac{A P}{B P}}
\]
从而
\[
A V: M V=b: q=C E: C K
\]
这样一来,画出直线 $A M, E K$, 则它们平行.

\section{$\S 146$}

由 $p q \sin s=a b$ 知 $p q$ 大于 $a b$, 再由 $p^{2}+q^{2}=a^{2}+b^{2}$ 知 $p$ 与 $q$ 比 $a$ 与 $b$ 靠得近, 即共轭直 径中正交直径之间的差别最大. 我们来求相等的共轭直径, 即 $q=p$. 此时我们有
\[
\begin{aligned}
& 2 p^{2}=a^{2}+b^{2}, \quad p=q=\sqrt{\frac{a^{2}+b^{2}}{2}} \\
& \sin s=\frac{2 a b}{a^{2}+b^{2}}, \quad \cos s=\frac{-a^{2}+b^{2}}{a^{2}+b^{2}}
\end{aligned}
\]
从而
\[
\sin \frac{1}{2} s=\sqrt{\frac{a^{2}}{a^{2}+b^{2}}}, \quad \cos \frac{1}{2} s=\sqrt{\frac{b^{2}}{a^{2}+b^{2}}}
\]
继而
\[
\tan \frac{1}{2} s=\frac{a}{b}=\tan \angle C E B, \quad \angle M C K=2 \angle C E B=\angle A E B
\]
进而
\[
C P=\frac{a}{\sqrt{2}}, \quad C M=\frac{b}{\sqrt{2}}
\]
我们得到彼此相等的共轭半径 $C M, C K$, 它们分别平行于弦 $A E$ 和 $B E$.

\section{$\S 147$}

如果取顶点 $A$ 作原点, 也即取 $A P=x, P M=y$, 则原来的 $x$ 变为现在的 $a-x$, 得到现 在的方程为
\[
y^{2}=\frac{b^{2}}{a^{2}}\left(2 a x-x^{2}\right)=\frac{2 b^{2}}{a} x-\frac{b^{2}}{a^{2}} x^{2}
\]
这里, 显然 $\frac{2 b^{2}}{a}$ 是椭圆的参数或 latus rectum. 记半参数, 即焦点处的纵标为 $c$, 记焦点至顶 点的距离 $A F=d$, 则
\[
\frac{b^{2}}{a}=c, \quad a-\sqrt{a^{2}-b^{2}}=d=a-\sqrt{a^{2}-a c}
\]
从而
\[
2 a d-d^{2}=a c, \quad a=\frac{d^{2}}{2 d-c}
\]
进而
\[
y^{2}=2 c x-\frac{c(2 d-c) x^{2}}{d^{2}}
\]
这是在直角坐标 $x, y$ 之下的椭圆方程, 这里横标被置于主轴 $A B$ 之上, 并以顶点 $A$ 为原 点, 且顶点 $A$ 至焦点 $F$ 的距离 $A F=d$, 半参数为 $c$. 这里还应该指出, $2 d$ 恒大于 $c$, 因为
\[
A C=a=\frac{d^{2}}{2 d-c}, \quad C D=b=d \sqrt{\frac{c}{2 d-c}}
\]
\section{$\S 148$}

如果 $2 d=c$, 则 $y^{2}=2 c x$. 这是我们前面见过的抛物线方程 (图 32). 将横标移动 $\frac{\alpha}{\beta}$, 前 面的方程 $y^{2}=\alpha+\beta x$ 就化为我们这里的方程. 设 $M A N$ 是抛物线, 其性质由横标 $A P=x$ 与纵标 $P M=y$ 间方程 $y^{2}=2 c x$ 描述. 因而焦点到顶点的距离 $A F=d=\frac{1}{2} c$, 半参数 $F H=$ $c$, 且对曲线上的任一点 $M$ 都有 $P M^{2}=2 F H \cdot A P$. 由此得到, 如果横标 $A P$ 无穷, 则纵标 $P M, P N$ 都无穷, 即曲线在轴 $A P$ 的两侧都趋向无穷. 横标取负值, 则纵标为虚数, 也即以 点 $A$ 为界,与 $T$ 同在一侧的横标点不对应曲线上的点.


【图,待补】
%%![](https://cdn.mathpix.com/cropped/2023_02_05_39e6b491fce1d5f0c077g-02.jpg?height=474&width=744&top_left_y=1362&top_left_x=461)

图 32

\section{$\S 149$}

$2 d=c$ 时椭圆方程变为抛物线方程, 可见, 抛物线是半轴 $a=\frac{d^{2}}{2 d-c}$ 变为无穷时的椭 圆,因而视轴 $a$ 为无穷, 椭圆的性质就变为抛物线的性质. 首先, 由 $A F=\frac{1}{2} c$ 得 

$FP=x-\frac{1}{2}c$

因此由焦点 $F$ 向曲线上点 $M$ 引直线 $F M$, 则
\[
F M^{2}=x^{2}-c x+\frac{1}{4} c^{2}+y^{2}=x^{2}+c x+\frac{1}{4} c^{2}
\]
从而
\[
F M=x+\frac{1}{2} c=A P+A F
\]
这是抛物线焦点的主要性质.

\section{$\S 150$}

椭圆长轴无限增大, 则成为抛物线. 因而我们把抛物线看成轴 $A C=a$ 无穷大, 即中心 $C$ 距顶点 $A$ 无穷远的椭圆. 引曲线上点 $M$ 处的切线 $M T$, 交轴于 $T$. 由
\[
C P: C A=C A: C T
\]
利用 $C P=a-x$, 得
\[
C T=\frac{a^{2}}{a-x}
\]
由此得 $A T=\frac{a x}{a-x}, a$ 为无穷大, $x$ 与 $a$ 相比, 可以不计, 可以 $a-x=a$, 从而 $A T=x=A P$.

这一点也可以用下面的方法证明: 由 $A T=\frac{a x}{a-x}$, 得 $A T=x+\frac{x^{2}}{a-x}$. 该表达式中分母为 无穷大, 分子为有限数, 因而分数可略去, 得到 $A T=A P=x$.

\section{$\S 151$}

因而从点 $M$ 向抛物线的无穷远中心引直线 $M C$, 则 $M C$ 平行于轴 $A C$. 因而它也是直 径, 它等分平行于切线 $M T$ 的所有弦. 例如, 画平行于切线 $M T$ 的弦 $m n$, 则直径 $M p$ 在 $p$ 点处等分它, 即拋物线的平行于轴 $A P$ 的任何一条直线都是斜角直径. 为讨论这类直径 的性质, 我们记 $M p=t, p m=u$, 并从 $m$ 向轴引垂线 $m s r$. 这样, 由 $P T=2 x, M T=$ $\sqrt{4 x^{2}+2 c x}$, 得
\[
\sqrt{4 x^{2}+2 c x}: 2 x: \sqrt{2 c x}=p m: p s: m s
\]
由此得
\[
p s=\frac{2 x u}{\sqrt{4 x^{2}+2 c x}}=u \sqrt{\frac{x}{2 x+c}}, \quad m s=u \sqrt{\frac{c}{2 x+c}}
\]
从而
\[
A r=x+t+u \sqrt{\frac{2 x}{2 x+c}}, \quad m r=\sqrt{2 c x}+u \sqrt{\frac{c}{2 x+c}}
\]
由 $m r^{2}=2 c \cdot A r$, 得
\[
2 c x+2 c u \sqrt{\frac{2 x}{2 x+c}}+\frac{c u^{2}}{2 x+c}=2 c x+2 c t+2 c u \sqrt{\frac{2 x}{2 x+c}}
\]
从而
\[
u^{2}=2 t(2 x+c)=4 F M \cdot t
\]
即
\[
p m^{2}=4 F M \cdot M p
\]
$\angle m p s$ 的正弦和余弦为
\[
\sqrt{\frac{c}{2 x+c}}=\sqrt{\frac{A F}{F M}}, \quad \sqrt{\frac{2 x}{2 x+c}}=\sqrt{\frac{A P}{F M}}
\]
从而
\[
\sin 2 \angle m p s=\frac{2 \sqrt{2 c x}}{2 x \pm c}=\frac{y}{F M}=\sin \angle M F P
\]
这样一来, $\angle m p s=\angle M T P=\frac{1}{2} \angle M F r$.

\section{$\S 152$}

由 $M F=A P+A F$ 及 $A P=A T$ 得 $F M=F T$, 因而 $\triangle M F T$ 等腰, $\angle M F r=2 \angle M T A$, 这是我们刚得到的. 又由 $M T=2 \sqrt{x\left(x+\frac{1}{2} c\right)}$, 得 $M T=2 \sqrt{A P \cdot F M}$. 因而由焦点 $F$ 向 切线引垂线 $F S$, 则
\[
M S=T S=\sqrt{A P \cdot F M}=\sqrt{A T \cdot T F}
\]
由此得 $A T: T S=T S: T F$. 从这一比式我们看到, 点 $S$ 位于过顶点 $A$ 垂直于轴的直线 $A S$ 上,但
\[
A S=\frac{1}{2} P M, \quad A S: T S=A F: F S
\]
从而 $F S=\sqrt{A F \cdot F M}$, 即 $F S$ 是 $A F$ 和 $F M$ 的比例中项, 此外还有
\[
A S: M S=A S: T S=F S: F M=\sqrt{A F}: \sqrt{F M}
\]
如果引切点 $M$ 处的法线 $M W$ 交轴于 $W$, 则
\[
P T: P M=P M: P W \text {, 也即 } 2 x: \sqrt{2 c x}=\sqrt{2 c x}: P W
\]
从而 $P W=c$. 我们得到, 轴上位于纵标 $P M$ 和法线 $W M$ 之间的线段 $P W$ 为常数, 等于半参 数或纵标 $F H$. 此外还有
\[
F W=F T=F M, \quad M W=2 \sqrt{A F \cdot F M}
\]
\section{$\S 153$}

现在我们讨论双曲线, 置横标于正交直径上时, 双曲线的方程为 
\[
\begin{aligned}
& y^{2}=\alpha+\beta x+\gamma x^{2}
\end{aligned}
\]
如果原点移动距离 $\frac{\beta}{2 \gamma}$, 也即, 使原点与中心重合, 则该方程变为 $y^{2}=\alpha+\gamma x^{2}$, 这里 $\gamma$ 必须 为正, $\alpha$ 可正可负, 坐标 $x, y$ 交换时, $\alpha$ 改变符号. 我们假定 $\alpha$ 为负, 即 $y^{2}=\gamma x^{2}-\alpha$, 显然
\[
x=\sqrt{\frac{\alpha}{\gamma}} \text { 或 } x=-\sqrt{\frac{\alpha}{\gamma}}
\]
时纵标 $y$ 都为零. 记中心为 $C$, 记曲线与轴的交点为 $A, B$ (图 33), 记半轴 $C A=C B=a$, 则 $a=\sqrt{\frac{\alpha}{\gamma}}, \alpha=\gamma a^{2}$, 从而
\[
y^{2}=\gamma x^{2}-\gamma a^{2}
\]
可见, $x^{2}<a^{2}$ 时 $y$ 为虚数, 也即, 轴上 $A$ 到 $B$ 这一段的 $x$ 不对应曲线上的点. $x^{2}>a^{2}$ 时纵 标连续增加, 并趋向无穷, 从而双曲线有 4 个分支: $A I, A i, B K, B k$, 它们相似相等. 这是 双曲线的主要性质.


【图,待补】
%%![](https://cdn.mathpix.com/cropped/2023_02_05_39e6b491fce1d5f0c077g-05.jpg?height=675&width=845&top_left_y=960&top_left_x=434)

图 33

\section{$\S 154$}

$x=0$ 时 $y^{2}=-\gamma a^{2}$, 即在中心处纵标为负, 所以跟椭圆不同, 双曲线没有实共轭轴, 为与椭圆类比, 我们取 $b \sqrt{-1}$ 为共轭虚轴, 则 $\gamma a^{2}=b^{2}, \gamma=\frac{b^{2}}{a^{2}}$. 因而, 记横标 $C P=x$, 纵标 $P M=y$, 则
\[
y^{2}=\frac{b^{2}}{a^{2}}\left(x^{2}-a^{2}\right)
\]
将 $b^{2}$ 换为 $-b^{2}$, 前面求得的椭圆方程 

$y^{2}=\frac{b^{2}}{a^{2}}\left(a^{2}-x^{2}\right)$

就变成双曲线方程. 由于它们之间的这种关系, 前面求出的椭圆的性质, 可以容易地转换 成双曲线的性质. 首先关于焦点至中心的距离, 由椭圆为 $\sqrt{a^{2}-b^{2}}$, 得双曲线为 $C F=$ $C G=\sqrt{a^{2}+b^{2}}$. 由此得
\[
F P=x-\sqrt{a^{2}+b^{2}}, \quad G P=x+\sqrt{a^{2}+b^{2}}
\]
利用这两个结果 $y^{2}=-b^{2}+\frac{b^{2} x^{2}}{a^{2}}$, 得
\[
\begin{aligned}
& F M=\sqrt{a^{2}+x^{2}+\frac{b^{2} x^{2}}{a^{2}}-2 x \sqrt{a^{2}+b^{2}}}=\frac{x \sqrt{a^{2}+b^{2}}}{a}-a \\
& G M=\sqrt{a^{2}+x^{2}+\frac{b^{2} x^{2}}{a^{2}}+2 x \sqrt{a^{2}+b^{2}}}=\frac{x \sqrt{a^{2}+b^{2}}}{a}+a
\end{aligned}
\]
因而对连接两焦点与曲线上一点 $M$ 的直线 $F M, G M$, 我们有
\[
F M+A C=\frac{C P \cdot C F}{C A}, \quad G M-A C=\frac{C P \cdot C F}{C A}
\]
又这两条直线的差 $G M-F M=2 A C$. 在椭圆, 是这两条线的和等于主轴 $A B$; 类似地, 我 们得到了, 双曲线是这两条线的差等于主轴 $A B$.

\section{$\S 155$}

可以确定切线 $M T$ 的位置. 由 $C P: C A=C A: C T$ 对二阶曲线都成立, 得
\[
C T=\frac{a^{2}}{x}, \quad P T=\frac{x^{2}-a^{2}}{x}=\frac{a^{2} y^{2}}{b^{2} x}
\]
从而
\[
M T=\frac{y}{b^{2} x} \sqrt{b^{4} x^{2}+a^{4} y^{2}}=\frac{y}{b x} \sqrt{a^{2} x^{2}+b^{2} x^{2}-a^{4}}
\]
由
\[
F M \cdot G M=\frac{a^{2} x^{2}+b^{2} x^{2}-a^{4}}{a^{2}}
\]
得 $M F=\frac{a y}{b x} \sqrt{F M \cdot G M}$, 由
\[
F T=\sqrt{a^{2}+b^{2}}-\frac{a^{2}}{x}, \quad G T=\sqrt{a^{2}+b^{2}}+\frac{a^{2}}{x}
\]
得
\[
F T: F M=a: x, \quad G T: G M=a: x
\]
从而 $F T: G T=F M: G M$, 这一比式表明, 切线 $M T$ 等分 $\angle F M G$, 从而 $\angle F M T=$ $\angle G M T . C M$ 的延长线是斜直径, 它等分所有平行于切线 $M T$ 的弦. 

\section{$\S 156$}

从中心 $C$ 向切线引垂线 $C Q$, 则
\[
T M: P T: P M=C T: T Q: C Q
\]
或
\[
\frac{a y}{b x} \sqrt{F M \cdot G M}: \frac{a^{2} y^{2}}{b^{2} x}: y=\frac{a^{2}}{x}: T Q: C Q
\]
从而
\[
T Q=\frac{a^{3} y}{b x \sqrt{F M \cdot G M}}, \quad C Q=\frac{a b}{\sqrt{F M \cdot G M}}
\]
从焦点 $F$ 向切线引垂线 $F S$, 则
\[
T M: P T: P M=F T: T S: F S
\]
或
\[
\frac{a y}{b x} \sqrt{F M \cdot G M}: \frac{a^{2} y^{2}}{b^{2} x}: y=\frac{a \cdot F M}{x}: T S: F S
\]
从而
\[
T S=\frac{a^{2} y \cdot F M}{b x \sqrt{F M \cdot G M}}, \quad F S=\frac{b \cdot F M}{\sqrt{F M \cdot G M}}
\]
同样地, 从焦点 $G$ 向切线引垂线 $G s$, 则
\[
T s=\frac{a^{2} y \cdot G M}{b x \sqrt{F M \cdot G M}}, \quad G s=\frac{b \cdot G M}{\sqrt{F M \cdot G M}}
\]
利用以上结果,得
\[
\begin{gathered}
T S \cdot T s=\frac{a^{4} y^{2}}{b^{2} x^{2}}=\frac{a^{2}\left(x^{2}-a^{2}\right)}{x^{2}}=C T \cdot P T \\
T S: C T=P T: T s
\end{gathered}
\]
和 $F S \cdot G s=b^{2}$. 由 $Q S=Q s$, 得
\[
Q S=\frac{T S+T s}{2}=\frac{a^{2} y(F M+G M)}{2 b x \sqrt{F M \cdot G M}}=\frac{a y \sqrt{a^{2}+b^{2}}}{b \sqrt{F M \cdot G M}}=Q s
\]
从而
\[
\begin{aligned}
C S^{2} & =C Q^{2}+Q S^{2} \\
& =\frac{a^{2} b^{4}+a^{4} y^{2}+a^{2} b^{2} y^{2}}{b^{2} \cdot F M \cdot G M} \\
& =\frac{a^{2} b^{4}\left(a^{2}+b^{2}\right)\left(b^{2} x^{2}-a^{2} b^{2}\right)}{b^{2} \cdot F M \cdot G M} \\
& =\frac{\left(a^{2}+b^{2}\right) x^{2}-a^{4}}{F M \cdot G M}=a^{2}
\end{aligned}
\]
跟椭圆一样, 我们得到直线 $C S=a=C A$. 我们还得到 
\[
\begin{gathered}
\text { Suffinile analyjiis (无穷分析引论 Snlraduclicn } \\
C Q+F S=\frac{b x \sqrt{a^{2}+b^{2}}}{a \sqrt{F M \cdot G M}}
\end{gathered}
\]
从而
\[
(C Q+F S)^{2}-C Q^{2}=\frac{b^{2} x^{2}\left(a^{2}+b^{2}\right)-a^{4} b^{2}}{a^{2} \cdot F M \cdot G M}=b^{2}
\]
因而, 如果从焦点 $F$ 平行于切线引直线 $F X$, 交垂线 $C Q$ 的延长线于 $X$, 则 $C X=$ $\sqrt{b^{2}+C Q^{2}}$. 这条性质也与椭圆的类似.

\section{$\S 157$}

从顶点 $A, B$ 引垂直于轴的直线, 交切线于 $V, v$, 则由
\[
A T=\frac{a(x-a)}{x}, \quad B T=\frac{a(x+a)}{x}
\]
得 $P T: P M=A T: A V=B T: B v$, 从而
\[
A V=\frac{b^{2}(x-a)}{a y}, \quad B v=\frac{b^{2}(x+a)}{a y}
\]
进而
\[
A V \cdot B v=\frac{b^{4}\left(x^{2}-a^{2}\right)}{a^{2} y^{2}}=b^{2}
\]
也即
\[
A V \cdot B v=F S \cdot G s
\]
我们还得到 $P T: T M=A T: T V=B T: T v$, 从而
\[
T V=\frac{b(x-a)}{x y} \sqrt{F M \cdot G M}, \quad T v=\frac{b(x+a)}{x y} \sqrt{F M \cdot G M}
\]
进而
\[
T V \cdot T v=\frac{a^{2}}{x^{2}} F M \cdot G M=F T \cdot G T
\]
类似地, 由此可以得到下面一系列另外的推论.

\section{$\S 158$}

由 $C T=\frac{a^{2}}{x}$, 知横标 $C P=x$ 越大, 线段 $C T$ 越短, 曲线趋向无穷时, 其切线过中心 $C$, $C T=0$. 已知
\[
\tan \angle P T M=\frac{P M}{P T}=\frac{b^{2} x}{a^{2} y}
\]
当点 $M$ 位于无穷远处, 即 $x=\infty$ 时, 我们有
\[
y=\frac{b}{a} \sqrt{x^{2}-a^{2}}=\frac{b x}{a}
\]
从而, 曲线趋向无穷时, 其切线过中心 $C$, 且与轴所成角 $\angle A C D$ 的正切为 $\frac{b}{a}$. 如果过顶点 $A$ 引垂直于轴的直线 $A D=b$, 则 $C D$ 向两头延长所得直线恒不与曲线相触, 但曲线伸得越 远越向它靠近,最终在无穷远处 $C D$ 与 $C I$ 相合. $C k$ 的情形同于 $C D$, 最终在无穷远处与 $B k$ 相合. 如果从另一面以相同的角度引直线 $K C i$, 那么延长到无穷远时, 它同分支 $B K$, $A i$ 重合. 一直一曲两线, 曲线越来越靠近直线, 到无穷远时与直线重合, 这条直线就称为 这条曲线的渐近线,直线 $I C k, K C i$ 都是双曲线的渐近线.

\section{$\S 159$}

双曲线的这两条渐近线相交于中心 $C$, 与轴构成的角 $\angle A C D=\angle A C d$, 该角的正切 等于 $\frac{b}{a}$, 该角的倍角 $\angle D C d$ 的正切等于 $\frac{2 a b}{a^{2}-b^{2}}$. 由此可知, 如果 $b=a$, 则这两条渐近线的 交角, 即 $\angle D C d$ 为直角, 称此时的双曲线为等轴双曲线. 由 $A C=a, A D=b$, 得 $C D=C d=$ $\sqrt{a^{2}+b^{2}}$. 因而从焦点 $G$ 向一条渐近线引垂线 $G H$, 则由 $C G=\sqrt{a^{2}+b^{2}}=C D$, 得
\[
C H=A C=B C=a, \quad G H=b
\]
\section{$\S 160$}

向两头延伸弦 $M P N=2 y$, 交渐近线于 $m, n$, 则
\[
\begin{gathered}
P m=P n=\frac{b x}{a} \\
C m=C n=\frac{x \sqrt{a^{2}+b^{2}}}{a}=F M+A C=G M-A C
\end{gathered}
\]
又
\[
M m=N n=\frac{b x-a y}{a}, \quad N m=M n=\frac{b x+a y}{a}
\]
从而利用 $a^{2} y^{2}=b^{2} x^{2}-a^{2} b^{2}$, 得
\[
M m \cdot N m=M m \cdot M n=\frac{b^{2} x^{2}-a^{2} y^{2}}{a^{2}}=b^{2}
\]
进而处处有
\[
M m \cdot N m=M m \cdot M n=N n \cdot N m=N n \cdot M n=b^{2}=A D^{2}
\]
由 $M$ 引平行于渐近线 $C d$ 的直线 $M r$, 则
\[
2 b: \sqrt{a^{2}+b^{2}}=M m: m r(M r)
\]
从而
\[
\begin{gathered}
m r=M r=\frac{(b x-a y) \sqrt{a^{2}+b^{2}}}{2 a b} \\
C m-m r=C r=\frac{(b x+a y) \sqrt{a^{2}+b^{2}}}{2 a b}
\end{gathered}
\]
进而
\[
M r \cdot C r=\frac{\left(b^{2} x^{2}-a^{2} y^{2}\right)\left(a^{2}+b^{2}\right)}{4 a^{2} b^{2}}=\frac{a^{2}+b^{2}}{4}
\]
或者由 $A$ 引平行于渐近线 $C d$ 的直线 $A E$, 则 $A E=C E=\frac{1}{2} \sqrt{a^{2}+b^{2}}$, 从而
\[
M r \cdot C r=A E \cdot C E
\]
这是双曲线的渐近线的主要性质.

\section{$\S 161$}

参见图 34, 取中心 $C$ 为原点, 置横标 $C P=x$ 于一条渐近线上, 取纵标 $P M=y$ 平行于 另一条渐近线, 则 $y x=\frac{a^{2}+b^{2}}{4}$, 且 $A C=B C=a, A D=A d=b$, 如果令 $A E=C E=h$, 则 $y x=h^{2}, y=\frac{h^{2}}{x}$, 因而 $x=0$ 时 $y=\infty$; 反之 $y=0$ 时 $x=\infty$. 过曲线上点 $M$ 引平行于任一直线 $G H$ 的直线 $Q M N R$, 令 $C Q=t, Q M=u$, 则
\[
G H: C H: C G=u: P Q: P M
\]
从而
\[
P Q=\frac{C H}{G H} u, \quad P M=\frac{C G}{G H} u
\]
进而
\[
y=\frac{C G}{G H} u, \quad x=t-\frac{C H}{G H} u
\]
把这两个值代入 $y x=h^{2}$, 得
\[
\frac{C G}{G H} t u-\frac{G H \cdot C G}{G H^{2}} \cdot u^{2}=h^{2}
\]
或
\[
u^{2}-\frac{G H}{C H} t u+\frac{G H^{2}}{C H \cdot C G} h^{2}=0
\]
该方程告诉我们, 纵标 $u$ 有两个值 $Q M$ 和 $Q N$, 和等于 $\frac{G H}{C H} t=Q R$, 积 $Q M$. $Q N=\frac{G H^{2}}{C H \cdot C G} h^{2}$


【图,待补】
%%![](https://cdn.mathpix.com/cropped/2023_02_05_39e6b491fce1d5f0c077g-10.jpg?height=307&width=850&top_left_y=1893&top_left_x=413)

图 34 

\section{$\S 162$}

由 $Q M+Q N=Q R$, 得 $Q M=R N, Q N=R M$. 因而, 如果点 $M, N$ 重合, 则 $Q R$ 为切线, 且被切点等分. 例如, 如果直线 $X Y$ 为双曲线的切线, 则切点 $Z$ 为 $X Y$ 的中点. 因此, 如果 从 $Z$ 引平行于另一条渐近线的直线 $Z V$, 则 $C V=V Y$, 这给了我们一个引双曲线上任意点 $Z$ 处切线的方法: 取 $V Y=C V$, 那么过点 $Y$ 和曲线上点 $Z$ 的直线就是双曲线上点 $Z$ 处的 切线.

由 $C V \cdot Z V=h^{2}=\frac{a^{2}+b^{2}}{4}$, 得
\[
C X \cdot C Y=a^{2}+b^{2}=C D^{2}=C D \cdot C d
\]
因此, 如果连直线 $D X, d Y$, 则它们平行. 由此我们得到一个画出曲线任意多条切线的简 便的方法.

\section{$\S 163$}

由矩形面积 $Q M \cdot Q N=\frac{G H^{2}}{C H \cdot C G} h^{2}$ 得知,平行于 $H G$ 的直线 $Q R$, 不管画在哪,矩形 面积 $Q M \cdot Q N$ 都相同,且
\[
Q M \cdot Q N=Q M \cdot M R=Q N \cdot N R=\frac{C H^{2}}{C H \cdot C G} h^{2}
\]
如果考虑平行于 $Q R$ 的切线, 由于其位于渐近线之间的部分被切点所等分, 记这一半为 $q$, 我们恒有
\[
Q M \cdot Q N=Q M \cdot M R=R N \cdot R M=R N \cdot N Q=q^{2}
\]
这是画在渐近线之间的双曲线的重要性质.

\section{$\S 164$}

双曲线由相背的两部分 $I A i$ 和 $K B k$ 构成. 上述性质并不仅仅适用于只与一部分相交 的直线, 也适用于与两部分都相交的直线. 过 $M$ 引平行于 $G h$ 且与相背的另一部分也相 交的直线 $M q r n$, 记 $C q=t, q M=u$, 则由 $\triangle C G h$ 和 $\triangle P M q$ 相似得
\[
P M=y=\frac{C G}{G h} u, \quad q P=x-t=\frac{C h}{G h} u
\]
从而 $x=t+\frac{C h}{G h} u$. 将 $x, y$ 代入 $x y=h^{2}$, 得
\[
\frac{C G}{G h} t u+\frac{C G \cdot C h}{G h^{2}} u^{2}=h^{2}
\]
或 
\[
\begin{aligned}
& u^{2}+\frac{G h}{C h} t u-\frac{G h^{2}}{C G \cdot C h} h^{2}=0
\end{aligned}
\]
\section{$\S 165$}

纵标 $u$ 有两个值 $q M,-q m$. 这里取渐近线 $C P$ 为轴, $-q n$ 位于 $C P$ 的另一侧, 因而为 负. 从所得方程我们得到, 这两个根的和
\[
q M-q n=-\frac{G h}{C h} t=-q r
\]
或 $q n-q M=q r$, 从而 $q M=r n, q n=r M$. 从所得方程我们还得到两根的积
\[
-q M \cdot q n=-\frac{G h^{2}}{C G \cdot C h} h^{2}
\]
也即
\[
q M \cdot q n=q M \cdot r M=r n \cdot q u=r n \cdot r M=\frac{G h^{2}}{C G \cdot C h} h^{2}
\]
因而不管平行于 $G h$ 的直线 $M n$ 位置如何, 这个矩形面积是不变的.

各类二阶线的基本性质, 我们就讲到这里. 如果加上各类二阶线的共同性质, 我们可 以说能得到无穷多条性质.

