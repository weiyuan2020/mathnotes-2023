\chapter{第一章曲线概述}
%%01p001-020
\section{$\S 1$}

通常视变量为可以取一切值的量, 因而几何上用无穷直线 (这里记为 $R S$, 图 1) 表示 变量. 在无穷直线 $R S$ 上取定一个点 $A$, 拿 $A$ 作为 $R S$ 上一切量的起始点, 则线段 $A P$ 的长 度就是变量的一个确定的值, 不同的 $A P$ 表示变量的不同的值, 整个的 $R S$ 表示变量的一 切值.


【图,待补】
%%![](https://cdn.mathpix.com/cropped/2023_02_05_37db6883222c7ef36acdg-01.jpg?height=89&width=596&top_left_y=951&top_left_x=549)

图 1

\section{$\S 2$}

设无穷直线 $R S$ 表示的变量为 $x$,那么 $R S$ 上的线段就可以表示 $x$ 的每一个确定的 值. 例如, 取点 $A$ 作 $P$, 则这长度为零的线段 $A P$ 就表示 $x=0$ 这个值, 点 $P$ 离点 $A$ 越远, 线 段 $A P$ 表示的 $x$ 值越大.

线段 $A P$ 的长为横标,横标表示变量 $x$ 的确定的值.

\section{$\S 3$}

由于无穷直线 $R S$ 从点 $A$ 向左向右都伸向无穷远, 所以 $x$ 的正值负值它都可以表示. 如果取右侧的 $x$ 为正,那么左侧的线段 $A P$ 就表示负的 $x$. 事实上,点 $P$ 离开 $A$ 右移时线 段 $A P$ 增长, 它所表示的 $x$ 值增大, 反之, 点 $P$ 左移, 则 $x$ 值减小; 当点 $P$ 与 $A$ 重合时, $x$ 值 为零; 点 $P$ 继续左移时, $x$ 值将小于零, 为负, 也即, 取 $A$ 点右侧的 $A P$ 为正, 则左侧的 $A P$ 为负,也可以取左侧为正, 那么右侧就为负.

\section{$\S 4$}

我们来看看, 用无穷直线表示变量 $x$ 的时候,几何上 $x$ 的函数怎样表示. 设 $y$ 是 $x$ 的 一个函数, 那么当 $x$ 为一个确定值时, 我们得到一个确定的 $y$ 值. 用无穷直线 RAS (图 2) 表示 $x$ 值, 并假定 $R S$ 上方的 $y$ 值为正, $R S$ 上的 $y$ 值为零, $R S$ 下方的 $y$ 值为负, 那么, 对 $A P$ 确定的每一个 $x$, 我们引 $R S$ 的垂线 $M P$, 使线段 $M P$ 等于对应的 $y$, 当 $y$ 值为正时, 垂线 $M P$ 在 $R S$ 上方, $y$ 值为负时, $P M$ 在 $R S$ 下方.


【图,待补】
%%![](https://cdn.mathpix.com/cropped/2023_02_05_37db6883222c7ef36acdg-02.jpg?height=273&width=622&top_left_y=347&top_left_x=515)

图 2

\section{$\S 5$}

图 2 所表示的这个函数, $x=0$ 时, $y=A B, x=A P$ 时, $y=P M, x=A D$ 时, $y=0 . x=$ $A P$ 时, 如果 $y$ 的值为负, 则垂线 $P M$ 在直线 $R S$ 下方, $x$ 为负值时类似, $y$ 值为正, 则用 $R S$ 上方的垂线表示它, $y$ 值为负, 则用 $R S$ 下方的垂线, 例如 Pm 表示它. 如果对某个 $x$ 值, 例 如 $x=A E$, 函数 $y=0$, 则垂线长为零.

\section{$\S 6$}

上述方法使 $R S$ 上的每一个点 $P$ 都有一个对应点 $M, A P$ 为一个 $x$ 值, $P M$ 垂直于 $R S$, 且 $P M$ 的长等于 $A P$ 所表示的那个 $x$ 所对应的 $y$ 值. 点 $M$ 的位置: $y$ 为正时,在 $R S$ 上 方; $y$ 为负时, 在 $R S$ 下方; $y$ 为零时, 在 $R S$ 上, 即 $P, M$ 重合, 图 2 上, 点 $D$ 和 $E$ 处, $P$ 和 $M$ 是重合的. 点 $M$ 全体构成一条直线或曲线, 这直线或曲线由函数 $y$ 决定. $x$ 的任何一个函 数 $y$, 我们都可以用这样的方法把它变成几何上的一条直线或曲线, 这条线的性质由函 数决定.

\section{$\S 7$}

由函数 $y$ 我们可以确定对应曲线上的每一点, 从而确定整个曲线. 曲线上对应于 $P$ 的点 $M$ 随 $P M$ 的确定而确定. 我们可以这样确定曲线上的每一点. 反之, 对于曲线上的每 一点 $M$, 直线 $R S$ 上都有一个点 $P$ 与它对应. 过曲线上点 $M$ 引纵标, 记垂足为 $P$, 这 $P$ 就是 $M$ 的对应点, 这线段 $A P$ 的长就是 $x$, 纵标 $P M$ 的长就是 $y$ 的值,即曲线上的点都由函数 $y$ 确定.

\section{$\S 8$}

让一个点机械地连续运动,可以画出很多不同的线. 每做一次这样的运动都得到一 条完整的具体的曲线. 但我们主要考虑由函数产生的曲线, 这种曲线更适于解析处理, 更 适于微积分. $x$ 的任何一个函数都给出一条线, 直线或曲线. 反之, 每条线也都决定一个函数. 可见, 我们研究的任何一条曲线, 其性质都由这样的一个函数确定: 从曲线上任意 一点 $M$ 向直线 $R S$ 引纵标 $M P$, 则数值等于线段 $A P$ 之长的 $x$ 所对应的函数值 $y$ 应该为纵 标 $M P$ 的长.

\section{$\S 9$}

从上面对曲线的描述看得出, 可将曲线分为连续的和不连续的, 不连续的也称为混 合的, 连续曲线可以用 $x$ 的一个确定的函数表示. 如果曲线的不同部分, 如 $B M, M D, D M$ 等由不同的函数确定, 比如 $B M$ 部分由一个函数确定, 而 $M D$ 部分由另一个函数确定, 等 等, 则称它为不连续的, 或者混合的, 或者不规则的. 不连续曲线不由一个单一的规律构 成,而是由几个连续部分合成.

\section{$\S 10$}

几何主要讲连续曲线. 后面我们将证明: 依某种不变规则机械地运动, 这样画出的曲 线都可用一个函数表示, 也即都是连续曲线. 设 $m E B M D M$ 是可以用 $x$ 的函数 $y$ 表示的 连续曲线, 那么在直线 $R S$ 上取定了 $A P$ 表示的 $x$, 则 $y$ 的值就等于纵标 $P M$ 的长.

\section{$\S 11$}

在曲线的这样的定义之下, 曲线方程中极常用的几个术语都可继续使用, 首先称表 示 $x$ 的直线为轴或笛卡儿直线.

称 $x$ 值的起始点为原点, 称轴上表示 $x$ 确定值的那一部分, 即 $A P$ 为横标. 从横标端 点 $P$ 垂直于轴画到曲线的直线 $P M$ 叫纵标, 也称纵标为直角线, 因为它与轴成直角, 也可 以取纵标与轴成斜角, 那时的纵标也称为斜角线. 只要不特别声明, 我们都采用直角 纵标.

\section{$\S 12$}

这样, 记横标 $A P$ 为 $x$, 即 $A P=x$, 则函数 $y$ 告诉我们纵标 $P M$ 的大小, 即 $P M=y$. 连 续曲线的性状都由一个函数 $y$ 表示, 即都由 $y, x$ 和常数的一个关系式表示. 轴 $R S$ 上的 $A S$ 部分表示正横标, $A R$ 部分表示负横标, $R S$ 上方为正纵标区域, $R S$ 下方为负纵标 区域.

\section{$\S 13$}

在 $x$ 的任何一个函数我们也都得到一条连续曲线, 事实上, 先让 $x$ 取从 0 到 $\infty$ 的所有正值, 求出对应于每一个 $x$ 的 $y$ 值, 并用纵标表示出来. $y$ 为正时纵标在轴的上方, 为负 时在轴的下方. 这样我们就得到了曲线 $B M M$ 部分. 然后让 $x$ 取从 0 到 $-\infty$ 的所有负值, 则对应的 $y$ 值确定曲线的 $B E m$ 部分. 两部分合起来就是函数描述的整个曲线.

\section{$\S 14$}

由 $y$ 是 $x$ 的函数知, 或 $y$ 为 $x$ 的显函数,或 $y$ 与 $x$ 由一个方程相联系. 总之,每条曲线 都可以由变量 $x$ 和 $y$ 所构成的方程表示. $x$ 是轴上的横标, 从原点 $A$ 算起, $y$ 是垂直于轴的 纵标. 纵标横标总称为直角坐标. 因而可以说曲线的性质由坐标方程决定.

\section{$\S 15$}

这样我们就把对曲线的研究归结成了对函数的研究. 因而可以根据函数对曲线进行 分类. 曲线也首先分为代数的和超越的, 纵标 $y$ 是横标 $x$ 的代数函数时, 曲线是代数的, 也即曲线的性质由坐标 $x$ 和 $y$ 的代数方程表示的时候, 称它为代数曲线. $x$ 和 $y$ 构成超越 方程, 或者 $y$ 是 $x$ 的超越函数, 这时的曲线叫超越曲线. 这样我们就把曲线分成了代数的 和超越的两大类.

\section{$\S 16$}

从 $x$ 得 $y$, 从 $y$ 得曲线, 因而为了考察曲线, 我们应该考虑这函数是单值的还是多值 的. 先假定 $y$ 是 $x$ 的单值函数, 也即 $y=P, P$ 是 $x$ 的某个单值函数. 这样对每一个确定的 $x$ 值我们都得到一个确定的 $y$ 值, 也即每一个横标都对应于一个纵标. 从而得到曲线的形

状是: 对轴 $R S$ 上的任意一点 $P$, 过 $P$ 垂直于 $R S$ 的直线 $P M$ 都与曲线相交, 并且只交于一 点 $M$. 这样一来, 轴上的每一点都对应曲线上一点, 而轴两面伸向无穷, 所以曲线也两面 伸向无穷. 换句话说, 从单值函数得到的曲线, 随 $x$ 轴两面无间断地伸向无穷, 图 2 所示 曲线 EBMDM 就是两面无间断地伸向无穷的.

\section{$\S 17$}

设 $y$ 为 $x$ 的二值函数, 或者 $P, Q$ 表示 $x$ 的单值函数, $y^{2}=2 P y-Q$, 从而 $y=\pm \sqrt{P^{2}-Q}$, 那么每一个横标 $x$ 将对应两个纵标 $y$. 这两个纵标, 或者同为实数, 或者同为虚数. $P^{2}>$ $Q$ 时同为实数, $P^{2}<Q$ 时同为虚数, 因而当两个 $y$ 值都为实数时, 横标 $A P$ 对应两个纵标 $P M, P M$, 也即过点 $P$ 与轴垂直的直线交曲线于 $M$ 和 $M$ 两点. 如果 $P^{2}<Q$, 则没有纵标 与横标对应, 也即过点 $P$ 垂直于横轴的直线同曲线不相交. 图 3 上点 $P$ 处就是这样. 由于 从 $P^{2}>Q$ 不能越过实与虚的衔接点 $P^{2}=Q$ 变为 $P^{2}<Q$, 因而在实纵标结束点处, 如 $C$ 和 $G$, 必有 $y=P \pm 0$,也即两个纵标相等,曲线向相反的两个方向弯曲. 


【图,待补】
%%![](https://cdn.mathpix.com/cropped/2023_02_05_37db6883222c7ef36acdg-05.jpg?height=380&width=730&top_left_y=279&top_left_x=478)

图 3

$\S 18$

从图 3 上我们看到, 当负的横标 $-x$ 在 $A C$ 和 $A E$ 之间时, 纵标是虚的, 此时 $P^{2}<Q$. 如果点 $E$ 继续左移, 纵标将重新变为实数 (但不能越过使 $Q^{2}=P$ 的点 $E$, 点 $E$ 处两个纵标 相等), 又重新是横标 $A P$ 对应两个纵标 $P m$ 和 $P m$, 到 $G$ 又两个纵标相等, 过 $G$ 纵标再次 变为虚数. 可见这种曲线是由彼此分离的两部分, 如 $M B D B M$ 和 $\mathrm{Fm} \mathrm{Hm}$ 或更多部分组 成. 但是应该认为作为总体来研究的这几部分共同构成一条连续的或规则的曲线, 因为 不同部分产生于同一个函数. 可见这种曲线具有如下的性质: 过横轴上一点引垂直于横 轴的直线 $M M$, 则 $M M$ 与曲线或者不相交, 或者相交于两点, 两点重合处, 如 $D, F, H$ 和 $I$ 处例外.

\section{$\S 19$}

如果 $y$ 是 $x$ 的三值函数, 也即如果 $y$ 由状如
\[
y^{3}-P y^{2}+Q y-R=0
\]
的方程决定,其中 $P, Q, R$ 是 $x$ 的单值函数,那么对每一个横标 $x$, 纵标 $y$ 将有三个值. 这 三个值或者全实, 或者一实两虚. 因此这纵标线与曲线必定或者相交于三点, 或者只相交 于一点, 两点重合处例外. 可见每一个横标至少对应于一个实纵标, 这意味着曲线随 $x$ 轴 一起向两面无穷延伸, 因而这曲线或者由一根连续曲线构成, 如图 4 , 或者由分离的两部 分构成,如图 5 , 或者由更多部分构成. 


【图,待补】
%%![](https://cdn.mathpix.com/cropped/2023_02_05_37db6883222c7ef36acdg-06.jpg?height=443&width=424&top_left_y=343&top_left_x=286)

图 4


【图,待补】
%%![](https://cdn.mathpix.com/cropped/2023_02_05_37db6883222c7ef36acdg-06.jpg?height=507&width=618&top_left_y=290&top_left_x=761)

图 5

\section{$\S 20$}

如果 $y$ 是 $x$ 的四值函数, 也即如果 $y$ 由方程
\[
y^{4}-P y^{3}+Q y^{2}-R y+S=0
\]
确定, 那么每一个 $x$ 值所对应的实 $y$ 值, 都或者为 4 个, 或者为 2 个, 或者为 0 个, 因而, 纵 标与这种函数所形成的曲线的交点个数, 或为 4 , 或为 2 , 或为 0 . 这几种情形图 6 上都有. 应该指出 $I$ 和 $O$ 两处,那里两个交点重合, 可见向左向右至无穷这曲线的分支个数都或 为 0 , 或为 2 , 或为 4 . 分支个数为 0 时, 曲线的向左向右部分都不产生伸向无穷的分支, 曲 线两面都封闭,被局限在某个区域之内,值数更多的函数,其曲线的性质都可以用这种方 法进行讨论.


【图,待补】
%%![](https://cdn.mathpix.com/cropped/2023_02_05_37db6883222c7ef36acdg-06.jpg?height=409&width=828&top_left_y=1442&top_left_x=414)

图 6

\section{$\S 21$}

当 $y$ 为 $n$ 值函数, 即 $y$ 由最高次数为 $n$ 的方程确定时, $y$ 的实值个数或为 $n$, 或为 $n-$ 2 , 或为 $n-4$, 或为 $n-6$, 等等. 纵标线与曲线的交点个数同于 $y$ 的实值个数. 这样一来, 如果一根纵标线与连续曲线的交点个数为 $m$,那么任一纵标线与这条曲线的交点个数同 $m$ 的差都为偶数. 即此时纵标线与曲线交点的个数不能是 $m+1, m-1$ 或 $m \pm 3$, 等等. 纵 标线与同一曲线的交点个数必定同为奇数或同为偶数. 也即知道了一根纵标线与曲线交 点个数的奇偶, 也就知道了全体纵标线与曲线交点个数的奇偶.

\section{$\S 22$}

这样,如果一根纵标线与曲线的交点个数为奇数, 那么就不存在与该曲线无交点的 纵标线, 即曲线在每一面都至少有一条伸向无穷的分支, 如果这时在某一面有多条伸向 无穷的分支, 那么这分支条数也必为奇数, 因为这时纵标线与曲线的交点个数不能为偶 数. 这时两面伸向无穷的分支条数的和为偶数. 当纵标线与曲线交点个数为偶数时, 每面 伸向无穷的分支数, 或为 0 , 或为 2 , 或为 4 , 等等. 两面伸向无穷的分支数的和也为偶数.

以上我们介绍了连续的规则曲线的重要性质, 可用来区别间断的不规则曲线. 

