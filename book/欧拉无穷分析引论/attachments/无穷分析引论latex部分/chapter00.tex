\chapter*{相关信息}
《无穷分析引论》上册18章,下册22章,附录6章。在数学史上地位显赫,是对数学发展影响最大的七部名著之一。

欧拉的《无穷小分析引论》首次把对数作为指数、把三角函数作为数值之比而不是作为一些线段的系统论述,次用函数概念作为中心和主线,把函数而不是曲线作为主要研究对象,使无穷小分析不再依赖几何性质。

《无穷小分析引论》除了是三角学研究的开端, 还对微积分进行了进一步的完善。

《无穷小分析引论》标志着微积分历史上的一个转折:以往的数学家们都以曲线作为微积分的主要研究对象,而欧拉则第一次把函数放到了中心的地位,并且是建立在函数的微分的基础之上。

可以说,从欧拉开始,在极大程度上摆脱了对几何直观的依赖,在逻辑上更为严瑾和便于分析。 数学开始逐渐摆脱对几何的依赖。欧拉冲破了古希腊人的思想框架,进一步向符号代数转化,几何问题常常反过来用代数方法解决,而欧拉对微积分的完善,实现了数学研究的基本方法由古希腊的几何演绎向以算术和代数的分析方法的转变。

可以说《无穷小分析引论》这本书确立了欧拉在数学界的地位,将大家从欧氏几何的桎梏中解救出来。这本书涉及了此书涉及了当时数学的各个领域和分支,包括分析学,几何学,代数学,微分方程,变分学, 数论等等。极大地促进了数学的发展。

\section*{七部影响世界数学历史发展的名著}

1.《几何原本》 作者是古希腊亚历山大前期的数学家欧几里得(公元前330-275).共13卷.《几何原本》开创了用公理方法建立起演绎体系的先河,这也是这本著作最主要的特色.中学阶段课程原《平面几何》、《立体几何》中的内容基本上就是2000年前的《几何原本》中的内容.因此,它对数学发展的影响超过了其他任何一部数学著作,它也造就了许多著名的数学家,牛顿就是深受其影响的一位.《几何原本》是一部“除了《圣经》之外再也没有比它发行量更大的书”.

2.《九章算术》 作者不详.大约成书于公元前3世纪至公元前1世纪.它是我国在数学方面流传最早的一部重要经典著作.它的出现标志着中国古代数学体系的形成.全书分为方田、栗米、衰分、少广、商功、均输、盈不足、方程、勾股等九章,分别对算术、几何、代数等各类数学知识进行了解析.后由数学家刘徽作注,使之变得更有条理.现流传下来的仅有5章,为南宋刻本,收藏于上海图书馆.

3.《代数学》 作者为阿拉伯数学家花拉子米,它不但第一次赋予这门学科以名称,他也因此被冠以“代数学之父”的称号,而且最先引入“还原(或移项)和取消(或对消)”这两种变换. 1859年,清代数学家李善兰和英国传教士韦烈亚力共同翻译过来,把书名定为《代数学》。“代数”一词就此产生并沿用至今.

4.《几何学》 作者为法国哲学家和数学家笛卡儿(1596-1650),发表于1637年.全书共分三卷.他证明了几何问题可归纳为代数形式问题,创立了解析几何学,开辟了数学新领域.

5.《自然科学的数学原理》 作者为英国伟大的科学家和思想家牛顿(1642-1727),1687年出版.他创立了对后世科学技术影响深远的微积分学说、微分方程理论.著名科学家杨振宁教授把这本巨著的出版作为近代科学在欧洲诞生的标志.

6.《无穷小分析引论》 作者为瑞士数学大师欧拉(1707-1783),1784年出版.它是第一部最系统的分析著作,是微积分发展史上的一个里程牌.

7.《算术探究》 德国“数学王子”高斯(1777-1855)在20岁时写成此书,1801年出版.它是近代数论的基石,全书共有七个部分.他把同余理论推向一个新阶段,发展了二项剩余理论,把型的理论系统化并加以扩展,在二项方程的研究方面也取得了重大成果.

另外一种说法是:

《从微分观点看拓扑》,作者米尔诺;《无穷小分析引论》,作者欧拉;《自然哲学之数学原理》,作者牛顿;《几何原本》,作者欧几里得;《数论报告》,作者希尔伯特;《算术研究》,作者高斯;《代数几何原理》,作者哈里斯。

\section*{排版说明}
依个人兴趣,做了适当排版(\LaTeX 模板选自钱院学辅),选取原书内容。原书分为上、下两册。排版时合为一册。章节暂时保留原有编号,以方便与原书参照。

原书部分图形依情况加入。

校对依时间缓慢进行,如有交流,可电邮:hnkznhb@126.com。

\section*{内容简介}
《无穷分析引论》是作为微积分预备教程,为弥补初等代数对于微积分的不足,为学生从有穷概念向无穷概念过渡而写,读者对象是准备攻读和正在攻读数学的学生、数学工作者和广大数学爱好者。《无穷分析引论》在数学史上地位显赫,是对数学发展影响最大的七部名著之一。

此书是在数学史上具有划时代意义的代表作,当时数学家们称欧拉为"分析学的化身".

欧拉在这本书里总结了大量丰富的数学成果, 其研究技巧也充满了启发性, 对后来的数学家产生了不可估量的巨大影响。

19世纪伟大数学家高斯(Gauss,1777-1855年)曾说:"研究欧拉的著作永远是了解数学的最好方法."

此书涉及了当时数学的各个领域和分支,包括分析学, 几何学, 代数学,微分方程, 变分学, 数论等等。


\section*{图书在版编目(ClP)数据}



%%
\chapter*{中译者的话}

本书在数学史上地位显赫,是对数学发展影响最大的七部名著之一.初版(1748年)至今虽已200多年,但大数学家A.Weil教授1979年称道其现实作用说:学生从它所能得到的益处,是现代的任何一本数学教科书都比不上的.笔者手边的俄、德、英译本依次出版于1961,1985,1988,这大概可视为其现实作用的一个证明.

欧拉贡献巨大,著述极为多产.本书是它著作中最杰出的,书中结果几乎或为他自己所得,或为他用自己的方法推出.他的作法是把最基本的东西解释得尽量清楚,讲明引导他得出结论的思路,而把进一步展开留给读者,使读者有机会驰骋自己的才能.这大概都是A.Weil教授前面那段话的根据.

本书是作为微积分预备教程,为弥补初等代数对于微积分的不足,为帮助学生从有穷概念向无穷概念过渡而写.读者对象是准备攻读和正在攻读数学的学生、数学工作者和广大数学爱好者.

本书从英译本转译,参考俄、德译本作了些订正和改动.

限于水平,中译文错误难免,敬希指正.

\section*{几段话}

1.高斯:“学习欧拉的著作,乃是认识数学的最好工具.”

2.拉普拉斯:“读读欧拉,他是我们大家的老师.”

3.波利亚很欣赏欧拉的作法:坦率地告诉人们引导他作出发明的思路.

4.AlbertoDou,S.J教授将欧拉的许多著作译成了西班牙文.他对本书的英译者说:“《无穷分析引论》是欧拉著作中最杰出的.”

5.A.Weil教授1979年在Rochester大学的一次讲演中说:“今天的学生从欧拉的《无穷分析引论》中所能得到的益处,是现代的任何一本数学教科书都比不上的.”

\section*{英译者序(节译)}

1979年10月,Andre Weil教授在Rochester大学,以欧拉的生平和工作为题,作了一次报告.报告中他向数学界着力陈述的一点是:今天的学生从欧拉的《无穷分析引论》中所能得到的益处,是现代的任何一本数学教科书都比不上的.我查到了该书的法、德、俄三个语种的译本,但查不到英文全译本,就是在这样的背景下,我着手对该书进行翻译的.

欧拉的序言中说得明白,这是一本微积分预备教程.书中有几处,那里的东西只提了一下,把处理留给了微积分,用微积分处理要简单容易许多.凡这种地方书中都有交待.

关于书名,欧拉原文中的无穷(Infinitorum)是复数.看来这复数主要指:无穷级数、无穷乘积和连分式三种无穷.因而书名应译为《有关几种无穷的分析引论》,不顺口,我译它为《无穷分析引论》.

任教于巴塞罗那大学的S.J.Alberto Dou教授将欧拉的很多著作译成了西班牙文,最近译者曾与他谈起过本书.我们就用那次谈话中他的一句话作为这段序言的结束:“在欧拉的著作中《无穷分析引论》最为杰出.”

\section*{作者序}

接触到的学生,他们学习无穷分析之所以遇到困难,往往是由于在必须使用无穷这一陌生概念时,初等代数刚学,尚未登堂入室.虽然无穷分析并不要求初等代数的全部知识和技能,问题是有些必备的东西,初等代数或者完全没讲,或者讲得不够详细.本书力求把这类东西讲得既充分又清楚,求得完全弥补初等代数对无穷分析的不足.书中还把相当多的难点化易,使得读者逐步地、不知不觉地掌握到无穷这一思想,有很多通常归无穷分析处理的问题,本书使用了代数方法.这清楚地表明了分析与代数两种方法之间的关系.

本书分上、下两册,上册讲纯分析,下册讲必要的几何知识,这是因为无穷分析的讲解常常伴以对几何的应用.别的书中都讲的一般知识本书上、下册都不讲.本书所讲是别处不讲的,或讲得太粗的,或虽讲但所用方法完全不同的.

整个无穷分析所讨论的都是变量及其函数,因此上册细讲函数,讲了函数的变换、分解和展开为无穷级数.对函数,包括属于高等分析的一些函数进行了分类.首先分函数为代数函数和超越函数.变量经通常的代数运算形成的函数叫代数函数,经别的运算或无穷次代数运算形成的函数叫超越函数.代数函数又分为有理函数和无理函数.对有理函数讲了分解它为因式和部分分式,分解为部分分式之和这种运算在积分学中有着重要应用.对无理函数给出了用适当的代换变它为有理函数的方法.无理函数和有理函数都可以展开成为无穷级数,但这种展开对超越函数用处最大.无穷级数的理论可用于高等分析,为此增加了几章,用于考察很多无穷级数的性质与和.其中有些级数的和不用无穷分析是很难求出的,其和为对数和弧度的级数就是.对数和弧度是超越量,可通过求双曲线下的和圆的面积确定,主要由无穷分析对它们进行研究.接下去从以底为变量的幂转向了以指数为变量的幂.作为以指数为变量之幂的逆,自然而有成果地得到了对数概念.对数不仅本身有着大量应用,而且由它可得到一般量的无穷级数表示.还讲了造对数表的简单方法.类似地,我们考察了弧度.弧度与对数虽然是两种完全不同的量,但它们却有着如此密切的关系,当一种为虚数形式时,可化为另一种,重复了几何中多倍角和等分角正弦和余弦的求法之后,从任意角的正弦余弦导出了极小角的正弦和余弦,并导出了无穷级数.由此,从趋于消失的角其正弦等于角度,余弦等于半径,我们可以通过无穷级数使任何一个角度等于它的正弦或余弦.这里我们得到了如此之多的各种各样的有限的和无穷的这种表达式,以至于无需再对其性质进行研究.对数有着它自己的特殊算法,这种算法应用于整个分析.我们推出了三角函数的算法,使得对三角函数的运算如同对数运算和代数运算一样地容易.从书中有几章的内容可以看出,三角函数算法在解决难题时,其应用范围是何等的广.事实上,这种例子从无穷分析中还可举出很多,日常的数学学习和数学工作中也会遇到很多.

分解分数函数为实部分分式在积分学中有着重要应用,而三角函数算法对分解分式为实部分分式有极大帮助,我们对它进行详细讨论的原因正在于此.接下去的讨论是分数函数展成的无穷级数——递推级数.讨论了它的和、通项和另外一些重要性质.递推级数考虑的是因式乘积的倒数,我们也考虑了展多因式,甚至无穷个因式的乘积为级数.这不仅可导致对无穷多个级数的研究,而且利用级数可表示成无穷乘积,我们找到了一些方便的数值表达式,用这些表达式可以容易地计算出正弦、余弦和正切的对数,利用展因式乘积为级数,我们推出了许多有关拆数为和这类问题的解.倘不利用这一点,看来分析对拆数为和是无能为力的.

本书涉及方面之广,完全可以写成几册书,因而我们力求简单明了,把最基本的东西解释得尽量清楚,而把进一步展开留给读者,使读者有机会驰骋自己的才能,自己来进一步发展分析.我坦率地告诉读者,本书含有许多全新的东西,并且从本书的很多地方可以得到重要的进一步的发现.

下册讨论的问题,一般地说都属于高等几何,处理方法同于上册.一般教科书讲这一部分时都从圆锥曲线开始,本书先讲曲线的一般理论,再讲圆锥曲线,为的是能够应用曲线理论去研究任何一种曲线.本书利用描述曲线的方程,而且只用这种方程来研究曲线.曲线的形状和基本性质都从方程推出.我觉得这种处理方法的优越性,在圆锥曲线上表现得最突出.即或有人对它应用分析方法,那也是显得生硬、不自然的.我们先从二阶曲线的一般方程解释了二阶曲线的一般性质.接下去根据有无伸向无穷的分支,也即是否介于某个有限区域之中,对二阶曲线进行了分类.对于无穷分支,我们进一步考虑分支的条数,并考虑各条分支有无渐近线.这样我们得到了通常的三种圆锥曲线.第一种是椭圆,它介于一个有限区域之中;第二种是双曲线,它有四条伸向无穷的分支,趋向两条渐近线;第三种是抛物线,有两条伸向无穷的分支,没有渐近线.

接下去,对三阶曲线用类似的方法,阐述了其一般性质,并将它分为12类,事实上是把牛顿的72种划分成了12类.对这一方法我们的描述是充分的,不难用它对更高阶曲线进行分类.书中用它对四阶曲线进行了分类.

在分阶进行考察之后,我们转向了寻求曲线的共同性质.讲了曲线的切线和法线的定义方法,也讲了用密切圆半径表示的曲率.虽然这些问题现在一般都用微积分来解决,但本书只在通常代数的基础上对它进行讨论,为的是使读者能够比较容易地从有穷分析过渡到无穷分析.我们也对曲线的拐点、尖点、二重点和多重点进行了研究.讲了如何从方程求出这些点,求法都不难.但我不否认用微分学的方法来求更容易.我们也讲到了关于二阶尖点这有争论的问题.二阶尖点,即有同朝向的两段弧收敛于它的尖点.我们讨论的深度不越出看法一致的范围.

加写了几章,用来讨论具有某些性质的曲线的求法.最后给出了与圆有关的几个问题的解.

几何中有几部分是学习无穷分析所必备的.有鉴于此,我们添上了一个附录,用计算的方式讲立体几何中有关立体和曲面的一些知识.讲了如何用三元方程表达曲面的性质,然后照曲线那样,根据方程的阶数将曲面分了类,并证明了只有一阶曲面才是平面.根据它伸向无穷的部分将二阶曲面分成了六类.对更高阶的曲面也可以用类似的方式进行分类.我们对两个曲面的交线进行了讨论.交线多数都不在一个平面上,我们讲了如何用方程表示交线.最后对曲面的切线和法面进行了一些讨论.

这里声明一点,书中很多东西是别人已经得到了的,恕我没有一一指出.本书力求简短,如果对问题的历史进行讨论,那将突破本书的篇幅限制.作者可聊以自慰的是,对别人已经得到了的东西,其中很多本书是用另一种方法进行讨论的.很希望多数读者从方法新和全新特别是全新的东西中得到益处.

 