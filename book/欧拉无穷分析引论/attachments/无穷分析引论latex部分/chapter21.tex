\chapter{第三章代数曲线的阶}

\section{$\S 47$}

曲线跟函数一样, 数量无穷, 不分成类, 就不能有条理地进行考察, 研究就不能进行. 我们已经把曲线分成了代数的和超越的, 但这不够, 要再分. 本书只考察代数曲线, 因而 我们只对代数曲线进行再分类, 这首先要选定一个特征, 作为分类的依据.

\section{$\S 48$}

这依据只能从表明曲线性质的函数或方程上面找. 到现在为止, 这是我们研究曲线 的唯一途径, 当然对代数曲线也没有别的路可走. 对两坐标间的函数, 上册我们已经进行 过几种分类. 这里我们首先想到的, 是拿值的个数作依据, 把单值函数产生的曲线归入第 一类, 二值函数产生的曲线归人第二类, 三值的归人第三类, 等等.

\section{$\S 49$}

这种分法虽显得自然,但稍作深究就会发现, 函数的值的个数与由它产生的曲线之 间, 很少有不变的关系. 函数的值的个数, 很大程度上是依赖于轴的位置的, 而讨论曲线 时允许轴的位置任意. 一个函数对一个轴是单值的, 对另一个轴可以是多值的. 这样同一 条曲线对不同的轴得归人不同的类, 这不符合分类要求. 例如, 方程 $a^{3} y=a^{2} x^{2}-x^{4}$ 产生 的曲线属第一类, 因为纵标 $y$ 是横标 $x$ 的单值函数. 但是移动坐标, 使新轴垂直旧轴, 则 同一条曲线的方程变成了 $y^{4}-a^{2} y^{2}+a^{3} x=0$, 曲线属第四类. 可见, 函数的值的个数不能 作为曲线分类的依据.

\section{$\S 50$}

试试看, 用项作依据, 项数为 2 的方程, 例如 $y^{m}-\alpha x^{n}=0$ 产生的曲线归第一类; 项数 为 3 的方程, 例如 $\alpha y^{m}+\beta y^{p} x^{q}+\gamma x^{n}=0$ 产生的曲线归第二类, 等等. 显然, 在这种分法之 下, 同一条曲线的类别也不唯一. 例如, 在 $\S 36$, 作为例子, 我们讨论过方程 $y^{2}-a x=0$ 产 生的曲线. 拿项作依据, 这条曲线既属第一类, 又属第四类, 因为改变轴和原点的位置, 可 使方程变为
\[
16 u^{2}-24 t u+9 t^{2}-55 a u+10 a t=0
\]
而且选择另外的轴, 另外的原点, 可使这条曲线属于第二类, 第三类, 第五类. 可见项数也 不能作为依据.

\section{$\S 51$}

如果用曲线的方程的阶作曲线分类的依据,那情形就不同了. 不管轴和原点的位置 以及坐标角如何变化, 曲线方程的阶都是不变的. 也即, 用坐标方程的阶作依据的话, 轴 和原点的位置及坐标角的变化都不影响曲线的类, 同一条曲线, 不管取它的特殊方程, 还 是通用方程, 或者最通用方程, 它都属于同一类. 因此, 用方程的阶作曲线分类的依据是 适宜的.

\section{$\S 52$}

方程按幂次分类,各项最高幂次为几就叫几阶方程,我们把它搬到曲线上来,方程各 项最高幂次为几, 就称曲线为几阶曲线,一阶通用方程为
\[
0=\alpha+\beta x+\gamma y
\]
以 $x, y$ 为坐标, 直角坐标、斜角坐标都可. 从该方程得到的曲线都属一阶曲线. 但是前面 我们看到了, 该方程产生的都是直线, 也即一阶曲线都为直线. 直线是最简单的曲线,一 阶曲线, “曲”字在这里名不符实了. 我们去掉“曲” 字, 只用一个线字, 线包含直线, 也包 含曲线,一阶线不含曲线, 只含直线.

\section{$\S 53$}

取上节方程中的 $x, y$ 为直角坐标、斜角坐标都可以,如果取纵标与轴成斜角 $\varphi$, 并记 $\varphi$ 的正弦和余弦为 $\mu$ 和 $\nu$, 则置
\[
y=\frac{u}{\mu}, \quad x=\frac{\nu u}{\mu}+t
\]
就得到直角坐标 $t, u$ 之间的方程
\[
0=\alpha+\beta t+\left(\frac{\beta \nu}{\mu}+\frac{\gamma}{\mu}\right) u
\]
前后两个方程都是通用方程,所以后一个的范围并不比前一个小. 也即坐标角为直角时 方程所包含的曲线并不比为斜角时少. 同样,阶数更高的方程,它所包含的曲线数目, 也 不因取直角为坐标角而减少,任何阶的通用方程都不因取定坐标角而减少所含曲线. 当 然也可以取直角为坐标角. 斜角坐标下任何阶通用方程的任何一条曲线, 都将包含在直 角坐标下的对应方程中.

%%02p021-040

\section{$\S 54$}

二阶线都包含在二阶通用方程
\[
0=\alpha+\beta x+\gamma y+\delta x^{2}+\varepsilon x y+\zeta y^{2}
\]
之中, $x, y$ 为直角坐标. 该方程表示的曲线都是二阶线. 二阶线是最简单的曲线, 一阶线 是直线不是曲线, 因而有人把二阶线叫做一阶曲线. 但二阶线的一个更为熟知的名称是 圆雉曲线. 圆雉曲线又分为圆、椭圆、抛物线和双曲线. 后面我们将从通用方程导出它们.

\section{$\S 55$}

三阶通用方程
\[
0=\alpha+\beta x+\gamma y+\delta x^{2}+\varepsilon x y+\zeta y^{2}+\eta x^{3}+\theta x^{2} y+\omega x y^{3}+\kappa y^{3}
\]
产生的曲线都是三阶线, $x, y$ 是直角坐标. 前面已经指出, 取 $x, y$ 为斜角坐标, 并不使方 程含有更多的曲线,这个方程中可任意取值的字母,比前节多得多,因而所含曲线形状的 种数也比前节多得多, 牛顿列出了这些形状.

\section{$\S 56$}

四阶通用方程
\[
\begin{aligned}
0= & \alpha+\beta x+\gamma y+\delta x^{2}+\varepsilon x y+\zeta y^{2}+\eta x^{3}+\theta x^{2} y+\omega x y^{2}+\kappa y^{3}+ \\
& \lambda x^{4}+\mu x^{3} y+\nu x^{2} y^{2}+\xi x y^{3}+o y^{4}
\end{aligned}
\]
产生的曲线都是四阶线. 这里取 $x, y$ 为直角坐标; 当取 $x, y$ 为斜角坐标时,方程也不含有 更多的曲线. 本节方程中有 15 个可任意取值的常数,因而四阶线的类型比三阶线又要多 出很多. 人们称二阶线为一阶曲线,因而也称四阶线为三阶曲线,称三阶线为二阶曲线.

\section{$\S 57$}

照以上所讲,我们可以列出五、六、七等各阶线的通用方程. 四阶线的通用方程加 上含
\[
x^{5}, x^{4} y, x^{3} y^{2}, x^{2} y^{3}, x y^{4}, y^{5}
\]
的项就是五阶线的通用方程, 共 21 项. 类似的,六阶线的通用方程含 28 项. 根据三角形规 则, $n$ 阶线的通用方程, 其项数为 $\frac{(n+1)(n+2)}{1 \times 2}$, 可任意取值的常数, 其个数等于项数.

\section{$\S 58$}

但可任意取值的常数, 其不同的取值可以给出相同的曲线. 前一章我们看到, 变化轴与原点的位置, 可以得到同一曲线的无穷多个方程, 也即, 同阶数的不同方程可以给出相 同的曲线. 因此, 根据通用方程对属于同一阶的曲线进行再分类时, 要十分注意, 不要把 同一条曲线分到两个或更多个类里去.

\section{$\S 59$}

曲线的阶由坐标间方程决定, 每一个坐标间的代数方程都告诉我们它所表示的曲线 属哪一阶. 无理方程应有理化, 分式方程应整式化, 有理化整式化之后的代数方程, 各项 最高次数, 即各项中 $x, y$ 次数之和的最大者, 就是原方程所表示的曲线的阶. 例如, 方程 $y^{2}-a x=0$ 给出的曲线是二阶的; 方程 $y^{2}=x \sqrt{a^{2}-x^{2}}$ (有理化之后是四阶的) 给出的曲 线是四阶的; 而方程 $y=\frac{a^{3}-a x^{2}}{a^{2}+x^{2}}$ 给出的曲线是三阶的, 因为整式化之后的方程 $a^{2} y+$ $x^{2} y-a^{3}=a x^{2}$ 最高次项 $x^{2} y$ 的次数为 3 .

\section{$\S 60$}

随纵标线与轴间夹角的变化, 同一方程表示的可以是多条不同的曲线. 例如, 方程 $y^{2}=a^{2}-x^{2}$, 在直角坐标下它表示的是圆, 在斜角坐标下它表示的是椭圆, 但这两条曲线 的阶相同, 因为变直角坐标为斜角坐标的变换不改变曲线的阶. 虽然纵标线与轴间夹角 大小的改变, 既不增大也不缩小各阶曲线方程所含曲线的范围, 但不指明坐标角, 方程表 示的曲线就确定不下来.

\section{$\S 61$}

曲线以方程的阶为阶, 但方程必须是不能分解成有理因式的, 方程的每个有理因式 都自成一个方程, 每个自成方程都产生一条曲线. 自成方程产生的曲线总体构成原方程 描述的曲线, 即可分解为有理因式的方程, 它含有的不是一条而是几条连续曲线, 每条连 续曲线都由自己的方程产生, 这些方程彼此无关, 但乘积等于原方程. 可分解成有理因式 的方程给出的不是一条, 而是几条连续曲线, 我们称这样的线为复合线.

\section{$\S 62$}

例如方程 $y^{2}=a y+x y-a x$, 看上去它表示的是二阶线. 移项得 $y^{2}-a y-x y+a x=$ 0 , 分解得 $(y-x)(y-a)=0$, 是方程
\[
y-x=0, \quad y-a=0
\]
的乘积. 这两个方程都是直线方程, 前一直线与轴成半直角, 后一直线平行于轴, 至轴的 距离为 $a$, 也即方程 
\[
y^{2}=a y+x y-a x
\]
表示的是两条直线所成的复合线.

方程
\[
y^{4}-x y^{3}-a^{2} x^{2}-a y^{3}+a x^{2} y+a^{2} x y=0
\]
给出的不是四阶线, 是复合线, 其左端可分解为 $(y-x)(y-a)\left(y^{2}-a x\right)$, 即该方程含有 三条线,两条直线和一条由方程 $y^{2}-a x=0$ 产生的曲线.

\section{$\S 63$}

复合线可以随意构成, 可以含有两条线, 也可以含有多条线; 可以都是直线, 可以都 是曲线, 也可以是直线和曲线,而且曲线形状不拘. 事实上, 不管多少条线,只要它们的方 程共轴,共原点,且都化成了同一端为零的形式,那么这些方程的乘积所成的复合方程, 就含有积中每个方程所表示的曲线.

图 16 是一个以 $C$ 为圆心, $A C=a$ 为半径的圆和一条过圆心 $C$ 的直线 $L N$. 我们可以 关于随便的轴求出包含这两条线的复合方程.


【图,待补】
%%![](https://cdn.mathpix.com/cropped/2023_02_05_00a02f82302d074ee0d6g-03.jpg?height=376&width=503&top_left_y=1073&top_left_x=591)

图 16

\section{$\S 64$}

例如, 取与直线 $L N$ 成半直角的直径 $A B$ 作轴, 取点 $A$ 作原点, 记横标 $A P=x$, 纵标 $P M=y$. 那么, 对直线上的点 $M$, 我们有 $P M=C P=a-x$, 因为这个点 $M$ 是在负纵标区, 所以我们有 $y=-a+x$, 或者 $y-x+a=0$. 对圆上的点 $M$, 我们有 $P M^{2}=A P \cdot P B$ 和 $B P=2 a-x$, 从而得到 $y^{2}=2 a x-x^{2}$, 或者 $y^{2}+x^{2}-2 a x=0$. 得到的两个方程相乘, 结果 为三阶复合方程
\[
y^{3}-y^{2} x+y x^{2}-x^{3}+a y^{2}-2 a x y+3 a x^{2}-2 a^{2} x=0
\]
它同时包含图中的圆和直线. 横标 $A P=x$ 对应三个纵标, 两个在圆上,一个在直线上. 例 如, 令 $x=\frac{1}{2} a$, 则
\[
y^{3}+\frac{1}{2} a y^{2}-\frac{3}{4} a^{2} y-\frac{3}{8} a^{3}=0
\]
该方程的一个因式为 $y+\frac{1}{2} a=0$, 作除法, 得另一个因式为 $y^{2}-\frac{3}{4} a^{2}=0$. 从而 $y$ 的三个 值为
\[
\text { I. } y=-\frac{1}{2} a, \quad \text { II. } y=\frac{\sqrt{3}}{2} a, \quad \text { III. } y=-\frac{\sqrt{3}}{2} a
\]
我们构造的复合方程表示的是由一个圆和一根直线所成的连续曲线.

\section{$\S 65$}

以上我们讲了复合线与非复合线的区别. 显然, 二阶线可以是连续曲线, 可以是含两 根直线的复合线, 这是因为二阶方程如果有因式, 这因式必为一阶的, 一阶方程表示的是 直线. 三阶线可以是非复合的, 可以是复合的. 这复合线的成员, 可以是一条直线与一条 二阶线, 也可以是三条直线. 四阶线可以是连续线, 即非复合的, 也可以是复合的. 这复合 线的成员, 可以是一条直线与一条三阶曲线, 可以是两条二阶线, 可以是一条二阶线与两 条直线, 最后可以是四条直线. 类似地, 可以列出五阶和更高阶复合线的可能成员. 可见, 任何阶的曲线都可由阶数比它低的非复合线复合而成, 复合方式可以不同, 但复合线成 员的阶数的和必等于复合线的阶数. 

