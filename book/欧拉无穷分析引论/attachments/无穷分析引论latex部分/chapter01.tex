%%01p001-020
\chapter{第一章 函数}

\section{$\S 1$}

常量是固定的保持不变的量.

常量可以取定一个数值, 一旦取定即保持常值不变. 在需要用符号表示常量时, 使用 拉丁字母表中开始部分的字母 $a, b, c$ 等. 这是分析与代数的不同. 代数的考察对象是固 定的量, 在代数中 $a, b, c$ 等代表已知数, $x, y, z$ 等代表末知数. 而分析中前者代表常量, 后 者代表变量.

\section{$\S 2$}

变量是不确定的,是可以取不同数值的量.

确定的量都只可以是一个数, 变量可以取每一个数. 也即确定的量, 或者常量与变量 的关系有如单个事物与一类事物. 一类事物包含这类事物的每一个, 变量包含每一个确 定的量. 变量通常用拉丁字母表中结尾部分的字母 $x, y, z$ 等表示.

\section{$\S 3$}

指定变量为某个确定的值,它就成了常量.

变量可以取任何数,因而它的确定方式是无穷的. 取不遍所有确定的数,这变量就依 然是变量, 不是常量, 这样变量就包容着正数和负数、整数和分数、无理数和超越数等这 一切数. 零和虚数也一样地在它的取值范围之中.

\section{$\S 4$}

变量的函数是变量、常量和数用某种方式联合在一起的解析表达式.

只含一个变量 $z$, 余者都为常量, 这样的解析表达式叫做 $z$ 的函数. 表达式
\[
a+3 z, a z-4 z^{2}, a z+b \sqrt{a^{2}-z^{2}}, z^{2}
\]
等就都是 $z$ 的函数. 

\section{$\S 5$}

变量的函数本身也是一个变量.

可以用任何一个确定的值来代替变量, 因而函数可以取无穷多个值. 又由于变量可 以取虚数值, 因而函数可以取任何值. 例如, 函数 $\sqrt{9-z^{2}}$, 如果限制 $z$ 只取实数值, 那么 $\sqrt{9-z^{2}}$ 就取不到大于 3 的值. 如果允许 $z$ 取虚数值, 那就没有 $\sqrt{9-z^{2}}$ 取不到的值. 例如, 可以让 $z$ 取 $5 \sqrt{-1}$. 但有时会遇到只是像函数的函数, 不管变量取什么值, 它总保持为常 数. 例如
\[
z^{0}, 1^{z}, \frac{a^{2}-a z}{a-z}
\]
它们样子像函数,但实际上都是常量.

\section{$\S 6$}

函数由变量与常量联合而成. 函数之间的基本区别就在于这联合方式.

联合方式决定于运算,运算规定量之间的关系. 这运算首先是代数运算, 即加、减、 乘、除、乘方和开方, 以及解方程. 其次是超越运算, 即指数运算, 对数运算, 以及积分学提 供的大量其他运算等.

这里指出两种简单的函数, 一种是倍数, 例如
\[
2 z, 3 z, \frac{3}{5} z, a z, \cdots
\]
再一种是幂, 例如
\[
z^{2}, z^{3}, z^{\frac{1}{2}}, z^{-1}, \cdots
\]
它们都只含有单一的一种运算. 下面我们对包含多于一种运算的表达式加以分类, 并赋 予每类一个名称.

\section{$\S 7$}

函数分为代数函数和超越函数, 前者只含代数运算, 后者含有超越运算. $z$ 的倍数, $z$ 的幂以及由前面所说的代数运算形成的任何一个表达式, 例如
\[
\frac{a+b z^{n}-c \sqrt{2 z-z^{2}}}{a^{2} z-3 b z^{3}}
\]
都是代数函数,代数函数常常不能显式表出. 例如由方程
\[
Z^{5}=a z^{2} Z^{3}-b z^{4} Z^{2}+c z^{3} Z-1
\]
确定的 $z$ 的函数 $Z$ 就不能显式表出. 虽然这个方程解不出, 但可以肯定这个 $Z$ 等于 $z$ 和常 数构成的某个表达式, 因而这个 $Z$ 是 $z$ 的函数. 关于超越函数要指出的一点是, 超越运算必须作用于变量. 如果超越运算只作用于常量, 这样的函数依旧是代数函数. 例如, 记半 径为 1 的圆的周长为 $c$, 这 $c$ 是个超越量. 对这个超越量 $c$, 表达式
\[
c+z, c z^{2}, 4 z^{c}
\]
等仍然是代数函数. 有人对 $z^{c}$ 是否为代数函数提出疑问, 也有人认为指数为无理数的幂, 如 $z^{\sqrt{2}}$, 不该归人代数函数, 并给它们起了个名字叫半超越函数. 这都无关紧要.

\section{$\S 8$}

代数函数又分为有理函数和无理函数. 有理函数其变量不受根号作用, 无理函数其 变量受到根号的作用.

有理函数只含有加、减、乘、除和整数次的乘方运算. 如
\[
a+z, a-z, a z, \frac{a^{2}+z^{2}}{a+z}, a z^{3}-b z^{5}
\]
等就都是 $z$ 的有理函数. 而表达式
\[
a+\sqrt{a^{2}-z^{2}},\left(a-2 z+z^{2}\right)^{\frac{1}{3}}, \frac{a^{2}-z \sqrt{a^{2}+z^{2}}}{a+z}
\]
就都是无理函数.

无理函数又分为显式的和隐式的. 显式无理函数, 如我们刚举出的例子, 是可以用根 号表示出来. 隐式无理函数是从方程产生的. 例如,方程
\[
Z^{7}=a z Z^{2}-b z^{5}
\]
确定的 $Z$ 就是 $z$ 的隐式函数. 代数理论还没有达到能够从该方程求出 $Z$ 的显式表达式这 样的完善程度, 允许使用根号也不行.

\section{$\S 9$}

有理函数又分为整函数和分数函数.

分母中不含变量 $z$, 且变量 $z$ 的指数中没有负数, 这样的有理函数叫整函数. 分母中含 有 $z$, 或者 $z$ 的指数中有负数, 这样的有理函数叫分数函数. 整函数的一般形状为
\[
a+b z+c z^{2}+d z^{3}+e z^{4}+f z^{5}+\cdots
\]
凡整函数都在该表达式之中, 由于几个分数可以合成为一个分数, 所以分数函数的形状 都为
\[
\frac{a+b z+c z^{2}+d z^{3}+e z^{4}+f z^{5}+\cdots}{\alpha+\beta z+\gamma z^{2}+\delta z^{3}+\varepsilon z^{4}+\zeta z^{5}+\cdots}
\]
这里须指出一点, 常量 $a, b, c, d, \cdots$ 和 $\alpha, \beta, \gamma, \delta, \cdots$ 可为正数, 可为负数; 可为整数, 可为 分数;可为无理数,甚至可为超越数. 这都不影响该表达式为分数函数. 

\section{$\S 10$}
接下来我们考虑单值函数和多值函数.

单值函数指, 从变量 $z$ 的每一个值都只得到一个确定的函数值; 多值函数指, 从变量 $z$ 的每一个值都可以得到多于一个确定的函数值. 有理函数中的整函数和分数函数都是单 值函数, 因为这类表达式, 每一个 $z$ 值都只产生一个函数值. 无理函数都是多值的, 根号 给出两个值. 超越函数就不同了, 有单值的, 也有多值的, 甚至有无穷多值的. 反正弦函数 就是无穷多值的, 其变量 $z$ 的每一个值都对应无穷多个角度.

我们用字母 $P, Q, R, S, T$ 等表示 $z$ 的单值函数.

\section{$\S 11$}

二值函数, 指从每一个 $z$ 值都得到函数 $Z$ 的两个值.

平方根, 例如 $\sqrt{2 z+z^{2}}$, 就是二值函数. 对每一个 $z$ 值, 表达式 $\sqrt{2 z+z^{2}}$ 都有一正一负 两个值. 一般地,如果 $Z$ 由二次方程
\[
Z^{2}-P Z+Q=0
\]
确定, 它就是一个二值函数. 当然, 这里假定 $P, Q$ 都是 $z$ 的单值函数. 从这个二次方程我 们得到
\[
Z=\frac{1}{2} P \pm \sqrt{\left(\frac{1}{4}\right) P^{2}-Q}
\]
也即每一个确定的 $z$ 值都对应两个确定的 $Z$ 值. 须指出, $Z$ 的两个值必定同为实数或同为 虚数,而且由方程的知识我们知道: 这两个值, 和等于 $P$, 积等于 $Q$.

\section{$\S 12$}

三值函数,指每一个 $z$ 值都给出函数的三个确定的值.

三次方程的解就是一个三值函数. 如果 $P, Q, R$ 是 $z$ 的单值函数,且
\[
Z^{3}-P Z^{2}+Q Z-R=0
\]
那么 $Z$ 就是 $z$ 的三值函数, 因为从 $z$ 的任何一个值都能得到 $Z$ 的三个值. $Z$ 的这三个值, 必 定或者全为实数, 或者一实两虚, 且这三个值, 和等于 $P$, 积等于 $R$, 两个两个之积的和等 于 $Q$.

\section{$\S 13$}

四值函数, 指每一个 $z$ 值都给出函数的四个确定的值.

四次方程的解就是一个四值函数. 如果 $P, Q, R, S$ 都是 $z$ 的单值函数,且
\[
Z^{4}-P Z^{3}+Q Z^{2}-R Z+S=0
\]
那么 $Z$ 就是 $z$ 的四值函数, 因为 $z$ 的每一个值都对应 $Z$ 的四个值. 这四个值, 或者都是实 的, 或者两实两虚, 或者都是虚的. 而且这四个值, 和等于 $P$, 积等于 $S$, 两个两个之积的和 等于 $Q$, 三个三个之积的和等于 $R$. 类似地可以定义五值函数、六值函数, 等等.

\section{$\$ 14$}

这样一来,如果 $Z$ 由方程
\[
Z^{n}-P Z^{n-1}+Q Z^{n-2}-R Z^{n-3}+S Z^{n-4}-\cdots=0
\]
确定,那么 $Z$ 就是 $z$ 的 $n$ 值函数, 从 $z$ 的每一个值都可以得到 $Z$ 的 $n$ 个值.

这里应该指出, $n$ 必须为整数. 也即要知道 $Z$ 是 $z$ 的 $n$ 值函数, 应先化 $Z$ 的方程为有理 形式, 这时 $Z$ 的最高次幂的次数为 $n, Z$ 就是 $z$ 的 $n$ 值函数, 从 $z$ 的每一个值就可以得到 $Z$ 的 $n$ 个值. 还应记住, $P, Q, R, S, \cdots$ 都应该是单值函数. 如果其中某一个是多值的,那么从 $z$ 的每一个值得到的 $Z$ 值的个数, 将比 $P, Q, R, S, \cdots$ 都是单值函数时多很多. $Z$ 值中虚数 的个数必定为偶数. 由此我们得到, 如果 $n$ 为奇数, 则 $Z$ 的值中至少有一个是实的, 如果 $n$ 是偶数, $Z$ 可以没有实值.

\section{$\S 15$}

如果 $z$ 的多值函数 $Z$ 恒有并且只有一个实值, 那么这个 $Z$ 就可以被看成单值函数, 并 且在多数情况下就可以当作单值函数来使用. 如
\[
\sqrt[3]{P}, \sqrt[5]{P}, \sqrt[7]{P}, \cdots
\]
就是这样的函数, $P$ 为 $z$ 的单值函数时, 它们都给出并且只给出一个实值, 其余的值都是 虚的. 因此形状如 $P^{\frac{m}{n}}$ 的函数, 不管 $m$ 为奇数还是偶数, 只要 $n$ 是奇数, 就可以当作单值函 数. 如果 $n$ 为偶数,那么 $P^{\frac{m}{n}}$ 或者没有实根, 或者有两个实数. 因此 $n$ 为偶数时, 表达式 $P^{\frac{m}{n}}$ 可以被当作二值函数. 这里要求 $\frac{m}{n}$ 为最简单分数.

\section{$\S 16$}

如果 $y$ 是 $z$ 的函数,那么 $z$ 也就是 $y$ 的函数.

$y$ 是 $z$ 的函数, 不管是单值的还是多值的, 那就有一个方程. 通过这个方程, $y$ 由 $z$ 和常 量决定. 通过这同一个方程, $z$ 也可以由 $y$ 和常量决定. 这样 $z$ 就可以等于由 $y$ 和常量构成 的表达式. 这就是说 $z$ 是 $y$ 的函数. 并且我们也可以得出从一个 $y$ 值能确定几个 $z$ 值. 可以 有这样的情形, $y$ 是 $z$ 的单值函数,但 $z$ 是 $y$ 的多值函数. 例如, $y, z$ 通过方程 $y^{3}=a y z-b z^{2}$ 相联系时, $y$ 是 $z$ 的三值函数,而 $z$ 是 $y$ 的二值函数. 

\section{$\S 17$}

如果 $y$ 和 $x$ 都是 $z$ 的函数,那么 $y$ 和 $x$ 就也互为对方的函数.

$y$ 是 $z$ 的函数, 从而 $z$ 也是 $y$ 的函数; 类似地 $z$ 也是 $x$ 的函数. 这两个函数 $z$ 相等, 由此 得到一个关于 $x, y$ 的方程. 通过这个方程, $y$ 和 $x$ 可互由对方表出, 也即互为对方的函数. 由于代数技巧的不足, 两个函数 $z$ 往往都不是显式的, 但这并不影响它们相等这一性质. 再者, 给定两个方程, 一个含 $y$ 和 $z$, 一个含 $x$ 和 $z$, 那么用传统的方法消去 $z$, 我们就也得到 一个表示 $x$ 和 $y$ 之间关系的方程.

\section{$\S 18$}

下面我们考虑特殊的几类函数. 先考虑偶函数. $z$ 取 $+k$ 和 $-k$ 时, 函数值相等的函数 叫偶函数.

$z^{2}$ 就是 $z$ 的一个偶函数. $z=k$ 和 $z=-k$ 时, 表达式 $z^{2}$ 的值相同, 都是 $z^{2}=k^{2}$. 类似地, $z^{4}, z^{6}, z^{8}$, 一般地, 只要 $m$ 为偶数, 不管为正为负, 幂 $z^{m}$ 都为偶函数. 另外, 由于当 $n$ 为奇数 时, 可以把 $z^{\frac{m}{n}}$ 当作单值函数, 所以 $m$ 为偶数 $n$ 为奇数时, $z^{\frac{m}{n}}$ 是偶函数. 进一步, 由偶次幂以 任何方式组成的函数仍然是偶函数, 例如
\[
\begin{aligned}
& Z=a+b z^{2}+c z^{4}+d z^{6}+\cdots \\
& Z=\frac{a+b z^{2}+c z^{4}+d z^{6}+\cdots}{\alpha+\beta z^{2}+\gamma z^{4}+\delta z^{6}+\cdots}
\end{aligned}
\]
都是 $z$ 的偶函数. 对分数指数也类似
\[
\begin{gathered}
Z=a+b z^{\frac{2}{3}}+c z^{\frac{2}{5}}+d z^{\frac{4}{7}}+\cdots \\
Z=a+b z^{-\frac{2}{3}}+c z^{-\frac{4}{3}}+d z^{-\frac{2}{5}}+\cdots \\
Z=\frac{a+b z^{\frac{2}{7}}+c z^{-\frac{4}{5}}+d z^{\frac{8}{3}}}{\alpha+\beta z^{\frac{2}{3}}+\gamma z^{-\frac{2}{5}}+\delta z^{\frac{4}{7}}}
\end{gathered}
\]
都是 $z$ 的偶函数. 而且这类表达式全是单值函数, 因而也称它们为单值偶函数.

\section{$\S 19$}

$z=+k$ 和 $z=-k$ 时取值完全相同的多值函数称为多值偶函数.

含 $Z$ 和 $z$ 的方程, $Z$ 的最高次数为 $n, Z$ 就是 $z$ 的 $n$ 值函数. 如果 $z$ 的次数都是偶数,那 么这种方程确定的就是 $Z$ 的多值偶函数. 如果
\[
Z^{2}=a z^{4} Z+b z^{2}
\]
那么 $Z$ 就是 $z$ 的二值偶函数; 如果
\[
Z^{3}-a z^{2} Z^{2}+b z^{4} Z-c z^{8}=0
\]
那么 $Z$ 就是 $z$ 的三值偶函数. 如果 $P, Q, R, S, T$ 等表示 $z$ 的单值偶函数,那么
\[
Z^{2}-P Z+Q=0
\]
确定的 $Z$ 就是 $z$ 的二值偶函数
\[
Z^{3}-P Z^{2}+Q Z-R=0
\]
确定的 $Z$ 就是 $z$ 的三值偶函数,类推.

\section{$\S 20$}

可见由常量和变量 $z$ 构成的偶函数, 不管单值的还是多值的, $z$ 的次数都必须是偶数. 我们已经举过一些这样的单值函数的例子, 再如, 表达式
\[
a+\sqrt{b^{2}-z^{2}}, a z^{2}+\sqrt[3]{a^{6} z^{4}-b z^{2}}, a z^{\frac{2}{3}}+\sqrt[3]{z^{2}+\sqrt{a^{4}-z^{4}}}
\]
等也是这样的函数.

因而可定义偶函数为 $z^{2}$ 的函数.

如果 $y=z^{2}$, 而 $Z$ 是 $y$ 的函数, 那么换 $y$ 为 $z^{2}, Z$ 就成了指数全为偶数的 $z$ 的函数. 须指 出, 作为 $y$ 的函数, $Z$ 中不能含有 $\sqrt{y}$ 或类似于 $\sqrt{y}$ 的, 使得换 $y$ 为 $z^{2}$ 时产生 $z$ 的奇次幂. 例如 $y+\sqrt{a y}$ 是 $y$ 的函数, 但换 $y$ 为 $z^{2}$, 它成为 $z^{2}+z \sqrt{a}$, 不是 $z$ 的偶函数. 排除掉这种情况, 我 们做成的偶函数就都是 $z^{2}$ 的函数. 这定义既适用又便于构成偶函数.

\section{$\S 21$}

换 $z$ 为 $-z$ 时, 其值变号的函数叫 $z$ 的奇函数.

$z$ 的奇次幂 $z^{1}, z^{3}, z^{5}, z^{7}, \cdots$ 及 $z^{-1}, z^{-3}, z^{-5}, \cdots$ 都是 $z$ 的奇函数. 当 $m, n$ 都是奇数时, $z^{\frac{m}{n}}$ 也是奇函数. 更一般地, 由这类幂组成的表达式, 如
\[
a z+b z^{3}, a z+a z^{-1}
\]
及
\[
z^{\frac{1}{3}}+a z^{\frac{3}{5}}+b z^{-\frac{5}{3}}
\]
等都是 $z$ 的奇函数. 奇函数的形式及性质的获得都可以比照着偶函数进行.

\section{$\S 22$}

$z$ 的偶函数乘上 $z$ 或 $z$ 的任何一个奇函数,得到的积为 $z$ 的奇函数.

设 $P$ 是 $z$ 的偶函数, 则换 $z$ 为 $-z$ 时, $P$ 的值不变, 这时换 $P z$ 的 $z$ 为 $-z$ 得 $-P z$. 即 $P z$ 是奇函数. 现在设 $P, Q$ 分别为 $z$ 的偶函数和奇函数, 由定义知, 换 $z$ 为 $-z$ 时, $P$ 的值不变, $Q$ 的值变为 $-Q$. 因而, 换 $P Q$ 的 $z$ 为 $-z$ 时, 其值变为 $-P Q$, 即 $P Q$ 是奇数. 例如 $a+$ $\sqrt{a^{2}+z^{2}}$ 是偶函数, $z^{3}$ 是奇函数, 所以乘积
\[
a z^{3}+z^{3} \sqrt{a^{2}+z^{2}}
\]
是 $z$ 的奇函数. 类似有
\[
z\left(\frac{a+b z^{2}}{\alpha+\beta z^{2}}\right)=\frac{a z+b z^{3}}{\alpha+\beta z^{2}}
\]
是 $z$ 的奇函数. 从这里的讨论中可以看到, 如果函数 $P, Q$ 一奇一偶, 则它们的商 $\frac{P}{Q}$ 和 $\frac{Q}{P}$ 都 是奇函数.

\section{$\S 23$}

两个奇函数,相乘相除, 结果都为偶函数.

设 $Q, S$ 都是 $z$ 的奇函数, 那么换 $z$ 为 $-z$ 时, $Q$ 变为 $-Q, S$ 变为 $-S$. 从而换 $z$ 为 $-z$ 时, 积 $P Q$ 和商 $\frac{Q}{S}$ 的值都不变, 也即它们都是 $z$ 的偶函数. 显然, 任何一个奇函数, 其平方是偶 函数,立方是奇函数,四次方是偶函数,类推.

\section{$\S 24$}

如果 $y$ 是 $z$ 的奇函数,则 $z$ 也是 $y$ 的奇函数.

由 $y$ 是 $z$ 的奇函数我们知道, 换 $z$ 为 $-z$ 时, $y$ 变为 $-y$. 现在将这个 $z$ 确定为 $y$ 的函数, 那么换 $y$ 为 $-y$ 时, $z$ 必定变为 $-z$, 也即 $z$ 是 $y$ 的奇函数. 例如 $y=z^{3}$, 这 $y$ 是 $z$ 的奇函数, 解出 $z$ 得 $z^{3}=y$ 或 $z=y^{\frac{1}{3}}$. 这个 $z$ 是 $y$ 的奇函数. 再如 $y=a z+b z^{3}$, 这个 $y$ 是 $z$ 的奇函数. 反 之,由方程 $b z^{3}+a z=y$ 得到的由 $y$ 表示的 $z$, 也是 $y$ 的奇函数.

\section{$\S 25$}

如果在确定 $y$ 为 $z$ 的函数的方程里, 各项中 $y, z$ 指数之和, 同为偶数或同为奇数,则 $y$ 为 $z$ 的奇函数.

在这样的方程中, 同时换 $z$ 为 $-z$, 换 $y$ 为 $-y$, 则各项或者都不变, 或者都变为负的. 这两种情况下方程都是不变的. 也即在两种情况下, $y$ 由 $z$ 确定的方式跟 $-y$ 由 $-z$ 确定的 方式都是一样的. 因此, $z$ 换为 $-z$ 时, $y$ 就换为 $-y$. 也即 $y$ 是 $z$ 的奇函数. 例如, 方程
\[
\begin{gathered}
y^{2}=a y z+b z^{2}+c \\
y^{3}+a y^{2} z=b y z^{2}+c y+d z
\end{gathered}
\]
中的 $y$ 都是 $z$ 的奇函数.

\section{$\S 26$}

如果 $Y$ 是 $y$ 的函数, $Z$ 是 $z$ 的函数,且 $Y$ 同 $y$ 及常数之间的关系跟 $Z$ 同 $z$ 及常数之间, 这两种关系完全相同,那么我们就称 $Y, Z$ 是相似函数.

例如,当
\[
Z=a+b z+c z^{2}, Y=a+b y+c y^{2}
\]
时, 这 $Z, Y$ 就是相似函数. 类似地, 在多值函数情况下, 当
\[
Z^{3}=a z^{2} Z+b, Y=a y^{2} Y+b
\]
时, 这 $Z, Y$ 也是相似函数. 由此我们得到, 当 $y$ 的函数 $Y$ 与 $z$ 的函数 $Z$ 相似时, 换 $y$ 为 $z$, 函 数 $Y$ 就变成了函数 $Z$. 这种相似, 我们把它说成是 $Y$ 与 $y$ 之间的函数关系同于 $Z$ 与 $z$ 之间 的函数关系. 变量 $y, z$ 之间有无依赖关系我们都可以使用这种说法. 例如, $y$ 的函数
\[
a y+b y^{3}
\]
相似于 $y+n$ 的函数
\[
a(y+n)+b(y+n)^{3}
\]
这里我们视 $y+n$ 为 $z$, 又 $z$ 的函数
\[
\frac{a+b z+c z^{2}}{\alpha+\beta z+\gamma z^{2}}
\]
相似于 $\frac{1}{z}$ 的函数
\[
\frac{a z^{2}+b z+c}{\alpha z^{2}+\beta z+\gamma}
\]
这里我们视 $\frac{1}{z}$ 为 $y$. 相似函数这一概念在整个高等分析中有着广泛的应用. 一元函数的 一般知识先讲这些, 进一步的解释将在后面的应用中给出. 

