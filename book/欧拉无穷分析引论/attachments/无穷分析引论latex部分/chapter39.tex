\chapter{第二十一章 超越曲线}

\section{$\S 506$}

到现在为止, 我们讨论的都是代数曲线. 代数曲线的特点是: 在任何轴上取定了横 标, 则对应的纵标都是横标的代数函数, 或者横标与纵标之间的关系由代数方程表示. 这 就是说, 如果纵标不能用横标的代数函数表示, 曲线就不是代数曲线. 非代数曲线统称之 为超越曲线. 也即横纵标间关系不能用代数方程表示的曲线叫超越曲线. 这样, 只要纵标 $y$ 是 $x$ 的超越函数,曲线就是超越的.

\section{$\S 507$}

上册我们讨论了两类超越函数, 即对数函数和三角函数, 如果纵标 $y$ 等于横标 $x$ 的 对数; 或者纵标 $y$ 等于用横标 $x$ 的正弦和余弦或正切表示的圆弧, 即 $y=\lg x, y=\arcsin x$, $y=\arccos x, y=\arctan x$; 再或者这对数这圆弧含于 $x, y$ 间的方程. 这三种情况下曲线都 是超越的. 但这只是超越曲线中的几种. 这几种之外还有无穷多个超越表达式, 它们的产 生在无穷分析中详细讲解,超越曲线的数目比代数曲线要多许多倍.

\section{$\S 508$}

代数函数以外的函数都是超越函数, 含超越函数的方程表示的都是超越曲线. 代数 方程, 或者是有理的, 不含整数指数以外的指数; 或者是无理的, 含分数指数. 但无理代数 方程都可化成有理的. 这样, 只要 $x, y$ 间的方程不是有理的, 也不能化成有理的, 它表示 的曲线就是超越的. 因而, 如果一个方程中含有这样的幂, 其指数既不是整数也不是分 数, 而且用任何方式也不能化它为有理数,那么这方程表示的曲线就是超越的. 由此我们 得到第一类, 也是最简单的一类超越曲线, 即方程中含有无理指数的超越曲线. 这种方程 既不含对数也不含圆弧, 纯粹是从无理数概念产生的, 在很大程度上这属于几何. 莱布尼 兹称这种曲线为次超越的, 意思是它处在代数曲线与超越曲线之间.

\section{$\S 509$}

方程 $y=x^{\sqrt{2}}$ 表示的就是次超越曲线,因为没有方法可以把这个方程化成有理的. 几 何上也没有办法可以把它表示的曲线画出来. 几何上画不出有理指数幂以外的幂. 可见 这类曲线与代数曲线的差别极大. 如果想近似地画出该曲线, 可以取 $\sqrt{2}$ 的近似值
\[
\frac{3}{2}, \frac{7}{5}, \frac{17}{12}, \frac{41}{29}, \frac{96}{70}, \cdots
\]
中的一个来代替 $\sqrt{2}$, 那么我们就得到靠近该曲线的代数曲线. 这代数曲线的阶数, 可以 或者是 3 , 或者是 7 , 或者是 17 , 或者是 41 , 等等. 因为 $\sqrt{2}$ 可以用分子分母都是无穷大的分 数有理表示, 所以应该认为 $y=x^{\sqrt{2}}$ 表示的是无穷大阶曲线, 也即它不是代数曲线. 此外, $\sqrt{2}$ 有两个值, 一正一负, 对应地得到两个 $y$ 值,也就得到两条曲线.

\section{$\S 510$}

如果要准确地画出这条曲线,那就必须借助于对数. 对 $y=x^{\sqrt{2}}$ 取对数,得 $\lg y=\sqrt{2} \lg x$, 即横标的对数乘 $\sqrt{2}$ 等于纵标的对数. 因而对横标 $x$ 的每一个值, 我们都可以利用对数表 得到对应的 $y$ 值. 例如 $x=0$, 则 $y=0 ; x=1$, 则 $y=1$. 这两个值都容易从方程得到; $x=2$, 则 $\lg y=\sqrt{2} \lg 2=\sqrt{2} \times 0.3010300$, 将 $\sqrt{2}=1.41421356$ 代入, 得 $\lg y=0.4257207$. 由 此近似地有 $y=2.665144: x=10$, 得 $\lg y=1.41421356$, 由此得 $y=25.954554$. 用这样 的方法, 对每一个横标 $x$ 我们都可以算出对应的纵标 $y$, 也就可以画出曲线. 这里一个前 提是横标 $x$ 都为正. 如果 $x$ 取负值, 那就确定不了 $y$ 值是实数还是虚数. 例如 $x=-1$ 时, 就确定不了 $(-1)^{\sqrt{2}}$ 是实数还是虚数, 此时 $\sqrt{2}$ 的近似值邦不了忙.

\section{$\S 511$}

含有虚指数的方程表示的曲线是超越的,这疑问就更小. 但是含虚数指数的表达式 表示的可以是确定的实数值. 我们已经见过这样的例子, 这里再举一个, 设
\[
2 y=x^{+\sqrt{-1}}+x^{-\sqrt{-1}}
\]
这里虽然 $x^{+\sqrt{-1}}$ 和 $x^{-\sqrt{-1}}$ 都是虚量,但它们的和是实数. 事实上, 设 $x$ 的双曲对数 $\ln x=$ $v$, 记双曲对数为 1 的数为 $\mathrm{e}$, 则 $x=\mathrm{e}^{v}$. 代入所给方程, 得
\[
2 y=\mathrm{e}^{+v \sqrt{-1}}+\mathrm{e}^{-v \sqrt{-1}}
\]
上册 $\S 138$ 推出了
\[
\frac{\mathrm{e}^{+v \sqrt{-1}}+\mathrm{e}^{-v \sqrt{-1}}}{2}=\cos v
\]
从而
\[
y=\cos v=\cos \ln x
\]
这样, 对给定的 $x$ 值, 先取它的双曲对数, 再在半径为 1 的圆上截一段等于这个对数的 弧, 那么这段弧的余弦就是纵标 $y$ 的值. 例如, 取 $x=2$, 即
\[
2 y=2^{+\sqrt{-1}}+2^{-\sqrt{-1}}
\]
则
\[
y=\cos \ln 2=\cos 0.6931471805599
\]
由等于 $3.1415926535$ 的弧所对角为 $180^{\circ}$, 利用黄金规则得等于 $\log 2$ 的弧所对角为 $39^{\circ} 41^{\prime} 51^{\prime \prime} 52^{\prime \prime \prime} 8^{\prime \prime \prime \prime}$, 它的余弦等于 $0.76923890136400$. 这个数就是横标 $x=2$ 所对应的纵 标 $y$ 的值. 这类表达式既含对数又含圆弧, 当然是超越的.

\section{$\S 512$}

这样,超越曲线中占第一位的是,其方程在代数量之外还含有对数的曲线. 这种曲线 中最简单的,其方程为
\[
\log \frac{y}{a}=\frac{x}{b} \text { 或 } x=b \log \frac{y}{a}
\]
这里对数取哪一种都可以, 因为用乘以常数 $b$ 的方法都可以把它们化成同一种. 用 $\log$ 表 示双曲对数,方程 $x=b \log \frac{y}{a}$ 表示的曲线叫对数曲线, 记双曲对数为 1 的数为 $\mathrm{e}$
\[
\mathrm{e}=2.71828182845904523536028
\]
得
\[
\mathrm{e}^{x: b}=\frac{y}{a} \text { 或 } y=a \mathrm{e}^{x: b}
\]
从这个方程易于看出对数曲线的性质. 事实上,依次用成算术级数的数代 $x$, 则对应的 $y$ 为成几何级数的数, 为便于作图, 令
\[
\mathrm{e}=m^{n}, \quad b=n c
\]
则
\[
y=a m^{x: c}
\]
其中 $m$ 为大于 1 的任何数, 取 $x$ 为
\[
x=0, c, 2 c, 3 c, 4 c, 5 c, 6 c, \cdots
\]
则 $y$ 为
\[
y=a, a m, a m^{2}, a m^{3}, a m^{4}, a m^{5}, a m^{6}, \cdots
\]
取 $x$ 为负值
\[
x=-c,-2 c,-3 c,-4 c,-5 c, \cdots
\]
则 $y$ 为
\[
y=\frac{a}{m}, \frac{a}{m^{2}}, \frac{a}{m^{3}}, \frac{a}{m^{4}}, \frac{a}{m^{5}}, \cdots
\]
\section{$\S 513$}

由此可见, 纵标 $y$ 恒正, 且随横标 $x$ 增加到正无穷而增长到无穷, 如图 101 所示. 横标 $x$ 减小到负无穷时, 轴 $A p$ 成为曲线的渐近线. 取 $A$ 作横标原点, 记此处纵标 $A B=a$, 记横 标 $A P=x$, 则纵标 
\[
P M=y=a m^{x ; c}=a \mathrm{e}^{x: b}
\]
从而
\[
\log \frac{y}{a}=\frac{x}{b}
\]
即横标 $A P$ 除以 $b$, 得到的商等于比 $\frac{P M}{A B}$ 的对数. 任取轴上另 外一点 $a$ 作原点,方程的形状不变. 事实上,记 $A a$ 为 $f$, 记 $a P$ 为 $t$, 则由 $x=t-f$ 得
\[
y=a \mathrm{e}^{(t-\rho): b}=a \mathrm{e}^{t: b}: \mathrm{e}^{f: b}
\]
记常数 $a: \mathrm{e}^{f: b}$ 为 $g$, 则 $y=g \mathrm{e}^{t: b}$, 由 $a b=g$ 得
\[
\frac{a P}{b}=\log \frac{P M}{a b}
\]

【图,待补】
%%![](https://cdn.mathpix.com/cropped/2023_02_05_d0626289dd3c515543f1g-10.jpg?height=462&width=371&top_left_y=358&top_left_x=1088)

图 101

因而对任何两条距离为 $P p$ 的纵标 $P M$ 和 $p m$ 我们有
\[
\frac{P p}{b}=\log \frac{P M}{p m}
\]
式中的常数 $b$ 类似于对数参数.

\section{$\S 514$}

对数曲线任何一点 $M$ 处的切线都易于确定. 事实上, 记点 $M$ 的横标 $A P=x$, 则点 $M$ 的纵标 $P M=a \mathrm{e}^{x: b}$, 对任何另外的纵标 $Q N$ 我们有
\[
Q N=a \mathrm{e}^{(x+u): b}=a \mathrm{e}^{x: b} \cdot \mathrm{e}^{u: b}
\]
$u$ 为 $P M$ 至 $Q N$ 的距离, 即 $P Q=u$. 对平行于轴的直线 $M L$, 有
\[
L N=(Q N-P M)=a \mathrm{e}^{x: b}\left(\mathrm{e}^{u: b}-1\right)
\]
过点 $N, M$ 画直线 $N M T$, 交轴于 $T$, 则
\[
L N: M L=P M: P T
\]
从而
\[
P T=u:\left(\mathrm{e}^{u: b}-1\right)
\]
上册我们证明了
\[
\mathrm{e}^{u: b}=1+\frac{u}{b}+\frac{u^{2}}{2 b^{2}}+\frac{u^{3}}{6 b^{3}}+\cdots
\]
从而
\[
P T=\frac{1}{\frac{1}{b}+\frac{u}{2 b^{2}}+\frac{u^{2}}{6 b^{3}}+\cdots}
\]
区间 $P Q=u$ 为零时点 $M$ 与 $N$ 重合, 直线 $N M T$ 就成了曲线的切线, 次切距 $P T=b$ 为常 数. 这是对数曲线的一条美妙的性质, 即次切距为常数, 恒等于对数参数 $b$.

\section{$\S 515$}

这里产生一个问题, 这样画出来的对数曲线完整吗? 在图 101 上画出来的分支 $\mathrm{MBm}$ 之外还有没有从轴的另一侧趋向无穷的分支. 前面讲了, 凡渐近线都有两个分支收 玫于它. 因此有人推定, 对数曲线由位于轴两侧的类似的两部分组成. 因而渐近线同时也 是直径. 但从方程 $y=a \mathrm{e}^{x^{i} b}$ 完全看不出这一点. 因为当 $\frac{x}{b}$ 为整数或者为分母是奇数的分 数时, $y$ 都有一个实值, 并且是正的. 但是当 $\frac{x}{b}$ 为分母是偶数的分数时, $y$ 有两个值, 一正 一负, 负值给出的点在渐近线的另一侧. 由此可见, 对数曲线在渐近线的另一侧还有一个 由离散点组成的无穷集合. 虽然相邻离散点间的距离无限的小, 给人一种它们是线一样 连续的印象, 但它们并不构成连续曲线. 这是难以置信的, 是悖论, 对代数曲线是不会产 生的. 这里还有更加难以置信的另一个悖论. 负数的对象是虚数 (这是明显的, 另外也可 以从 $\log (-1)$ 比 $\sqrt{-1}$ 为有限数得到证实), 因而 $\log (-n)$ 是虚数, 记它为 $\mathrm{i}$ 一个个数的平 方的对数等于这个数自身对数的两倍, 因而我们有 $\log (-n)^{2}=2 \log (-n)=2$ i. 但 $\log n^{2}$ 是实数, 等于 $2 \log n$. 可见实数 $\log n^{2}$ 的一半为实数 $\log n$, 另一半为虚数 i. 由此前一步, 每个数都由不同的两部分组成, 一实一虚, 各等于原数的一半. 类似地, 每个数都由不同 的三部分组成, 各等于原数的三分之一; 每个数都由四部分组成, 各等于原数的四分之 一. 类推, 这不同的部分中只有一部分是实数, 这一点如何与数的通常概念相一致, 还不 清楚.

\section{$\S 516$}

如果承认前面所说, 则数 $a$ 的两半为 $\frac{a}{2}+\log (-1)$ 和 $\frac{a}{2}$, 因为前半的两倍等于
\[
a+2 \log (-1)=a+\log (-1)^{2}=a+\log 1=a
\]
这里应该指出,虽然 $\log (-1) \neq 0$,但
\[
+\log (-1)=-\log (-1)
\]
事实上,由 $-1=\frac{+1}{-1}$ 得
\[
\log (-1)=\log (+1)-\log (-1)=-\log (-1)
\]
类似地,由 $\sqrt[3]{1}$ 不仅仅是 1 , 并且是 $\frac{-1 \pm \sqrt{-3}}{2}$, 我们有
\[
3 \log \frac{-1 \pm \sqrt{-3}}{2}=\log 1=0
\]
从而数 $a$ 的三个三分之一为
\[
\frac{a}{3}, \frac{a}{3}+\log \frac{-1+\sqrt{-3}}{2} \text { 和 } \frac{a}{3}+\log \frac{-1-\sqrt{-3}}{2}
\]
这每个三分之一的三倍都等于 $a$. 这看上去是不能接受的. 为消除疑问我们提出再一个 难以置信的情形: 每个数都有无穷多个对数, 其中只有一个是实的. 例如 1 的对数, 在零 之外就还有
\[
2 \log (-1), 3 \log \frac{-1 \pm \sqrt{-3}}{2}, 4 \log (-1) \text { 和 } 4 \log (\pm \sqrt{-1})
\]
等无穷多个虚数, 它们都可以从 1 开方得到. 下面的考虑使得这里的情形比前面要直观, 设 $x=\log a$, 则 $a=\mathrm{e}^{x}$, 从而
\[
a=1+x+\frac{x^{2}}{2}+\frac{x^{3}}{3}+\frac{x^{4}}{24}+\cdots
\]
这个方程的次数为无穷, 因而有无穷多个根就不足为奇了, 这样我们就解决了最后这个 难以置信, 但第一个难以置信, 即对数曲线在轴的另一侧有无穷多个离散点, 还仍然难以 置信.

\section{$\S 517$}

但是, 借助方程 $y=(-1)^{x}$ 可对无穷多个这种点的存在进行明显得多的证明. $x$ 是偶 数或分子为偶数的分数时 $y$ 都等于 $1 . x$ 是奇数或分子分母都为奇数的分数时 $y$ 都等 于 $-1$. 在其余情况下, 即 $x$ 为分母为偶数的分数或无理数时 $y$ 都为虚数. 可见, 方程 $y=$ $(-1)^{x}$ 给出无穷多个离散点. 它们位于轴的两侧, 至轴的距离都为 1 , 它们中任何两个都 不相接, 又任何相邻的两个的距离都小于任何给定的正数. 横标的两个值, 不管离得多 近, 它们中间都有不只一个, 而是无穷多个分母是奇数的分数. 这每一个分数都给出满足 所给方程的一个点. 这些点位于平行于轴至轴的距离为 1 的两条直线上, 这两条直线上 没有不含方程 $y=(-1)^{x}$ 的离散点的区间. 事实上是, 这两条直线上的每一个区间都含有 方程 $y=(-1)^{x}$ 的无穷多个离散点. 这种异常情形也发生于方程 $y=(-a)^{x}$ 和与这个方 程类似的其他方程, 即 $y$ 等于负数的变指数幂的方程, 应该指出这种异常情形只产生于 超越曲线.

\section{$\S 518$}

含变指数的方程, 也即由对数方程化成的方程, 它们表示的曲 线叫指数曲线, 指数曲线也具有对数曲线的性质. 例如, 方程 $y=$ $x^{x}$, 也即 $\log y=x \log x$ 所表示的曲线即为皆数曲线. $x=0$, 则 $y=$ $1 ; x=1$, 则 $y=1 ; x=2$, 则 $y=4 ; x=3$, 则 $y=27 ; \cdots$. 图 102 上 BDM 是该曲线关于轴 $A P$ 的图形. 取 $A C=1$, 则 $A B=C D=1 . A, C$ 之间 的点对应的纵标都小于 1 , 例如 $x=\frac{1}{2}$ 时


【图,待补】
%%![](https://cdn.mathpix.com/cropped/2023_02_05_d0626289dd3c515543f1g-12.jpg?height=356&width=239&top_left_y=1804&top_left_x=1240)

图 102 
\[
\begin{gathered}
\text { Sufinite analysies 无穷分析引论 Sulxaduclion } \\
\qquad y=\frac{1}{\sqrt{2}}=0.7071068
\end{gathered}
\]
横标
\[
x=\frac{1}{\mathrm{e}}=0.36787944
\]
时纵标 $y=0.6922005$ 为最小值, 这一点后面将给以证明. 为考察该曲线在点 $B$ 另一侧 的情形, 应使 $x$ 值为负, 则 $y=\frac{1}{(-x)^{x}}$. 从而曲线的这一部分完全由收玫于轴的离散点组 成, 这里的收玫同于向渐近线的收玫. 这些点因 $x$ 的奇偶而位于轴的两侧. 又, 对应于 $x$ 为分母为偶数的分数, 也有无穷多个离散点落在 $A P$ 轴之下. 例如, $x=\frac{1}{2}$ 时, $y=+\frac{1}{\sqrt{2}}$, $y=-\frac{1}{\sqrt{2}}$

这样, 连续曲线 $M D B$ 从点 $B$ 处开始一反代数曲线的性质, 突然中断, 改由离散点组 成. 这里我们更清楚地看到了位于直线两侧至直线距离相等的点的存在. 如果不承认它 们存在,那就得承认整个曲线在点 $B$ 处突然中止,这与连续规律矛盾.

\section{$\S 519$}

在可借助对数画出其图形的另外无穷多条曲线中, 有些借助对数实现起来不那么容易, 但可借助适当的代 换使得容易实现,方程
\[
x^{y}=y^{x}
\]
表示的曲线就是这样的. 由方程显然可见,纵标 $y$ 等于横 标 $x$ 时, 即在与轴成半直角的直线上方程满足. 我们还看 到方程 $x^{y}=y^{x}$ 比 $x=y$ 更一般. 后者是前者的一部分, 满 足前者的坐标可以不满足后者, 例如 $x=2, y=4$ 就是, 因 而如图 103 所示,所给方程除了直线 EAF 必定还有另外的分支. 为了求出这另外的分支, 又得到方程所示的整个 曲线, 我们令 $y=t x$, 得 $x^{t x}=t^{x} x^{x}$, 开 $x$ 次方得


【图,待补】
%%![](https://cdn.mathpix.com/cropped/2023_02_05_d0626289dd3c515543f1g-13.jpg?height=390&width=460&top_left_y=1270&top_left_x=1053)

图 103 
\[
x^{t}=t x \text { 或 } x^{t-1}=t
\]
也即
\[
x=\frac{1}{t^{t-1}} \text { 或 } y=\frac{t}{t^{t-1}}
\]
令 $t-1=\frac{1}{u}$, 则
\[
x=\left(1+\frac{1}{u}\right)^{u} \text { 或 } y=\left(1+\frac{1}{u}\right)^{u+1}
\]
这样, 除了直线 $E A F$, 就还有分支 $R S$. 这 $R S$ 渐近于直线 $A G$ 和 $A H$, 并且以 $A F$ 为直径,$R S$ 交 $A F$ 于 $C$, 从而 $A B=B C=\mathrm{e}, \mathrm{e}$ 是对数为 1 的数. 此外还有满足所给方程的无穷多个 离散点. 这离散点与直线 $A F$ 以及曲线 $R S$ 合起来构成满足方程的全体. 可见有无穷多个 由 $x$ 和 $y$ 构成的数对满足方程 $x^{y}=y^{x}$. 下面是这种有理数对中的几个
\[
\begin{array}{cl}
x=2, & y=4 \\
x=\frac{3^{2}}{2^{2}}=\frac{9}{4}, & y=\frac{3^{3}}{2^{3}}=\frac{27}{8} \\
x=\frac{4^{3}}{3^{3}}=\frac{64}{27}, & y=\frac{4^{4}}{3^{4}}=\frac{256}{81} \\
x=\frac{5^{4}}{4^{4}}=\frac{625}{256}, & y=\frac{5^{5}}{4^{5}}=\frac{3125}{1024}
\end{array}
\]
等. 称这种数对中的 $x$ 和 $y$ 为甲和乙, 则甲的乙次幂等于乙的甲次幂. 例如
\[
\begin{gathered}
2^{4}=4^{2}=16 \\
\left(\frac{9}{4}\right)^{\frac{27}{8}}=\left(\frac{27}{8}\right)^{\frac{9}{4}}=\left(\frac{3}{2}\right)^{\frac{27}{4}} \\
\left(\frac{64}{27}\right)^{\frac{256}{81}}=\left(\frac{256}{81}\right)^{\frac{64}{27}}=\left(\frac{4}{3}\right)^{\frac{256}{27}}
\end{gathered}
\]
等.

\section{$\S 520$}

虽然这一条和类似于这一条的另外一些曲线的无穷多个点, 可以用代数方法求出 来, 但它们都并不是代数曲线, 因为它们各自都有用代数方法根本求不出来的无穷多个 另外的点. 现在我们转向另一类超越曲线. 这类曲线要用到圆弧. 为了计算不因大量的符 号而复杂化, 我们只取用单位圆的弧. 虽然“化圆为方不能” 尚末得证, 但易于证明用到 圆弧的曲线不是代数曲线. 我们只考虑这类方程中最简单的, 例如 $\frac{y}{a}=\arcsin \frac{x}{c}$, 即纵标 与正弦为 $\frac{x}{c}$ 的圆弧成正比. 由于同一个正弦 $\frac{x}{c}$ 对应于无穷多个弧, 可见这里的纵标 $y$ 是 无穷多值函数. 纵标线与另外一些直线都在无穷多个点处与这里的曲线相交. 这是它显 然地区别于代数曲线的性质. 设 $s$ 是正弦为 $\frac{x}{c}$ 的最短的弧, $\pi$ 表示等于半圆周的弧, 则 $\frac{y}{a}$ 的值为
\[
\begin{gathered}
s, \pi-s, 2 \pi+s, 3 \pi-s, 4 \pi+s, 5 \pi-s, \cdots \\
-\pi-s,-2 \pi+s,-3 \pi-s,-4 \pi+s,-5 \pi-s, \cdots
\end{gathered}
\]
因而, 如果如图 104 取直线 $C A B$ 作轴, 取 $A$ 为横标原点, 那么 $x=0$ 时, 在 $A$ 的一侧我们有 纵标 $A A^{1}=\pi a, A A^{2}=2 \pi a, A A^{3}=3 \pi a, \cdots$. 在另一侧有 $A A^{-1}=+\pi a, A A^{-2}=+2 \pi a$, $A A^{-3}=+3 \pi a, \cdots$, 曲线经过这每一点. 如果取横标 $A P=x$, 则纵标交曲线于无穷多个点 $M$, 我们有 $P M^{1}=a s, P M^{2}=a(\pi-s), P M^{3}=a(2 \pi-s), \cdots$. 整个曲线由无穷多个相似的 部分 $A E^{1} A^{1}, A^{1} F^{1} A^{2}, A^{2} E^{2} A^{3}, A^{3} F^{2} A^{4}, \cdots$ 组成, 过点 $E, F$ 平行于轴 $B C$ 的每一条直径都是这条曲线的直径. 这时当然有 $A C=A B=c$, 且线段 $E^{1} E^{2}$, $E^{2} E^{3}, E^{1} E^{-1}, E^{-1} E^{-2}$ 及 $F^{1} F^{1}, F^{1} F^{-2}, F^{-1} F^{-2}$ 都等于 $2 a \pi$. 莱布 尼兹称这条曲线为正弦曲线, 因为借助它可以求出任何弧的正 弦. 事实上, 由
\[
\frac{y}{a}=\arcsin \frac{x}{c}
\]
取正弦,得
\[
\frac{x}{c}=\sin \frac{y}{a}
\]
令
\[
\frac{y}{a}=\frac{1}{2} \pi-\frac{z}{a}
\]
则
\[
\frac{x}{c}=\cos \frac{z}{a}
\]
这样同时也就得到了余弦曲线.


【图,待补】
%%![](https://cdn.mathpix.com/cropped/2023_02_05_d0626289dd3c515543f1g-15.jpg?height=783&width=352&top_left_y=288&top_left_x=1145)

图 104

\section{$\S 521$}

类似地可以得到正切曲线, 正切曲线的方程为 $y=\arctan x$, 这里为简单起见, 取 $a=$ $1, c=1$. 对方程取正切, 得
\[
x=\tan y=\frac{\sin y}{\cos y}
\]
这条曲线的形状可以从正切的性质推出. 它有无穷多条相平行的渐近线. 同样地, 由方程
\[
y=\operatorname{arcsec} x \text { 也即 } x=\sec y=\frac{1}{\cos y}
\]
可以画出正割曲线,它有无穷多条伸向无穷远的分支. 这类曲线中极为有名的一条是旋 轮线, 也称为摆线. 它由沿直线滚动的圆周上的一点画出, 其直角坐标方程为
\[
y=\sqrt{1-x^{2}}+\arccos x
\]
这条曲线之所以值得特别注意, 原因有两个,一个是它容易画, 再一个是它具有许多重要 的性质,但是这些性质中的大多数,都必须应用无穷小分析才能进行阐述,我们只对可以 从方程直接推出的主要几条进行简单的考察.

\section{$\S 522$}

假定如图 105 所示, 圆 $A C B$ 在直线 $E A$ 上滚动, 为了使考察具有更为一般的意义, 我 们把圆周上的一点 $B$ 换成直径延长线上的一点 $D$, 考察圆滚动时点 $D$ 画出的曲线 $D d$. 设 圆的半径 $C A=C B=a$, 距离 $C D=b$, 又设开始时刻 $D$ 点位于距 $A E$ 最远处, 且圆滚到了 $a Q b R$. 记距离 $A Q$ 为 $z$, 则弧 $a Q=z$. 用半径 $a$ 除这段弧, 得 $\angle a c Q=\frac{z}{a}$, 而画出曲线的点此时位于 $d$. 因而有 $c d=b$, 且
\[
\angle d c Q=\pi-\frac{z}{a}
\]
$d$ 是所求曲线上的点, 从 $d$ 先向直线 $A Q$ 画垂线 $d p$, 再向直线 $Q R$ 画垂线 $d n$, 则
\[
d n=b \sin \frac{z}{a}, \quad c n=-b \cos \frac{z}{a}
\]
从而
\[
Q n=d p=a+b \cos \frac{z}{a}
\]

【图,待补】
%%![](https://cdn.mathpix.com/cropped/2023_02_05_d0626289dd3c515543f1g-16.jpg?height=477&width=292&top_left_y=288&top_left_x=1166)

图 105

延长直线 $d n$ 到交直线 $A D$ 于 $P$, 记坐标
\[
D P=x, \quad P d=y
\]
则
\[
\begin{gathered}
x=b+c n \text { 或 } x=b-b \cos \frac{z}{a} \\
y=A Q+d n=z+b \sin \frac{z}{a}
\end{gathered}
\]
由
\[
b \cos \frac{z}{a}=b-x
\]
得
\[
\begin{gathered}
b \sin \frac{z}{a}=\sqrt{2 b x-x^{2}} \\
z=a \arccos \left(1-\frac{x}{b}\right)=a \arcsin \frac{\sqrt{2 b x-x^{2}}}{b}
\end{gathered}
\]
将这两个值代入 $y$ 的右端, 得
\[
y=\sqrt{2 b x-x^{2}}+a \arcsin \frac{\sqrt{2 b x-x^{2}}}{b}
\]
如果取 $A D$ 为轴, $C$ 为原点, 并记 $b-x$ 为 $t$, 则
\[
\sqrt{2 b x-x^{2}}=\sqrt{b^{2}-t^{2}}
\]
从而得到 $t, y$ 间的方程
\[
y=\sqrt{b^{2}-t^{2}}+a \arccos \frac{t}{b}
\]
$b=a$ 时, 该方程给出的为通常的旋轮线, $b$ 大于 $a$ 和小于 $a$ 时给出的, 分别称为短幅旋轮 线和长幅旋轮线. 不管哪种情况下, $y$ 都是 $x$ 或 $t$ 的无穷多值函数. 也即, 只要 $x$ 或 $t$ 不使 $\sqrt{2 b x-x^{2}}$ 或 $\sqrt{b^{2}-t^{2}}$ 为虚数, 任何一条平行于 $A Q$ 的直线就都交曲线于无穷多个点.

\section{$\S 523$}

另外两种旋轮线是圆外旋轮线和圆内旋轮线. 参见 图 106. $D$ 是动圆 $A C B$ 圆外或圆内一点. 动圆 $A C B$ 在不动圆 $A O Q$ 的圆周上滚动时点 $D$ 画出曲线 $D d$. 设不动圆的半径 $O A=c$, 动圆的半径 $C A=C B=a$, 点 $D$ 至动圆圆心的距离 $C D=b$. 取直线 $O D$ 作轴, 我们来求曲线 $D d$. 动圆从点 $O, C$, $D$ 在同一条直线上的起始位置滚动到 $Q c R$ 处时, 滚过弧 $A Q=z$, 从而 $\angle A O Q=\frac{z}{c}$. 此时我们有弧 $Q a=A Q=z$, 因而
\[
\angle a c Q=\frac{z}{a}=\angle R c d
\]
距离 $c d=C D=b$, 点 $d$ 在曲线 $D d$ 上, 从点 $d$ 向轴画垂线 $d P$, 从点 $c$ 画轴 $O D$ 的垂线 $c m$ 和平行线 $c n$. 由


【图,待补】
%%![](https://cdn.mathpix.com/cropped/2023_02_05_d0626289dd3c515543f1g-17.jpg?height=524&width=434&top_left_y=451&top_left_x=1085)

图 106
\[
\angle R c n=\angle A O Q=\frac{z}{c}
\]
得
\[
\angle d c n=\frac{z}{c}+\frac{z}{a}=\frac{(a+c) z}{a c}
\]
从而
\[
\begin{aligned}
& d n=b \sin \frac{(a+c) z}{a c} \\
& c n=b \cos \frac{(a+c) z}{a c}
\end{aligned}
\]
再由 $O C=O c=a+c$, 得
\[
\begin{aligned}
c m & =(a+c) \sin \frac{z}{c} \\
O m & =(a+c) \cos \frac{z}{c}
\end{aligned}
\]
进而, 记坐标 $O P=x, P d=y$, 则
\[
\begin{aligned}
& x=(a+c) \cos \frac{z}{c}+b \cos \frac{(a+c) z}{a c} \\
& y=(a+c) \sin \frac{z}{c}+b \sin \frac{(a+c) z}{a c}
\end{aligned}
\]
由此可见, 如果 $\frac{a+c}{a}$ 是有理数, 则 $\frac{z}{c}$ 与 $\frac{(a+c) z}{a c}$ 有公度, $z$ 可消去, 得到的是 $x, y$ 间的代 数方程, 否则这样画出的曲线是超越的.

此外, 应该指出, $a$ 为负值时, 动圆在不动圆之内, 得到的是圆内的旋轮线. 通常取常 数 $b$ 等于 $a$, 即 $D$ 在动圆圆周上, 此时得到的是给了专门名称的圆外旋轮线或圆内旋轮线. 我们推出的是更为一般情况下的方程. 从我们的一般易于导出我们的特殊, 我们的做 法是便捷的,使平方 $x^{2}, y^{2}$ 相加, 得
\[
x^{2}+y^{2}=(a+c)^{2}+b^{2}+2 b(a+c) \cos \frac{z}{a}
\]
$a, c$ 可公度时从该方程消去 $z$ 更为容易.

\section{$\S 524$}

两个圆的半径 $a, c$ 可公度,曲线为代数的,这两种情况之外,应该指出的再一种情况 是 $b=-a-c$, 即曲线上的点 $D$ 在不动圆的圆心 $O$, 由 $b=-a-c$ 得
\[
x^{2}+y^{2}=2(a+c)^{2}\left(1-\cos \frac{z}{a}\right)=4(a+c)^{2} \cos ^{2} \frac{z}{2 a}
\]
从而
\[
\cos \frac{z}{2 a}=\frac{\sqrt{x^{2}+y^{2}}}{2(a+c)}
\]
又由
\[
\begin{aligned}
& x=(a+c)\left(\cos \frac{z}{c}-\cos \frac{(a+c) z}{a c}\right) \\
& y=(a+c)\left(\sin \frac{z}{c}-\sin \frac{(a+c) z}{a c}\right)
\end{aligned}
\]
得
\[
\begin{gathered}
\frac{x}{y}=-\tan \frac{(2 a+c) z}{2 a c} \\
\sin \frac{(2 a+c) z}{2 a c}=\frac{x}{\sqrt{x^{2}+y^{2}}}
\end{gathered}
\]
进而
\[
\cos \frac{(2 a+c) z}{2 a c}=\frac{-y}{\sqrt{x^{2}+y^{2}}}
\]
将
\[
\sqrt{x^{2}+y^{2}}=2(a+c) \cos \frac{z}{2 a}
\]
代入,得
\[
\begin{gathered}
x=2(a+c) \cos \frac{z}{2 a} \sin \frac{(2 a+c) z}{2 a c} \\
y=-2(a+c) \cos \frac{z}{2 a} \cos \frac{(2 a+c) z}{2 a c}
\end{gathered}
\]
假如, 设 $c=2 a$, 则
\[
x=6 a \cos \frac{z}{2 a} \sin \frac{z}{a}
\]
\[
\begin{aligned}
& y=-6 a \cos \frac{z}{2 a} \cos \frac{z}{a} \\
& \sqrt{x^{2}+y^{2}}=6 a \cos \frac{z}{2 a}
\end{aligned}
\]
令
\[
\cos \frac{z}{2 a}=q
\]
则
\[
\sin \frac{z}{2 a}=\sqrt{1-q^{2}}, \quad \sin \frac{z}{a}=2 q \sqrt{1-q^{2}}, \quad \cos \frac{z}{a}=2 q^{2}-1
\]
从而
\[
\begin{gathered}
q=\frac{\sqrt{x^{2}+y^{2}}}{6 a} \\
y=-6 a q\left(2 q^{2}-1\right)=\left(1-2 q^{2}\right) \sqrt{x^{2}+y^{2}}=\left(1-\frac{x^{2}+y^{2}}{18 a^{2}}\right) \sqrt{x^{2}+y^{2}}
\end{gathered}
\]
或
\[
18 a^{2} y=\left(18 a^{2}-x^{2}-y^{2}\right) \sqrt{x^{2}+y^{2}}
\]
令 $18 a^{2}=f^{2}$,两边平方得六次方程
\[
\left(x^{2}+y^{2}\right)^{3}-2 f^{2}\left(x^{2}+y^{2}\right)^{2}+f^{4} x^{2}=0
\]
这里我们要讨论的不是代数曲线, 是超越曲线, 因而不继续. 下面讨论同时含有对数和圆 弧的曲线.

\section{$\S 525$}

前面我们已经求出了方程
\[
2 y=x^{+\sqrt{-1}}+x^{-\sqrt{-1}}
\]
表示的曲线, 见图 107. 那里我们把这个方程化成了 $y=\arccos \log x$. 化成的这个方程可写成
\[
\arccos y=\log x \text { 或 } x=\mathrm{e}^{\arccos y}
\]
如果取直线 $A P$ 作轴, 取 $A P$ 上点 $A$ 作横标原点, 显然, 在点 $A$ 下 面, 在横标为负的区域中, 曲线没有任何连续部分. 但轴 $A P$ 交曲 线于无穷多个点 $D$, 且这些点 $D$ 到点 $A$ 的距离成几何级数. 其中有 距 $A$ 越来越远的无穷多个点
\[
A D=\mathrm{e}^{\frac{\pi}{2}}, A D^{1}=\mathrm{e}^{\frac{3 \pi}{2}}, A D^{2}=\mathrm{e}^{\frac{5 \pi}{2}}, A D^{3}=\mathrm{e}^{\frac{7 \pi}{2}}, \cdots
\]
还有距 $A$ 越来越近的无穷多个点
\[
A D^{-1}=\mathrm{e}^{-\frac{\pi}{2}}, A D^{-2}=\mathrm{e}^{-\frac{3 \pi}{2}}, A D^{-3}=\mathrm{e}^{-\frac{5 \pi}{2}}, \cdots
\]

【图,待补】
%%![](https://cdn.mathpix.com/cropped/2023_02_05_d0626289dd3c515543f1g-19.jpg?height=572&width=268&top_left_y=1490&top_left_x=1206)

图 107

此外,曲线左右两侧至轴的距离 $A B=A C=1$, 在点 $B, C$ 处,曲线与平行于轴的直线相切, 切点为无穷多个点 $E$, 点 $F$. 点 $E$ 至点 $B$ 的距离, 点 $F$ 至点 $C$ 的距离都成几何级数. 可见,曲线在逼近直线 $B C$ 时形成无穷多个弯曲, 最终与 $B C$ 重合. 也即这条曲线的渐近线不是 无穷直线, 而是有限直线 $B C$. 这是这条曲线与代数曲线截然不同的一个特点.

\section{$\S 526$}

形状各色各样列举不尽的螺线也属于这样的超越曲线, 它们或者只要求角度, 或者 要求角度的同时还要求对数. 如图 108 所示, 螺线都有一个作为中心的确定的点 $C$, 螺线 绕中心而成, 圈数无穷. 螺线的性质易于用距离与角度间的方程表示. 距离是螺线上任一 点 $M$ 至中心 $C$ 的距离 $C M$, 角度是 $C M$ 与给定了位置的直线 $C A$ 所成的 $\angle A C M$. 设 $\angle A C M=s$, 即设 $s$ 是单位圆上 $\angle A C M$ 所对的弧, 又设距离 $C M=z . s, z$ 间的每一个方程 表示的都是一条螺线. 对直线 $C M$ 的同一个位置, $\angle A C M$ 在 $s$ 之外还有 $2 \pi+s, 4 \pi+s$, $6 \pi+s, \cdots$ 及 $-2 \pi+s,-4 \pi+s, \cdots$ 无穷多个值. 用这无穷多个值中的每一个代替方程中 的 $s$, 我们都得到距离 $C M$ 的一个不同的值. 也即 $C M$ 的延长线与曲线交于无穷多个点. 当然这里要 $z$ 不是虚数. 我们先看最简单的情形, $z=a s$. 此时对直线 $C M$ 的同一个位置我 们得到 $z$ 值 $a(2 \pi+s), a(4 \pi+s), a(6 \pi+s), \cdots$ 和 $-a(2 \pi-s),-a(4 \pi-s),-a(6 \pi-$ $s), \cdots$ 换方程中的 $s$ 为 $\pi+s$ 时直线 $C M$ 的位置不变, 但 $z$ 的值变负. 因而应该把值 $-a(\pi+$ $s),-a(3 \pi+s),-a(5 \pi+s), \cdots$ 和 $a(\pi-s), a(3 \pi-s), a(5 \pi-s), \cdots$ 加到已经列出的 $z$ 值 上去. 这是最简单的螺线, 其形状如图 109 所示. 曲线与直线 $A C$ 相切于点 $C$, 且两个分支 从点 $C$ 出发绕 $C$ 无穷多圈趋向无穷, 每圈都在垂直于 $A C$ 的直线 $B C$ 上相交, 即直线 $B C B$ 是该曲线的直径. 人们称这样的曲线为阿基米德螺线. 方程 $z=a$ 表明, 一旦曲线被准确 画出,它与过点 $C$ 的任何一条直线的交点就都可求出.


【图,待补】
%%![](https://cdn.mathpix.com/cropped/2023_02_05_d0626289dd3c515543f1g-20.jpg?height=304&width=292&top_left_y=1547&top_left_x=321)

图 108


【图,待补】
%%![](https://cdn.mathpix.com/cropped/2023_02_05_d0626289dd3c515543f1g-20.jpg?height=575&width=656&top_left_y=1412&top_left_x=725)

图 109

\section{$\$ 527$}

跟方程 $z=a s(s, z$ 表示直角坐标时, 该方程给出的是直线) 给出阿基米德螺线一样,

%%13p241-260
$z, s$ 间的别的代数方程也给出无穷多条螺线, 只要这方程使 $s$ 对应的 $z$ 值都为实数. 例如, 方程 $z=\frac{a}{s}$ (它类似于以渐近线为轴的双曲线方程) 就给出约翰 - 伯努里称之为双曲螺线 的螺线. 该螺线绕中心 $C$ 无穷多圈后, 在无穷远处收玫于作为渐近线的直线 $A A$. 又例如, 方程 $z=\sqrt{s}$ ( $s$ 为负时距离为虚数), 对 $s$ 的每一个值我们都得到两个 $z$ 值, 一正一负, 曲线 绕点 $C$ 无穷多圈, 再例如, 方程 $z=a \sqrt{n^{2}-s^{2}}, s$ 不在 $-n$ 与 $n$ 之间时 $z$ 为虚数, $s$ 在 $-n$, $+n$ 之间时曲线是有限的, 如图 110 所示. 过中心 $C$ 引与轴 $A C B$ 所成角为 $n$ 的两条直线 $E F, E F$, 则相交于点 $C$ 的这两条直线都是曲线的切线, 由线的形状为双纽线 $A C B C A$. 用 这种方法可得到无穷多种别的形状的超越曲线, 篇幅不允许, 不再继续讨论.


【图,待补】
%%![](https://cdn.mathpix.com/cropped/2023_02_05_c02c95ceeae1a896f812g-01.jpg?height=371&width=488&top_left_y=772&top_left_x=606)

图 110

\section{$\S 528$}

如果不限于 $z, s$ 间的代数方程, 把超越方程也包 括进来, 那这讨论就更加地没有边了. 值得特别注意 的一条这种曲线, 其方程为 $s=n \log \frac{z}{a}$, 方程表明 $s$ 与 距离 $z$ 的对数成正比,因此称这条曲线为对数螺线. 对 数螺线之所以值得特别注意, 是因为它有很多重要的 性质. 主要的一条是, 过点 $C$ 的直线与曲线的交角都相 等. 为从方程推出这条性质, 参见图 111 , 令 $\angle A C M=s$, 距离 $C M=z$, 则 $s=n \log \frac{z}{a}, z=a \mathrm{e}^{\frac{s}{n}}$, 取更大的角$\angle A C N=s+v$, 则距离 $C N=a \mathrm{e}^{\frac{1}{i}} \mathrm{e}^{\frac{v}{n}}$, 以点 $C$ 为中心画长度等于 $z v$ 的弧 $M L$, 则距离


【图,待补】
%%![](https://cdn.mathpix.com/cropped/2023_02_05_c02c95ceeae1a896f812g-01.jpg?height=369&width=503&top_left_y=1400&top_left_x=1029)

图 111 
\[
L N=a \mathrm{e}^{\frac{s}{n}}\left(\mathrm{e}^{\frac{v}{n}}-1\right)=a \mathrm{e}^{\frac{1}{n}}\left(\frac{v}{n}+\frac{v^{2}}{2 n^{2}}+\frac{v^{3}}{6 n^{3}}+\cdots\right)
\]
由此得
\[
\frac{M L}{L N}=\frac{v}{\frac{v}{n}+\frac{v^{2}}{2 n^{2}}+\frac{v^{3}}{6 n^{3}}+\cdots}=\frac{n}{1+\frac{v}{2 n}+\frac{v^{2}}{6 n^{2}}+\cdots}
\]
两角之差 $\angle M C N=v$ 为零时, $\frac{M L}{L N}$ 成为半径 $C M$ 与曲线所成角的正切. 从而 $v=0$ 时, $\angle A C M$ 的正切等于 $n$, 即 $\angle A C M$ 的正切是常数. 如果 $n=1$, 则 $\angle A C M$ 为半直角, 此时的 对数螺线称为半直角对数螺线. 

