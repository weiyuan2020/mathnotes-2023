\chapter{第十八章 曲线的相似性和仿射性}

\section{$\S 435$}

每个曲线方程中, 直角坐标 $x, y$ 之处,还应包含一个或几个常量 ( $\S 437$ 称常量为参 数 中一译者), 比如 $a, b, c$ 等, 也称常量为线常量. 曲线方程中常量变量次数之和, 各项 相等. 如果一项为 $n$ 个量 (常量和变量) 的积, 其他各项都为 $n$ 个量相乘. 否则, 不同类的 量相比较, 是无法进行的. 因此, 在每个曲线方程中, 常量与变量一起, 次数的和各项相 等. 常量可以取 1 或别的数. 按照以上说法, 一个方程中, 如果没有常量, 则变量 $x, y$ 次数 的和, 各项相同, 也即必为齐次方程. 我们讲了, 齐次方程表示的不是曲线, 而是交于一点 的几条直线.

\section{$\S 436$}

我们先考虑变量 $x$ 和 $y$ 之外只含一个常量 $a$ 的方程. $x, y$ 和 $a$ 这三个量的次数之和, 各项相同. 常量 $a$ 不同的值给出不同的曲线. 这样的曲线有无穷多条, 它们不同的只是大 小, 形状完全相似. 我们称同一个方程所表示的这种曲线为彼此相似. 它们之间的不同, 不同于半径不同的圆之间的不同.

\section{$\S 437$}

为了清楚地认识这种相似, 我们考虑一个完全确定的方程
\[
y^{3}-2 x^{3}+a y^{2}-a^{2} x+2 a^{2} y=0
\]
在变量 $x$ 和 $y$ 之外, 它只含一个常量 $a$. 我们称这个 $a$ 为参数. 图 88 上, 直线 $A B$ 为轴, 坐标 $A P=x, P M=y$. 设参数 $a$ 的值为 $A C$, 又设 $A C=a$ 时方程给出的曲线为 $A M B$. 现在我们 给参数 $a$ 以另外一个值 $a c$, 此时方程给出的曲线为 $a m b$, 如图 89 所示. 我们称这里的曲线 $A M B$ 和 $a m b$ 彼此相似. 保持 $A C=a, A P=x, P M=y$, 如果 $a c=\frac{1}{n}, A C=\frac{a}{n}$, 则取 $a p=$ $\frac{1}{n}, A P=\frac{x}{n}, p m=\frac{1}{n}, P M=\frac{y}{n}$. 依次用 $\frac{a}{n}, \frac{x}{n}, \frac{y}{n}$ 代替 $a, x, y$, 则所得方程等于原方程各 项都除以 $n^{3}$, 即方程不变.


【图,待补】
%%![](https://cdn.mathpix.com/cropped/2023_02_05_a9d9b884b46ab350bed3g-13.jpg?height=421&width=256&top_left_y=290&top_left_x=473)

图 88


【图,待补】
%%![](https://cdn.mathpix.com/cropped/2023_02_05_a9d9b884b46ab350bed3g-13.jpg?height=407&width=252&top_left_y=304&top_left_x=975)

图 89

\section{$\S 438$}

相似曲线具有一条极能说明其相似性的性质: 如果横标 $A P$ 与 $a p$ 之比等于参数 $A C$ 与 $a c$ 之比, 则纵标 $P M$ 与 $p m$ 之比也等于参数 $A C$ 与 $a c$ 之比. 也即, 如果取
\[
A P: a p=A C: a c
\]
则
\[
P M: p m=A C: a c
\]
由此得
\[
A P: P M=a p: p m
\]
这告诉我们, 相似曲线之间,除了大小,其他性质都相同. 例如,如果取横标 $A P, a p$ 同调, 即其比等于参数 $A C, a c$ 之比, 则不只是纵标 $P M, p m$, 而是用类似方式画出的一切对应 线, 甚至弧 $A M, a m$ 之比, 也都等于参数 $A C, a c$ 之比. 且对应面积 $A P M, a p m$ 之比等于参 数平方 $A C^{2}, a c^{2}$ 之比. 如果任取两个同调点 $O, o$, 即 $A O: a o=A C: a c$, 并从这两点分别向 曲线画直线 $O M, o m$, 使 $\angle A O M$ 和 $\angle a O m$ 相等, 则
\[
O M: o m=A C: a c
\]
最后, 由于相似, 同调点 $M, m$ 处切线与轴所成角相等, 且密切半径之比等于参数 $A C 与$ $a c$ 之比.

\section{$\S 439$}

由此可见, 方程 $y^{2}=2 a x-x^{2}$ 表示的一切圆彼此相似. 同样,方程 $y^{2}=a x$ 表示的全体 曲线, 也即全体抛物线彼此相似. 由于在表示相似曲线的这些方程中, $x, y, a$ 次数的和, 各项相等,因而从方程解出的 $y$ 是量 $a$ 和 $x$ 的一次齐次函数,反之,如果 $P$ 是量 $a$ 和 $x$ 的一 次函数,则方程 $y=P$ 含有无数条彼此相似的曲线. 参数 $a$ 每取定一个值,就得到其中一 条. 同样地,从相似曲线的这类方程可以得到,横标 $x$ 是 $a$ 和 $y$ 的一次齐次函数,参数 $a$ 是 $x$ 和 $y$ 的一次齐次函数. 

\section{$\S 440$}

任给一条曲线 $A M B$, 都可另外画出无数条曲线 $a m b$ 与它相似. 任取一个数, 例如 $\frac{1}{n}$, 作为同调部分间的比值. 如果所给曲线 $A M B$ 是对轴 $A B$ 和直角坐标 $A P, P M$ 画出 的, 那么在相似轴 $a b$ 上我们取横标 $a p$, 使 $A P: a p=1: n$, 再自点 $p$ 作垂线 $p m$, 使 $P M:$ $p m=1: n$. 这样得到的 $m$ 就在相似线 $a m b$ 上. 点 $M$ 和 $m$ 是同调的. 也可以从一个选定的 点 $O$ 出发来画相似线. 取一个点 $o$ 作点 $O$ 的对应点. 作 $\angle a o m=\angle A O M$, 截取线段 om, 使 $O M: o m=1: n$

这点 $m$ 就在相似线 $a m b$ 上. 用这种方法, 对取定的任一个比 $1: n$ 都可画出相似线. 人们 制作了工具,用它可以画出与给定曲线以任何比相似的曲线.

\section{$\S 441$}

不难从给定曲线的方程求出与它相似的曲线的方程. 设给定曲线 $A M$ 的方程是坐标 $A P=x$ 和 $P M=y$ 间的. 我们来求与它相似的曲线 $a m$ 的方程. 设相似曲线的横标 $a p=X$, 纵标 $p m=Y$, 则从 $x: X=1: n, y: Y=1: n$, 得
\[
x=\frac{X}{n}, \quad y=\frac{Y}{n}
\]
将这两个值代入 $x, y$ 间方程, 得到的就是相似曲线的 $X, Y$ 间方程. 如果在新方程中只考 虑坐标 $X, Y$ 和字母 $n$, 则各项次数都为零. 如果用 $n$ 的适当的幂乘方程. 以便去掉分母, 则得到的方程中, $X, Y, n$ 的次数的和各项相等. 前面我们讲了, 相似曲线的方程中, 两个 坐标和由它的变化而得到不同相似曲线的那个常量, 这三个量的次数的和, 各项相等. 这 是判断一个方程所表示的是否为相似曲线的一个准则.

\section{$\S 442$}

相似曲线, 同调的横标和纵标按相同的比伸长或缩短. 如果横标按一个比伸缩, 纵标 按另一个比, 这两个曲线就不是相似的. 这样的两个曲线虽不相似, 但有着某种关系. 我 们称它们为仿射. 仿射包含相似, 相似是仿射的特殊情形. 纵标比、横标比相等的仿射是 相似. 任何曲线 $A M B$, 我们都可以求出它无穷多个仿射线 $a m b$ (图 88,89). 求法是, 先取 横标 $a p$, 使 $A P: a p=1: m$, 再引纵标 $p m$, 使 $P M: p m=1: n$. 比不同, 所得仿射线不同. 同时改变比 $1: m$ 和 $1: n$, 或只改变其中一个, 我们就得到无穷多个仿射于原曲线 $A M B$ 的曲线. 

\section{$\S 443$}

假定所给曲线 $A M B$ 由直角坐标 $A P=x, P M=y$ 间的方程表示. 根据上节所讲,其仿 射线 $a m b$ 由坐标 $a p=X, p m=Y$ 间方程表示. 由两坐标间的关系
\[
x: X=1: m, \quad y: Y=1: n
\]
得
\[
x=\frac{X}{m}, \quad y=\frac{Y}{n}
\]
将这两个值代入 $x, y$ 间方程, 得到的就是仿射线的 $X, Y$ 之间的通用方程. 为进一步考察 所得方程的性质, 我们假定所给曲线 $A M B$ 的方程中, $y$ 是 $x$ 的函数, 记为 $P$, 即 $y=P$. 这

样, 如果换 $P$ 中 $x$ 为 $\frac{X}{m}$, 则 $P$ 成为 $X$ 和 $m$ 的零次函数. 即仿射线的通用方程中 $\frac{Y}{n}$ 是 $X$ 和 $m$ 的零次函数, 也即 $Y$ 和 $n$ 的零次函数等于 $X$ 和 $m$ 的零次函数.

\section{$\S 444$}

相似曲线与仿射曲线的主要差别是: 对一根轴或一个固定的点相似的曲线, 对任何 另外的同调轴, 或任何另外的同调点, 依然相似. 而仿射曲线, 对两根轴仿射, 对另外的轴 或另外的同调点, 可以不仿射. 还应该指出, 跟彼此相似的曲线一样, 彼此仿射的曲线也 同阶同类. 下面我们用几种大家熟悉的曲线作例子, 对所说进行解释.

\section{$\S 445$}

设所给曲线为圆,方程为 $y^{2}=2 c x-x^{2}$. 为求相似于所给圆的曲线, 将
\[
x=\frac{X}{n}, \quad y=\frac{Y}{n}
\]
代入方程, 得全体相似于圆的曲线的 $X, Y$ 间通用方程
\[
\frac{Y^{2}}{n^{2}}=\frac{2 c X}{n}-\frac{X^{2}}{n^{2}}
\]
也即
\[
Y^{2}=2 n c X-X^{2}
\]
由此显见, 相似于圆的曲线都是圆,直径 $2 n c$ 取值不同, 圆不同. 为求仿射于圆的曲线, 将
\[
x=\frac{X}{m}, \quad y=\frac{Y}{n}
\]
代入圆的方程, 得
\[
\frac{Y^{2}}{n^{2}}=\frac{2 c X}{m}-\frac{X^{2}}{m^{2}}
\]
或 
\[
m^{2} Y^{2}=2 m n^{2} c X-n^{2} X^{2}
\]
这是椭圆的关于一根主轴的通用方程. 由此可见, 凡椭圆都与圆仿射, 进而椭圆彼此仿 射. 用同样的方法可以得到, 凡双曲线彼此仿射. 椭圆有两根主轴, 主轴之比相等的椭圆 相似. 同样,主轴之比相等的双曲线也相似.

\section{$\S 446$}

至于抛物线, 方程为 $y^{2}=c x$, 显然相似于它的曲线是抛物线, 抛物线彼此相似. 我们 来求仿射于抛物线的曲线, 令
\[
y=\frac{Y}{n}, \quad x=\frac{X}{m}
\]
得方程 $Y^{2}=\frac{n^{2} c}{m} X$. 这也是抛物线方程. 可见, 仿射于抛物线的曲线同时也相似于抛物线. 也即对抛物线来说, 相似与仿射范围相同. 方程只有两项的曲线, 如 $y^{3}=c^{2} x, y^{3}=c x^{2}$, $y^{2} x=c^{3}$ 等, 情况都是这样. 我们得到, 双曲线之间、抛物线之间、仿射与相似都是同时的. 别的曲线则不然. 我们看到了, 圆和椭圆它们之间可以仿射而不相似.

\section{$\S 447$}

从 $x, y$ 的方程, 不管它含有多少个常量 $a, b, c, \cdots$, 只要使每一个常量都取确定的值, 我们就得到一条确定的曲线. 如果只让常量中的一个, 比如 $a$ 变动,那么对每一个不同的 $a$ 值得到一条不同的曲线, 总数是无穷多条, 且它们都相似. 这是在别的常量都为确定值 都不变动的前提之下, 否则, 这里的曲线就不相似. 如果 $a$ 之外, 允许 $b$ 也变动, 那么对每 一个固定的 $a$, 由 $b$ 的变动又得到无穷多个曲线. $a, b$ 都变动时, 我们得到的不同曲线的数 目是无穷多个无穷多. 如果 $a, b$ 之外, 常量 $c$ 再变动, 那么对每一对固定的 $a$ 和 $b$, 我们就 又得到无穷多个曲线. 这样允许变动的常量个数越多,我们得到的不同曲线的数目, 其无 穷的幂次就越高.

\section{$\S 448$}

对只让一个常量变动时,一个方程所产生的那无穷多个曲线, 我们再作些考察. 当轴 和原点固定, 这时方程给出的不只是无穷个曲线, 还给出这每条曲线的位置. 这无穷多曲 线充满某个区域, 也即这区域中每一点都有这无穷多曲线中的线通过. 这无穷多曲线相 似与否, 可据前面所讲作出判断, 可以有这样的情况, 这无穷多曲线不只相似且相等, 不 同的只是位置. 例如, $a$ 变动时方程
\[
y=a+\sqrt{2 c x-x^{2}}
\]
给出的无穷多个半径为 $c$, 圆心在垂直于轴的一条直线上的圆就是.

\section{$\S 449$}

反之, 如果依照某种规律, 把一个曲线画在一个平面 的无穷多个不同位置上, 我们也可以求出一个方程, 使得 让方程中的一个常量变化, 就得到画出的这无穷多个曲 线, 假定如图 90 所示, 画在无穷多位置上的曲线是半径 为 $c$ 的圆. 画这无穷多个圆的规律是, 让顶点 $A, a$ 都在称 为准线的一条给定曲线 $\mathrm{Aad}$ 上, 且直径 $a b$ 都平行于轴 $A B$. 为了求出含这无穷多个圆的方程, 取准线上任一点 $a$, 从它向主轴画垂线 $a K$. 记 $A K=a$. 由于准线已给, 我 们有 $K a$ 过 $a$. 记 $K a=A$, 则 $A$ 是 $a$ 的一个确定的函数. 再 从 $a$ 引圆的直径 $a b$, 使平行于主轴, 这新圆的顶点 $a$ 在准 线上. 从新圆上任取一点 $m$,画纵标 $m P=y$, 对应横标为 $A P=x$. 这样我们得到


【图,待补】
%%![](https://cdn.mathpix.com/cropped/2023_02_05_a9d9b884b46ab350bed3g-17.jpg?height=541&width=442&top_left_y=440&top_left_x=1050)

图 90
\[
a p=x-a, \quad p m=y-A
\]
如果令 $a p=t, p m=u$, 那么由圆的性质我们有 $u^{2}=2 c t-t^{2}$. 将 $t=x-a, u=y-A$ 代入, 得
\[
(y-A)^{2}=2 c(x-a)-(x-a)^{2}
\]
这是我们所说意义下沿准线 $\mathrm{Aad}$ 的圆的通用方程. 让 $A$ 所依赖的 $a$ 变化, 从该方程即可 得所有的圆.

\section{$\S 450$}

类似地, 如果把沿准线 $A a d$ 画的圆, 换为另一种曲线 $a m b$, 也使顶点或横标原点在准 线上,轴保持平行,那就也可以求出包含画出的无穷多个这另一种曲线的方程. 设这另一 种曲线的方程是坐标 $a p=t, p m=u$ 间的,并取平行于 $a b$ 的直线 $A B$ 作所有曲线的主轴, 也作准线 $A a d$ 的主轴. 照用过的方法. 令 $A K=a, K a=A$, 则 $A$ 是 $a$ 的函数. 记横标 $A P=$ $x$, 纵标 $P m=y$, 得 $t=x-a, u=y-A$. 将这两个值代入给定的方程, 就得到包含所有曲线 $a m b$ 的通用方程. 对 $a$ 的每一个确定的值,我们得到通用方程所表示的无穷多个曲线 $a m b$ 中的一个. 例如, 如果曲线 $a m b$ 是方程为 $u^{2}=c t$ 的抛物线, 则顶点在准线 $A a d$ 上,轴 平行于直线 $A B$ 的这无穷多个相等的抛物线的方程为
\[
(y-A)^{2}=c(x-a)
\]
\section{$\S 451$}

前面我们让曲线顶点沿着作为准线的曲线移动时, 保持曲线的轴相平行. 如果顶点 依旧在准线上,但轴不是相平行, 而是依另外的规律移动,那么含无穷多个这种曲线的方程, 要更为一般. 为说得更清楚一些, 先假定曲线的顶点 $A$ 沿圆弧 $A a$ 移动时轴 $a b$ 恒指向 圆心 $O$, 如图 91 所示. 这里是曲线 $A M B$ 跟它的轴 $B A O$ 一起绕点 $O$ 所作的转动给出无穷 多个曲线 $A M B$. 无穷多个这种曲线, 也可用含有一个常数的方程表示.


【图,待补】
%%![](https://cdn.mathpix.com/cropped/2023_02_05_a9d9b884b46ab350bed3g-18.jpg?height=430&width=503&top_left_y=448&top_left_x=567)

图 91

\section{$\S 452$}

设不变的半径 $A O=a O=c$, 可变动的角即 $\angle A O a=\alpha$. 从我们的一条曲线 $a m b$ 的任 一点 $m$, 向主轴 $O A B$ 引垂线 $m P$, 则 $O P=x, P m=y$. 再从这 $m$ 向 $a m b$ 的轴 $a b$ 引垂线 $m p$, 记 $a p=t, p m=u$. 假定 $a m b$ 的 $t, u$ 间方程已给. 自点 $P$ 引平行于 $O b$ 的直线 $P s$, 交纵 标 $m p$ 的延长线于 $s$, 则
\[
p s=x \sin \alpha, \quad O p-P s=x \cos \alpha
\]
又由
\[
\angle P m s=\angle A O a=\alpha
\]
得
\[
P s=y \sin \alpha, \quad m s=y \cos \alpha
\]
由此得
\[
O p=c+t=x \cos \alpha+y \sin \alpha, \quad m p=u=y \cos \alpha-x \sin \alpha
\]
这样,将
\[
t=x \cos \alpha+y \sin \alpha-c, \quad u=y \cos \alpha-x \sin \alpha
\]
代入所给 $t, u$ 间方程, 就得 $x, y$ 间的通用方程, 变动角 $\alpha$ 就得到所有的曲线 $a m b$.

\section{$\S 453$}

设曲线 $A M B$ 的顶点沿任何一条准线 $A a L$ 移动. 此时轴 $a b$ 的移动使 $\angle A O a$ 以某种 方式依赖于点 $a$. 参见图 92. 例如顶点为 $a$ 时, 记 $A K=a, K a=A$, 记 $\angle A O a=\alpha$. 由于准线 已给, 所以 $A$ 是 $a$ 的某个已知函数, $\alpha$ 的正弦和余弦也是 $a$ 的函数. 在这里的记号之下我 们有 
\[
K O=\frac{A}{\tan \alpha}, \quad O a=\frac{A}{\sin \alpha}
\]
从 $a m b$ 的任一点 $m$ 先向主轴 $A O$ 引垂线 $m P$, 再向 $a m b$ 的轴引垂线 $m p$,再向 $a m b$ 的轴引 垂线 $m p$, 令 $A P=x, P m=y ; a p=t, p m=u$. 假定原曲线坐标 $t, u$ 间方程已给. 从它我们可 以求出变动曲线的 $x, y$ 间通用方程.


【图,待补】
%%![](https://cdn.mathpix.com/cropped/2023_02_05_a9d9b884b46ab350bed3g-19.jpg?height=436&width=373&top_left_y=529&top_left_x=644)

图 92

\section{$\S 454$}

为此,自 $P$ 向 $m p$ 的延长线引垂线 $P s$, 它平行于曲线的轴 $a b O$. 由 $\angle P m s=\angle A O a=$ $\alpha$, 知
\[
P s=y \sin \alpha, \quad m s=y \cos \alpha
\]
又由
\[
O P=a+\frac{A}{\tan \alpha}-x
\]
得
\[
p s=a \sin \alpha+A \cos \alpha-x \sin \alpha
\]
和
\[
O p-P s=a \cos \alpha+\frac{A \cos \alpha}{\tan \alpha}-x \cos \alpha
\]
从而
\[
O p=a \cos \alpha+\frac{A \cos \alpha}{\tan \alpha}-x \cos \alpha+y \sin \alpha=\frac{A}{\sin \alpha}-t
\]
进而
\[
\begin{gathered}
t=A \sin \alpha-a \cos \alpha+x \cos \alpha-y \sin \alpha \\
u=-a \sin \alpha-A \cos \alpha+x \sin \alpha+y \cos \alpha
\end{gathered}
\]
或
\[
\begin{aligned}
& t=(x-a) \cos \alpha-(y-A) \sin \alpha \\
& u=(x-a) \sin \alpha+(y-A) \cos \alpha
\end{aligned}
\]

将这两个表达式代入 $t, u$ 间方程, 就得到所求的 $x, y$ 间方程. 不管依什么规律将曲线 $a m b$ 在平面上画出无穷多次, 用这里的方法都可以求出这无穷多个曲线的通用方程.

\section{$\S 455$}

这样我们就做到了用一个方程表示只是位置不同的无穷多个相同曲线,这里原曲线 的 $t, u$ 间方程不含可变化的常数 $a$. 如果 $t, u$ 间方程含有一个或几个依赖于 $a$ 的常数,那 么所得通用方程所包含的无穷多条曲线, 就不一定全都相似. 如果 $t, u$ 间方程中的 $u$ 是 $t$ 和 $f$ 的一次齐次函数, $f$ 依赖于 $a$, 那么这无穷多个曲线全都相似,否则, 将不全相似.

\section{$\S 456$}

作为具体说明, 我们举个例子. 如图 93 所示, $A B, a B, a m B, \cdots$ 是过点 $B$, 圆心在 $A E$ 上的无穷多 个圆. 这些圆类似于地图上的经线. 从点 $B$ 向直线 $A C$ 引垂线, 记 $B C=c$, 这 $c$ 是不变的. 现在我们对这 无穷多个圆中的一个 $a m B$ 进行考察. 引纵标 $m P$, 则 $C P=x, P m=y$. 对选定的一个圆,这半径是常 量, 但对所有的圆来说, 这半径就是一个变量. 记 $a E=B E=a$, 则
\[
C E=\sqrt{a^{2}-c^{2}}, \quad P E=x+\sqrt{a^{2}-c^{2}}
\]

【图,待补】
%%![](https://cdn.mathpix.com/cropped/2023_02_05_a9d9b884b46ab350bed3g-20.jpg?height=385&width=539&top_left_y=918&top_left_x=956)

图 93

由 $P E^{2}+P m^{2}=a^{2}$, 得
\[
y^{2}+x^{2}+2 x \sqrt{a^{2}-c^{2}}+a^{2}-c^{2}=a^{2}
\]
也即
\[
y^{2}=c^{2}-2 x \sqrt{a^{2}-c^{2}}-x^{2}
\]
如果把方程中应变化的常量取为线段 $C E=a$, 则得到更为简单的方程
\[
y^{2}=c^{2}-2 a x-x^{2}
\]
使 $a$ 变化, 就得到过点 $B$, 圆心在直线 $A E$ 上的所有的圆. 类似地, 可以用一个方程表示遵 守某种规律的所有曲线, 只要使本来不变化的常量适当地进行变化.

%%11p201-220

