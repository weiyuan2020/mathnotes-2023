\chapter{第二章坐标变换}

\section{$\S 23$}

给定一个关于坐标 $x, y$ 的方程, 那么取一条直线 RS 作轴, 并在 $R S$ 上取一个点 $A$ 作 原点, 我们就可以像图 2 那样把这个方程描述的曲线画出来. 反之, 有了画出的曲线, 我 们就可以列出描述这曲线性质的坐标方程. 但是列这个方程时有两个东西是随意的: $R S$ 本身的位置和原点 $A$ 在 $R S$ 上的位置. 位置有无穷多种取法, 因而对应于这同一条曲线的 方程也就有无穷多个. 可见方程不同,曲线不一定不同, 但曲线不同,方程一定不同.

\section{$\S 24$}

改变轴的位置, 改变原点在轴上的位置, 都可以得到描述同一条曲线性质的无穷多 个方程. 这些方程是相联系的, 由其中任何一个都可以推出其他. 事实上, 当坐标方程已 知, 它所描述的曲线已经画出时, 任取一条直线作轴, 在取定的轴上任取一点作原点, 我 们都可以推出曲线对所取轴的坐标方程. 本\章我们讲从一条曲线的一个方程求这条曲线 关于另一个轴、另一个原点的另一个方程的方法. 用这种方法可以求出一条曲线的所有 方程. 这种方法还可用来判断不同方程所给曲线是否相同.

\section{$\S 25$}

给定一个 $x, y$ 的方程, 如图 7, 取直线 $R S$ 作轴, 取点 $A$ 作原点, 用 $x$ 表示横标 $A P$, 用 $y$ 表示纵标 $P M$, 那么根据方程我们就可以画出它所描述的曲线 $C B M$. 先保持轴 $R S$ 不 变, 取 $R S$ 上异于 $A$ 的点 $D$ 作原点, 记点 $M$ 在新原点下的横标 $D P$ 为 $t$, 纵标 $P M$ 依旧为 $y$ 不变, 现在我们要进行的是, 求出曲线 $C B M$ 关于 $t, y$ 的方程. 记线段 $A D=f$, 这里的 $D$ 在 $A$ 点左侧, $A D$ 在原来的负横标区, $D P=t=f+x$, 从而 $x=t-f$, 以 $t-f$ 替换关于 $x, y$ 的 方程中的 $x$, 我们就得到描述同一条曲线 $C B M$ 的关于 $t, y$ 的方程. 由于 $A D=f$ 可以任 取,所以我们已经得到了无穷多个方程, 它们都描述同一条曲线. 


【图,待补】
%%![](https://cdn.mathpix.com/cropped/2023_02_05_37db6883222c7ef36acdg-09.jpg?height=395&width=581&top_left_y=293&top_left_x=533)

图 7

\section{$\S 26$}

设曲线交轴 $R S$ 于某一点, 比如 $C$, 如果取这个点 $C$ 作原点, 则横标 $C P=0$ 时, 纵标也 为 0 , 当然这里假定点 $C$ 只对应一个纵标, 曲线与轴的交点, 包括点 $C$, 都可以从曲线的方 程求得, 方法是: 置曲线方程中的 $y$ 为 0 , 得 $x$ 的方程, 这 $x$ 的方程的解就是曲线与轴的交 点, 曲线与轴相交处 $y$ 为 0 , 因而令 $y=0$, 解方程就得到曲线与轴的所有交点.

\section{$\S 27$}

加大或减小横标 $x$, 也即以 $x-f$ 替换 $x$, 则原点移动, 新的原点 $D$ 在 $A$ 点左侧时 $f$ 为 正, 在点 $A$ 右侧时 $f$ 为负.

记 $A P=x, P M=y$, 则方程给出的曲线为 $L B M$ (图 8). 原点左移之后, 我们再将轴平 行下移, 取平行于原轴 $R S$ 的 $r s$ 为新轴, 取新轴 $r s$ 上的点 $D$ 为新原点, 新轴在原纵标的负 值区域中, 与原轴的距离 $A F=g$, 取 $D F=A G=f$, 记新轴时曲线上点 $M$ 的横标 $D Q=t$, 纵 标 $Q M=u$,则
\[
t=D F+F Q=f+x, \quad u=P M+P Q=y+g
\]
从而
\[
x=t-f, \quad y=u-g
\]
将所给方程中的 $x$ 和 $y$ 换为 $t-f$ 和 $u-g$, 得到关于 $t, u$ 的方程, 它描述的也是所给曲线.

\section{$\S 28$}

$f, g$ 都任取, 取法都无穷, 因而不同方程的个数比只移动原点又多一层无穷, 它们描 述的全都是同一条曲线. 我们看到, 两个方程, 一个是关于 $x, y$ 的, 一个是关于 $t, u$ 的, 它 们虽然不同, 但一个是另一个坐标增大、减小的结果. 这样的两个方程描述的曲线是同一 条,用这样的方法可以得到描述同一条曲线的无穷多个不同方程. 

【图,待补】
%%![](https://cdn.mathpix.com/cropped/2023_02_05_37db6883222c7ef36acdg-10.jpg?height=662&width=944&top_left_y=34&top_left_x=356)

图 8

\section{$\S 29$}

使新轴 $r s$ 与旧轴 $R S$ 交于原点 $A$, 且相垂直,即新旧轴相垂直, 且原点重合 (图 9). 轴 为 $R S$ 时,曲线 $L M$ 的方程是关于横标 $A P=x$ 和纵标 $P M=y$ 的. 从曲线上的点 $M$ 向新轴 $r s$ 引垂线 $M Q$, 记新横标 $A Q=t$, 新纵标 $Q M=u$, 由 $A P M Q$ 为矩形, 得 $t=y, u=x$. 这样, 将原方程中的 $x$ 换成 $u, y$ 换成 $t$, 就得到关于 $u$ 和 $t$ 的方程, 也即原来的横标 $x$ 变成了现在 的纵标 $Q M=u$, 原来的纵标 $y$ 变成了现在的横标 $A Q=t$. 换成新轴,方程不变, 只是横标 纵标互换, 而纵标横标统称为坐标, 可以不加区别, 不指明谁纵谁横, 给定一个 $x, y$ 的方 程,取 $x$ 还是取 $y$ 为横标,曲线是一样的.


【图,待补】
%%![](https://cdn.mathpix.com/cropped/2023_02_05_37db6883222c7ef36acdg-10.jpg?height=395&width=450&top_left_y=1315&top_left_x=625)

图 9

\section{$\S 30$}

上节我们假定新轴 $r s$ 的 $A s$ 部分为正横标,轴 $r s$ 的右侧为正纵标区域,这正负的选择 是任意的,如果取轴的 $A r$ 部分为正横标,则 $A Q=-t$, 应该用 $-t$ 代替 $x, y$ 方程中的 $y$. 继 而, 如果取 $r s$ 轴的右侧为负纵标区,则应该用 $-u$ 代替 $x$. 可见,坐标方程中的一个或两个 坐标都变成负的,曲线的性质不变. 请注意,对方程的所有变换都是这样. 

\section{$\S 31$}

设新轴 $r s$ 与旧轴 $R S$ 相交成某个角, 即 $\angle S A s$, 交于原点 $A$, 新轴也取 $A$ 为原点, 即新 旧原点重合 (图 10), 又设对旧轴 $R S$, 曲线 $L M$ 关于横标 $A P=x$ 和纵标 $P M=y$ 的方程已 给, 现在我们要从这根曲线关于旧轴 $R S$ 的方程, 求出它关于新轴 $r s$ 的方程. 从曲线上的 点 $M$ 向新轴引垂线 $M Q$, 我们寻求的是关于横标 $A Q=t$ 和纵标 $M Q=u$ 的方程. 设 $\angle S A s=q, \sin q=m, \cos q=n$, 则 $m^{2}+n^{2}=1$. 从点 $P$ 向新坐标线引垂线 $P p$ 和 $P q$, 由 $A P=x$ 得
\[
P p=x \sin q, \quad A p=x \cos q
\]
由 $\angle P M Q=\angle P A Q=q$, 以及 $P M=y$, 得
\[
P q=Q p=y \sin q, \quad M q=y \cos q
\]
从而
\[
\begin{gathered}
A Q=t=A p-Q p=x \cos q-y \sin q \\
Q M=u=P p+M q=x \sin q+y \cos q
\end{gathered}
\]

【图,待补】
%%![](https://cdn.mathpix.com/cropped/2023_02_05_37db6883222c7ef36acdg-11.jpg?height=407&width=625&top_left_y=1084&top_left_x=525)

图 10

\section{$\S 32$}

由 $\sin q=m, \cos q=n$, 得
\[
t=n x-m y, \quad u=m x+n y
\]
从而
\[
\begin{aligned}
& n t+m u=n^{2} x+m^{2} x=x \\
& n u-m t=n^{2} y+m^{2} y=y
\end{aligned}
\]
这样一来, 将 $x, y$ 方程中的 $x$ 换为 $n t+m u, y$ 换为 $n u-m t$, 就得到我们所要的 $t, u$ 的 方程. 这里轴 $r s$ 的 $A s$ 部分为正的横标部分, 纵标线 $Q M$ 所在部分为正的纵标部分. 这里 还假定了 $\angle S A s$ 在负的纵标部分, 如果 $A s$ 在 $A S$ 的上方, 那么计算时应取 $\angle S A s=q$ 为负. 

\section{$\S 33$}

考虑新轴 $r s$ 和新轴上原点 $D$ 的位置都任意的情形 (图 11). 设曲线 $L M$ 关于旧轴 $R S$ 的以横标 $A P=x$ 和纵标 $P M=y$ 为变量的方程已知, 我们要进行的是, 从 $L M$ 的已知方程 求出 $L M$ 的关于新轴 $r s$ 的以 $t$ 和 $u$ 为变量的方程. 从曲线上任意一点 $M$ 引新轴的垂线 $M Q$, 分别记横标 $D Q$ 和纵标 $Q M$ 为 $t$ 和 $u$. 为了求出以 $t, u$ 为变量的方程, 我们从新的横 标原点 $D$ 向旧轴引垂线 $D G$, 记 $A G=f, D G=g$. 再过点 $D$ 引旧轴 $R S$ 的平行线 $D O, O$ 为 $D O$ 与旧纵标线 $P M$ 延长线的交点, 这样我们有
\[
M O=y+g, \quad D O=G P=x+f
\]
最后, 记 $\angle O D Q=q$, 记 $\sin q=m, \cos q=n$, 则 $m^{2}+n^{2}=1$.


【图,待补】
%%![](https://cdn.mathpix.com/cropped/2023_02_05_37db6883222c7ef36acdg-12.jpg?height=412&width=467&top_left_y=888&top_left_x=604)

图 11

\section{$\S 34$}

从点 $O$ 分别向新轴和纵标线 $M Q$ 引垂线 $O p$ 和 $O q$. 由 $\angle O M Q=\angle O D Q$ 和 $D O=$ $x+f, M O=y+g$, 得
\[
\begin{gathered}
O p=Q q=(x+f) \sin q=m x+m f \\
D p=(x+f) \cos q=n x+n f
\end{gathered}
\]
和
\[
\begin{gathered}
O q=Q p=(y+g) \sin q=m y+m g \\
M q=(y+g) \cos q=n y+n g
\end{gathered}
\]
从而
\[
\begin{aligned}
D Q & =t=n x+n f-m y-m g \\
Q M & =u=m x+m f+n y+n g
\end{aligned}
\]
进而利用 $m^{2}+n^{2}=1$ 得
\[
n t+m u=x+f, \quad n u-m t=y+g
\]
由此得
\[
x=m u+n t-f, \quad y=n u-m t-g
\]
将 $x$ 和 $y$ 代入以 $x, y$ 为变量的方程, 我们就得到曲线 $L M$ 的以 $t, u$ 为变量的方程.

\section{$\S 35$}

曲线 $L M$ 所在平面的任何一根轴 $r s$ 都是图 11 的特例, 因而 $L M$ 的坐标间方程无例外 的全都包含在我们所求出的 $t, u$ 间的方程之中. $f, g$ 和决定 $m, n$ 的角的取法都无穷, 即我 们所求出的 $t, u$ 间所含方程个数无穷, 它们都描述曲线 $L M$. 因此称我们求出的 $t, u$ 间方 程为曲线 $L M$ 的通用方程, 因为它包含曲线 $L M$ 的一切方程.

\section{$\S 36$}

从前面所讲我们看到, 不同方程描述的曲线可以是同一条. 两个方程描述的是否为 同一条曲线, 怎样判断呢? 这里我们讲一种判断方法. 设两个方程, 第一个是 $x, y$ 间的, 第二个是 $t, u$ 间的, 先令第一个方程中的
\[
x=m u+n t-f, \quad y=n u-m t-g
\]
其中 $m, n$ 满足关系式 $m^{2}+n^{2}=1$. 再看能否决定 $f, g$ 和 $m, n$, 使得代换后的方程等于第二 个方程. 能, 则两个方程描述的是同一条曲线, 不能, 则不是.

例 用我们的方法判断方程
\[
\begin{gathered}
y^{2}-a x=0 \\
16 u^{2}-24 t u+9 t^{2}-55 a u+10 a t=0
\end{gathered}
\]
描述的是否为同一条曲线.

这是两个看不出有什么关系的方程, 将
\[
x=m u+n t-f, \quad y=n u-m t-g
\]
代入第一个方程,得
\[
n^{2} u^{2}-2 m n t u+m^{2} t^{2}-2 n g u+2 m g t+g^{2}-m a u-n a t+a f=0
\]
为了判断得到的这个方程能否化成第二个方程, 把它们分别乘以 $n^{2}$ 和 16 , 使得它们的第 一项相同,乘得的结果为
\[
16 n^{2} u^{2}-24 n^{2} t u+9 n^{2} t^{2}-55 n^{2} a u+10 n^{2} a t=0
\]
$16 n^{2} u^{2}-32 m n t u+16 m^{2} t^{2}-32 n g u+32 m g t+16 g^{2}-16 m a u-16 n a t+16 a f=0$

我们看看可以确定 $f, g, m, n$, 使这两个方程的哪些项相等. 先分别使 $t u$ 和 $t^{2}$ 的系数 相等, 得 $24 n^{2}=32 m n, 9 n^{2}=16 m^{2}$, 都给出 $3 n=4 m$, 将 $m^{2}=1-n^{2}$ 代入 $9 n^{2}=16 m^{2}$, 得 $25 n^{2}=16$, 从而
\[
n=\frac{4}{5}, \quad m=\frac{3}{5}
\]
已经有三项相等了, 再分别使 $u$ 和 $t$ 的系数相等, 得
\[
55 n^{2} a=32 n g+16 m a, \quad 10 n^{2} a=32 m g-16 n a
\]
解出 $g$, 看是否相等, 从前一等式得
\[
g=\frac{55 n a}{32}-\frac{m a}{2 n}=\frac{11 a}{8}-\frac{3 a}{8}=a
\]
从后一等式得
\[
g=\frac{5 n^{2} a}{16 m}+\frac{n a}{2 m}=\frac{a}{3}+\frac{2 a}{3}=a
\]
两个 $g$ 相等. 这样, 我们有五项相等了, 只剩下能否决定 $f$, 使 $g^{2}+a f=0$, 取 $f=-a$ 即 可. 我们得到结论,所给的两个方程描述的是同一条曲线.

\section{$\S 37$}

虽然很不相同的两个方程描述的可以是同一条曲线,但常常从方程的不同就完全可 以断定它们描述的不是同一条曲线. 两个方程,阶 (各项中变量次数和的最大数) 不同, 描述的曲线就肯定不同.

代换
\[
x=m u+n t-f, \quad y=n u-m t-g
\]
不改变方程的阶, 因此 $t, u$ 间的另一种阶的方程描述的肯定是另一条不同的曲线.

\section{$\S 38$}

即两个方程,一个是 $x, y$ 间的,一个是 $t, u$ 间的, 如果阶不同, 我们立即可以总结结 论, 它们描述的曲线不同. 这样需要我们加以区别的, 就只剩下同阶方程了. 即我们只需 对同阶方程进行前面的考察. 但是当阶数增大时考察是很麻烦的. 后面我们将讲一种规 则,用它立即可以说出两个方程描述的曲线是否相同.

\section{$\S 39$}

曲线通用方程的求法也可用于直线, 设 $L M$ 不是曲线, 而是平行于轴 $R S$ 的直线 (图 12), 这时不管横标怎么取, 对应的纵标都为常数, 即恒有 $y=a$. 这 $y=a$ 就是平行于 轴的直线的方程. 现在我们对任意轴 $r s$ 来求该直线的通用方程. 记 $D G$ 为 $g$, 记 $\angle O D s$ 的 正弦为 $m$, 余弦为 $n$, 记横标 $D Q$ 为 $t$, 纵标 $M Q$ 为 $u$, 则由
\[
y=n u-m t-g
\]
得
\[
n u-m t-g-a=0
\]
这就是直线的通用方程, 把它乘以 $k$, 令 $n k=\alpha, m k=-\beta,(g+a) k=-b$, 得直线的方程
\[
\alpha u+\beta t+b=0
\]
这是 $t, u$ 间的一阶通用方程. 可见两个坐标间的任何一个一阶方程描述的都是直线, 不 是曲线. 


【图,待补】
%%![](https://cdn.mathpix.com/cropped/2023_02_05_37db6883222c7ef36acdg-15.jpg?height=302&width=579&top_left_y=292&top_left_x=536)

图 12

\section{$\S 40$}

我们得到结论: 坐标 $x, y$ 间的状如
\[
\alpha x+\beta y-a=0
\]
的方程,描述的都是直线. 这直线关于轴 $R S$ 的位置, 可用下面的方法确定 (图 13): 先令 $y=0$, 求得直线与轴的交点, 为 $A C=\frac{a}{\alpha}$. 再令 $x=0$, 求得直线在原点处的纵标值为 $y=$ $\frac{a}{\beta}$. 这样, 我们有了直线上的两个点 $B$ 和 $C$, 直线的位置就确定了. 我们来验证, 过点 $B$ 和 $C$ 的直线 $L M$ 确实满足所给方程. 设 $A P=x$ 是任何一个横标,其对应的纵标 $M P=y$, 则 由 $\triangle C P M$ 与 $\triangle C A B$ 相似, 得
\[
C P: P M=C A: A B
\]
也即
\[
\left(\frac{a}{\alpha}-x\right): y=\frac{a}{\alpha}: \frac{a}{\beta}
\]
从而
\[
\frac{a y}{\alpha}=\frac{a a}{\alpha \beta}-\frac{a x}{\beta}
\]
或
\[
\alpha x+\beta y=a
\]
正是所给方程.


【图,待补】
%%![](https://cdn.mathpix.com/cropped/2023_02_05_37db6883222c7ef36acdg-15.jpg?height=349&width=464&top_left_y=1793&top_left_x=594)

图 13 

\section{$\S 41$}

$\alpha=0$ 或 $\beta=0$, 上面的推导不能进行. 不过这种情形易于讨论. 如果 $\alpha=0, y=a$, 显然该 方程描述的是直线, 这直线平行于轴, 至轴的距离为 $a$. 如果 $\beta=0, x=a$, 该方程描述的也 显然是直线, 这直线垂直于轴, 至原点的距离为 $a$. 此时所有纵标都对应于同一个横标, 这里横标是不变量. 从以上的讨论,我们完全清楚了, 坐标间方程是如何规定直线的.

\section{$\S 42$}

到现在为止,描述曲线时,我们所用的坐标都是相垂直的. 纵标线与横标线成斜角 时,我们也可以对它求出方程规定的曲线. 反之,曲线的性质也都可以用斜角坐标方程描 述. 同一条曲线, 对每一个坐标角(包括直角和斜角)因轴和原点位置的变化,而有无穷 多个方程,这无穷多个方程可以用一个通用方程包含.坐标角的个数无穷,因而通用方程 的个数也无穷. 这无穷多个通用方程也可以用一个方程包含, 我们称这个方程为最通用 方程. 坐标角固定为直角时,最通用方程就成了我们讨论过的通用方程.

\section{$\S 43$}

设曲线 $L M$ 关于直角坐标 $A P=x, P M=y$ 的方程已知,如图 14 所示. 保持轴 $R S$ 和原 点 $A$ 不变, 改坐标角为 $\varphi$, 我们来求新坐标下 $L M$ 的方程. 从点 $M$ 向轴 $R S$ 引直线 $M Q$, 使 与轴 $R S$ 的夹角 $\angle M Q A$ 为 $\varphi$, 记 $\sin \varphi=\mu, \cos \varphi=\nu$. 在新坐标下, 点 $M$ 的横标为 $A Q=t$, 纵 标为 $Q M=u$. 从 $\triangle P M Q$ 得
\[
\frac{y}{u}=\mu, \quad \frac{P Q}{u}=v=\frac{t-x}{u}
\]
从而
\[
u=\frac{y}{\mu}, \quad t=v u+x=\frac{\nu y}{\mu}+x
\]
进而
\[
x=t-\nu u, \quad y=\mu u
\]
代入 $x, y$ 间方程, 就得到坐标角为 $\varphi$ 时,曲线 $L M$ 的 $t, u$ 间方程.


【图,待补】
%%![](https://cdn.mathpix.com/cropped/2023_02_05_37db6883222c7ef36acdg-16.jpg?height=282&width=503&top_left_y=1908&top_left_x=586)

图 14 

\section{$\S 44$}

反之, 从曲线 $L M$ 的斜角坐标 $A Q, M Q$ 间的方程, 也可以求出其直角坐标 $A P, M P$ 间 的方程. 设纵横坐标线 $M Q, A Q$ 的夹角为 $\varphi$, 记 $\varphi$ 的正弦为 $\mu$, 余弦为 $\nu$. 又设 $A Q=t$, $Q M=u$ 间的方程已给. 从点 $M$ 向轴 $R S$ 引直角纵标线 $M P$, 记横标 $A P=x$, 纵标 $M P=y$, 我们有
\[
u=\frac{y}{\mu}, \quad t=\frac{\nu y}{\mu}+x
\]
代入 $t, u$ 间方程, 就得到我们所要的 $x, y$ 间方程.

\section{$\S 45$}

设曲线 $L M$ 关于直角坐标 $A P=x, P M=y$ 的方程已给, 下面我们求它的最通用方 程. 取直线 $r s$ 为新轴, 取它上面的点 $D$ 作新原点, 记新纵标线 $M T$ 与 $r s$ 的夹角 $\angle D T M=$ $\varphi$, 记 $\varphi$ 的正弦为 $\mu$, 余弦为 $\nu$. 我们要做的是, 求出 $L M$ 的关于新横标 $D T$ 和新纵标 $T M$ 的 方程. 从点 $D$ 引旧轴 $R S$ 的垂线 $D G$, 记 $A G=f, D G=g$. 过点 $D$ 引旧轴 $R S$ 的平行线 $D O$, 记 $\angle O D s$ 的正弦为 $m$, 余弦为 $n$. 像我们做过的那样, 自点 $M$ 引新轴 $r s$ 的垂线 $M Q$, 记 $D Q=t, Q M=u$, 斜角坐标为 $D T=r, T M=s$.

首先我们有
\[
t=r-\nu s, \quad u=\mu s
\]
其次当然有
\[
x=m u+n t-f, \quad y=n u-m t-g
\]
从而
\[
x=n r-(n \nu-m \mu) s-f, \quad y=-m r+(\mu m+\nu m) s-g
\]
其中 $n \nu-m \mu$ 是 $\angle A V M$ 的余弦, $\angle A V M$ 是新纵标线与旧轴 $R S$ 所成的角, 而 $\mu m+\nu m$ 是 $\angle A V M$ 的正弦. 将得到的 $x, y$ 的表达式代入原方程, 结果为新坐标 $r, s$ 间的方程, 这就是 我们所要的曲线 $L M$ 的最通用方程.


【图,待补】
%%![](https://cdn.mathpix.com/cropped/2023_02_05_37db6883222c7ef36acdg-17.jpg?height=397&width=672&top_left_y=1764&top_left_x=516)

图 15 

\section{$\S 46$}

由于把 $x, y$ 间方程变为 $r, s$ 间方程的代换是一次的,所以 $r, s$ 间的最通用方程与原 $x, y$ 间的方程同阶. 可见轴、原点和斜坐标角的变化, 全都不影响曲线的阶数. 描述同一 条曲线的直角坐标方程和斜角坐标方程, 虽然都变化无穷, 但方程的阶数, 既不减小也不 增大. 因此, 两个方程, 不管别的方面何其相似, 只要阶数不同, 它们描述的曲线就一定不 相同. 

