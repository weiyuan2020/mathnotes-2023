\chapter{第二十章 列方程}

\section{$\S 486$}

前章讨论曲线的交点, 主要的一环是列出高次方程. 那里是从两条曲线列出一个方 程, 所得方程的根指出这两条曲线的交点. 反之, 可以从两条曲线的交点求出方程的根. 当一个方程的根应该用线表示的时候, 用曲线交点求根这个方法特别有用. 因为画出了 适用于这一目的两条曲线, 就可以指出其交点, 从交点向轴引垂线, 垂足处的横标就是方 程的根. 用这种方法求出的横标都是根, 但方程可以有求出横标以外的根. 这是该方法的 不足.

\section{$\S 487$}

如果给了末知量 $x$ 的代数方程, 要得到它的根, 那就先求出两条曲线的变量 $x, y$ 间 的两个方程, 使得从这两个方程消去纵标 $y$ 就得到所给代数方程, 然后在同轴同原点之 下画求得的两条曲线, 当然也就有了这两条曲线的交点. 从交点向轴画垂线, 垂足处横标 都为所给方程的根. 只要方程不含交点以外的根, 用这个方法就可以把它的根都求出来.

\section{$\S 488$}

我们要讲的是求两条曲线, 并从求得的两条曲线列出给 定方程. 我们先看看求得了两条曲线之后怎样列方程. 假定求 得的两条线为直线 $E M$ 和 $F M$, 交点为 $M$, 如图 97 所示. 取直 线 $E F$ 作轴, 取点 $A$ 作原点. 从 $A$ 引轴的垂线 $A B C$, 交 $E M$ 于 $B$, 交 $F M$ 于 $C$. 记 $A E=a, A F=b, A B=c, A C=d$. 记横标 $A P=x$, 纵标 $P M=y$. 这样从直线 $E M$ 得 $a: c=(a+x): y$, 或 $a y=c(a+x)$, 从直线 $F M$ 得 $b: d=(b-x): y$, 或 $b y=d(b-x)$. 从得到的这两个方程消去 $y$, 得


【图,待补】
%%![](https://cdn.mathpix.com/cropped/2023_02_05_a0e82ac431ec1349400cg-16.jpg?height=287&width=404&top_left_y=1592&top_left_x=1105)

图 97 
\[
b c(a+x)=a d(b-x)
\]
或
\[
x=\frac{a b d-a b c}{b c+a d}=\frac{a b(d-c)}{b c+a d}
\]
这样根据两条直线的交点可列出线性方程 
\[
x==\frac{a b(d-c)}{b c+a d}
\]
线性方程都可以化为这种形式.

$\S 489$

除了直线, 最容易画出来的线是圆. 因而接下 去我们假定求得的两条线为直线和圆, 看看从它 们的交点如何列出方程. 取 $A P$ 作轴, $A$ 作原点, 画 直线 $E M$ 如图 98. 记 $A E=a, A B=b$. 记坐标 $A P=$ $x, P M=y$. 这样我们有 $a: b=(a+x): y$, 或 $a y=$ $b(a+x)$. 这是直线方程. 设圆的半径 $C M=c$, 从圆 心 $C$ 向轴画垂线 $C D$, 记 $A D=f, C D=g$, 则 $D P=$ $x-f, P M-C D=y-g$.


【图,待补】
%%![](https://cdn.mathpix.com/cropped/2023_02_05_a0e82ac431ec1349400cg-17.jpg?height=302&width=562&top_left_y=596&top_left_x=973)

图 98

由圆的性质我们有
\[
C M^{2}=D P^{2}+(P M-C D)^{2}
\]
由此得圆的方程
\[
c^{2}=x^{2}-2 f x+f^{2}+y^{2}-2 g y+g^{2}=(x-f)^{2}+(y-g)^{2}
\]
从直线方程得 $y=\frac{a b+b x}{a}$, 从而
\[
y-g=\frac{a(b-g)+b x}{a}=b-g+\frac{b x}{a}
\]
代入圆的方程, 得
\[
c^{2}=x^{2}-2 f x+f^{2}+(b-g)^{2}+\frac{2 b(b-g) x}{a}+\frac{b^{2} x^{2}}{a^{2}}
\]
或
\[
a^{2} x^{2}+2 a b(b-g) x+a^{2}(b-g)^{2}+b^{2} x^{2}-2 a^{2} f x+a^{2} f^{2}-a^{2} c^{2}=0
\]
该方程的根可从直线与圆的交点 $M$ 和 $m$ 求出. 从 $M$ 和 $m$ 向轴画垂线 $M P$ 和 $m p, A P$ 和 Ap 都是所求的 $x$ 值.

\section{$\S 490$}

上节求得的方程包含所有的二次方程, 因而从它可以推出二次方程的一般列法, 设 给定的二次方程为
\[
A x^{2}+B x+C=0
\]
乘以 $\frac{a^{2}+b^{2}}{A}$, 使 $x^{2}$ 的系数与前节相同,得
\[
\left(a^{2}+b^{2}\right) x^{2}+\frac{B\left(a^{2}+b^{2}\right) x}{A}+\frac{C\left(a^{2}+b^{2}\right)}{A}=0
\]
令 $x$ 的系数与前节方程的相等,得
\[
2 A a b(b-g)-2 A a^{2} f=B\left(a^{2}+b^{2}\right)
\]
从而
\[
a f=b(b-g)-\frac{B\left(a^{2}+b^{2}\right)}{2 A a}
\]
代入令常数项相等所得等式
\[
a^{2}(b-g)^{2}+a^{2} f^{2}-a^{2} c^{2}=\frac{C\left(a^{2}+b^{2}\right)}{A}
\]
得
\[
\left(a^{2}+b^{2}\right)(b-g)^{2}-\frac{B b(b-g)\left(a^{2}+b^{2}\right)}{A a}+\frac{B^{2}\left(a^{2}+b^{2}\right)^{2}}{4 A^{2} a^{2}}-a^{2} c^{2}=\frac{C\left(a^{2}+b^{2}\right)}{A}
\]
从而
\[
b-g=\frac{B b}{2 A a} \pm \sqrt{\frac{a^{2} b^{2}}{a^{2}+b^{2}}+\frac{C}{A}-\frac{B^{2}}{4 A^{2}}}
\]
$a, b, c$ 尚末确定,应取这三个量使得
\[
\frac{a^{2} c^{2}}{a^{2}+b^{2}}+\frac{C}{A}-\frac{B^{2}}{4 A^{2}}+\frac{B^{2} b^{2}}{4 A^{2} a^{2}}
\]
为正. 否则, $b-g$, 也即 $C D$ 为虚数.

\section{$\S 491$}

不妨令 $b=0$, 这样则
\[
g=\sqrt{c^{2}+\frac{-B^{2}+4 A C}{4 A^{2}}}, \quad f=-\frac{B}{2 A}
\]
方程 $A x^{2}+B x+C=0$ 应该有实根, 因而 $B^{2}>4 A C, \frac{B^{2}-4 A C}{4 A^{2}}$ 为正, 令这正数等于 $c^{2}$, 则由
\[
c=\frac{\sqrt{B^{2}-4 A C}}{2 A}
\]
得 $g=0, a$ 不再需要计算. 因而直线 $E M$ 与轴 $A P$ 重合, 在
\[
A D=-\frac{B}{2 A}
\]
的条件下, 圆心与点 $D$ 重合, 以这个圆心为圆心, 以
\[
c=\frac{\sqrt{B^{2}-4 A C}}{2 A}
\]
为半径的圆与轴的交点, 就是我们的方程的根. 为避免列出的方程中出现无理式, 令 $g=$ $c-\frac{k}{2 A}$, 则
\[
c^{2}-\frac{2 c k}{2 A}+\frac{k^{2}}{4 A^{2}}=c^{2}+\frac{-B^{2}+4 A C}{4 A^{2}}
\]
从而
\[
c=\frac{k^{2}+B^{2}-4 A C}{4 k A}, \quad g=\frac{B^{2}-4 A C-k^{2}}{4 k A}
\]
这样还有一个量 $k$ 任我们选定. 不管 $k$ 取什么值, 由于 $C M$ 在轴上, 所以圆的画法都应如 下: 取 $A D=-\frac{B}{2 A}$, 作垂线 $C D=\frac{B^{2}-4 A C-k^{2}}{4 A k}$, 以 $C$ 为圆心, $\frac{B^{2}-4 A C+k^{2}}{4 A k}$ 为半径画 圆, 这个圆与轴的交点就是我们的方程的根. 如果令 $k=-B$, 取 $A D=-\frac{B}{2 A}$, 则 $C D=\frac{C}{B}$, 以 $C$ 为心所画圆的半径为
\[
\frac{-B^{2}+2 A C}{2 A B}=-\frac{B}{2 A}+\frac{C}{B}
\]
等于 $A D+C D$. 这个方法在实用上是很方便的.

\section{$\S 492$}

现在考虑相交的两个圆, 如图 99 所示. 对第一个圆, 记 $A D=a, C D=b$, 记半径 $C M=c$. 那么如果令 $A P=x, P M=y$, 则
\[
D P=a-x, \quad C D-P M=b-y
\]
从而由圆的性质得
\[
x^{2}-2 a x+a^{2}+y^{2}-2 b y+b^{2}=c^{2}
\]
类似地, 对第二个圆, 记 $A d=f, d c=g$, 记半径 $c M=h$, 得
\[
x^{2}-2 f x+f^{2}+y^{2}+2 g y+g^{2}=h^{2}
\]
这两个方程相减,得
\[
2(f-a) x+a^{2}-f^{2}-2(b+g) y+b^{2}-g^{2}=c^{2}-h^{2}
\]
从而


【图,待补】
%%![](https://cdn.mathpix.com/cropped/2023_02_05_a0e82ac431ec1349400cg-19.jpg?height=491&width=292&top_left_y=1018&top_left_x=1144)

图99
\[
y=\frac{a^{2}+b^{2}-f^{2}-g^{2}-c^{2}+h^{2}-2(a-f) x}{2(b+g)}
\]
进而
\[
b-y=\frac{b^{2}+2 b g-a^{2}+f^{2}+g^{2}+c^{2}-h^{2}+2(a-f) x}{2(b+g)}
\]
又
\[
a-x=\frac{2 a(b+g)-2(b+g) x}{2(b+g)}
\]
代入已知等式 $(a-x)^{2}+(b-y)^{2}=c^{2}$, 得
\[
\begin{aligned}
& +4(a-f)^{2} x^{2}-4(a+f)(b+g)^{2} x+(b+g)^{4}+2\left(a^{2}-c^{2}\right)(b+g)^{2}+ \\
& 4(b+g)^{2} x^{2}-4(a-f)\left(a^{2}-f^{2}\right) x+2\left(f^{2}-h^{2}\right)(b+g)^{2}+ \\
& 4(a-f)\left(c^{2}-h^{2}\right) x+\left(a^{2}-c^{2}-f^{2}+h^{2}\right)^{2}=0
\end{aligned}
\]
可见,用上面导出的方程列出方程
\[
A x^{2}+B x+C=0
\]
的方式有无穷多种. 但要记住, 次数高于 2 的方程不能用两个圆的交点列出, 因为两个圆 的交点最多为两个. 同一个二次方程, 既可用直线与圆相交列出, 亦可用两圆相交列出. 当然,除非两圆情况下的 $a, b, f, g, c, h$ 极易确定, 人们更喜欢的是用直线与圆.

\section{$\S 493$}

现在我们考虑圆与抛物线的相交, 参见图 100, 从圆 心 $C$ 向 $A P$ 引垂线 $C D$, 记 $A D=a, C D=b$, 记圆的半径 $C M=c$, 那么在直角坐标 $A P=x, P M=y$ 之下, 该圆的方 程为 $(x-a)^{2}+(y-b)^{2}=c^{2}$. 设抛物线的轴 $F B$ 垂直于轴 $A P$, 记 $A E=f, E F=g$, 记抛物线参数为 $2 h$,那么由抛物 线的性质得 $E P^{2}=2 h(E F+P M)$, 或者使我们的记号为 $(x-f)^{2}=2 h(g+y)$, 从而
\[
y=\frac{(x-f)^{2}}{2 h}-g, \quad y-b=\frac{(x-f)^{2}}{2 h}-(b+g)
\]

【图,待补】
%%![](https://cdn.mathpix.com/cropped/2023_02_05_a0e82ac431ec1349400cg-20.jpg?height=424&width=426&top_left_y=592&top_left_x=1087)

图 100

将这个值代入圆的方程, 消去 $y$, 得
\[
\frac{(x-f)^{4}}{4 h^{2}}-\frac{(b+g)(x-f)^{2}}{h}+(b+g)^{2}+(x-a)^{2}=c^{2}
\]
或
\[
\left.\begin{array}{l}
x^{4}-4 f x^{3}+6 f^{2} x^{2}-4 f^{3} x+f^{4} \\
-4 h(b+g)+8 h(b+g)-4 f^{2} h(b+g) \\
+4 h^{2}-8 a h^{2}+4 h^{2}(b+g)^{2} \\
+4 a^{2} h^{2} \\
-4 c^{2} h^{2}
\end{array}\right\}=0
\]
横标 $A P, A p, A p, A p$ 就是这个方程的根, 纵标过交点 $M, m, m, m$.

\section{$\S 494$}

该方程含六个常数 $a, b, c, f, g, h$, 但 $b, g$ 以 $b+g$ 形式出现, 视 $b+g$ 为一个常数,那 就成了五个常数. 令 $C D+E F=b+g=k$, 则方程成为
\[
\begin{aligned}
& x^{4}-4 f x^{3}+6 f^{2} x^{2}-4 f^{3} x+f^{4} \\
& -4 h k+8 h k-4 f^{2} h k \\
& +4 h^{2}-8 a h^{2}+4 h^{2} k^{2} \\
& +4 a^{2} h^{2} \\
& -4 c^{2} h^{2}
\end{aligned}
\]
凡四次方程都可化成这种形式. 事实上, 设给定的方程为
\[
x^{4}-A x^{3}+B x^{2}-C x+D=0
\]
%%12p221-240
比较系数, 从
\[
4 f=A \text {, 得 } f=\frac{1}{4} A
\]
从
\[
6 f^{2}-4 h k+4 h^{2}=B, \quad \text { 得 } \frac{3}{8} A^{2}-4 h k+4 h^{2}=B
\]
进而
\[
k=\frac{3 A^{2}}{32 h}+h-\frac{B}{4 h}
\]
从
\[
4 f^{3}-8 f h k+8 a h^{2}=C
\]
得
\[
\frac{1}{16} A^{3}-\frac{3}{16} A^{3}-A h^{2}+\frac{1}{2} A B+8 a h^{2}=C
\]
进而
\[
a=\frac{A^{3}}{64 h^{2}}+\frac{A}{4}-\frac{A B}{16 h^{2}}+\frac{C}{8 h^{2}}
\]
最后,使常数项相等得
\[
\left(f^{2}-2 h k\right)^{2}+4 a^{2} h^{2}-4 c^{2} h^{2}=D
\]
但
\[
\begin{gathered}
f^{2}-2 h k=\frac{B}{2}-2 h^{2}-\frac{A^{2}}{8} \\
2 a h=\frac{A^{3}}{32 h}+\frac{A h}{2}-\frac{A B}{8 A}+\frac{C}{4 h}
\end{gathered}
\]
将这两个值代入上式, 得 $c, h$ 间方程. 从所得方程易于解出 $c, h$, 取它们的实值.

\section{$\S 495$}

四次方程的第二项都可消去. 下面我们讨论消去了第二项的方程
\[
x^{4}+B x^{2}-C x+D=0
\]
此时我们有 $f=0, k=h-\frac{B}{2 h}, a=\frac{C}{8 h^{2}}$, 由
\[
2 h k-f^{2}=2 h^{2}-\frac{B}{2}, \quad 2 a h=\frac{C}{4 h}
\]
得
\[
4 h^{4}-2 B h^{2}+\frac{1}{4} B^{2}+\frac{C^{2}}{16 h^{2}}-4 c^{2} h^{2}=D
\]
去分母,得
\[
64 c^{2} h^{4}=C^{2}+4 B^{2} h^{2}-32 B h^{4}+64 h^{6}-16 D h^{2}
\]
开方, 得
\[
\begin{aligned}
& 8 c h^{2}=\sqrt{4 h^{2}\left(B-4 h^{2}\right)^{2}+C^{2}-16 D h^{2}}
\end{aligned}
\]
应保证 $c, h$ 都为实数, 令 $c=h-\frac{B+q}{4 h}$, 得
\[
C^{2}-16 D h^{2}-8 B h^{2} q+32 h^{4} q-4 h^{2} q^{2}=0
\]
为满足所提要求, 分 $D$ 为正为负两种情形进行考虑.

I

$D$ 为正, 设 $D=E^{2}$, 则要列出的方程为
\[
x^{4}+B x^{2}-C x+E^{2}=0
\]
令 $q=0$, 得 $c=\frac{4 h^{2}-B}{4 h}, h^{2}=\frac{C^{2}}{16 E^{2}}, h=\frac{C}{4 E}$, 从而
\[
c=\frac{C^{2}-4 B E}{4 C E}
\]
且
\[
k=c=\frac{C^{2}-4 B E}{4 C E}, \quad a=\frac{2 E^{2}}{C}, \quad f=0
\]
II

$D$ 为负, 设 $D=-E^{2}$, 要列出的方程为
\[
x^{4}+B x^{2}-C x-E^{2}=0
\]
我们有
\[
64 c^{2} h^{4}=C^{2}+4 h^{2}\left(4 h^{2}-B\right)^{2}+16 E^{2} h^{2}
\]
由
\[
c=\frac{\sqrt{C^{2}+4 h^{2}\left(4 h^{2}-B\right)^{2}+16 E^{2} h^{2}}}{8 h^{2}}
\]
知, 不管 $h$ 取什么值, 该方程给出的 $c$ 都为实数, 因而 $h$ 可以任取. 这样在每种情况下我们 都取 $h$,使得 $c$ 最容易求得.

这样做了之后, 跟过去一样,我们有
\[
\begin{gathered}
A E=f=0, \quad C D+E F=k=\frac{4 h^{2}-B}{4 h} \\
A D=a=\frac{C}{8 h^{2}}
\end{gathered}
\]
如果 $E=0$, 则要列出的方程为
\[
x^{3}+B x+C=0
\]
该方程的列出依赖有名的 Backer 规则.

\section{$\S 496$}

任取两条二阶线, 也即两条圆雉曲线, 它们关于同一根轴、同一个横标原点的方程可 写成
\[
a y^{2}+b x y+c x^{2}+d y+e x+f=0
\]
\[
\begin{aligned}
& \alpha y^{2}+\beta y x+\gamma x^{2}+\delta y+\varepsilon x+\zeta=0
\end{aligned}
\]
我们用前面讲过的方法来消去 $y$, 为此将这两个方程与 $\$ 479$ 的两个方程
\[
\begin{gathered}
P+Q y+R y^{2}=0 \\
p+q y+r y^{2}=0
\end{gathered}
\]
相比较. 我们看到, $P, p$ 是 $x$ 的二次函数, $Q, q$ 是 $x$ 的一次函数, $R, r$ 是常数. 可见作为结 果的那个方程将是双二次的. 这样, 从两条圆雉曲线的交就列不出次数高于双二次的方 程. 前面讲了从圆和抛物线的交列出的是双二次方程. 二阶线与一条直线可以有两个交 点, 与两条直线可以有四个交点. 两条直线作为整体可看成一条二阶线. 由此可见, 两条 二阶线可以有四个交点.

\section{$\S 497$}

现在我们考虑一条二阶线和一条三阶线的交,这两条线的方程为
\[
\begin{gathered}
P+Q y+R y^{2}=0 \\
p+q y+r y^{2}+s y^{3}=0
\end{gathered}
\]
这里 $P, Q$ 依次是 $x$ 的二次和一次函数, $R$ 是常数; $p, q, r$ 依次是 $x$ 的三、二、一次函数, $s$ 是 常数. 从 $\$ 480$ 的讨论我们看到, 从这两个方程消去 $y$, 所得方程是六阶的. 可见从圆雉曲 线与三阶线的交列不出次数高于 6 的方程. 这一结论也可以从三阶线的性质推出, 三阶 线与一条直线有三个交点, 两条直线作为整体可视为一条二阶线, 这二阶线与三阶线的 交点是 6 个.

\section{$\S 498$}

将前面讲的消去 $y$ 的方法或与直线相交的方法应用到更高阶方程上去,显然地我们 得到, 从两条三阶线的交可列出 9 阶方程, 从两条四阶线的交可列出不高于 16 阶的方程. 一般地, 从阶数分别为 $m$ 和 $n$ 的两条曲线的交可列出阶数不高于 $m n$ 的各阶方程. 例如, 我们要列出 100 阶方程, 这时的两条曲线, 可以都是 10 阶的, 可以一个 20 阶的和一个 5 阶 的, 等等. 只要这两个阶数的乘积等于 100 , 就都可以. 如果要列出的方程的阶数是素数, 或者是没有方便因数的数, 那么利用从列出的方程可列出阶数比它低的各阶方程这一 点, 可取两个方便的, 但积大于要列出的方程的阶数大的因数, 并列出相应的方程. 例如, 对 39 阶方程, 可以取阶数为 6 和 7 的两条曲线, 从这两条曲线可列出 42 阶方程. 这比从 3 阶和 13 阶曲线列出 39 阶方程要简单.

\section{$\S 499$}

可见, 从交点横标为所给方程实根的两条曲线, 列出所给方程, 这方式有很多种, 甚 至有无穷多种. 在这无穷多种方式中, 我们取两条最简单最容易画出的曲线. 当然首先要交点给出的根都为实数. 没有虚交点的两条曲线即可满足这一要求. 前面我们看到了, 一 条曲线方程中的纵标 $y$ 可表示成横标 $x$ 的线性函数时, 交点就都是实的. 由于这条单值 的纵标都是实的, 所以不管另一条曲线有多少个虚纵标, 它们也不会有虚交点. 在列方程 的过程中, 我们总取一条曲线的方程为 $P+Q y=0, P$ 和 $Q$ 都是 $x$ 的某个函数.

\section{$\S 500$}

这样,不管给定的方程是什么, 我们都取两条曲线中的一条, 其方程的形状为 $P+$ $Q y=0$. 另一条的方程应该是这样的, 换 $y$ 为 $-\frac{P}{Q}$ 就成为给定方程. 由此可见, 反过来, 换 给定方程中的 $-\frac{P}{Q}$ 为 $y$, 得到的就是另一条曲线的方程. 例如, 给定的方程为
\[
x^{4}+A x^{3}+B x^{2}+C x+D=0
\]
我们取方程 $a y=x^{2}+b x$. 表示的抛物线作两条曲线中的一条. 由这抛物线方程得 $x^{2}=$ $a y-b x$ 代入给定方程的四次和三次项, 得
\[
\begin{aligned}
& x^{4}=a^{2} y^{2}-2 a b x y+b^{2} x^{2} \\
& A x^{3}=+A a x y-A b x^{2}
\end{aligned}
\]
最后得二阶线方程
\[
a^{2} y^{2}+a(A-2 b) x y+\left(B-A b+b^{2}\right) x^{2}+C x+D=0
\]
该曲线与曲线 $a y=x^{2}+b x$ 的交点的横标是给定方程的根.

\section{$\S 501$}

上节的两条曲线, 常量 $a, b$ 可为任意值, 因而有无穷多种变化. 我们可以进一步增加 这无穷多种变化, 事实上, 由第一个方程得 $x^{2}-a y+b x=0$, 由该方程得 $a c x^{2}-a^{2} c y+$ $a b c x=0$, 把这个方程加到第二个方程上去, 得到二阶线的更为一般的方程. 它同第一条 曲线的交点同样指出给定方程的根. 这样用来列方程的两条曲线, 其方程就成了
\[
\begin{gathered}
\mathrm{I} \\
a y=x^{2}+b x \\
\quad \text { II } \\
a^{2} y^{2}+a(A-2 b) x y+\left(B-A b+b^{2}+a c\right) x^{2}-a^{2} c y+(C+a b c) x+D=0
\end{gathered}
\]
方程 II 包含所有的圆雉曲线, 应特别注意量
\[
A^{2}-4 B-4 a c
\]
它为正、为零、为负, 曲线依次为双曲线、为抛物线、为椭圆. 如果
\[
b=\frac{1}{2} A, \quad a^{2}=B-\frac{1}{4} A^{2}+a c
\]
从而
\[
c=a+\frac{A^{2}}{4 a}-\frac{B}{a}
\]
则曲线为圆,此时方程为
\[
a^{2} y^{2}+a^{2} x^{2}-\left(a^{3}+\frac{A^{2} a}{4}-B a\right) y+\left(C+\frac{A^{2} a}{2}+\frac{A^{3}}{8}-\frac{A B}{2}\right) x+D=0
\]
也唄
\[
\begin{aligned}
& \left(y-\frac{a}{2}-\frac{A}{8}+\frac{B}{2 a}\right)^{2}+\left(x+\frac{c}{2 a^{2}}+\frac{A}{4}+\frac{A^{3}}{16 a^{2}}-\frac{A B}{4 a^{2}}\right)^{2} \\
= & \left(\frac{a}{2}+\frac{A}{3}-\frac{B}{2 a}\right)^{2}+\left(\frac{C}{2 a^{2}}+\frac{A}{4}+\frac{A^{3}}{16 a^{2}}-\frac{A B}{4 a^{2}}\right)^{2}-\frac{D}{a^{2}}
\end{aligned}
\]
右端是圆朿径的平方.

\section{$\S 502$}

这样我们就有无穷多条圆雉曲线,它们同抛物线 $a y=x^{2}+b x$ 的交点,都给出给定方 程的根. 从这无穷多条中任取两条,当然这每一条与 $a y=x^{2}+b x$ 的交点都给出给定方程 的根,因而任取的这两条的交点也给出给定方程的根. 这样,从圆和拋物线可列出给定方 程 (这是我们讨论过的), 从两条抛物线, 从一条抛物线与一个椭圆或一条双曲线, 从两个 椭圆, 从一个椭圆与一条抛物线也都可以列出给定方程. 如果对更高阶曲线应用这一方 法,那列出给定方程的方式将大为增加.

\section{$\S 503$}

类似地, 取方程为 $y=P$ 的抛物线作两条曲线中的第一条,可以列出更高阶的方程. 例如,要列出的方程为
\[
x^{12}-f^{10} x^{2}+f^{9} g x-g^{12}=0
\]
取四阶抛物方程 $x^{4}=a^{3} y$ 作第一条, 那么将 $x^{12}=a^{9} y^{3}$ 代入, 得三阶方程
\[
a^{9} y^{3}-f^{10} x^{2}+f^{9} g x-g^{12}=0
\]
将第一个方程 $x^{4}-a^{3} y=0$ 的某个倍数加到这个方程上去,可得到无穷多条四阶线,其中 任何两条都给出所给方程的根.

\section{$\S 504$}

如果用上面的方法列出给定方程时不那么容易, 可以乘给定方程 $x$, 或者 $x^{2}$, 或者 $x^{3}$, 或者 $x$ 的某个更高次幂. 这增加了为零的根, 这种根对应的交点, 其横标为原点, 易于 从给定方程的真根中区别出来. 这样给定方程的次数增加了, 但往往却变得容易列出. 例 如.给定的为三次方程
\[
x^{3}+A x^{2}+B x+C=0
\]
令 $x^{2}=a y$, 这里一条曲线为抛物线,另一条必为双曲线. 事实上,换 $x^{2}$ 为 $a y$, 得
\[
a x y+A a y+B x+C=0
\]
将 $c x^{2}-a c y=0$ 加上去,得更一般的方程
\[
a x y+c x^{2}+a(A-c) y+B x+C=0
\]
它也恒为双曲线方程, 因而利用圆、椭圆或抛物线将更为方便. 此时要以 $x$ 乘给定方 程, 得
\[
x^{4}+A x^{3}+B x^{2}+C x=0
\]
将这个方程与前面列出的双二次方程相比较, 得 $D=0$. 这个方程恒可以用圆和抛物线 列出.

\section{$\S 505$}

由于任何阶的任何方程都可以用两条代数曲线的交点列出, 且方式有无穷多种, 因 而可用任何一条曲线代替另一曲线. 这就产生一个问题, 怎样才能保证给定方程能借助 一条给定曲线列出. 这里首先应该指出, 给定曲线应该是这样的, 它的纵标可表示成 $x$ 的 单值函数, 以保证列方程的过程中不产生虚交点. 当我们关心的只是给定方程的一个根 时, 通常要加上这个条件. 有这样的情形, 虽然对应于曲线的一段弧上某个点的横标是真 根, 但是这段弧没有交点. 所以如此, 是由于这根对应的是虚交点, 或者对应的是曲线的 另一个分支的交点. 列方程, 这是一个虽有意思但用途不大的问题. 有关它的基础性的东 西我们已经讲得够详细, 对这个问题的讨论不再继续.

