\chapter{第七章 伸向无穷的分支}

\section{$\S 166$}

如果曲线有一支或一部分伸向无穷, 则曲线上无穷远处点的直角坐标 $x, y$ 中必至少 有一个为无穷. 设曲线上无穷远处某点坐标 $x, y$ 都不为无穷, 则该点至原点的距离 $\sqrt{x^{2}+y^{2}}$ 为有限数, 这与无穷远矛盾. 因此, 当曲线有伸向无穷的分支时, 则必然是或者 有限的横标对应无限的实纵标, 或者无限的横标对应实的有限的或无限的纵标, 我们就 以此为起点对曲线的伸向无穷分支进行研究.

\section{$\S 167$}

给定 $x, y$ 间的一个代数方程, 阶数不拘, 比如为 $n$. 我们看它的 $x, y$ 次数之和为 $n$ 的 那些项所成的部分, 即
\[
\alpha y^{n}+\beta y^{n-1} x+\gamma y^{n-2} x^{2}+\delta y^{n-3} x^{3}+\cdots+\xi x^{n}
\]
该表达式可分解成状如 $A x+B y$ 的或实或虚的线性因式的积. 如果其中有虚线性因式, 则虚线性因式的个数必为偶数, 并且可以配对, 每对的积为状如
\[
A^{2} y^{2}-2 A B x y \cos \varphi+B^{2} x^{2}
\]
的实二次因式. 只要 $x, y$ 中有一个为 $\infty$, 这种二次因式的值就等于 $\infty^{2}$, 这是因为 $A, B$ 都 不为零, 项 $2 A B y x \cos \varphi$ 恒小于另外两项的和 $A^{2} y^{2}+B^{2} x^{2}$. 换句话说, 当 $x$, 或 $y$, 或 $x, y$ 同时为无穷时, 因式
\[
A^{2} y^{2}-2 A B x y \cos \varphi+B^{2} x^{2}
\]
不等于零, 不等于有限数, 甚至不等于无穷大, 它等于 $\infty^{2}, \infty^{2}$ 比 $\infty$ 大无穷大.

\section{$\S 168$}

如果方程次数最高各项的和
\[
\alpha y^{n}+\beta y^{n-1} x+\gamma y^{n-2} x^{2}+\cdots+\xi x^{n}
\]
不含实线性因式, 则 $n$ 必为偶数, 且和式为状如
\[
A^{2} y^{2}-2 A B x y \cos \varphi+B^{2} x^{2}
\]
的因式的积. 因而, 如果 $x$, 或 $y$, 或 $x, y$ 同时为无穷, 则和式的值不等于有限的量, 也不等 于 $\infty^{m}$ (这里的 $m$ 小于 $n$ ), 而等于 $\infty^{n}$. 方程其余的项, 也即 $x, y$ 次数和小于 $n$ 的项, 它们给 出的无穷, 其次数都低于 $n$, 不能等于 $n$ 次无穷. 因此, 只要 $x, y$ 中有一个取值为无穷, 方 程就不能成立.

\section{$\S 169$}

由此得出, 如果坐标 $x, y$ 间的方程, 其次数最高各项的和没有实线性因式, 它给出的 曲线就没有伸向无穷的分支, 整个曲线就像椭圆和圆那样, 处于一个有限的范围之中. 二 阶线通用方程
\[
\alpha y^{2}+\beta x y+\gamma x^{2}+\delta y+\varepsilon x+\zeta=0
\]
次数最高各项的和为 $\alpha y^{2}+\beta x y+\gamma x^{2} \cdot \beta^{2}$ 小于 $4 \alpha \gamma$ 时这个和没有实线性因式, 此时该方程 给出的曲线为椭圆, 没有伸向无穷的分支.

\section{$\S 170$}

为了解释得更清楚, 我们把 $x, y$ 间方程的项按次数分组, 把次数最高, 也即次数为 $n$ 的项归为第一组, 把次数为 $n-1$ 的项归为第二组, 把次数为 $n-2$ 的项归为第三组, 类推, 到既不含 $x$ 也不含 $y$ 的项, 也即到常数项为止. 记第一组, 也即次数最高的组为 $P$, 记第二 组为 $Q$, 记第三组为 $R$, 记第四组为 $S$, 等等.

\section{$\S 171$}

这样, 次数最高组 $P$ 没有实线性因式时, 方程 $P+Q+R+S+\cdots=0$ 表示的曲线就没 有伸向无穷的分支. 现在我们考虑次数最高组只有一个实线性因式的情形, 即 $P=$ $(a y-b x) M, M$ 为 $x, y$ 的 $n-1$ 次函数, 且 $M$ 没有任何实线性因式. 此时, 如果置 $x$, 或 $y$, 或 $x, y$ 同时为无穷, 则 $M=\infty^{n-1}$. 当然 $Q$ 也可以是 $M$ 这样的无穷, 而 $R, S$ 等则都只能是 次数更低的无穷. 因而,如果 $a y-b x$ 等于有限量或零, 方程
\[
P+Q+R+\cdots=0
\]
就可以成立,此时的曲线就伸向无穷.

\section{$\S 172$}

设 $a y-b x=p, p$ 是这样一个有限量: 当曲线伸向无穷时
\[
p M+Q+R+S+\cdots=0
\]
也即
\[
p=\frac{-Q-R-S-\cdots}{M}
\]
由 $M$ 是比 $R, S, \cdots$ 阶数更高的无穷大, 知分数 $\frac{R}{M}, \frac{S}{M}, \cdots$ 都为零, 因而 $p=\frac{-Q}{M}$. 这样一来,变量 $x, y$ 为无穷时, 分数 $\frac{-Q}{M}$ 给出 $p$ 值, 但是由 $a y-b x=p$, 得 $y=\frac{b x+p}{a}$, 又由 $x=\infty$ 时 $\frac{p}{a x}=0$, 得 $\frac{y}{x}=\frac{b}{a}+\frac{p}{a x}=\frac{b}{a}$. 这样一来,曲线伸向无穷时 $y=\frac{b x}{a}$.

\section{$\S 173$}

$Q, M$ 同为 $n-1$ 次齐次函数, $\frac{-Q}{M}$ 为零次函数,且 $y=\frac{b x}{a}$ 时, 它等于常数 $p$; 或者由 $y$ 比 $x$ 等于 $b: a$, 以 $b$ 替换 $\frac{-Q}{M}$ 中的 $y$, 以 $a$ 替换 $x$, 我们也得到函数 $\frac{-Q}{M}$ 的值 $p$, 有了 $p$ 也就 有了曲线伸向无穷时包含在方程 $P+Q+R+S+\cdots=0$ 中的 $a y-b x=p$.

\section{$\S 174$}

可见,曲线伸向无穷的部分由方程 $a y-b x=p$ 表示. 该方程表示的是直线,这直线在 无穷远处与曲线重合, 这直线是曲线的渐近线, 曲线向远处延伸时越来越靠近这直线, 最 终相重合. 从 $x$ 或 $y$ 等于 $\infty$ 时, 方程 $P+Q+R+S+\cdots=0$ 变为 $a y-b x=p$, 我们看到直 线向两头无限延伸, 最终都与曲线重合, 也即曲线有相背的两个伸向无穷远的分支, 一支 与直线的一头重合, 另一支与直线的另一头重合.

\section{$\S 175$}

上面我们看到, 方程 $P+S+Q+R+\cdots=0$ 的最高次部分 $P$ 只有一个实因式时, 曲线 具有伸向无穷的两个分支, 这两个分支从两头靠近我们称它为渐近线的直线. 现在我们 假定最高次部分 $P$ 有两个实线性因式 $a y-b x$ 和 $c y-d x$, 即 $P=(a y-b x)(c y-d x) M$, $M$ 是 $n-2$ 次齐次函数,这里应该区别两个线性因式相等和不相等两种情形.

\section{$\S 176$}

假定两因式不相等, 此时对无穷的横标或纵标方程
\[
(a y-b x)(c y-d x) M+Q+R+S+\cdots=0
\]
成立是在 $a y-b x$ 或 $c y-d x$ 为有限数时. 设 $a y-b x=p$, 由 $p$ 为有限数, 知在无穷远处有 $\frac{y}{x}=\frac{b}{a}$, 跟前面类似,我们得到
\[
p=\frac{-Q-R-S-\cdots}{(c y-d x) M}=\frac{-Q}{(c y-d x) M}
\]
该表达式是 $x$ 的零次函数, 因此令 $\frac{y}{x}=\frac{b}{a}$, 或者分别换 $x, y$ 为 $a, b$, 我们就得到常数 $p$ 的值. 由两线性因式不相等, 知 $b c-a d$ 不为零, 由 $M$ 没有实根, 知 $M$ 不为零, 因而此时有
\[
p=\frac{-Q}{(b c-a d) M}
\]
从而 $Q$ 为零, 或 $Q$ 含有因式 $a y-b x$, 则 $p$ 为有限值, 甚至或者为零.

\section{$\S 177$}

由前面的情形知, $P$ 的线性实因式 $a y-b x$ 使曲线有一条渐近线,其位置由方程 $a y-$ $b x=p$ 决定. 同样地, 另一个线性实因式 $c y-d x$ 也使曲线有一条渐近线, 其位置由方程

$c y-d x=q$ 决定, $q=\frac{-Q}{(a y-b x) M}$, 其中 $x, y$ 分别换成 $c, d$. 这样一来, 曲线共有两条渐近 线, 四条伸向无穷并最终与渐近线相合的分支. 这正是前面讨论过的双曲线的情形. 可 见, 二阶线方程 $\alpha y^{2}+\beta x y+\gamma x^{2}+\varepsilon x+\zeta=0$, 如果其最高次项部分 $\alpha y^{2}+\beta x y+\gamma x^{2}$ 有两个 不同的实线性因式,也即如果 $\beta^{2}$ 大于 $4 \alpha \gamma$,则这方程表示的是双曲线.

\section{$\S 178$}

假定线性因式 $a y-b x, c y-d x$ 相等, 即 $P=(a y-b x)^{2} M$, 则由 $P+Q+R+$ $S+\cdots=0$ 得
\[
(a y-b x)^{2}=\frac{-Q-R-S-\cdots}{M}
\]
由于 $Q, R, S$ 依次是 $n-1, n-2, n-3$ 次函数, $M$ 是 $n-2$ 次函数,因而变量无穷时我们有 $\frac{S}{M}=0$. 从而
\[
(a y-b x)^{2}=-\frac{Q}{M}-\frac{R}{M}=-\frac{Q}{M(\mu y+\nu x)}(\mu y+\nu x)-\frac{R}{M}
\]
但 $\frac{Q}{M(\mu y+\nu x)}$ 和 $\frac{R}{M}$ 是 $x, y$ 的零次函数, 又对无穷值有 $y: x=b: a$, 因而换 $\frac{y}{x}$ 为 $\frac{b}{a}$, 也即 分别换 $x, y$ 为 $a, b$ 时, 这两个零次函数都为常数.

\section{$\S 179$}

作代换, 令
\[
\frac{Q}{M(\mu y+\nu x)}=A, \quad \frac{R}{M}=B
\]
则
\[
(a y-b x)^{2}=-A(\mu y+\nu x)-B
\]
该方程所表示的曲线与 $P+Q+R+S+\cdots=0$ 所表示的曲线在无穷远处重合. $\mu, \nu$ 任意, 我们令 $\mu=b, \nu=a$, 取 

$a y-b x=u \sqrt{a^{2}+b^{2}}, \quad b y+a x=t \sqrt{a^{2}+b^{2}}$}

改变坐标后, 我们的曲线方程变为
\[
u^{2}+\frac{A t}{\sqrt{a^{2}+b^{2}}}+\frac{B}{a^{2}+b^{2}}=0
\]
显然, 这是抛物线方程. 因而所求曲线延伸到无穷时将与抛物线重合. 这样, 它只有两个 伸向无穷的分支, 其渐近线不是直线,而是前面方程所表示的抛物线.

\section{$\S 180$}

上节假定 $A$ 不为零. 如果 $A=0$, 即 $Q$ 为零, 或 $Q$ 被 $a y-b x$ 除得尽, 则方程成为 $u^{2}+$ $\frac{B}{a^{2}+b^{2}}=0$, 不再是抛物线方程. 我们对该方程的三类情形分别进行讨论. 第一类, $B$ 为 负, 设 $\frac{B}{a^{2}+b^{2}}=-f^{2}$. 此时方程 $u^{2}-f^{2}=0$ 含有两个方程, $u-f=0$ 和 $u+f=0$, 表示的是 两条平行直线. 有如第一种情形, 这两条直线都是渐近线, 因而曲线有四条伸向无穷并与 这两条直线相合的分支.

\section{$\S 181$}

第二类情形, $B$ 为正, 记为 $+f^{2}$. 此时方程 $u^{2}+f^{2}=0$ 为不可能, 所以曲线没有伸向无 穷的分支, 而是整个地位于一个有限区域之中, 即方程 $P+Q+R+S+\cdots=0$ 表示的曲 线没有伸向无穷的分支, 最高次部分 $P$ 没有实线性因式时没有这种分支, $P$ 有实线性因 式时也没有这种分支,这一点我们已经见到了, 以后还会多次见到.

\section{$\S 182$}

第三类情形, $B=0$, 它介于前两类情形之间, 要弄清此时曲线的形状, 需考察次数更 低的部分. 由 $P+Q+R+S+\cdots=0$ 和 $P=(a y-b x)^{2} M$, 得到在无穷远处有
\[
\frac{y}{x}=\frac{b}{a}, \quad(a y-b x)^{2}+\frac{Q}{M}+\frac{R}{M}+\frac{S}{M}+\frac{T}{M}+\cdots=0
\]
照前面进行代换那样,令
\[
\frac{y}{x}=\frac{b}{a}, \quad \frac{Q}{M}=A(b y+a x), \quad \frac{R}{M}=B
\]
再由 $S, T, V, \cdots$ 依次为 $(n-3),(n-4), \cdots$ 次函数, 及 $M$ 是 $n-2$ 次函数, 我们有
\[
\frac{S(b y+a x)}{M}=C, \quad \frac{T(b y+a x)^{2}}{M}=D, \quad \frac{V(b y+a x)^{3}}{M}=E, \quad \cdots
\]
从而
\[
(a y-b x)^{2}+A(b y+a x)+B+\frac{C}{b y+a x}+\frac{D}{(b y+a x)^{2}}+\frac{E}{(b y+a x)^{3}}+\cdots=0
\]
该方程表示的是这样一条曲线, $b y+a x$ 无穷大时, 其无穷远部分与 $P+Q+R+$ $S+\cdots=0$ 所表示的曲线相合. 曲线伸向无穷时, 虽然 $b y+a x$ 为无穷, 但 $(a y-b x)^{2}$ 可以 为有限数, 可以为无穷, 为无穷时小于 $\infty^{2}$.

\section{$\S 183$}

把轴变换到求得的渐近线上, 分别置横标与纵标为
\[
\frac{a x+b y}{\sqrt{a^{2}+b^{2}}}=t \text { 和 } \frac{a y-b x}{\sqrt{a^{2}+b^{2}}}=u
\]
并简记 $\sqrt{a^{2}+b^{2}}=g$, 我们得到方程
\[
u^{2}+\frac{A t}{g}+\frac{B}{g^{2}}+\frac{C}{g^{3} t}+\frac{D}{g^{4} t^{2}}+\frac{E}{g^{5} t^{3}}+\cdots=0
\]
在我们讨论的情况下有 $A=0, B=0$, 从而
\[
u^{2}+\frac{C}{g^{3} t}+\frac{D}{g^{4} t^{2}}+\frac{E}{g^{5} t^{3}}+\cdots=0
\]
如果 $C$ 不等于零, 那么当 $t$ 为无穷时, 与项 $\frac{C}{g^{3} t}$ 相比较, 项 $\frac{D}{g^{4} t^{2}}+\frac{E}{g^{5} t^{3}}+\cdots$ 都可略去, 我们 得到
\[
u^{2}+\frac{C}{g^{3} t}=0
\]
该方程表示的曲线, $t=\infty$ 时与所求曲线相合, 从该方程得 $u=\pm \sqrt{\frac{-C}{g^{3} t}}$, 由此知曲线有两 个分支,从两面趋向于轴的同一部分.

\section{$\S 184$}

如果再加上 $C=0$, 则方程为
\[
u^{2}+\frac{D}{g^{4} t^{2}}=0
\]
这里又分成 $D$ 为正、为负和为零三种情形. $D$ 为正, 方程不成立, 曲线没有伸向无穷的分 支, 而是完全地位于一个有界的范围之内. $D$ 为负, 记为 $\frac{D}{g^{4}}=-f^{2}$, 则 $u^{2}=\frac{f^{2}}{t^{2}}$, 因而 $t=\infty$ 和 $t=-\infty$ 时纵标 $u$ 都取两个趋向于零的值, 一正一负. 这就是说, 曲线有四个分支, 从两 头两面收玫于轴. $D=0$, 应该取方程 $u^{2}+\frac{E}{g^{5} t^{3}}=0$. 这类似于前节讨论过的情形, 应对方程 $P+Q+R+S+\cdots=0$ 的更多的项进行考察. 

\section{$\S 185$}
现在假定方程
\[
P+Q+R+S+\cdots=0
\]
的最高次项部分 $P$ 有三个实线性因式. 如果这三个因式相异,那么前面关于单个实线性 因式所讲的, 就对这三个因式中的每一个都适用. 因而曲线有六条趋向无穷的分支,它们 收敛于三条渐近线. 如果三个因式中有两个相等, 那么刚才所说仍适用于这第三个不相 等的因式, 前面关于两个相等因式所讲的, 适用于这里的两个相等因式, 只剩下三个实线 性因式都相等的情形了, 设 $P=(a y-b x)^{3} M$,要方程
\[
P+Q+R+S+\cdots=0
\]
在无穷远处成立, 则 $(a y-b x)^{3}$ 必为有限值或小于无穷值 $\infty^{3}$, 从而 $P$ 所达到的无穷将小 于 $\infty^{n}$, 在无穷远处有 $\frac{y}{x}=\frac{b}{a}$.

\section{$\S 186$}

为讨论这一情形, 首先要看 $Q$, 看 $Q$ 是否以 $a y-b x$ 为因式. 这里要指出,如果 $Q$ 为 零, 则认为 $Q$ 以 $a y-b x$ 为因式, 因为零以任何因式为因式. 先假定 $a y-b x$ 不是 $Q$ 的因 式. 由 $Q, M$ 分别为 $n-1$ 和 $n-3$ 次函数, 知 $\frac{Q}{(a x+b y)^{2} M}$ 为零次函数. 因而置 $\frac{y}{x}=\frac{b}{a}$, 则 这零次函数为常数, 记为 $A$, 我们得到 $(a y-b x)^{3}+A(a x+b y)^{2}=0$, 后继项在无穷远处 与 $A(a x+b y)^{2}$ 相比较略去.

\section{$\S 187$}

该方程所表示的曲线,伸向无穷远时与方程
\[
P+Q+R+S+\cdots=0
\]
表示的曲线相合. 为了进一步了解这条曲线, 我们取新的横标和纵标为
\[
t=\frac{a x+b y}{g}, \quad u=\frac{a y-b x}{g}
\]
其中 $\sqrt{a^{2}+b^{2}}=g$, 在新坐标之下方程为
\[
u^{3}+\frac{A t^{2}}{g}=0
\]
$t=\infty$ 时该方程给出曲线
\[
P+Q+R+\cdots=0
\]
的伸向无穷的部分. 因而确定了曲线 $u^{3}+\frac{A t^{2}}{g}=0$ 的形状, 也就确定了曲线 $P+Q+$ $R+\cdots=0$ 伸向无穷部分的形状. 下一章我们专门讨论这种渐近曲线.

\section{$\S 188$}

如果 $a y-b x$ 是 $Q$ 的因式, 这又分为 $(a y-b x)^{2}$ 是否同时也为 $Q$ 的因式两种情形. 假定 $(a y-b x)^{2}$ 不是 $Q$ 的因式, 记 $\frac{y}{x}=\frac{b}{a}$ 时零次函数 $\frac{Q}{(a y-b x)(a x+b y) M}$ 的值为常数 $A$, 我们得到
\[
(a y-b x)^{3}+A(a y-b x)(a x+b y)+\frac{R}{M}+\frac{S}{M}+\cdots=0
\]
这里置 $\frac{y}{x}=\frac{b}{a}$, 则 $R$ 被 $a y-b x$ 除得尽时, $\frac{R}{M}$ 成为 $B(a y-b x), R$ 不被 $a y-b x$ 除得尽时, $\frac{R}{M}$ 成为 $B(a x+b y)$, 不管 $R$ 是否被 $a y-b x$ 除得尽, $\frac{S}{M}$ 都成为常数 $C$. 如果照前面那样变 换成新坐标 $t$ 和 $u$, 则方程成为
\[
u^{3}+\frac{A t u}{g}+\frac{B u}{g^{2}}+\frac{C}{g^{3}}=0
\]
或
\[
u^{3}+\frac{A t u}{g}+\frac{B t}{g^{2}}+\frac{C}{g^{3}}=0
\]
由于这里只考虑 $t=\infty$ 的情形, 所以最后一项都略去, 从而前一方程成为
\[
u^{3}+\frac{A t u}{g}+\frac{B u}{g^{2}}=0
\]
给出两条渐近线 $u=0$ 和 $u^{2}+\frac{A t}{g}=0$, 前一条为直线, 后一条为抛物线, 后一方程 $t=\infty$ 时, 如果 $u$ 有限, 则由有限与无穷相比较可略去, 得 $\frac{A t u}{g}+\frac{B t}{g^{2}}=0$, 从而 $u=\frac{-B}{A g}$, 这是一条直 线; 如果 $u$ 为无穷,则由第三项可略去, 得
\[
u^{2}+\frac{A t}{g}=0
\]
这是抛物线方程, 这样两种情况下各自的两条渐近线, 都是一条为抛物线, 一条为直线, 因此这两种情况无需区分.

\section{$\S 189$}

现在设 $(a y-b x)^{2}$ 是 $Q$ 的因式, 那么对应 $a y-b x$ 是和不是 $R$ 的因式, 完成前面的运 算, 我们得到 $t, u$ 间的方程
\[
u^{3}+\frac{A u^{2}}{g}+\frac{B u}{g^{2}}+\frac{C}{g^{3}}=0, \quad u^{3}+\frac{A u^{2}}{g}+\frac{B t}{g^{2}}=0
\]
前一情形: 当方程
\[
u^{3}+\frac{A u^{2}}{g}+\frac{B u}{g^{2}}+\frac{C}{g^{3}}=0
\]
的三个根都为实数时, 是三条平行直线; 有两个根是虚数时, 是唯一的渐近直线. 指出一 点, 三条平行渐近线, 可以两条重合, 可以三条重合, 后一种情形
\[
u^{3}+\frac{A u^{2}}{g}+\frac{B t}{g^{2}}=0
\]
%%05p081-100
如果 $t=\infty$, 则 $u$ 也必为无穷方程才能成立. $u$ 无穷时, 与 $u^{3}$ 比较项 $\frac{A u^{2}}{g}$ 可以略去, 我们 得到
\[
u^{3}+\frac{B t}{g^{2}}=0
\]
这是三阶渐近曲线方程.

\section{$\S 190$}

如果 $A=0, B=0, C=0$, 则应考虑方程 $P+Q+R+S+\cdots=0$ 的后续项, 它们给出 方程
\[
u^{3}+\frac{D}{g^{4} t}+\frac{E}{g^{5} t^{2}}+\frac{F}{g^{6} t^{3}}+\cdots=0
\]
只要这里的 $D$ 不等于零, 第三及其以后的项就都可以略去, 得 $u^{3}+\frac{D}{g^{4} t}=0 . D=0$, 得 $u^{3}+$ $\frac{E}{g^{5} t^{2}}=0$, 加上 $E=0$, 得 $u^{3}+\frac{F}{g^{6} t^{3}}=0$, 类推, 这些方程确定的曲线, 当 $t=\infty$ 时都与方程 $P+Q+R+S+\cdots=0$ 确定的曲线重合. 又这些方程都含有奇次幂 $u^{3}$, 因而都有实根, 都 有伸向无穷的分支, 在每种情况下直线 $u=0$ 都是渐近线, 因为曲线
\[
u^{3}+\frac{D}{g^{4} t}=0, \quad u^{3}+\frac{D}{g^{5} t^{2}}=0, \quad \cdots
\]
都以它为渐近线.

\section{$\S 191$}

同是收玫于直线的分支, 可以极为不同, 对它们的区别应细加考察. 我们通过确定一 些简单曲线来进行考察, 这些简单曲线都收玫于原曲线的渐近线. 例如, 当方程
\[
u^{3}+\frac{A u^{2}}{g}+\frac{B u}{g^{2}}+\frac{C}{g^{3}}=0
\]
的三个根都为实数时, 虽然我们知道有三条平行渐近线, 但是关于曲线伸向无穷的分支, 它究竟是以 $u=\frac{C}{t}$ 为方程的双曲线, 还是别的形状的, 比方说以 $u=\frac{C}{t^{2}}$ 或 $u=\frac{C}{t^{3}}$ 为方程的 曲线, 我们还完全不清楚. 为了弄清这一点, 取方程的下一项 $\frac{D}{g^{4} t}$, 它为零, 则取再下一项 $\frac{E}{g^{5} t^{2}}$, 它也为零, 则继续往下取 $\frac{F}{g^{6} t^{3}}$. 设最近的不为零的后继项为 $\frac{K}{t^{k}}$, 由方程
\[
P+Q+R+\cdots=0
\]
的次数为 $n$, 知 $k$ 不大于 $n-3$, 设表达式
\[
u^{3}+\frac{A u^{2}}{g}+\frac{B u}{g^{2}}+\frac{C}{g^{3}}
\]
分解成因式为 $(u-\alpha)(u-\beta)(u-\gamma)$, 则
\[
(u-\alpha)(u-\beta)(u-\gamma)-\frac{K}{t^{k}}=0
\]
令 $u-\alpha=\frac{I}{t^{u}}$, 它表示一条渐近线,我们有
\[
\frac{I}{t^{\mu}}\left(\alpha-\beta+\frac{1}{t^{\mu}}\right)\left(\alpha-\gamma+\frac{1}{t^{\mu}}\right)=\frac{K}{t^{\mu}}
\]
$t$ 趋向无穷,得
\[
\frac{(\alpha-\beta)(\alpha-\gamma) I}{t^{\mu}}=\frac{K}{t^{k}}
\]
\section{$\S 192$}

如果 $\alpha$ 既不等于 $\beta$ 也不等于 $\gamma$, 则该方程成立, 此时我们有
\[
I=\frac{K}{(\alpha-\beta)(\alpha-\gamma)}, \quad \mu=k
\]
根 $u=\alpha$ 给出渐近曲线
\[
u-\alpha=\frac{K}{(\alpha-\beta)(\alpha-\gamma) t^{k}}
\]
如果三个根都不相等, 则每个根给出一条这样的渐近线. 如果有两个根相等, 比如 $\beta=\alpha$, 则这两条渐近线合而为一, 此时我们有
\[
\frac{I^{2}(\alpha-\gamma)}{t^{2} \mu}=\frac{K}{t^{k}}
\]
由此得
\[
I^{2}=\frac{K}{\alpha-\gamma}, \quad 2 \mu=k
\]
因而双重渐近线的方程为
\[
(u-\alpha)^{2}=\frac{K}{(\alpha-\gamma) t^{k}}
\]
如果三个根都相等, 则三条渐近线合为一条,其方程为
\[
(u-\alpha)^{3}=\frac{K}{t^{k}}
\]
\section{$\S 193$}

考虑方程 $P+Q+R+S+\cdots=0$ 的最高次项部分 $P$ 有四个实线性因式的情形, 它们 或者都不相等, 或者有两个相等, 或者有三个相等, 或者四个都相等, 前三种情况下伸向 无穷的分支及其渐近线都可由前节得知, 唯一需要讨论的情况是四个都相等. 此时我们 令 $P=(a y-b x)^{4} M, M$ 为 $n-4$ 次函数, 照我们做过的那样, 为得到常数, 令 $\frac{y}{x}=\frac{b}{a}$, 为改变轴,令
\[
t=\frac{a x+b y}{g}, \quad u=\frac{a y-b x}{g}
\]
其中 $g=\sqrt{a^{2}+b^{2}}$, 这样得到渐近线 $t, u$ 间方程. 首先, 如果 $Q$ 不被 $a y-b x$ 整除, 得到 的是
\[
u^{4}+\frac{A t^{3}}{g}=0
\]
\section{$\S 194$}

如果 $Q$ 被 $a y-b x$ 整除, 但不被 $(a y-b x)^{2}$ 整除,得到的是
\[
u^{4}+\frac{A t^{2} u}{g}+\frac{B t^{2}}{g^{2}}=0
\]
$t=\infty$ 时, 纵标 $u$ 可以有限可以无限, 得到的两条渐近线为直线 $u+\frac{B}{g A}=0$ 和曲线 $u^{3}+$ $\frac{A t^{2}}{g}=0$. 为进一步考察渐近直线, 我们取最靠近的非零项, 设为 $\frac{K}{t^{k}}$, 得方程
\[
u+\frac{B}{g A}+\frac{g K}{A t^{k+2}}=0
\]
横标 $t=\infty$ 时, 该方程表示的曲线与所求曲线相合.

\section{$\S 195$}

设 $Q$ 被 $(a y-b x)^{2}$ 整除, 但不被 $(a y-b x)^{2}$ 整除. 对 $R$ 被和不被 $a y-b x$ 整除两种情 形分别进行讨论. $R$ 被 $a y-b x$ 整除,得
\[
u^{4}+\frac{A t u^{2}}{g}+\frac{B t u}{g^{2}}+\frac{C t}{g^{3}}=0
\]
$R$ 不被 $a y-b x$ 整除,得
\[
u^{4}+\frac{A t u^{2}}{g}+\frac{B t^{2}}{g^{2}}+\frac{C t}{g^{3}}=0
\]
从第一个方程, $u$ 有穷得
\[
u^{2}+\frac{B u}{g A}+\frac{C}{g^{2} A}=0
\]
$u$ 无穷得
\[
u^{2}+\frac{A t}{g}=0
\]
这前一个方程, 如果它的两个根为实数, 且不相等, 则得到的是两条平行直线; 如果它的 两个根为虚数, 则没有伸向无穷的分支. 这后一个方程 $u^{2}+\frac{A t}{g}=0$ 给出的当然是渐近抛 物线. 

第二个方程
\[
u^{4}+\frac{A t u^{2}}{g}+\frac{B t^{2}}{g^{2}}=0
\]
( $t=\infty$ 时, 与 $\frac{B t^{2}}{g^{2}}$ 比较, $\frac{C t}{g^{3}}$ 略去) 包含两个状如 $u^{2}+a t=0$ 的方程, $A^{2}>A B$ 时, 得两条渐 近抛物线; $A^{2}=4 B$ 时, 这两条渐近抛物线重合为一条; $A^{2}<4 B$ 时, 为虚根, 没有伸向无 穷的分支.

\section{$\S 196$}

如果 $Q$ 被 $(a y-b x)^{3}$ 整除,那么由 $R$ 和 $S$ 被和不被 $a y-b x$ 整除,得方程
\[
\begin{gathered}
u^{4}+\frac{A u^{3}}{g}+\frac{B u^{2}}{g^{2}}+\frac{C u}{g^{3}}+\frac{D}{g^{4}}=0 \\
u^{4}+\frac{A u^{3}}{g}+\frac{B u^{2}}{g^{2}}+\frac{C t}{g^{3}}=0 \\
u^{4}+\frac{A u^{3}}{g}+\frac{B u t}{g^{2}}+\frac{C t}{g^{3}}=0 \\
u^{4}+\frac{A u^{3}}{g}+\frac{B t^{2}}{g^{2}}=0
\end{gathered}
\]
第一个方程, 如果根都为实数, 且不相同, 则它表示四条相平行的直线, 如果有两个或更 多个根相等, 则有同样条数的直线重合为一条, 如果有两个或四个虚根,则平行直线减少 两条或不存在. 在第二个方程中, $t=\infty$ 时, 纵标 $u$ 为无穷, 方程成为 $u^{4}+\frac{C t}{g^{3}}=0$, 这是一条 四阶渐近曲线, 从第三个方程可以得到有限值, 我们有 $u+\frac{c}{g B}=0$. 我们还有 $u^{3}+\frac{B t}{g^{2}}=0$, 这是三阶渐近线. 最后, 第四个方程, 当 $t=\infty$ 时 $u$ 无穷, 它变成 $u^{4}+\frac{B t}{g^{2}}=0$. 该方程 $B$ 为正 时不可能, $B$ 为负时给出两条共顶点方向相反的抛物线, 它们都在无穷远处与原曲线 重合.

\section{$\S 197$}

从以上所讲已经清楚, 当最高次部分 $P$ 含有更多相等因式时讨论应如何进行. 至于 不相等因式, 它们可以分别对待, 每个确定一条渐近直线. 有两个因式相等时, 照 $\S 178$ 和接下去几节中讲的那样处理. 类似地, 有三个因式相等时, 按 $\$ 185$ 和接下去几节讲的 办. 四个因式相等的情况我们刚讨论过, 从这些讨论我们可以去处理有更多个因式相等 的情形. 我们已经看到了曲线伸向无穷分支的多样性, 对局限于有限范围中的曲线, 其多 样性我们末触及. 

