\chapter{第十七章 依据其他性质求曲线}

\section{$\S 391$}
前章依纵标性质求曲线的直角或斜角坐标间方程. 本章求曲线方程, 不直接依据平 行纵标, 而是依据其他性质. 例如, 依据从一点向曲线所引直线的性质, 参见图 81. 设这 一点为 $C$, 从它向曲线引直线 $C M$ 和 $C A$, 使这两条直线具有某种性质, 我们就依这两条 直线的性质来求曲线的方程.


【图,待补】
%%![](https://cdn.mathpix.com/cropped/2023_02_05_94c61b3fb31a70215b66g-13.jpg?height=330&width=584&top_left_y=919&top_left_x=548)

图 81

\section{$\S 392$}

用二元方程表示曲线的方法有很多种. 这里我们取从给定点 $C$ 到曲线的直线 $C M$ 的 长作为一个变元, 当然另一个变元应能确定直线 $C M$ 的位置. 为此我们取过点 $C$ 的一条 直线 $C A$ 作轴, 取 $\angle A C M$ 或这个角的某个函数作另一个变元. 记直线 $C M=z$, 记 $\angle A C M=\varphi$, 使方程含有 $\varphi$ 的正弦或正切. 显然, 如果有了 $z$ 与 $\sin \varphi$ 或 $\tan \varphi$ 之间的方程, 曲线 $A M N$ 也就完全确定, 因为对任何一个 $\angle A C M$ 我们都有直线 $C M$ 的长, 因而也就决定曲线上 一点.

\section{$\S 393$}

我们对曲线的这一表示方法做更详细些的考察. 先设长度 $z$ 等于 $\varphi$ 的正弦的某个函 数. 如果这个函数是单值的, 则指明直线 $C M$ 与曲线只交于一点, 因为 $\angle A C M=\varphi$ 对应直 线 $C M$ 的唯一的值. 但是如果 $\varphi$ 增加两个直角, 则过点 $C$ 的直线 $C M$ 位置不变, 只是方向 相反. 这样, 虽然 $z$ 是 $\varphi$ 的正弦的单值函数, 但我们得到了直线 $C M$ 与曲线的另一个交点. 假设 $P$ 是 $\varphi$ 的正弦的这样的函数, 即 $z=P$, 并由它得到了曲线上的点 $M$, 如图 82 所示. 现在我们使 $\varphi$ 增加两个直角, 也即这个角的正弦变负, 从而 $P$ 变为 $Q$, 得 $z=Q$. 取 $C m=Q$, 得 $C M$ 的延长线与曲线的另一个交点 $m$.


【图,待补】
%%![](https://cdn.mathpix.com/cropped/2023_02_05_94c61b3fb31a70215b66g-14.jpg?height=227&width=486&top_left_y=432&top_left_x=585)

图 82

\section{$\S 394$}

这样, 虽然 $P$ 是 $\varphi$ 的正弦的单值函数, 但是只要 $Q \neq-P$, 过点 $C, \angle A C M=\varphi$ 的直线 $C M$ 就交曲线于 $M$ 和 $m$ 两点. 因而, 如果每条直线 $C M$ 与曲线都只交于一点, 则函数 $P$ 必 为 $\sin \varphi$ 的奇函数, 为 $\cos \varphi$ 的奇函数也可以, 也即, $P$ 为 $\angle A C M=\varphi$ 的正弦或余弦的奇函 数时, 曲线 $z=P$ 与过点 $C$ 的每条直线都只交于一点.

\section{$\S 395$}

与过点 $C$ 的直线都只相交于一点的曲线, 其方程为 $z=P, P$ 为 $\varphi$ 的正弦和余弦的奇 函数, 也即这函数 $P$, 当 $\varphi$ 的正弦和余弦的值都反号时, 函数值也反号. 这种曲线的直角 坐标方程易于求出. 参见图 81 , 如果自点 $M$ 向轴 $C A$ 引垂线 $M P$, 又如果取 $C P=x, P M=$ $y$, 则 $\frac{y}{z}=\sin \varphi, \frac{x}{z}=\cos \varphi$, 因而, 如果 $P$ 是 $\frac{x}{z}$ 和 $\frac{y}{z}$ 的奇函数, 方程 $z=P$ 就包含满足要求 的所有曲线, 这种方程中最简单的是
\[
z=\frac{\alpha x}{z}+\frac{\beta y}{z}+\frac{\gamma z}{x}+\frac{\delta z}{y}
\]
更高次数的为
\[
z=\frac{\alpha x}{z}+\frac{\beta y}{z}+\frac{\gamma z}{x}+\frac{\delta z}{y}+\frac{\varepsilon x^{3}}{z^{3}}+\frac{\xi x^{2} y}{z^{3}}+\frac{p x y^{2}}{z^{3}}+\frac{\theta y^{3}}{z^{3}}+\frac{e x^{2}}{y z}+\frac{k y^{2}}{x z}+\frac{\lambda y z}{x^{2}}+\cdots
\]
\section{$\S 396$}

以 $z$ 除该方程, 则方程中 $z$ 的次数全为偶数, 而 $z=\sqrt{x^{2}+y^{2}}$, 因而换 $z^{2}$ 为 $x^{2}+y^{2}$, 消 去 $z$ 得到的是 $x, y$ 间的有理方程. 这有理方程是 $x, y$ 的 $-1$ 次函数, 等于 1 或等于一个常 数 $C$. 设 $P$ 为这样的函数, 即 $c=P$, 从而 $\frac{1}{c}=\frac{1}{p}$, 但 $\frac{1}{p}$ 为 $x, y$ 的一次函数. 这样我们得到, 如果 $x, y$ 的一个一次函数等于常数, 那么这个方程表示的曲线与过点 $C$ 的直线就只有一 个交点.

\section{$\S 397$}

设 $P$ 和 $Q$ 分别为 $x, y$ 的 $n$ 次和 $n+1$ 次函数, 则 $\frac{Q}{P}$ 为一次函数. 这样, 我们这里讨论的 曲线就都含于方程 $\frac{Q}{P}=c$ 或 $Q=c P$ 之中. 也即我们的曲线, 其通用方程为
\[
\begin{aligned}
& \alpha x^{n+1}+\beta x^{n} y+\gamma x^{n-1} y^{2}+\delta x^{n-2} y^{3}+\varepsilon x^{n-3} y^{4}+\cdots \\
= & c\left(A x^{n}+B x^{n-1} y+C x^{n-2} y^{2}+D x^{n-3} y^{3}+\cdots\right)
\end{aligned}
\]
$n$ 为任何数. 由此得与过点 $C$ 的直线只有一个交点的各阶线, 其方程依次为

$\mathrm{I}$

$\alpha x+\beta y=c$

II

$\alpha x^{2}+\beta x y+\gamma y^{2}=c(A x+B y)$

III

$\alpha x^{3}+\beta x^{2} y+\gamma x y^{2}+\delta y^{3}=c\left(A x^{2}+B x y+C y^{2}\right)$

IV

$\alpha x^{4}+\beta x^{3} y+\gamma x^{2} y^{2}+\delta x y^{3}+\varepsilon y^{4}=c\left(A x^{3}+B x^{2} y+C x y^{2}+D y^{2}\right)$

等.

\section{$\S 398$}

I 是直线通用方程, 直线与过给定点的直线最多都只有一个交点. II 是圆雉曲线通 用方程. 圆雉曲线通过点 $C$ 时, 这点 $C$ 是通过点 $C$ 的所有直线与圆雉曲线的共同交点, 因 而不作为交点计算. 圆雉曲线与直线的交点不多于两个, 所以过圆雉曲线上点 $C$ 的直线 与圆曲线的交点不多于一个. 类似地, 过点 $C$ 的更高阶曲线与过点 $C$ 的所有直线也都以 点 $C$ 为共同交点, 我们也不把它作为交点计算, 因而上节方程所表示的曲线与过自身的 点 $C$ 的直线的交点都不多于一个. 这样我们就列出了与过点 $C$ 的直线的交点不多于一个 的所有代数曲线的方程.

\section{$\S 399$}

下面我们考虑与过点 $C$ 的直线或者有两个交点或者没有交点的曲线, 没有交点时, 决定交点的方程的根为虚数. 由于对任何 $\angle A C M=\varphi$ 直线 $C M=z$ 都应该有两个值, 所以 它由二次方程确定. 设
\[
z^{2}-P z+Q=0
\]
$P, Q$ 是 $\varphi$, 或者 $\varphi$ 的正弦或余弦的函数. 直线 $C M$ 与曲线应该只相交于两点, 为 $M$ 和 $N$, 所以不只 $P, Q$ 应该是 $\varphi$ 的单值函数, 并且 $\varphi$ 增加两个直角时也不应该得到任何新的交点. 事情将是这样的, 如果 $P$ 是 $\varphi$ 的正弦和余弦的奇函数, 也即让正弦和余弦取负值, 则 $P$ 也 取负值,且此时 $Q$ 应该是同一正弦和余弦的偶函数.

\section{$\S 400$}

记直角坐标 $C P=x, P M=y$, 则
\[
\frac{y}{z}=\sin \varphi, \quad \frac{x}{z}=\cos \varphi
\]
因而 $P$ 是 $\frac{x}{z}$ 和 $\frac{y}{z}$ 的奇函数, $Q$ 是 $\frac{x}{z}$ 和 $\frac{y}{z}$ 的偶函数. 由此得 $\frac{P}{z}$ 是 $x, y$ 的有理函数, 是 $-1$ 次 齐次函数. 类似地, $\frac{Q}{Z^{2}}$ 是 $x, y$ 的有理函数, 是 $-2$ 次齐次函数. 因而,如果 $L, M, N$ 依次是 $x, y$ 的 $n+2$ 次, $n+1$ 次和 $n$ 次齐次函数, 则分数 $\frac{M}{L}$ 和 $\frac{N}{L}$ 分别可代替 $\frac{P}{z}$ 和 $\frac{Q}{z^{2}}$. 由 $z^{2}-P z+$ $Q=0$ 得 $1-\frac{P}{z}+\frac{Q}{z^{2}}=0$, 从而与过点 $C$ 的直线交于两点的直线,其通用方程为
\[
1-\frac{M}{L}+\frac{N}{L}=0 \text { 或 } L-M+N=0
\]
其中
\[
P=\frac{M z}{L}, \quad Q=\frac{N z^{2}}{L}=\frac{N\left(x^{2}+y^{2}\right)}{L}
\]
由 $z=\sqrt{x^{2}+y^{2}}$ 知, $P$ 是 $x, y$ 的无理函数, $Q$ 是零次有理函数.

\section{$\S 401$}

现在容易求出与过给定点 $C$ 的直线有两个交点或没有交点的任何阶线. 例如,求二 阶线, 此时应置 $n=0$, 得圆雉曲线的通用方程
\[
\alpha x^{2}+\beta x y+\gamma y^{2}-\delta x-\varepsilon y+\zeta=0
\]
因而过任何一点 $C$ 的任何一条直线与圆雉曲线都或者有两个交点, 或者没有交点. 然而 有这样的情形, 某条直线与这曲线只交于一点, 但在过点 $C$ 的无穷多条直线中这样的直 线只有一条或两条, 所以这种例外无关紧要. 也可以把这种例外解释为另一个交点在无 穷远处,这样这例外就完全不影响我们结论的普遍性.

\section{$\S 402$}

为了弄清这种例外究竟什么时候发生, 我们化 $x$ 和 $y$ 间的方程为 $z$ 和 $\angle A C M=\varphi$ 间 方程. 由于 $y=z \sin \varphi$ 和 $x=z \cos \varphi, x$ 和 $y$ 间的方程化为
\[
z^{2}\left(\alpha \cos ^{2} \varphi+\beta \sin \varphi \cos \varphi+\gamma \sin ^{2} \varphi\right)-z(\delta \cos \varphi+\varepsilon \sin \varphi)+\zeta=0
\]
由此可见, $z^{2}$ 的系数等于零时就只有一个交点.
\[
\alpha+\beta \tan \varphi+\gamma \tan ^{2} \varphi=0
\]
时 $z^{2}$ 的系数为零. 如果该方程有两个实根, 则这曲线与两条过点 $C$ 的直线只有一个交点. 但该方程的根决定我们曲线的渐近线. 可见, 双曲线与平行于渐近线的直线只交于一点, 过点 $C$ 的这样的直线只有两条. 在抛物线时只有一根平行于其轴的直线是这种例外. 但 是如果圆雉曲线是椭圆, 那么过任何一点的任何一条直线就都或者与椭圆不相交, 或者 只交于一点.

\section{$\S 403$}

置 $n=1$, 得满足要求的三阶线的通用方程
\[
\alpha x^{3}+\beta x^{2} y+\gamma x y^{2}+\delta y^{3}+\varepsilon x^{2}+\zeta x y-p y^{2}+\theta x+\iota y=0
\]
该方程包含的每一条三阶线, 只要点 $C$ 在它上面, 它就满足要求. 如果 $x=0$, 则 $y=0$. 类 似地, 满足要求的四阶线, 点 $C$ 不仅要在四阶线上, 还必须是四阶线的二重点. 也即, 有二 重点的四阶线, 取二重点为 $C$ 时就满足要求. 但是如果点 $C$ 是四阶线的三重点, 那么过它 的直线与四阶线的交点就只有一个. 这属于开始时考虑过的那类曲线. 同样地, 如果 $C$ 是 三重点, 五阶线也可满足要求, 类推. 但要记住, 如果过点 $C$ 的直线平行于渐近直线, 或平 行于抛物渐近线的轴, 则都将只有一个交点, 另一个交点在无穷远处.

\section{$\S 404$}

各阶线与直线的交点的个数 (包括虚交点和无穷远交点) 都等于阶数. 我们前面所 讲跟这条性质是相符合的. $n$ 阶线与过点 $C$ 的直线的交点个数, 包括实交点、虚交点、无穷 远交点和点 $C$ 本身, 总数也是 $n$. 因而 $n$ 阶线与过点 $C$ 的直线在点 $C$ 之外的交点个数为 2 , 则点 $C$ 必为 $n-2$ 重点.

\section{$\S 405$}

有了以上结果, 对于 $z$ 的两个值 $C M, C N$ 之间的关系问题, 我们就容易或者给出解 答, 或者证明给不出解答. 由于 $z$ 的两个值 $C M$ 和 $C N$ 是方程 $z^{2}-P z+Q=0$ 的两个根, 所 以它们的和 $C M+C N=P$, 积 $C M \cdot C N=Q$. 首先, 我们求和 $C M+C N$ 为常数的曲线, 此 时函数 $P$ 应该为常数. 从与过点 $C$ 的直线交于两点的曲线的讨论我们有
\[
P=\frac{M z}{L}=\frac{M \sqrt{x^{2}+y^{2}}}{L}
\]
$(\S 400)$. 该表达式中含有无理量, 不能是常数, 因而在我们的曲线中没有 $C M+C N$ 恒为 常数的. 

\section{$\S 406$}

但是如果把与过点 $C$ 的直线只交于两点的曲线, 换成与过点 $C$ 的直线的交点多于两 个的曲线, 要求这交点中恒有满足 $C M+C N$ 为常数的 $M$ 和 $N$ 这样两个点. 这样的曲线有 无穷多, 令 $P$ 等于所说的常数 $C M+C N=a$, 则 $z^{2}-a z+Q=0, Q$ 为函数 $\frac{N z^{2}}{L}$. 该方程中含 有无理量,有理化,得
\[
a^{2} z^{2}=\left(z^{2}+Q\right)^{2} \text { 或 } a^{2}=z^{2}\left(1+\frac{N}{L}\right)^{2}
\]
或
\[
a^{2} L^{2}=\left(x^{2}+y^{2}\right)\left(L^{2}+2 L N+N^{2}\right)
\]
其中 $L$ 为 $x$ 和 $y$ 的 $n+2$ 次齐次函数, $N$ 为 $x$ 和 $y$ 的 $n$ 次齐次函数,令
\[
L=x^{2}+y^{2}, \quad N=\pm b^{2}
\]
代入得
\[
a^{2}\left(x^{2}+y^{2}\right)=\left(x^{2}+y^{2} \pm b^{2}\right)^{2}
\]
它表示的是满足要求的最简单的曲线, 是以点 $C$ 为圆心的两个同心圆, 是四阶复合线, 令
\[
L=\alpha x^{2}+\beta x y+\gamma y^{2}, \quad N=\pm b^{2}
\]
得
\[
a^{2}\left(\alpha x^{2}+\beta x y+\gamma y^{2}\right)^{2}=\left(x^{2}+y^{2}\right)\left(\alpha x^{2}+\beta x y+\gamma y^{2} \pm b^{2}\right)^{2}
\]
它表示的是六阶线, 是满足要求的最简单的连续线. 令 $\alpha=1, \beta=0, \gamma=0$, 得
\[
y^{2}+x^{2}=\frac{a^{2} x^{4}}{x^{4} \pm 2 b^{2} x^{2}+b^{4}}
\]
或
\[
y=\frac{x \sqrt{a^{2} x^{2}-x^{4}+2 b^{2} x^{2}-b^{4}}}{x^{2} \pm b^{2}}
\]
\section{$\S 407$}

如果从满足要求的曲线中,把与过点 $C$ 的直线的交点多于两个的曲线去掉,那就没 有满足要求的曲线. 也即没有这样的连续曲线, 过点 $C$ 的直线与它只相交于两点 $M$ 和 $N$, 且 $C M+C N$ 为常数. 但是, 如果我们改为要求 $C M \cdot C N$ 为常数,那么一个显然的解是圆, 圆心位置任意. 具有这种性质的曲线有无穷多. 这时的 $Q$ 应该为常数, 它应该等于 $C M$. $C N$, 记为 $a^{2} \cdot Q=\frac{N z^{2}}{L}$ 是 $x$ 和 $y$ 的有理函数,不矛盾.

\section{$\S 408$}

令 $ \frac{N z^{2}}{L}=a^{2} $或 $ L=\frac{N z^{2}}{a^{2}}=\frac{N\left(x^{2}+y^{2}\right)}{a^{2}}, $ 则满足要求的曲线的方程为 
\[
\frac{N\left(x^{2}+y^{2}\right)}{a^{2}}-M+N=0 \text { 或 } M a^{2}=N\left(x^{2}+y^{2}+a^{2}\right)
\]
其中 $M$ 和 $N$ 都是 $x$ 和 $y$ 的齐次函数, 次数为 $n+1$ 和 $n$. 由此得
\[
\frac{M}{N}=\frac{x^{2}+y^{2}+a^{2}}{a^{2}}
\]
为 $x$ 和 $y$ 的一次函数. 这是与过点 $C$ 的直线只相交于两点 $M$ 和 $N$, 且乘积 $C M \cdot C N$ 恒为 常数 $a^{2}$ 这样的曲线的通用方程.

\section{$\S 409$}

由于 $\frac{M}{N}$ 是 $x$ 和 $y$ 的一次齐次函数, 所以最简情况是 $\frac{M}{L}=\frac{\alpha x+\beta y}{a}$, 由此得方程
\[
x^{2}+y^{2}-a(\alpha x+\beta y)+a^{2}=0
\]
这是圆的方程, 是直角坐标下圆的通用方程. 显然对任何一点 $C$ 它都满足要求. 这是从欧 几里得的《原本》中, 也即从初等几何中已经知道了的. 圆以外的圆雉曲线都不满足要 求. 阶数更高的各阶曲线中都有满足这种要求的曲线. 因而这样的曲线有无穷多条. 满足 这种要求的三阶线的通用方程为
\[
\frac{\alpha x^{2}+\beta x y+\gamma y^{2}}{a(\delta x+\varepsilon y)}=\frac{x^{2}+y^{2}+a^{2}}{a^{2}}
\]
或
\[
(\delta x+\varepsilon y)\left(x^{2}+y^{2}\right)-a\left(\alpha x^{2}+\beta x y+\gamma y^{2}\right)+a^{2}(\delta x+\varepsilon y)=0
\]
类似地,阶数更高的各阶曲线中都有满足这种要求的曲线.

\section{$\S 410$}

现在我们求与过点 $C$ 的直线相交于两点, 且这两点到 $C$ 的距离的平方和为常数, 即 $C M^{2}+C N^{2}=2 a^{2}$ 的曲线. 由 $C M+C N=P$ 和 $C M \cdot C N=Q$, 得 $C M^{2}+C N^{2}=P^{2}-2 Q$, 因 而应该有
\[
P^{2}-2 Q=2 a^{2} \text { 或 } Q=\frac{P^{2}-2 a^{2}}{2}
\]
由 $P=\frac{M z}{L}, Q=\frac{N z^{2}}{L}$, 我们有 $\frac{2 N z^{2}}{L}=\frac{M^{2} z^{2}}{L^{2}}-2 a^{2}$. 从而 $N=\frac{M^{2}}{2 L}-\frac{a^{2} L}{z^{2}}$, 由于 $L, M$ 和 $N$ 依次 为 $x$ 和 $y$ 的 $n+2, n+1$ 和 $n$ 次函数,该方程是没有问题的. 取 $L$ 和 $M$ 为这样的函数时得 
\[
\begin{aligned}
& \text { शिख } \\
& \text { Finfinile analljisis (无穷分析引论). Fulraduclicun } \\
& N=\frac{M^{2}}{2 L}-\frac{a^{2} L}{z^{2}}
\end{aligned}
\]
由此得所求曲线的通用方程为
\[
L-M+\frac{M^{2}}{2 L}-\frac{a^{2} L}{z^{2}}=0
\]
或
\[
2 L^{2}\left(x^{2}+y^{2}\right)-2 L M\left(x^{2}+y^{2}\right)+M^{2}\left(x^{2}+y^{2}\right)-2 a^{2} L^{2}=0
\]
$M=0$ 时该方程给出的是以点 $C$ 为圆心的圆. 显然它满足我们的要求.

\section{$\S 411$}

$n+1=0$ 时, $M$ 为常数, 记为 $2 b ; L$ 为一次函数, 记为 $L=\alpha x+\beta y$. 这样我们得到四阶 方程
\[
(\alpha x+\beta y)^{2}\left(x^{2}+y^{2}-a^{2}\right)-2 b(\alpha x+\beta y)\left(x^{2}+y^{2}\right)+2 b^{2}\left(x^{2}+y^{2}\right)=0
\]
令
\[
L=x^{2}+y^{2}, \quad M=2(\alpha x+\beta y) a
\]
代入通用方程, 再除以 $2 x^{2}+2 y^{2}$, 得四阶线的另一个方程
\[
\left(x^{2}+y^{2}\right)^{2}-2 a(\alpha x+\beta y)\left(x^{2}+y^{2}\right)+2 a^{2}(\alpha x+\beta y)^{2}-a^{2}\left(x^{2}+y^{2}\right)=0
\]
只要该方程不被 $x^{2}+y^{2}$ 除得尽,换 $M$ 为 $2 M$, 它化为
\[
L^{2}\left(x^{2}+y^{2}\right)-2 L M\left(x^{2}+y^{2}\right)+2 M^{2}\left(x^{2}+y^{2}\right)-a^{2} L^{2}=0
\]
这是 $2 n+6$ 次方程. 也即对任何偶数阶我们都得到了满足要求的曲线的方程. 又, 如果 $L$ 被 $x^{2}+y^{2}$ 除得尽, 即 $L=\left(x^{2}+y^{2}\right) N$, 其中 $N$ 为 $x$ 和 $y$ 的任何 $n$ 次齐次函数, 我们得到另 一个通用方程
\[
N^{2}\left(x^{2}+y^{2}\right)^{2}-2 M N\left(x^{2}+y^{2}\right)+2 M^{2}-a^{2} N^{2}\left(x^{2}+y^{2}\right)=0
\]
阶数为 $2 n+4$. 这样对每个为偶数的阶, 我们都得到两个方程, 它们表示的曲线都满足要 求, 对阶数 6 这两个方程为
\[
\begin{aligned}
& \left(\alpha x^{2}+\beta x y+\gamma y^{2}\right)^{2}\left(x^{2}+y^{2}-a^{2}\right)-2 a(\delta x+\varepsilon y)\left(x^{2}+\right. \\
& \left.y^{2}\right)\left[\alpha x^{2}+\beta x y+\gamma y^{2}-a(\delta x+\varepsilon y)\right]=0
\end{aligned}
\]
和
\[
\begin{aligned}
& (\delta x+\varepsilon y)^{2}\left(x^{2}+y^{2}\right)\left(x^{2}+y^{2}-a^{2}\right) \\
= & 2 a\left(\alpha x^{2}+\beta x y+\gamma y^{2}\right)\left((\delta x+\varepsilon y)\left(x^{2}+y^{2}\right)-a\left(\alpha x^{2}+\beta y^{2}+\gamma y^{2}\right)\right)
\end{aligned}
\]
奇阶线都不满足这里的要求.

\section{$\S 412$}

把平方和 $C M^{2}+C N^{2}$ 为常数改为
\[
C M^{2}+C M \cdot C N+C N^{2}
\]
或更一般地

%%10p181-200
 $C M^{2}+n \cdot C M \cdot C N+C N^{2}$

为常数,曲线的求法类似,由于
\[
C M^{2}+n C M \cdot C N+C N^{2}=P^{2}+(n-2) Q
\]
我们令 $P^{2}+(n-2) Q=a^{2}$, 得 $Q=\frac{a^{2}-P^{2}}{n-2}$, 该方程并不带来什么困难. 由
\[
P=\frac{M z}{L}, \quad Q=\frac{N z^{2}}{L}
\]
得
\[
\frac{M^{2} z^{2}}{L^{2}}+\frac{(n-2) N z^{2}}{L}=a^{2}
\]
解出 $N$ 得
\[
N=\frac{a^{2} L}{(n-2) z^{2}}-\frac{M^{2}}{(n-2) L}
\]
由方程为 $L-M+N=0$ 的曲线满足 $C M^{2}+n C M \cdot C N+C N^{2}$ 为常数, 得
\[
(n-2) L^{2} z^{2}-(n-2) L M z^{2}+a^{2} L^{2}-M^{2} z^{2}=0
\]
或者由 $z^{2}=x^{2}+y^{2}$ 得
\[
a^{2} L^{2}+\left(x^{2}+y^{2}\right)\left[(n-2) L^{2}-(n-2) L M-M^{2}\right]=0
\]
其中 $L$ 和 $M$ 都是 $x$ 和 $y$ 的函数, 次数分别为 $m+2$ 和 $m+1$. 令 $N$ 为任何 $m$ 次齐次函数, 并 令 $L=\left(x^{2}+y^{2}\right) N$, 得另一个通用方程
\[
a^{2}\left(x^{2}+y^{2}\right) N^{2}+(n-2)\left(x^{2}+y^{2}\right) N^{2}-(n-2)\left(x^{2}+y^{2}\right) M N-M^{2}=0
\]
\section{$\S 413$}

$n=2$ 时,条件成为 $(C M+C N)^{2}=a^{2}$, 方程为
\[
a^{2} L^{2}=\left(x^{2}+y^{2}\right) M^{2} \text { 或 } M^{2}=a^{2}\left(x^{2}+y^{2}\right) N^{2}
\]
这两个方程都是齐次的,因而都至少包含两个 $\alpha y=\beta x$ 状的方程. 如此,要满足要求,过点 $C$ 的直线的条数也必须至少为 2 . 即问题无解. $(C M+C N)^{2}=a^{2}$, 也即 $C M+C N=a$, 此时 问题无解,这是我们前面已经指了出来的. 如果取 $n=-2$,则成了要求差的平方 $(C M-$ $(N)^{2}$ 为常数, 也即要求差 $M N$ 本身为常数. 此时得到两个方程
\[
a^{2} L^{2}=\left(x^{2}+y^{2}\right)(2 L-M)^{2}
\]
和
\[
a^{2}\left(x^{2}+y^{2}\right) N^{2}=\left[2\left(x^{2}+y^{2}\right) N-M\right]^{2}
\]
$N=1, M=2 b x$ 时得到最简单的解, 这时方程为
\[
a^{2}\left(x^{2}+y^{2}\right)=4\left(x^{2}+y^{2}-b x\right)^{2}
\]
或者令 $a^{2}=8 c^{2}$,得
\[
\left(x^{2}+y^{2}\right)^{2}=2\left(c^{2}+b x\right)\left(x^{2}+y^{2}\right)-b^{2} x^{2}
\]
从而
\[
x^{2}+y^{2}=c^{2}+b x \pm c \sqrt{c^{2}+2 b x}
\]
进而
\[
y=\sqrt{c^{2}+b x+x^{2} \pm c \sqrt{c^{2}+2 b x}}
\]
\section{$\S 414$}

因而与过点 $C$ 的直线交于两点 $M$ 和 $N$, 且 $M N$ 恒为常数,这样的曲线有无穷多条. 显然以点 $C$ 为圆心的圆都满足这里的要求, 线段 $M N$ 为圆的直径. 圆是 $M=0$ 时从通用方 程得到的. 继圆之后, 满足要求的方程为
\[
a^{2}\left(x^{2}+y^{2}\right)=4\left(x^{2}+y^{2}-b x\right)^{2}
\]
和
\[
a^{2} x^{2}=\left(x^{2}+y^{2}\right)(2 x-2 b)^{2}
\]
的四阶线. 为便于考察这些曲线的形状, 我们回到 $z$ 和 $\varphi$ 之间的方程. 由于 $x^{2}+y^{2}=z^{2}$, $x=z \cos \varphi, y=z \sin \varphi$, 再令 $a=2 c$, 则第一个方程化为
\[
c^{2} z^{2}=\left(z^{2}-b z \cos \varphi\right)^{2} \text { 或 } b \cos \varphi \pm c=z
\]
第二个方程化为
\[
c^{2}(\cos \varphi)^{2}=(z \cos \varphi-b)^{2} \text { 或 } z=\frac{b}{\cos \varphi} \pm c
\]
利用这两个表达式可以容易地得到这些曲线的形状.

\section{$\S 415$}

为画出方程 $z=b \cos \varphi \pm c$ 所表示的曲线 (图 83,84,85), 引过点 $C$ 的直线 $A C B$, 于 $A C B$ 上先取 $C D=b$, 再在 $D$ 的两侧取 $A, B$, 使 $D A=D B=c$, 则点 $A, B$ 都是所求曲线上的 点. 过点 $C$ 任意地画一条直线 $N C M$, 并从点 $D$ 向 $N C M$ 引垂线 $D L$, 在 $N C M$ 上 $L$ 的两侧 取点 $M$ 和 $N$, 使 $L M=L N=c$, 则这点 $M, N$ 在所求曲线上, 线段 $M N$ 的长为 $2 c$, 符合于问 题的要求.


【图,待补】
%%![](https://cdn.mathpix.com/cropped/2023_02_05_a9d9b884b46ab350bed3g-02.jpg?height=378&width=366&top_left_y=1743&top_left_x=171)

图 83


【图,待补】
%%![](https://cdn.mathpix.com/cropped/2023_02_05_a9d9b884b46ab350bed3g-02.jpg?height=376&width=350&top_left_y=1746&top_left_x=646)

图 84


【图,待补】
%%![](https://cdn.mathpix.com/cropped/2023_02_05_a9d9b884b46ab350bed3g-02.jpg?height=378&width=402&top_left_y=1745&top_left_x=1072)

图 85

应该指出, 如果 $C D=b<c$, 则曲线在点 $C$ 处有共轭点 ( $\S 277$ ), 如图 83 所示. 如果 $b=c$, 则曲线在点 $C$ 处有尖点. 线段 $A C$ 的长为零, 如图 84 所示.

最后, 如果 $b>c$, 则点 $A$ 在点 $C$ 和点 $B$ 之间, 曲线在点 $C$ 处有结点, 或二重点, 如图 85 所示. 这三条线都以 $A C B$ 为直径, 且垂直于 $A C B$ 的 $E C F$ 的长为 $2 c$.

\section{$\S 416$}

上节所作四阶线是封闭的, 有界的, 现在作满足同样条件的伸向无穷的四阶线, 方程 为 $z=\frac{b}{\cos \varphi} \pm c$. 参见图 86. 过点 $C$ 画直线 $C A B$, 取 $C D=b$, 取 $D A=D B=c$, 则点 $A, B$ 在所 求曲线上. 过点 $D$ 引垂直于 $A B$ 的直线 $E D F$. 任画直线 $C L$, 则 $C L=\frac{b}{\cos \varphi}, \varphi$ 为 $\angle D C L$. 在 $C L$ 上 $L$ 两侧取点 $M, N$, 使 $L M=L N=c$, 则 $M, N$ 在所求曲线上. 这样画出的是古希腊尼 科梅德斯 (Nicomedes) 的蚌线. $C$ 为极点, $E F$ 为渐近线, 四个分枝在无穷远处与该渐近 线重合. 称 $h B h$ 部分为蚌线的外线, $g A g$ 部分为蚌线的内线, 这两部分之外还有一个共轭 点 $C$.


【图,待补】
%%![](https://cdn.mathpix.com/cropped/2023_02_05_a9d9b884b46ab350bed3g-03.jpg?height=313&width=671&top_left_y=997&top_left_x=495)

图 86

\section{$\S 417$}

前两节画出的是四阶线. 具有这种性质的更高阶线, 需要多少条,都容易求得出. 只 要 $P$ 是 $\varphi$ 的正弦和余弦的奇函数,方程 $z=b P \pm c$ 所表示的曲线就与过点 $C$ 的直线相交 于两点, 记这两点为 $M, N$, 线段 $M N$ 就必定恒等于 $2 c$. 这些曲线都属于蚌线类, 并且代替 直线 $E F$ 可以取方程 $z=b P$ 表示的任何曲线作准线. 前面我们看到了这个方程所表示的 曲线与过点 $C$ 的直线都只有一个交点. 因此, 由于长度 $c$ 任意, 对一条曲线 $z=b P$ 我们就 可以画出无穷多条具有所要性质的曲线.

\section{$\S 418$}

参见图 87 , 作为例子,任画!一条曲线 $C E D L F$, 使过点 $C$ 的直线与它都相交于一点, 如 $D, L$. 那么在每一条这样的直线, 例如 $C L$ 上于交点 $L$ 的两侧取线段 $L M=L N=c$, 则 $M, N$ 为所求曲线上的点. 让 $C L$ 保持过点 $C$ 连续移动, 可使点 $M, N$ 画出曲线 $A M P C Q B N R C$. 这曲线当然满足与过点 $C$ 的每一条直线的交点 $M, N$ 都具有性质 $M N$ 恒为常数. 这里应该指出, 如果曲线 $C E D F$ 是过点 $C$ 的圆,则画出的曲线为 $\S 414$ 求出的 四阶线.


【图,待补】
%%![](https://cdn.mathpix.com/cropped/2023_02_05_a9d9b884b46ab350bed3g-04.jpg?height=380&width=364&top_left_y=413&top_left_x=632)

图 87

\section{$\S 419$}

前面我们解决了求这样的曲线 $A M N$ 的问题, 它与过点 $C$ 的直线必相交于两点 $M$, $N$, 使得 $C N-C M$ 或 $C M^{2}-2 C M \cdot C N+C N^{2}$ 恒为常数. 我们再简单地考虑一下 $C M^{2}+$ $C M \cdot C N+C N^{2}$ 为常数的情形. 令 $\S 412$ 方程中的 $n=1$, 得
\[
a^{2} L^{2}=\left(x^{2}+y^{2}\right)\left(L^{2}-L M+M^{2}\right)
\]
其中 $x$ 和 $y$ 的函数 $L, M$ 分别为 $m+1, m$ 次函数. 我们也得到方程
\[
a^{2}\left(x^{2}+y^{2}\right) N^{2}=\left(x^{2}+y^{2}\right)^{2} N^{2}-\left(x^{2}+y^{2}\right) M N+M^{2}
\]
其中 $M, N$ 都为 $x, y$ 的齐次函数,且 $M$ 的次数比 $N$ 大 1 .

\section{$\S 420$}

首先, 令 $M=0$ 得到的是以 $C$ 为圆心的圆. 此时过圆心 $C$ 到曲线的直线都相等, 满足 要求. 其次,令第一个方程中的 $M=b, L=x$, 得
\[
a^{2} x^{2}=\left(x^{2}+y^{2}\right)\left(x^{2}-b x+b^{2}\right)
\]
或
\[
y^{2}=\frac{x^{2}\left(a^{2}-b^{2}+b x-x^{2}\right)}{b^{2}-b x+x^{2}}
\]
这种曲线中, 除掉圆, 这是最简单的. 再次, 令第二个方程中的 $N=1, M=b x$, 得
\[
a^{2}\left(x^{2}+y^{2}\right)=\left(x^{2}+y^{2}\right)^{2}-b x\left(x^{2}+y^{2}\right)+b^{2} x^{2}
\]
或
\[
x^{2}+y^{2}=\frac{1}{2} b x+\frac{1}{2} a^{2} \pm \sqrt{\frac{1}{4} a^{4}+\frac{1}{2} a^{2} b x-\frac{3}{4} b^{2} x^{2}}
\]
也为四阶线, 也满足要求. 

\section{$\S 421$}

上面讨论的是 $z$ 的两个值 $C M, C N$ 的次数不高于 2 的问题. 现在我们讨论次数更高 的问题. $z$ 的两个值依然从方程 $z^{2}-P z+Q=0$ 得到, 其中
\[
P=\frac{M z}{L}, \quad Q=\frac{N z^{2}}{L}
\]
$L, M, N$ 都是 $x$ 和 $y$ 的齐次函数, 次数依次为 $n+2, n+1$ 和 $n, x$ 为横标 $C P, y$ 为纵标 $P M$. 这里要讨论的第一个问题是, 交点 $M, N$ 满足 $C M^{3}+C N^{3}=a^{3}$. 由方程 $z^{2}-P z+Q=0$ 的 性质得 $C M^{3}+C N^{3}=P^{3}-3 P Q$, 因而应该有 $P^{3}-3 P Q=a^{3}$. 由于 $P^{3}$ 和 $P Q$ 都是无理量, 该等式不能成立. 也即, 没有满足 $C M^{3}+C N^{3}=a^{3}$ 的曲线. 这结论是在只有两个交点的前 提下推出的. 如果允许交点个数多于 2 , 那么令 $Q=\frac{P^{3}-a^{3}}{3 P}$, 取 $P$ 为 $\varphi$ 的正弦和余弦的任 何函数,就可以求出满足要求的无穷多条曲线.

\section{$\S 422$}

如果要曲线满足
\[
C M^{4}+\mathrm{CN}^{4}=a^{4}
\]
则应该有
\[
P^{4}-4 P^{2} Q+2 Q^{2}=a^{4}
\]
该方程不含无理量, 求解可以进行, 应该得到
\[
Q=P^{2} \pm \sqrt{\frac{1}{2} P^{4}+\frac{1}{2} a^{4}}
\]
该函数虽含根号, 但可视为单值函数, 因为 $\sqrt{\frac{1}{2} P^{4}+\frac{1}{2} a^{4}}$ 取加号时所得 $z$ 值为虚数. 这 样我们有
\[
\frac{N z^{2}}{L}=\frac{M^{2} z^{2}}{L^{2}}-\sqrt{\frac{M^{4} z^{4}}{2 L^{4}}+\frac{1}{2} a^{4}}
\]
由 $L-M+N=0$ 或
\[
z^{2}-\frac{M z^{2}}{L}+\frac{N z^{2}}{L}=0
\]
得
\[
z^{2}-\frac{M z^{2}}{L}+\frac{M^{2} z^{2}}{L^{2}}-\sqrt{\frac{M^{4} z^{4}}{2 L^{4}}+\frac{1}{2} a^{4}}=0
\]
有理化得
\[
\frac{z^{4}}{L^{4}}\left(L^{2}-L M+M^{2}\right)^{2}=\frac{M^{4} z^{4}}{2 L^{4}}+\frac{1}{2} a^{4}
\]
或
\[
\left(x^{2}+y^{2}\right)^{2}\left(2\left(L^{2}-L M+M^{2}\right)^{2}-M^{4}\right)=a^{4} L^{4}
\]
这是满足要求的曲线的通用方程.

\section{$\S 423$}

$\S 372$ 中那个更为容易的方法也可以用来解决这类问题. 由于 $C M \cdot C N=Q$ 和 $Q=$ $\frac{N z^{2}}{L}$, 如果记 $C M, C N$ 中的一个为 $z$, 则另一个为 $\frac{Q}{z}=\frac{N z}{L}$, 因而, 条件为
\[
C M^{n}+C N^{n}=a^{n}
\]
时得
\[
u^{b}=\frac{u T}{u^{2} u N}+u z \text { 或 } z^{n}=\frac{a^{n} L^{n}}{L^{n}+N^{n}}
\]
该方程, $n$ 为偶数时有理, 满足所提条件; $n$ 为奇数时, 应两边平方, 使有理化. 这使交点个 数加倍, 不符合只交于两点的要求, 也即无解. 例如, 条件为
\[
C M^{2}+C N^{2}=a^{2}
\]
得
\[
z^{2}=x^{2}+y^{2}=\frac{a^{2} L^{2}}{L^{2}+N^{2}}
\]
由 $L-M+N=0$ 知, 这结果与 $\S 410$ 所得
\[
x^{2}+y^{2}=\frac{a^{2} L^{2}}{(L-M)^{2}+L^{2}}
\]
是一致的. 一般地,对偶数 $n$, 条件为 $C M^{n}+C N^{n}=a^{n}$ 时得
\[
z^{n}=\left(x^{2}+y^{2}\right)^{\frac{n}{2}}=\frac{a^{n} L^{n}}{L^{n}+N^{n}}=\frac{a^{n} L^{n}}{L^{n}+(M-L)^{n}}
\]
其中 $L, M, N$ 都是 $x$ 和 $y$ 的函数, 次数依次为 $m+2, m+1$ 和 $m$.

\section{$\S 424$}

这同样的解也可以利用 $C M+C N=P$ 得到. 如果令 $C M, C N$ 中的一个为 $z$, 则另一个 为 $P-z$, 因而 $C M^{n}+C N^{n}$ 应该为常数时, 则 $z^{n}+(P-z)^{n}=a^{n}$. 由
\[
P=\frac{M z}{L}, \quad Q=\frac{N z^{2}}{L}
\]
又及 $L-M+N=0$, 得
\[
z^{n}+\frac{z^{n}(M-L)^{n}}{L^{n}}=a^{n}
\]
也即
\[
z^{n}=\frac{a^{n} L^{n}}{L^{n}+(M-L)^{n}} \text { 或 } z^{n}=\frac{a^{n} L^{n}}{L^{n}+N^{n}}
\]
消去 $L$ 得
\[
z^{n}=\frac{a^{n}(M-N)^{n}}{(M-N)^{n}+N^{n}}
\]
$n$ 为偶数时, 这些方程都给出满足要求的曲线. $n$ 为奇数时, 当然有两个交点 $M$ 和 $N$, 满足 $C M^{n}+C N^{n}=a^{n}$. 但是还有另外两个点也具有这一性质, 也即每条过点 $C$ 的直线都双倍 地满足要求.

\section{$\S 425$}

有了这些讨论, 我们就可以比较容易地来解决一些相当困难的问题. 比如, 求一条曲 线, 过点 $C$ 的每条直线都与它交于两点, 记为 $M$ 和 $N$, 这两点满足
\[
\begin{aligned}
& C M^{n}+C N^{n}+\alpha C M \cdot C N\left(C M^{n-2}+C N^{n-2}\right)+ \\
& \beta C M^{2} \cdot C N^{2}\left(C M^{n-4}+C N^{n-4}\right)+\cdots=a^{n}
\end{aligned}
\]
设两个值中的一个 $C M=z$, 则另一个
\[
C N=\frac{Q}{z}=\frac{N z}{L}
\]
将这两个值代入上式, 得所求曲线的方程
\[
z^{n}\left(L^{n}+N^{n}+\alpha L N\left(L^{n-2}+N^{n-2}\right)+\beta L^{2} N^{2}\left(L^{n-4}+N^{n-4}\right)+\cdots\right)=\alpha^{n} L^{n}
\]
由 $L-M+N=0$, 及 $L, M, N$ 都是 $x$ 和 $y$ 的齐次函数, 次数依次为 $m+2, m+1$ 和 $m$, 得 $L=M-N$, 或 $N=M-L$. 这样, 在所给情况下我们可以得到无穷多个解.

\section{$\S 426$}

现在我们把考察对象改为与过定点 $C$ 的直线都有三个交点的曲线. 这类曲线的通用 方程为
\[
z^{3}-P z^{2}+Q z-R=0
\]
其中 $z$ 是曲线上任何一点到 $C$ 的距离, $P, Q, R$ 是 $\angle A C M=\varphi$, 或者它的正弦和余弦的函 数. 由于前面讲过的原因, 为保证交点个数不多于 $3, P$ 和 $R$ 应该是 $\sin \varphi$ 和 $\cos \varphi$ 的奇函 数, 而 $Q$ 应该是 $\sin \varphi$ 和 $\cos \varphi$ 的偶函数. 取直角坐标 $C P=x, P M=y$, 则 $x^{2}+y^{2}=z^{2}$. 如果 $K, L, M$ 和 $N$ 都表示 $x$ 和 $y$ 的齐次函数, 次数依次为 $n+3, n+2, n+1$ 和 $n$, 则
\[
P=\frac{L z}{K}, \quad Q=\frac{M z^{2}}{K}, \quad R=\frac{N z^{3}}{K}
\]
这样得所求曲线的直角坐标通用方程
\[
K-L+M-N=0
\]
这方程清楚地告诉我们,点 $C$ 是曲线的 $n$ 重点.

\section{$\S 427$}

首先, 该方程包含所有的三阶线, 这里要求点 $C$ 不在曲线上. 其次, 该方程包含所有的四阶线, 这里要求点 $C$ 在曲线上. 再次, 具有一个二重点的五阶线属于该方程, 这里要 求点 $C$ 为重点. 类似地, 更高阶线属于该方程, 则 $n+3$ 阶线必具有一个 $n$ 重点.

\section{$\S 428$}

设 $p, q, r$ 是从方程
\[
z^{3}-P z^{2}+Q z-R=0
\]
得到的三个 $z$ 值, 这里 $\angle C A M=\varphi$ 为任何值, 那么由方程的性质我们有
\[
P=p+q+r, \quad Q=p q+p r+q r, \quad R=p q r
\]
由 $P, Q$ 不能用 $x$ 和 $y$ 有理表示, 显见, 谈不上 $p+q+r$ 或 $p q r$ 为常数的曲线, 也求不出 $p$, $q, r$ 的奇函数为常数的曲线, 但可以使它们的偶函数为常数. 例如, 使
\[
p q+p r+q r=a^{2}
\]
此时我们有
\[
Q=\frac{M z^{2}}{K}=a^{2}
\]
也即 $M\left(x^{2}+y^{2}\right)=a^{2} K$. 将这个 $K$ 值代入方程 $K-L+M-N=0$, 得满足要求的所有曲线 的通用方程
\[
M\left(x^{2}+y^{2}\right)-a^{2} L+a^{2} M-a^{2} N=0
\]
消去 $M$ 得
\[
\left(x^{2}+y^{2}\right) K-\left(x^{2}+y^{2}\right) L+a^{2} K-\left(x^{2}+y^{2}\right) N=0
\]
\section{$\S 429$}

用同样的方式可以解决另外一些类似的问题. 例如, 求与过点 $C$ 的直线交于三点, 且
\[
p^{2}+q^{2}+r^{2}=a^{2}
\]
的曲线,由
\[
p^{2}+q^{2}+r^{2}=P^{2}-2 Q, \quad P=\frac{L z}{K}
\]
和
\[
Q=\frac{M z^{2}}{K}
\]
得
\[
\frac{L^{2} z^{2}}{K^{2}}-\frac{2 M z^{2}}{K}=a^{2}
\]
也即
\[
\left(x^{2}+y^{2}\right) L^{2}-2\left(x^{2}+y^{2}\right) K M=a^{2} K^{2}
\]
与过点 $C$ 的直线交于三点的曲线有通用方程 $K-L+M-N=0$, 这里重要的一点是 $x$ 和 $y$ 的最高次数比最低次数高 3 . 为从这两个方程得到我们所要的方程, 先以 $2\left(x^{2}+y^{2}\right) K$ 乘 $K-L+M-N=0$, 再与刚得到的方程
\[
\left(x^{2}+y^{2}\right) L^{2}-2\left(x^{2}+y^{2}\right) K M=a^{2} K^{2}
\]
相加, 得消去了 $M$ 的方程
\[
2\left(x^{2}+y^{2}\right) K^{2}-2\left(x^{2}+y^{2}\right) K L+\left(x^{2}+y^{2}\right) L^{2}-a^{2} K^{2}-2\left(x^{2}+y^{2}\right) K N=0
\]
其中次数最高项 $2\left(x^{2}+y^{2}\right) K^{2}$ 的次数为 $2 n+8$, 次数最低项 $2\left(x^{2}+y^{2}\right) K N$ 的次数为 $n+5$. 最高最低之差为 3 , 具有重要的那一点.

\section{$\S 430$}

由于最高次项、最低次项都不能为零, 为求最简单的曲线, 我们令 $n=0, N=b^{3}, K=$ $x\left(x^{2}+y^{2}\right), L=0$, 得方程
\[
2\left(x^{2}+y^{2}\right)^{3} x^{2}-a^{2} x^{2}\left(x^{2}+y^{2}\right)^{2}-2 b^{3} x\left(x^{2}+y^{2}\right)^{2}=0
\]
除以 $2 x\left(x^{2}+y^{2}\right)^{2}$,得
\[
x\left(x^{2}+y^{2}\right)-\frac{1}{2} a^{2} x-b^{3}=0
\]
阶数为 3 , 改 $L=0$ 为
\[
L=2 c\left(x^{2}+y^{2}\right)
\]
得四阶方程
\[
x^{2}\left(x^{2}+y^{2}\right)-2 c x\left(x^{2}+y^{2}\right)+2 c^{2}\left(x^{2}+y^{2}\right)-\frac{1}{2} a^{2} x^{2}-b^{3} x=0
\]
或
\[
x^{2}\left(x^{2}+y^{2}\right)+(2 c-x)^{2}\left(x^{2}+y^{2}\right)=a^{2} x^{2}+2 b^{3} x
\]
用类似的方法可得到很多满足条件的更高阶的曲线.

\section{$\S 431$}

也可以求出 $p^{4}+q^{4}+r^{4}$ 为常数的曲线, 由于
\[
p^{4}+q^{4}+r^{4}=P^{4}+4 P^{2} Q+2 Q^{2}+4 P R
\]
我们令
\[
P^{4}-4 P^{2} Q+2 Q^{2}+4 P R=c^{4}
\]
这样我们有
\[
z^{4}\left(L^{4}-4 K L^{2} M+2 K^{2} M^{2}+4 K^{2} L N\right)=c^{4} K^{4}
\]
或
\[
4 K^{2} L N z^{4}=c^{4} K^{4}-z^{4}\left(L^{4}-4 K L^{2} M+2 K^{2} M^{2}\right)
\]
解出 $N$, 代入 $K-L+M-N=0$, 得到的就是所要曲线的通用方程.

\section{$\S 432$}

条件 $p^{4}+q^{4}+r^{4}=c^{4}$ 和 $p^{2}+q^{2}+r^{2}=a^{2}$ 可以同时被满足, 为此,应该有 
\[
z^{2} L^{2}-2 z^{2} K M=a^{2} K^{2}
\]
或
\[
2 z^{2} K M=z^{2} L^{2}-a^{2} K^{2}
\]
由
\[
4 K^{2} L N z^{4}=c^{4} K^{4}-L^{4} z^{4}+4 K L^{2} M z^{4}-2 K^{2} M^{2} z^{4}
\]
我们有
\[
4 K^{2} L z^{4}=c^{4} K^{4}+L^{4} z^{4}-2 a^{2} K^{2} L^{2} z^{2}-2 K^{2} M^{2} z^{4}
\]
和
\[
4 K^{2} L M z^{4}=2 K L^{3} z^{4}-2 a^{2} K^{3} L z^{2}
\]
将 $M$ 和 $N$ 的值代入方程 $K-L+M-N=0$, 也即
\[
4 K^{3} L z^{4}-4 K^{2} L^{2} z^{4}+4 K^{2} L M z^{4}-4 K^{2} L N z^{4}=0
\]
得所求曲线的方程

$4 K^{3} L z^{4}-4 K^{2} L^{2} z^{4}+2 K L^{3} z^{4}-2 a^{2} K^{3} L z^{2}-c^{4} K^{4}-L^{4} z^{4}+2 a^{2} K^{2} L^{2} z^{2}+2 K^{2} M^{2} z^{4}=0$ 由
\[
K M z^{2}=\frac{1}{2} L^{2} z^{2}-\frac{1}{2} a^{2} K^{2}
\]
我们有
\[
2 K^{2} M^{2} z^{4}=\frac{1}{2} L^{4} z^{4}-a^{2} K^{2} L^{2} z^{2}+\frac{1}{2} a^{4} K^{4}
\]
这就是说,所求曲线的通用方程为
\[
8 K^{3} L z^{4}-8 K^{2} L^{2} z^{4}+4 K L^{3} z^{4}-4 a^{2} K^{3} L z^{2}-2 c^{4} K^{4}-L^{4} z^{4}+2 a^{2} K^{2} L^{2} z^{2}+a^{4} K^{4}=0
\]
\section{$\S 433$}

由于 $x$ 和 $y$ 的齐次函数 $K$ 的次数比 $L$ 大 1 , 取 $K=z^{2}, L=b x$, 可求出同时满足 $p^{2}+$ $q^{2}+r^{2}=a^{2}$ 和 $p^{4}+q^{4}+r^{4}=c^{4}$ 的三交点最简曲线,所得方程为
\[
8 b x z^{6}-8 b^{2} x^{2} z^{4}+4 b^{3} x^{3} z^{2}-4 a^{2} b x z^{4}-2 c^{4} z^{4}-b^{4} x^{4}+2 a^{2} b^{2} x^{2} z^{2}+a^{4} z^{4}=0
\]
由 $z^{2}=x^{2}+y^{2}$ 知该方程有理, 表示的是七阶线, $C$ 是它的四重点. 令 $K=x, L=b$, 可以得 到满足条件的另一个七阶线. 得到的方程为
\[
8 b x^{3} z^{4}-8 b^{2} x^{2} z^{4}+4 b^{3} x z^{4}-4 a^{2} b x^{3} z^{2}-2 c^{4} x^{4}-b^{4} z^{4}+2 a^{2} b^{2} x^{2} z^{2}+a^{4} x^{4}=0
\]
也即
\[
z^{4}=\frac{4 a^{2} b x^{3} z^{2}-2 a^{2} b^{2} x^{2} z^{2}+2 c^{4} x^{4}-a^{4} x^{4}}{8 b x^{3}-8 b^{2} x^{2}+4 b^{3} x-b^{4}}
\]
由此得

$z^{2}=\frac{2 a^{2} b x^{3}-a^{2} b^{2} x^{2} \pm x^{2} \sqrt{\left(2 b x-b^{2}\right)\left[2 c^{4}\left(b^{2}-2 b x+4 x^{2}\right)-2 a^{4}\left(b^{2}-2 b x+2 x^{2}\right)\right]}}{b(2 x-b)\left(4 x^{2}-2 b x+b^{2}\right)}$ 


【图,待补】
%%![](https://cdn.mathpix.com/cropped/2023_02_05_a9d9b884b46ab350bed3g-11.jpg?height=129&width=924&top_left_y=77&top_left_x=373)

\section{$\S 434$}

可以进一步考察与过点 $C$ 的直线交于四点的曲线, 并从中求出满足某些条件的曲 线. 但从前面的讨论中可知这不会遇到什么困难, 这类问题的各种结果都可立即得到. 如 果解不存在也能立即判明, 因而于此我们不再停留. 下面转向关于曲线的另一个题目. 

