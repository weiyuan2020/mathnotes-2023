\chapter{第二章 函数变换}

\section{$\S 27$}

改变函数形式的方法有两种,一种是替换变量,一种是保持原有变量, 直接改写表达 式.

从代数中我们知道, 同一个量可以有不同的表达式. 例如
\[
2-3 z+z^{2}, a^{3}+3 a^{2} z+3 a z^{2}+z^{3}, \frac{2 a^{2}}{a^{2}-z^{2}}, \frac{1}{\sqrt{1+z^{2}}-z}
\]
可分别改写为
\[
(1-z)(2-z),(a+z)^{3}, \frac{a}{a-z}+\frac{a}{a+z}, \sqrt{1+z^{2}}+z
\]
这改写成的表达式都与原表达式完全等价,不同的只是形式.

变换函数的再一种方法是替换变量, 也称为换元, 即用另一个变量 $y$ 代替变量 $z$. 当 然, $y$ 与 $z$ 有着一定的关系. 例如, 用 $y$ 替换 $a-z$, 则 $z$ 的函数 $a^{4}-4 a^{3} z+6 a^{2} z^{2}-4 a z^{3}+z^{4}$ 变为 $y$ 的函数 $y^{4}$; 令 $z=\frac{a^{2}-y^{2}}{2 y^{2}}$, 则 $z$ 的无理函数 $\sqrt{a^{2}+z^{2}}$ 变为 $y$ 的有理函数 $\frac{a^{2}+y^{2}}{2 y}$. 下一 章讨论这种换元变换, 本章讨论不换元的改写表达式变换.

\section{$\S 28$}

将整函数分解成因式, 从而将它表示为乘积,这常常带来方便.

分解成因式的整函数,其性质变得明显, 一眼就看得出 $z$ 的哪些值使它变为零,例如 将函数
\[
6-7 z+z^{3}
\]
表示为乘积
\[
(1-z)(2-z)(3+z)
\]
那么使该函数变为零的三个值显然是 $z=1, z=2$ 和 $z=-3$. 但是要从原式 $6-7 z+z^{3}$ 看 出这几点性质, 可就远非这么容易. 我们用 $z$ 的最高次数来区别因式. 称 $z$ 的最高次数为 1 的因式为线性因式. 线性因式的形状为 
\[
f+g z
\]
称 $z$ 的最高次数为 2 的因式为二次因式, 其形状为
\[
f+g z+h z^{2}
\]
称 $z$ 的最高次数为 3 的因式为三次因式, 其形状为
\[
f+g z+h z^{2}+i z^{3}
\]
类推. 显然,二次因式是两个线性因式的积,三次因式是三个线性因式的积. 类推,由此可 见 $z$ 的最高次数为 $n$ 的整函数含有 $n$ 个线性因式. 这样, 即使一个整函数的因式中有二 次、三次或更高次的,我们也说得出它的线性因式的总个数.

\section{$\S 29$}

对于 $z$ 的整函数 $Z$, 求出了方程 $Z=0$ 的所有的根, 就等于求出了整函数 $Z$ 的所有线 性因式.

如果 $z=f$ 是方程 $Z=0$ 的根,则 $z-f$ 除得尽 $Z$, 也即 $z-f$ 就是 $Z$ 的因式. 这样,求出 了方程 $Z=0$ 的所有的根
\[
z=f, z=g, z=h
\]
等,也就把函数 $Z$ 分解成了线性因式,也就有了 $Z$ 的乘积形式
\[
Z=(z-f)(z-g)(z-h) \cdots
\]
这里有一点要注意, 如果 $Z$ 中 $z$ 的最高次幂的系数不是 $+1$, 那么乘积 $(z-f)(z-g) \cdots$ 应 乘上这个系数才等于 $Z$. 也即, 如果
\[
Z=A z^{n}+B z^{n-1}+C z^{n-2}+\cdots
\]
则
\[
Z=A(z-f)(z-g)(z-h) \cdots
\]
如果
\[
Z=A+B z+C z^{2}+D z^{3}+E z^{4}+\cdots
\]
$Z=0$ 的根为 $f, g, h, \cdots$, 则
\[
Z=A\left(1-\frac{z}{f}\right)\left(1-\frac{z}{g}\right)\left(1-\frac{z}{h}\right) \cdots
\]
反之,如果 $z$ 的函数 $Z$ 有因式 $z-f$ 或 $1-\frac{z}{f}$,那么把 $Z$ 中的 $z$ 换成 $f$, 则因式 $z-f$ 或 $1-\frac{z}{f}$ 变为零, 从而函数 $Z$ 变为零.

\section{$\S 30$}

线性因式可实可虚. 如果函数 $Z$ 有虚因式, 则虚因式的个数必为偶数.

整函数 $Z$ 的线性因式由方程 $Z=0$ 的根给出, 实根给出实因式,虚根给出虚因式. 而 方程的虚根个数恒为偶数, 因而函数 $Z$, 或者没有虚因式, 或者有 2 个, 或者有 4 个, 或者 有 6 个, 等等. 如果函数 $Z$ 只有两个虚因式,那么这两个虚因式的积必定是实二次因式. 

这是因为记全体实因式的积为 $P$, 则这两个虚因式的积 $\frac{Z}{P}$ 必定是实的. 同样, 如果函数 $Z$ 有 4 个, 6个或者 8 个虚因式, 那么它们的积也恒为实的. 因为这每一个积都等于 $Z$ 除上相 应的全体实因式的积.

\section{$\S 31$}

如果 $Q$ 是 4 个虚线性因式的实乘积,那么这个 $Q$ 可以表示成两个实二次因式的乘 积.

这个 $Q$ 的形状为
\[
z^{4}+A z^{3}+B z^{2}+C z+D
\]
假定 $Q$ 不能表示出两个实因式的乘积,那么它必定可以表示成形状为
\[
z^{2}-2(p+q \sqrt{-1}) z+r+s \sqrt{-1}
\]
和
\[
z^{2}-2(p-q \sqrt{-1}) z+r-s \sqrt{-1}
\]
这样两个虚二次因式的和. 这是因为这两个因式的积必须等于实的 $z^{4}+A z^{3}+B z^{2}+C z+$ D. 从这两个虚二次因式我们得到下面 4 个虚线性因式
\[
\begin{aligned}
& \text { I. } z-(p+q \sqrt{-1})+\sqrt{p^{2}+2 p q \sqrt{-1}-q^{2}-r-s \sqrt{-1}} \\
& \text { II. } z-(p+q \sqrt{-1})-\sqrt{p^{2}+2 p q \sqrt{-1}-q^{2}-r-s \sqrt{-1}} \\
& \text { III. } z-(p-q \sqrt{-1})+\sqrt{p^{2}-2 p q \sqrt{-1}-q^{2}-r+s \sqrt{-1}} \\
& \text { IV. } z-(p-q \sqrt{-1})-\sqrt{p^{2}-2 p q \sqrt{-1}-q^{2}-r+s \sqrt{-1}}
\end{aligned}
\]
为简单起见, 我们令
\[
t=p^{2}-q^{2}-r, u=2 p q-s
\]
I , III 相乘, 积为
\[
\begin{aligned}
& z^{2}-\left(2 p-\sqrt{2 t+2 \sqrt{t^{2}+u^{2}}}\right) z+p^{2}+q^{2}- \\
& p \sqrt{2 t+2 \sqrt{t^{2}+u^{2}}}+q \sqrt{-2 t+2 \sqrt{t^{2}+u^{2}}}+\sqrt{t^{2}+u^{2}}
\end{aligned}
\]
是实的. 类似地, II, IV 的积
\[
\begin{aligned}
& z^{2}-\left(2 p+\sqrt{2 t+2 \sqrt{t^{2}+u^{2}}}\right) z+p^{2}+ \\
& q^{2}+p \sqrt{2 t+2 \sqrt{t^{2}+u^{2}}}-q \sqrt{-2 t+2 \sqrt{t^{2}+u^{2}}}+\sqrt{t^{2}+u^{2}}
\end{aligned}
\]
也是实的. 这样从我们的假定, $Q$ 不能表示为两个实二次因式出发, 推出 $Q$ 依然能表示成 两个实二次因式的乘积.

\section{$\$ 32$}

不管 $z$ 的整函数 $Z$ 有多少个虚线性因式, 我们都可以把它们配成对, 使得每一对的乘积都是实的.

虚根的个数恒为偶数, 记为 $2 n$. 又虚根对应的全体因式的乘积是实的. 如果只有两 个虚根, 那么由前一节知它们的积为实; 如果有 4 个虚线性因式, 由前一节知, 它们的积 可以表示成两个状如 $f_{z}{ }^{2}+g z+h$ 的实二次因式的积. 虽然这证明方法不能推向更高次, 但是不管虚线性因式的个数是多少, 它们都具有这种性质是无疑的. 因而可以用 $n$ 个实 二次因式代替 $2 n$ 个虚线性因式. 这样我们得出了结论: $z$ 的整函数可以表示成实线性因 式和实二次因式的乘积. 这里没作严格证明, 后面给出的一种严格些的证法, 是把状如
\[
a+b z^{n}, a+b z^{n}+c z^{2 n}, a+b z^{n}+c z^{2 n}+d z^{3 n}
\]
等的函数实际地分解成实二次因式.

\section{$\S 33$}

如果 $z$ 的整函数 $Z$ 在 $z=a$ 时取值 $A$, 在 $z=b$ 时取值 $B$, 那么对 $A, B$ 之间的任何一个 值 $C, a, b$ 之间都存在一个 $c$, 使得 $z=c$ 时, $Z$ 取值为 $C$.

$Z$ 是 $z$ 的单值函数,对 $z$ 的每一个实值, $Z$ 都有一个实值与之对应. 我们这里, $z=a$ 时, $Z$ 取值 $A ; z=b$ 时, $Z$ 取值 $B$. 函数 $Z$ 从 $A$ 变到 $B$ 的过程不能跳过 $A, B$ 之间的任何一个值. 如果方程 $Z-A=0, Z-B=0$ 各有一个实根, 那么对 $A, B$ 之间的任何一个值 $C$, 方程 $Z-$ $C=0$ 就也有一个实根.

因此, 如果表达式 $Z-A, Z-B$ 各有一个实线性因式, 那么对 $A, B$ 之间的任何一个 $C$, 表达式 $Z-C$ 就也有一个实线性因式.

\section{$\S 34$}

如果整函数 $Z$ 中 $z$ 的最大指数是奇数,记作 $2 n+1$, 那么, 函数 $Z$ 至少有一个实线性 因式.

这个 $Z$ 的形状显然为
\[
z^{2 n+1}+\alpha z^{2 n}+\beta z^{2 n-1}+\gamma z^{2 n-2}+\cdots
\]
$z=\infty$ 时, 与第一项相比, 其他项都可忽略, 我们有 $Z=(\infty)^{2 n+1}=\infty$, 从而 $Z-\infty$ 有实线 性因式 $z-\infty$. 类似地, $z=-\infty$ 时, $z=(-\infty)^{2 n+1}=-\infty$, 从而 $z+\infty$ 有实线性因式 $z+\infty$. 由 $Z-\infty$ 和 $Z+\infty$ 都有实线性因式,知 $Z+C$ 也必有线性因式, 只要 $C$ 在 $+\infty$ 和 $-\infty$ 之间, 也即 $C$ 可以是任何一个实数, 或正, 或负, 或者为零. 因此 $C=0$ 时, 函数 $Z$ 有实线性 因式 $z-c$,数 $c$ 在 $+\infty$ 和 $-\infty$ 之间, 它或正, 或负,或者为零.

\section{$\S 35$}

如果整函数 $Z$ 中 $z$ 的最大指数是奇数, 那么, $Z$ 的实线性因式的个数, 必定或为 1 , 或 为 3 , 或为 5 , 或为 7 , 等等. 

前节证明了, 这个 $2 n+1$ 次整函数 $Z$ 至少有一个实线性因式 $z-c$. 如果 $Z$ 有另一个 因式 $z-d$, 那么, 用 $(z-c)(z-d)$ 除 $Z$ 得到的商, 是 $2 n-1$ 次整函数. 由前节知, 这商至 少有一个实线性因式, 即 $Z$ 的实线性因式个数, 如果多于 1 , 那它至少是 3 个. 类似地, 如 果多于 3 个, 那它至少是 5 个, 类推, 我们得到奇次整函数的实线性因式个数为奇数. 而这 个 $Z$ 的线性因式总个数为 $2 n+1$, 也为奇数, 可见它的虚线性因式的个数为偶数.

\section{$\S 36$}

$z$ 的最大指数为偶数 $2 n$ 的整函数 $Z$, 其实线性因式的个数, 必定或为 2 , 或为 4 , 或为 6 , 等等.

假定这个 $Z$ 有 $2 m+1$ 个实线性因式, $2 m+1$ 是奇数,那么,用这 $2 m+1$ 个因式的积 除 $Z$, 得到的商中 $z$ 的最大指数是 $2 n-2 m-1$, 这是个奇数. 因为这个商, 从而 $Z$ 有另外一 个实线性因式. 即 $Z$ 的实线性因式个数为 $2 m+2$, 为偶数. 加上前一节的结论, 我们又一 次证明了整函数的虚根个数为偶数.

\section{$\S 37$}

如果函数 $Z$ 中 $z$ 的最大指数为偶数, 且常数 $A$ 的符号为负, 那么, 这个 $Z$ 至少有两个 实线性因式.

这里的这个 $Z$ 形状为
\[
z^{2 n} \pm \alpha z^{2 n-1} \pm \beta z^{2 n-2} \pm \cdots \pm \gamma z-A
\]
前面讲过, 令 $z=\infty$, 则 $Z=\infty$, 这里令 $z=0$, 则 $Z=-A$. 因而 $Z-\infty$ 有实因式 $z-\infty ; Z+$ $A$ 有因式 $z-0$. 从而由 0 位于 $-\infty$ 与 $A$ 之间, 知 $Z+0$ 必有实线性因式 $z-c, c$ 在 0 与 $\infty$ 之间. 又由 $z=-\infty$ 时, $Z=\infty$, 得 $Z-\infty$ 有因式 $z+\infty, Z+A$ 有因式 $z+0$. 这样, 我们得 到 $Z+0$ 有实线性因式 $z+d, d$ 在 0 与无穷之间. 命题得证. 我们得到了:对于这里的 $Z$, 方 程 $Z=0$ 至少有两个实根,一正一负. 例如,方程
\[
z^{4}+\alpha z^{3}+\beta z^{2}+\gamma z-a^{2}=0
\]
就有一正一负两个实根.

\section{$\S 38$}

分数函数, 如果分子中 $z$ 的最高次数比分母中的不小,,那么, 这个分数函数可以表示 成一个整函数与一个新的分数函数这样两部分的和. 这个新的分数函数, 分子中 $z$ 的最 高次数比分母中的小.

如果分母中 $z$ 的最高次数比分子中的小, 那么用分母按通常的方法除分子, 除到商 中的幂要为负数时停止. 这个商就由一个整函数和一个分数函数组成, 且分数函数分子 中 $z$ 的最高次数比分母的低. 例如对分数函数 $\frac{1+z^{4}}{1+z^{2}}$, 作除法, 得 
\[
\frac{1+z^{4}}{1+z^{2}}=z^{2}-1+\frac{2}{1+z^{2}}
\]
仿照算术,我们称分子次数比分母次数不小的分数函数为假分数函数, 以区别于分子次 数小于分母次数的真分数函数. 从而本节所讲可陈述为假分数函数可表示成一个整函数 与一个真分数函数的和. 新的表示用通常的除法得到.

\section{$\S 39$}

分母是两个互质因式乘积的分数函数, 可分解成分别以这两个因式为分母的两个分 数函数之和.

虽然这里的分解方法对真假分数函数都适用,但我们主要是把它用于真分数函数, 即可以把分母是两个互质因式的真分数函数, 分解成分别以这两个因式为分母的两个新 的真分数函数之和, 且分法唯一. 我们只举例, 不证明, 从例子完全可以看清. 考虑分数函 数
\[
\frac{1-2 z+3 z^{2}-4 z^{3}}{1+4 z^{4}}
\]
它的分母 $1+4 z^{4}$ 等于乘积
\[
\left(1+2 z+2 z^{2}\right)\left(1-2 z+2 z^{2}\right)
\]
我们要做的是, 把该函数分解成分别以 $1+2 z+2 z^{2}$ 和 $1-2 z+2 z^{2}$ 为分母的两个分数函 数的和. 为了求出这两个真分数函数, 我们假定它们的分子分别为 $\alpha+\beta z$ 和 $\gamma+\delta z$. 根据 假定我们有
\[
\frac{1-2 z+3 z^{2}-4 z^{3}}{1+4 z^{4}}=\frac{\alpha+\beta z}{1+2 z+2 z^{2}}+\frac{\gamma+\delta z}{1-2 z+2 z^{2}}
\]
把右端的两个分数函数加起来, 结果得
\[
\begin{array}{l|l}
\text { 分子为 } \\
\alpha-2 \alpha z+2 \alpha z^{2}+\beta z-2 \beta z^{2}+2 \beta z^{3}+ & \\
\gamma+2 \gamma z+2 \gamma z^{2}+\delta z+2 \delta z^{2}+2 \delta z^{3}
\end{array} \mid 1+4 z^{4}
\]
求得的和, 其分母与原分数函数的相等, 因而分子也应相等. 由于末知数( 即 $\alpha, \beta, \gamma$, $\delta)$ 的个数与分子的项数相等, 我们唯一地得到如下的四个方程
1) $\alpha+\gamma=1$
2) $-2 \alpha+\beta+2 \gamma+\delta=-2$
3) $2 \alpha-2 \beta+2 \gamma+2 \delta=3$
4) $2 \beta+2 \delta=-4$

1) 和 4) 分别给出 $\alpha+\gamma=1$ 和 $\beta+\delta=-2$. 由 2) 和 4) 得 $\alpha-\gamma=0$, 由 1) 和 3 ) 得 $\delta-$ $\beta=1$. 最后我们得到
\[
\alpha=\frac{1}{2}, \gamma=\frac{1}{2}, \beta=-\frac{5}{4}, \delta=-\frac{3}{4}
\]
从而函数 
\[
\frac{1-2 z+3 z^{2}-4 z^{3}}{1+4 z^{4}}
\]
分解成了
\[
\frac{\frac{1}{2}-\frac{5}{4} z}{1+2 z+2 z^{2}}+\frac{\frac{1}{2}-\frac{3}{4} z}{1-2 z+2 z^{2}}
\]
由于引进的末知数的个数恒等于分子的项数, 所以一定求得出我们所要的解. 但要记住 一点, 一个前提是分母的因式必须是互质的.

\section{$\S 40$}

分数函数 $\frac{M}{N}$ 可以分解成 $N$ 的不相同线性因式个数那么多个, 状如 $\frac{A}{p-q z}$ 的简分式.

设分数函数 $\frac{M}{N}$ 为真分数函数, 即 $M, N$ 都是整函数,且 $M$ 的次数比 $N$ 的小,那么当 $N$ 分解成了不相同线性因式时, 表达式 $\frac{M}{N}$ 就可以分解成 $N$ 的不相同线性因式个数那么多个 分式, 因为每一个因式都是一个部分分式的分母. 因此, 如果 $p-q z$ 是 $N$ 的一个因式, 它必 定是一个部分分式的分母, 而这个分式分子的次数应该小于分母 $p-q z$ 的次数, 所以分子 应该是常数. 从而我们得到分母 $p+q z$ 对应的简分式为 $\frac{A}{p-q z}$. 这种分式全体的和等于分 数函数 $\frac{M}{N}$.

例: 我们考虑分数函数
\[
\frac{1+z^{2}}{z-z^{3}}
\]
分母的线性因式为 $z, 1-z$ 和 $1+z$, 该函数可分解为三个简分式
\[
\frac{A}{z}+\frac{B}{1-z}+\frac{C}{1+z}=\frac{1+z^{2}}{z-z^{3}}
\]
作为分子的常数 $A, B, C$ 待定. 求这三个简分式的和, 得公分母为 $z-z^{3}$. 由分子的和应该 等于 $1+z^{2}$, 我们得到方程
\[
\left.\begin{array}{l}
A+B z-A z^{2}+ \\
C z+B z^{2} \\
-C z^{2}
\end{array}\right\}=1+z^{2}=1+0 \cdot z+z^{2}
\]
比较两端, 得到关于末知数 $A, B, C$ 的方程
1) $A=1$
2) $B+C=0$
3) $-A+B-C=1$

方程的个数与末知数的个数相等. 将 1 ) 代入 3 ) 得 $B-C=2$, 从而 
\[
A=1, B=1, C=-1
\]
这样我们得到
\[
\frac{1+z^{2}}{z-z^{3}}
\]
的分解式为
\[
\frac{1}{z}+\frac{1}{1-z}-\frac{1}{1+z}
\]
不管分数函数 $\frac{M}{N}$ 的分母 $N$ 有多少个线性因式,只要它们都不相同,就都可以用这样的方 式把它分解为分母因式个数那么多个简分式. 如果分母的因式中有相同的, 那时的分解 方法与这里的不同, 对此, 将在后面的内容中作介绍.

\section{$\S 41$}

分母 $N$ 的每一个线性因式都给出分数函数 $\frac{M}{N}$ 分解式的一个简分式. 这一节我们讲单 个简分式的求法.

设 $p-q z$ 是 $N$ 的一个线性因式,即
\[
N=(p-q z) S
\]
这里 $S$ 是 $z$ 的整函数. 记 $p-q z$ 和 $S$ 所对应的分式分别为 $\frac{A}{p-q z}$ 和 $\frac{P}{S}$, 那么由 $\S 39$ 我们有
\[
\frac{M}{N}=\frac{A}{p-q z}+\frac{P}{S}=\frac{M}{(p-q z) S}
\]
从而
\[
\frac{P}{S}=\frac{M-A S}{(p-q z) S}
\]
由该等式知 $\frac{M-A S}{p-q z}$ 等于 $P$, 即 $p-q z$ 是 $M-A S$ 的因式. 由此知 $z=\frac{p}{q}$ 时 $M-A S$ 为零, 也 即将 $M$ 和 $S$ 中的 $z$ 换为 $\frac{p}{q}$ 时 $M-A S=0$, 从而 $A=\frac{M}{S}$. 这样我们得到了分式 $\frac{A}{p-q z}$ 的分子 $A$, 等于 $z=\frac{p}{q}$ 时 $\frac{M}{S}$ 的值. 当 $\frac{M}{N}$ 的分母 $N$ 分解成了不同的线性因式时,求出对应于每一个 因式的简分式,分式 $\frac{M}{N}$ 就等于这些简分式的和.

例如考虑上节举过的例子
\[
\frac{1+z^{2}}{z-z^{3}}
\]
这里 $M=1+z^{2}, N=z-z^{3}$. 取线性因式 $z$, 则 $S=1-z^{2}$. 当 $z=0$ 时, $A=\frac{1+z^{2}}{1-z^{2}}=1$, 简分式 

为 $\frac{1}{z}$; 取因式 $1-z$, 则 $S=z+z^{2}$. 当 $1-z=0$ 时, $A=\frac{1+z^{2}}{z+z^{3}}=1$, 简分式为 $\frac{1}{1-z}$; 取因式 $1+z$, 则 $S=z-z^{2}$. 当 $1+z=0$ 时, $A=\frac{1+z^{2}}{z-z^{2}}=-1$, 简分式为 $\frac{-1}{1+z}$. 最后得
\[
\frac{1+z^{2}}{z-z^{3}}=\frac{1}{z}+\frac{1}{1-z}-\frac{1}{1+z}
\]
与上节求得的结果一致.

\section{$\S 42$}

$P$ 的次数小于 $(p-q z)^{n}$ 的次数时, 分数函数 $\frac{P}{(p-q z)^{n}}$ 可分解为部分分式的和
\[
\frac{A}{(p-q z)^{n}}+\frac{B}{(p-q z)^{n-1}}+\frac{C}{(p-q z)^{n-2}}+\cdots+\frac{K}{p-q z}
\]
这里的分子全部是常数.

由 $P$ 的次数小于 $n$, 知 $P$ 的形状为
\[
a+\beta z+\gamma z^{2}+\delta z^{3}+\cdots+\chi z^{n-1}
\]
共 $n$ 项. $P$ 应该等于部分分式和的分子. 部分分式和的分母为 $(p-q z)^{n}$, 分子为
\[
A+B(p-q z)+C(p-q z)^{2}+D(p-q z)^{3}+\cdots+K(p-q z)^{n-1}
\]
这分子的次数为 $n-1$, 末知数 (即 $A, B, C, \cdots, K$ ) 的个数为 $n$, 与 $P$ 的项数相同. 这样由
\[
\begin{aligned}
\frac{P}{(p-q z)^{2}}= & \frac{A}{(p-q z)^{n}}+\frac{B}{(p-q z)^{n-1}}+\frac{C}{(p-q z)^{n-2}}+ \\
& \frac{D}{(p-q z)^{n-3}}+\cdots+\frac{K}{(p-q z)}
\end{aligned}
\]
我们可以求出 $A, B, C, \cdots$. 下面我们讲它们的求法.

\section{$\S 43$}

对应于因式 $(p-q z)^{2}$ 的部分分式的求法.

$\S 41$ 讲了对应于不相重线性因式的部分分式的求法. 现在我们假定分母 $N$ 有相同的 两个线性因式. 即有因式 $(p-q z)^{2}$. 上一节告诉我们, 对应于该因式的部分分式形状为
\[
\frac{A}{(p-q z)^{2}}+\frac{B}{p-q z}
\]
记
\[
N=(p-q z)^{2} S
\]
则
\[
\frac{M}{N}=\frac{M}{(p-q z)^{2} S}=\frac{A}{(p-q z)^{2}}+\frac{B}{p-q z}+\frac{P}{S}
\]
$\frac{P}{S}$ 是对应于分母其余因式的部分分式的总和. 由上面等式得
\[
\frac{P}{S}=\frac{M-A S-B(p-q z) S}{(p-q z)^{2} S}
\]
从而
\[
P=\frac{M-A S-B(p-q z) S}{(p-q z)^{2}}=\text { 整函数 }
\]
由此得知 $(p-q z)^{2}$ 为 $M-A S-B(p-q z) S$ 的因式. 当然 $p-q z$ 也为它的因式. 因此 $p-$ $q z=0$, 也即 $z=\frac{p}{q}$ 时 $M-A S-R(p-q z) S=0$, 从而
\[
M-A S=0, A=\frac{M}{S}
\]
也即 $A$ 等于 $z=\frac{p}{q}$ 时 $\frac{M}{S}$ 的值. $A$ 求了, 我们再来求 $B$. 由 $M-A S-B(p-q z) S$ 被 $(p-q z)^{2}$ 除得尽, 知 $p-q z$ 为 $\frac{M-A S}{p-q z}-B S$ 的因式. 从而 $z=\frac{p}{q}$ 时
\[
\frac{M-A S}{p-q z}=B S
\]
由此得
\[
B=\frac{M-A S}{(p-q z) S}=\frac{1}{p-q z}\left(\frac{M}{S}-A\right)
\]
要指出的是, 因为 $M-A S$ 被 $p-q z$ 除得尽, 所以应先做除法, 然后再将 $z$ 换为 $\frac{p}{q}$. 或者, 令
\[
\frac{M-A S}{p-q z}=T
\]
则 $B$ 等于 $z=\frac{p}{q}$ 时 $\frac{T}{S}$ 的值.

有了 $A, B$, 我们就可以写出对应于因式 $(p-q z)^{2}$ 的部分分式
\[
\frac{A}{(p-q z)^{2}}+\frac{B}{p-q z}
\]
例 1 考虑分数函数
\[
\frac{1-z^{2}}{z^{2}\left(1+z^{2}\right)}
\]
这里二重因式为 $z^{2}, S=1+z^{2}, M=1-z^{2}$. 记 $z^{2}$ 产生的部分分式为
\[
\frac{A}{z^{2}}+\frac{B}{z}
\]
则 
\[
A=\frac{M}{S}=\frac{1-z^{2}}{1+z^{2}}
\]
置 $z=0$, 得
\[
A=1
\]
又 $M-A S=-2 z^{2}$. 除它以线性因式 $z$, 得 $T=-2 z$. 从而
\[
B=\frac{T}{S}=\frac{-2 z}{1+z^{2}}
\]
置 $z=0$, 得
\[
B=0
\]
这样由 $z^{2}$ 产生的是一个单个的部分分式 $\frac{1}{z^{2}}$.

例 2 考虑分数函数
\[
\frac{z^{3}}{(1-z)^{2}\left(1+z^{4}\right)}
\]
二重因式为 $(1-z)^{2}$, 它产生的部分分式形状为
\[
\frac{A}{(1-z)^{2}}+\frac{B}{1-z}
\]
这里 $M=z^{3}, S=1+z^{4}$, 则
\[
A=\frac{M}{S}=\frac{z^{3}}{1+z^{4}}
\]
置 $1-z=0$, 即 $z=1$, 得
\[
A=\frac{1}{2}
\]
又
\[
M-A S=Z^{3}-\frac{1}{2}-\frac{1}{2} z^{4}=-\frac{1}{2}+z^{3}-\frac{1}{2} z^{4}
\]
除它以 $1-z$, 得
\[
T=-\frac{1}{2}-\frac{1}{2} z-\frac{1}{2} z^{2}+\frac{1}{2} z^{3}
\]
从而
\[
B=\frac{T}{S}=\frac{-1-z-z^{2}+z^{3}}{2+2 z^{4}}
\]
置 $z=1$, 得
\[
B=\frac{-1}{2}
\]
所求部分分式为
\[
\frac{1}{2(1-z)^{2}}-\frac{1}{2(1-z)}
\]
%%02p021-040
\section{$\S 44$}

再讲对应于因式 $(p-q z)^{3}$ 的部分分式的求法. 即分数函数 $\frac{M}{N}$ 的对应于 $N$ 的因式 $(p-$ $q z)^{3}$ 的部分分式
\[
\frac{A}{(p-q z)^{3}}+\frac{B}{(p-q z)^{2}}+\frac{C}{p-q z}
\]
的求法.

记
\[
N=(p-q z)^{3} S
\]
记 $S$ 产生的分式为 $\frac{P}{S}$, 则
\[
P=\frac{M-A S-B(p-q z) S-C(p-q z)^{2} S}{(p-q z)^{3}}=\text { 整函数 }
\]
因而 $M-A S-B(p-q z) S-C(p-q z)^{2} S$ 被 $p-q z$ 除得尽. 从而 $p-q z=0$, 也即 $z=\frac{p}{q}$ 时 它为零. 由此得 $z=\frac{p}{q}$ 时 $M-A S=0$, 即 $A$ 等于 $z=\frac{p}{q}$ 时 $\frac{M}{S}$ 的值.

由求得的 $A, M-A S$ 被 $p-q z$ 除得尽, 令
\[
\frac{M-A S}{p-q z}=T
\]
则 $T-B S-C(p-q z) S$ 被 $(p-q z)^{2}$ 除得尽. 从而 $p-q z=0$ 时 $T-B S=0$. 由此得 $z=\frac{p}{q}$ 时 $B=\frac{T}{S}$, 也即 $B$ 等于 $z=\frac{p}{q}$ 时 $\frac{T}{S}$ 的值.

对求出的 $B, T-B S$ 被 $p-q z$ 除得尽. 令
\[
\frac{T-B S}{p-q z}=V
\]
则 $V-C S$ 被 $p-q z$ 除得尽, 即 $p-q z=0$ 时 $V-C S=0$. 从而 $z=\frac{p}{q}$ 时 $C=\frac{V}{S}$. 即 $C$ 等于 $z=\frac{p}{q}$ 时 $\frac{V}{S}$ 的值.

至此 $A, B, C$ 全求了出来, 也即求出了由分母的因式 $(p-q z)^{3}$ 产生的部分分式
\[
\frac{A}{(p-q z)^{3}}+\frac{B}{(p-q z)^{2}}+\frac{C}{p-q z}
\]
例 3 考虑函数
\[
\frac{z^{2}}{(1-z)^{3}\left(1+z^{2}\right)}
\]
分母的三重因式 $(1-z)^{2}$ 产生的部分分式形状为
\[
\frac{A}{(1-z)^{3}}+\frac{B}{(1-z)^{2}}+\frac{C}{1-z}
\]
对于该分数函数我们有 $M=z^{2}, S=1+z^{2}$. 首先 $1-z=0$ 或 $z=1$ 时
\[
A=\frac{Z^{2}}{1+z^{2}}=\frac{1}{2}
\]
其次,令
\[
T=\frac{M-A S}{1-z}=\frac{\frac{1}{2} z^{2}-\frac{1}{2}}{1-z}=-\frac{1}{2}-\frac{1}{2} z
\]
得 $z=1$ 时
\[
B=\frac{-\frac{1}{2}-\frac{1}{2} z}{1+z^{2}}=-\frac{1}{2}
\]
再次,令
\[
V=\frac{T-B S}{1-z}=\frac{T+\frac{1}{2} S}{1-z}
\]
得
\[
V=\frac{-\frac{1}{2} z+\frac{1}{2} z^{2}}{1-z}=-\frac{1}{2} z
\]
从而 $z=1$ 时
\[
C=\frac{V}{S}=\frac{-\frac{1}{2} z}{1+z^{2}}=-\frac{1}{4}
\]
我们求得了对应于分母因式 $(1-z)^{3}$ 的部分分式为
\[
\frac{1}{2(1-z)^{3}}-\frac{1}{2(1-z)^{2}}-\frac{1}{4(1-z)}
\]
\section{$\S 45$}

现在讲对应于因式 $(p-q z)^{n}$ 的部分分式的求法, 即分数 $\frac{M}{N}$ 的对应于 $N$ 的因式 $(p-$ $q z)^{n}$ 的部分分式

的求法.
\[
\frac{A}{(p-q z)^{n}}+\frac{B}{(p-q z)^{n-1}}+\frac{C}{(p-q z)^{n-2}}+\cdots+\frac{K}{p-q z}
\]
记分母
\[
N=(p-q z)^{n} Z
\]
仿照前两节的推导, 我们依次得到

1) $A=\frac{M}{Z}$, 其中 $z=\frac{p}{q}$. 记 $P=\frac{M-A Z}{p-q z}$, 则

2) $B=\frac{P}{Z}$, 其中 $z=\frac{p}{q}$. 记 $Q=\frac{P-B Z}{p-q z}$, 则

3) $C=\frac{Q}{Z}$, 其中 $z=\frac{p}{q}$. 记 $R=\frac{Q-C Z}{p-q z}$, 则

4) $D=\frac{R}{Z}$, 其中 $z=\frac{p}{q}$. 记 $S=\frac{R-D Z}{p-q z}$, 则

5) $E=\frac{S}{Z}$,其中 $z=\frac{p}{q}$

类推. 照此法依次算出 $A, B, C, D$ 等, 我们就求得了 $\frac{M}{N}$ 的对应于分母 $N$ 的因式 $(p-q z)^{n}$ 的 部分分式.

例 4 考虑分数函数
\[
\frac{1+z^{2}}{z^{5}\left(1+z^{3}\right)}
\]
因式 $z^{5}$ 产生的部分分式形状为
\[
\frac{A}{z^{5}}+\frac{B}{z^{4}}+\frac{C}{z^{3}}+\frac{D}{z^{2}}+\frac{E}{z}
\]
这里
\[
M=1+z^{2}, Z=1+z^{3}, \frac{p}{q}=0
\]
依次进行计算得
\[
A=\frac{M}{Z}=\frac{1+z^{2}}{1+z^{3}}
\]
将 $z=0$ 代入, 得
\[
A=1
\]
记
\[
P=\frac{M-A Z}{z}=\frac{z^{2}-z^{3}}{z}=z-z^{2}
\]
则
\[
B=\frac{P}{Z}=\frac{z-z^{2}}{1+z^{3}}
\]
将 $z=0$ 代入, 得
\[
B=0
\]
记
\[
Q=\frac{P-B Z}{z}=\frac{z-z^{2}}{z}=1-z
\]
则 
\[
C=\frac{Q}{Z}=\frac{1-z}{1+z^{3}}
\]
将 $z=0$ 代入,得
\[
C=1
\]
记
\[
R=\frac{Q-C Z}{z}=\frac{-z-z^{3}}{z}=-1-z^{2}
\]
则
\[
D=\frac{R}{Z}=\frac{-1-z^{2}}{1+z^{3}}
\]
将 $z=0$ 代入, 得
\[
D=-1
\]
记
\[
S=\frac{R-D Z}{z}=\frac{-z^{2}+z^{3}}{z}=-z+z^{2}
\]
则
\[
E=\frac{S}{Z}=\frac{-z+z^{2}}{1+z^{3}}
\]
将 $z=0$ 代入,得
\[
E=0
\]
所求部分分式为
\[
\frac{1}{z^{5}}+\frac{0}{z^{4}}+\frac{1}{z^{3}}-\frac{1}{z^{2}}+\frac{0}{z} \\
\]
\section{$\S 45a$}

一般分数函数 $\frac{M}{N}$ 的部分分式的求法.

先将分母分解成线性因式,这线性因式也可以是虚的,对单重因式, 用 $\S 41$ 的方法 逐个求出它们所对应的部分分式. 如果线性因式中有二重的或更多重的, 则把相同的集 中在一起,写成 $(p-q z)^{n}$ 的形式. 然后用 $\S 45$ 的方法求出它对应的部分分式. 这样我们 就把对应于每一个线性因式的部分分式都求出来了. 把求得的部分分式加起来, 就得到 函数 $\frac{M}{N}$ 的部分分式. 这里假定 $\frac{M}{N}$ 为真分式. 如果 $\frac{M}{N}$ 为假分式, 则应先把整函数部分分离 出来,然后求出真分式部分的部分分式. 最后把两部分加起来就得到 $\frac{M}{N}$ 的最简表达式.

(1) 欧拉原书误编了两个 $\S 46$, 参照俄译本改这第一个 $\S 46$ 为 $\S 45 \mathrm{a}$, 保留下一个 — 中译者. 说明一点, 对假分式也可以先不分离出整函数部分, 而直接求部分分式. 事实上, 从前面 给出的部分分式求法中我们看到, 分子 $M$ 乘上或除上分母的整倍数, 对结果都不产生影 响.

例 5 求函数
\[
\frac{1}{z^{3}(1-z)^{2}(1+z)}
\]
的最简表达式.

先取单因式 $1+z$. 此时 $\frac{p}{q}=-1, M=1, Z=z^{3}-2 z^{4}+z^{5}$. 对应于 $1+z$ 的部分分式形 状为 $\frac{A}{1+z}$.

由 $A=\frac{1}{z^{3}-2 z^{4}+z^{5}}$, 将 $z=-1$ 代入, 得 $A=-\frac{1}{4}$, 即 $1+z$ 产生的部分分式为
\[
\frac{-1}{4(1+z)}
\]
再取重因式 $(1-z)^{2}$, 此时 $\frac{p}{q}=1, M=1, Z=z^{3}+z^{4}$. 对应于该重因式的部分分式形状 为
\[
\frac{A}{(1-z)^{2}}+\frac{B}{1-z}
\]
由 $A=\frac{1}{z^{3}+z^{4}}$, 将 $z=1$ 代入, 得 $A=\frac{1}{2}$.

记
\[
P=\frac{M-\frac{1}{2} Z}{1-z}=\frac{1-\frac{1}{2} z^{3}-\frac{1}{2} z^{4}}{1-z}=1+z+z^{2}+\frac{1}{2} z^{3}
\]
则
\[
B=\frac{P}{Z}=\frac{1+z+z^{2}+\frac{1}{2} z^{3}}{z^{3}+z^{4}}
\]
将 $z=1$ 代入,得
\[
B=\frac{7}{4}
\]
即 $(1-z)^{2}$ 产生的部分分式为

最后取三重因式 $z^{3}$, 此时
\[
\frac{1}{z(1-z)^{2}}+\frac{7}{4(1-z)}
\]
\[
\frac{p}{q}=0, M=1, Z=1-z-z^{2}+z^{3}
\]
所求部分分式的形状为 

由

将 $z=0$ 代入, 得

记

则
\[
\begin{gathered}
\frac{A}{z^{3}}+\frac{B}{z^{2}}+\frac{C}{z} \\
A=\frac{M}{Z}=\frac{1}{1-z-z^{2}+z^{3}}
\end{gathered}
\]
\[
A=1
\]
\[
\begin{aligned}
& P=\frac{M-Z}{z}=1+z-z^{2}
\end{aligned}
\]
\[
B=\frac{P}{Z}=\frac{1+z-z^{2}}{1-z-z^{2}+z^{3}}
\]
将 $z=0$ 代入,得
\[
B=1
\]
记
\[
Q=\frac{P-Z}{z}=2-z^{2}
\]
则
\[
C=\frac{Q}{Z}
\]
将 $z=0$ 代入,得
\[
C=2
\]
结果得函数
\[
\frac{1}{z^{3}(1-z)^{2}(1+z)}
\]
的分解式为
\[
\frac{1}{z^{3}}+\frac{1}{z^{2}}+\frac{2}{z}+\frac{1}{2(1-z)^{2}}+\frac{7}{4(1-z)}-\frac{1}{4(1+z)}
\]
没有整函数部分, 因为函数不是假分式. 

