\chapter{第十七章 应用递推级数求根}

\section{$\S 332$}

著名数学家丹尼尔 -贝努里有一篇研究任意次方程求根的文章, 发表在彼得堡科学 院通报第三卷上. 这篇文章给出了一种用递推级数求代数方程根的近似值的方法, 近似 程度很高. 本章我们详细介绍这一方法, 它常常很有用, 但对有的方程, 这个方法无效, 用 它求不出根. 我们先考虑递推级数与这个方法有密切关系的一些性质, 以便能有效地应 用这一方法.

\section{$\S 333$}

递推级数都由有理分式产生. 设分式
\[
\frac{a+b z+c z^{2}+d z^{3}+e z^{4}+\cdots}{1-\alpha z-\beta z^{2}-\gamma z^{3}-\delta z^{4}-\cdots}
\]
产生的递推级数为
\[
A+B z+C z^{2}+D z^{3}+E z^{4}+F z^{5}+\cdots
\]
则系数
\[
\begin{gathered}
A=a \\
B=\alpha A+b \\
C=\alpha B+\beta A+c \\
D=\alpha C+\beta B+\gamma A+d \\
E=\alpha D+\beta C+\gamma B+\delta A+e
\end{gathered}
\]
第八章我们讲了, 通项, 也即 $z^{n}$ 的系数的求法是: 先把有理分式表示成部分分式, 部分分 式是以分母
\[
1-\alpha z-\beta z^{2}-\gamma z^{3}-\cdots
\]
的因式为分母的公式. 

\section{$\S 334$}

通项主要决定于分母线性因式的性质,决定于线性因式中有无虚的,有无相同的. 我 们对不同情形分别进行讨论, 先考虑分母的线性因式都是实的, 且都不相同的情形. 记此 时分母的线性因式为
\[
(1-p z)(1-q z)(1-r z)(1-s z) \cdots
\]
记所给分数分解成的部分分式为
\[
\frac{\mathfrak{X}}{1-p z}+\frac{\mathfrak{B}}{1-q z}+\frac{\mathfrak{S}}{1-r z}+\frac{\mathfrak{D}}{1-s z}+\cdots
\]
则递推级数的通项为
\[
z^{n}\left(\mathfrak{X} p^{n}+\mathfrak{B} q^{n}+\left(\mathfrak{C}^{n}+\mathfrak{D} s^{n}+\cdots\right)\right.
\]
记它为 $P z^{n}$, 也即记 $z^{n}$ 的系数为 $P$. 记 $P z^{n}$ 后继项的系数为 $Q, R, \cdots$, 则递推级数为
\[
A+B z+C z^{2}+D z^{3}+\cdots+P z^{n}+Q z^{n+1}+R z^{n+2}+\cdots
\]
\section{$\S 335$}

我们继续写这递推级数的项到很多, 也即让 $n$ 很大. 两数, 一大一小, 大数的幂比小 数的幂更大. 设不相等的 $p, q, r, \cdots$ 中 $p$ 最大, 那么 $n$ 很大时, 与 $\mathfrak{A} p^{n}$ 相比较, $\mathfrak{B} q^{n}, \mathfrak{C}^{n} r^{n}, \cdots$ 都可忽略不计. 因而, $n$ 很大时, 我们可以取, 或者至少近似地可以取
\[
P=\mathfrak{A} p^{n}
\]
类似地,可以取
\[
Q=\mathfrak{X} p^{n+1}
\]
从而
\[
\frac{Q}{P}=p
\]
由此可见, 级数继续到很多项时, 这第很多项与其前一项的比, 就是 $p, q, r, \cdots$ 中最大 的 $p$ 的近似值.

\section{$\S 336$}

这样,如果分式
\[
\frac{a+b z+c z^{2}+d z^{3}}{1-\alpha z-\beta z^{2}-\gamma z^{3}-\delta z^{4}-\cdots}
\]
分母的线性因式都是实的, 且不相同, 又如果分母的线性因式中 $z$ 的最大系数为 $p$, 那么, 从分式的递推级数, 我们就可以求出线性因式 $1-p z$, 在求该因式的过程中, 分子的系数 $a, b, c, d, \cdots$ 不起作用. 事实上, 不管分子的系数取什么值, 求出的最大数 $p$ 的值都是相同 的. $n$ 很大时, 我们得到 $p$ 的近似值, $n$ 越大近似程度越好, $p$ 比 $q, r, s, \cdots$ 它们大得越多, 近似程度也越好, $n$ 趋向无穷时, 我们得到 $p$ 的真值. 最后, $p$ 为正为负, 我们的方法是一样 的,因为 $p$ 为正为负, 其幂都是增加的.

\section{$\S 337$}

以上我们讲了应用递推级数求代数方程根的方法. 知道了分母
\[
1-\alpha z-\beta z^{2}-\gamma z^{3}-\delta z^{4}-\cdots
\]
的因式,也就知道了方程
\[
1-\alpha z-\beta z^{2}-\gamma z^{3}-\delta z^{4}-\cdots=0
\]
的根. $1-p z$ 为因式, 则 $z=\frac{1}{p}$ 为根. 用递推级数求得的是最大的数 $p$, 因而得到的是方程
\[
1-\alpha z-\beta z^{2}-\gamma z^{3}-\cdots=0
\]
的最小的根. 令 $z=\frac{1}{x}$, 方程化为
\[
x^{m}-\alpha x^{m-1}-\beta x^{m-2}-\gamma x^{m-3}-\cdots=0
\]
那么,我们得到的就是这个方程最大的根 $x=p$.

\section{$\S 338$}

这样一来,如果给了方程
\[
x^{m}-\alpha x^{m-1}-\beta x^{m-2}-\gamma x^{m-3}-\cdots=0
\]
并已知其根都为实数, 且不相同, 那么最大根的求法是: 先根据所给方程的系数写出分式
\[
\frac{a+b z+c z^{2}+d z^{3}+\cdots}{1-\alpha z-\beta z^{2}-\gamma z^{3}-\cdots}
\]
再列出这个分式的递推级数, 分子或者级数前若干项的系数任意. 记列出的递推级数为
\[
A+B z+C z^{2}+D z^{3}+\cdots+P z^{n}+Q z^{n+1}+\cdots
\]
则分数 $\frac{Q}{P}$ 就是所给方程的最大根, $n$ 越大,近似程度越高.

例 1 求方程
\[
x^{2}-3 x-1=0
\]
的最大根.

先写出分式
\[
\frac{a+b z}{1-3 z-z^{2}}
\]
取该分式递推级数前两项的系数为 1 和 2 , 则级数的系数为
\[
1,2,7,23,76,251,829,2738, \cdots
\]
从而分数
\[
\frac{2738}{829}
\]
就是所给方程最大根的近似值. 化成小数, 为
\[
\text { 3. } 3027744
\]
最大真根为
\[
\frac{3+\sqrt{13}}{2}=3.3027756
\]
比我们求得的近似值只大百万分之一. 我们指出, 随 $n$ 依次增大, 分数 $\frac{Q}{P}$ 比真根大与比真 根小交替.

例 2 方程
\[
3 x-4 x^{3}=\frac{1}{2}
\]
的根是角的正弦,这每个角的三倍的正弦都为 $\frac{1}{2}$.

改写方程为
\[
1-6 x+8 x^{3}=0
\]
我们求它的最小根,因而不需换 $x$ 为 $\frac{1}{z}$. 写出分式
\[
\frac{a+b x+c x^{2}}{1-6 x+8 x^{3}}
\]
为便于递推级数后继系数的列出, 我们取开始三系数为 $0,0,1$. 这样, 递推级数的系数为
\[
0,0,1,6,36,208,1200,6912,39808,229248, \cdots
\]
最小根的近似值为
\[
\frac{39808}{229248}=\frac{311}{1791}=0.1736460
\]
真值为 $\sin 10^{\circ}$, 从三角函数表中查得 $\sin 10^{\circ}=0.1736482$, 比我们算出的值大 $\frac{22}{10000000}$.

令 $x=\frac{1}{2} y$, 所给方程化为
\[
1-3 y+y^{3}=0
\]
该方程的根求起来更容易. 类似地, 我们得到系数
\[
0,0,1,3,9,26,75,216,622,1791,5157, \cdots
\]
最小根的近似值为
\[
y=\frac{1791}{5157}=\frac{199}{573}=0.3472949
\]
从而
\[
x=\frac{y}{2}=0.1736475
\]
这后一个近似值与真值的差是前一个的约三分之一. 

例 3 求例 2 中方程
\[
0=1-6 x+8 x^{3}
\]
的最大根. 置 $x=\frac{y}{2}$, 得
\[
y^{3}-3 y+1=0
\]
由该方程产生的递推级数,其递推尺度为 $0,3,-1$, 任意取定开始三个系数,得到的级数 的系数为
\[
1,1,1,2,2,5,4,13,7,35,8,98,-11, \cdots
\]
这系数中有负值, 表明最大根是负的, 实际上, 最大根为
\[
x=-\sin 70^{\circ}=-0.9396926
\]
我们让任意的开始三系数中也包含负值, 例如
\[
1-2+4-7+14-25+49-89+172-316+605-\cdots
\]
由此得
\[
y=\frac{-605}{316}, x=-\frac{605}{632}=-0.957
\]
偏离真值太大.

\section{$\S 339$}

例 3 所得偏离真值太大, 其主要原因是方程的三个真根
\[
-\sin 70^{\circ},+\sin 50^{\circ},+\sin 10^{\circ}
\]
里面, $\sin 50^{\circ}$ 比最大根 $-\sin 70^{\circ}$ 小得太少. 在我们的计算中, $\sin 50^{\circ}$ 的幂与 $-\sin 70^{\circ}$ 的 幂相比较, 还没有达到可以忽略的程度. 偏离真值太大的另一个原因是, 求到的值随项的 推移而太大太小交替. 退一步取
\[
y=\frac{-316}{172}
\]
则
\[
x=\frac{-158}{172}=\frac{-79}{86}=-0.918
\]
这是因为最大根的幂正负交替, 因而第二个根的幂交替地与它相加相减. 要第二个根的 影响可以忽略, 就需求出级数的很多项.

\section{$\S 340$}

再一种补救的方法是, 作适当的交换,把根的距离拉开. 例如,方程
\[
0=1-6 x+8 x^{3}
\]
以 $-\sin 70^{\circ}, \sin 50^{\circ}, \sin 10^{\circ}$ 为根, 作代换 $x=y-1$, 则所得方程
\[
0=8 y^{3}-24 y^{2}+18 y-1
\]
以 $1-\sin 70^{\circ}, 1+\sin 50^{\circ}, 1+\sin 10^{\circ}$ 为根. 对应于原方程最大根 $-\sin 70^{\circ}$, 新方程中 $1-\sin 70^{\circ}$ 是最小根. 原方程中的中间根 $\sin 50^{\circ}$, 对应于新方程的最大根 $1+\sin 50^{\circ}$. 用 变换的方法, 可以把任何一个根变成最大根或最小根. 从而就可以用前面的方法求出它. 由于 $1-\sin 70^{\circ}$ 远小于另外两个根, 用递推级数可以很容易地算出它.

例 4 求方程
\[
0=8 y^{3}-24 y^{2}+18 y-1
\]
的最小根. 1 减去这个最小根得 $\sin 70^{\circ}$.
\[
\text { 令 } y=\frac{1}{2} z \text {, 得 }
\]
\[
0=z^{3}-6 z^{2}+9 z-1
\]
求最小根的递推级数, 其递推尺度为 $9,-6,1$; 求最大根的递推级数, 其递推尺度为 $6,-9,1$. 求最小根的递推级数,其系数为
\[
1,1,1,4,31,256,2122,17593,145861, \cdots
\]
$z$ 的近似值为
\[
z=\frac{17593}{145861}=0.12061483
\]
从而
\[
y=0.06030741
\]
由此得
\[
\sin 70^{\circ}=1-y=0.93969258
\]
甚至最后一位也是真值相同. 从这个例子中我们看到, 求根时变量替换的作用是很大的, 配合上变量替换,用递推级数法, 就不仅可以求出最大和最小根, 而且可以求出任何一个 根.

\section{$\S 341$}

给定一个方程, 已知它的一个根很靠近数 $k$. 这时令 $x-K=y$ 或 $x=y+K$, 我们得到 一个新方程. $x-K$ 是新方程的最小根. 因为它比别的根小很多, 所以可从递推级数很容 易地求得. 把 $K$ 加到求得的根上去, 就得到原方程的根. 这一技巧甚至在方程有复根时也 可以使用.

\section{$\S 342$}

特别地, 符号相反数值相等的两个根, 不用上节技巧, 它们中任何一个都不能从递推 级数求出. 比如方程有根 $p$ 和 $-p, p$ 为最大根. 此时即使把级数继续到无穷, 也求不出 $p$ 来. 我们来看一个具体例子. 方程
\[
x^{3}-x^{2}-5 x+5=0
\]
有根 $\sqrt{5}$ 和 $-\sqrt{5}$, 且 $\sqrt{5}$ 为最大根. 用来求最大根的级数, 其递推尺度为 $1,5,-5$, 其系数为
\[
1,2,3,8,13,38,63,188,313,938,1563, \cdots
\]
邻项比不趋向任何常数. 请注意, 隔项比趋向最大根的平方, 即近似地有
\[
5=\frac{1563}{313}=\frac{938}{188}=\frac{313}{63}
\]
实际上, 只要隔项比趋向常数, 这常数必为所求根的平方. 为了求出根 $x=\sqrt{5}$, 我们作替 换 $x=y+2$,得方程
\[
1-3 y-5 y^{2}-y^{3}=0
\]
产生该方程最小根的级数, 其系数为
\[
1,1,1,9,33,145,609,2585,10945, \cdots
\]
最小根的近似值为
\[
\frac{2585}{10945}=0.2361
\]
2. 2361 近似地等于原方程的最大根 $\sqrt{5}$.

\section{$\S 343$}

产生递推级数的分式, 其分子的选取完全随意, 但选取适当与否, 对计算的难易影响 很大, 照 $\S 334$ 的假定, 我们的递推级数, 其通项为
\[
z^{n}\left(\mathfrak{A} p^{n}+\mathfrak{B} q^{n}+\left(r^{n}+\cdots\right)\right.
\]
其中的 $\mathfrak{A}, \mathfrak{B}, \mathfrak{C}, \cdots$ 决定于分式的分子. $\mathfrak{U}$ 的大小决定着计算最大根 $p$ 的速度的快慢. $\mathfrak{A}$ 完全不出现时, 即使把级数延长得再远, 也求不出最大根. 分子中含有因式 $1-p z$ 时, 就是 这种不出现的情形. 此时 $1-p z$ 将从计算中消失. 例如, 方程
\[
x^{3}-6 x^{2}+10 x-3=0
\]
的最大根是 3. 取分式
\[
\frac{1-3 z}{1-6 z+10 z^{3}-3 z^{3}}
\]
则递推级数的递推尺度为 $6,-10,3$, 系数为
\[
1,3,8,21,55,144,377, \cdots
\]
邻项比不趋向 $\frac{1}{3}$, 实际上, 这个级数是由分式
\[
\frac{1}{1-3 z+z^{2}}
\]
展成的,邻项比趋向的是方程
\[
x^{3}-3 x+1=0
\]
的最大根.

%%13p241-260

\section{$\S 344$}

可以选取分子, 使得通过递推级数, 能够求出方程的任何一个根. 从分母中分出所求 根对应的因式, 取其余因式的乘积作分子, 就能达到我们的目的. 以刚才讨论过的方程为 例,如果取 $1-3 z+z^{2}$ 作分子,那么分式
\[
\frac{1-3 z+z^{2}}{1-6 z+10 z^{2}-3 z^{3}}
\]
产生的递推级数, 是一个几何级数,其系数为
\[
1,3,9,27,81,243, \cdots
\]
由此立即得到根 $x=3$. 实际上这分式是
\[
\frac{1}{1-3 z}
\]
可见, 如果取允许任选的开始几项成几何级数, 且等于方程的根的幂, 那么整个递推 级数将成为几何级数. 这个级数就给出方程的根. 这个根可以既不是最大根也不是最小 根.

\section{$\S 345$}

为保证从递推级数求得的根一定是最大的或最小的, 选取的分子应该与分母没有公 因式. 取 1 作分子就可以保证这一点. 这样, 级数的第一项为 1 , 后续项就完全由递推尺度 决定. 这样做我们一定可以得到方程的最大根或最小根.

例如,给定方程
\[
y^{3}-3 y+1=0
\]
我们求它的最大根. 递推尺度为 $0,3,-1$, 取第一项为 1 , 我们得到递推级数的系数为
\[
\begin{aligned}
& 1-0+3-1+9-9+28-27+90-109+297- \\
& 517+1000-1848+3517-6544+\cdots
\end{aligned}
\]
邻项比为负, 收玫于常数, 表明最大根是负的. 近似地有
\[
y=\frac{-6544}{3517}=-1.651741
\]
这个根应该等于 - 1.867 938 52. 这里收玫如此之慢的原因, 前面讨论过, 是因为有另一 个数值比最大根小得不多的正根存在.

关于递推级数求根法, 我们讲了一般道理,讨论了会引起困难的特殊情况, 介绍了提 高计算速度的技巧, 还举了一些例子, 大家已经看清楚了这一方法的作用, 剩下要讲的还 有两点, 即方程有重根和虚根的情形. 假定分式 
\[
\begin{gathered}
\text { Sulfinite analyjisis } \\
\qquad \frac{a+b z+c z^{2}+d z^{3}+\cdots}{1-\alpha z-\beta z^{2}-\gamma z^{3}-\delta z^{4}-\cdots}
\end{gathered}
\]
的分母含有因式 $(1-p z)^{2}$, 其余的因式 $1-q z, 1-r z, \cdots$ 都是单重的. 该分式递推级数的 通项为
\[
z^{n}\left((n+1) \mathfrak{X} p^{n}+\mathfrak{B} p^{n}+\left(5 q^{n}+\cdots\right)\right.
\]
$n$ 增大, 我们看看这通项在 $p$ 是和不是最大根时的情形. $p$ 是最大根, 由于系数 $n+1$ 的存 在, $\mathfrak{B} p^{n}+\mathfrak{S} q^{n}+\cdots$ 消失得不如单根时快; $p$ 不是最大根, 比如 $q>p$, 此时 $(n+1) \mathfrak{A} p^{n}$ 的 消失, 与 $\mathfrak{5} q^{n}$ 相比也不快. 总之, 有重根时, 求最大根的计算量要更大.

例 5 方程
\[
x^{3}-3 x^{2}+4=0
\]
重根 $z$ 为最大根.

我们用前面的方法来求它的最大根. 考虑分式
\[
\frac{1}{1-3 z+4 z^{3}}
\]
其递推级数的系数为
\[
1,3,9,23,57,135,313,711,1593, \cdots
\]
我们看到用前项除后项, 商都大于 2 . 这原因可从通项中找. 从 $z^{n}$ 的系数中去掉 $\varsigma^{n} \cdots$, 剩 下的为
\[
(n+1) \mathfrak{X} p^{n}+\mathfrak{B} p^{n}
\]
从 $z^{n+1}$ 的系数去掉相应部分, 剩下的为
\[
(n+2) \mathfrak{X} p^{n+1}+\mathfrak{B} p^{n+1}
\]
后被前除,得
\[
\frac{(n+2) \mathfrak{U}+\mathfrak{B}}{(n+1) \mathfrak{U}+\mathfrak{B}^{p}}
\]
只要 $n$ 不增加到无穷, 该式恒大于 $p$.

例 6 方程
\[
x^{3}-x^{2}-5 x-3=0
\]
中 $-1$ 为重根, 最大根为 3 .

我们用递推级数求最大根, 递推尺度为 $1,5,3$, 系数为
\[
1,1,6,14,47,135,412,1228, \cdots
\]
这个级数很快地就给出 3 . 这是因为 $-1$ 的幂即使乘上 $n+1$, 与 3 的幂相比, 变小的速度 也是快的.

例 7 方程
\[
x^{3}+x^{2}-8 x-12=0
\]
根为 $3,-2,-2$.

对该方程使用递推级数求根法, 最大根的出现要慢得多. 递推级数的系数为
\[
1,-1,9,-5,65,3,457,347,3345,4915, \cdots
\]
要得到根 3 , 需将这系数继续到很远. 

\section{$\S 347$}

类似地, 如果有三个因式相同, 即分母的一个因式为 $(1-p z)^{3}$, 其余的因式为 $1-q z$, $1-r z, \cdots$, 则递推级数的通项为
\[
z^{n}\left(\frac{(n+1)(n+2)}{1 \cdot 2} 9\left(p^{n}+(n+1) \mathfrak{B} p^{n}+5 p^{n}+\mathfrak{D} q^{n}+5 r^{n}+\cdots\right)\right.
\]
如果 $p$ 是最大根, 又如果 $n$ 够大, 使得幂 $q^{n}, r^{n}, \cdots$ 与 $p^{n}$ 相比可以忽略. 那么从递推级数, 我 们得到根的近似值为
\[
\frac{\frac{1}{2}(n+2)(1+3) \mathfrak{A}+(n+2) \mathfrak{B}+\mathfrak{5}}{\frac{1}{2}(n+1)(n+2) \mathfrak{M}+(n+1) \mathfrak{B}+\mathfrak{5}} p
\]
除非 $n$ 极大, 即除非 $n$ 为无穷大, 这个分数将恒大于 $p$, 事实上, 它等于
\[
p+\frac{(n+2) \mathfrak{A}+\mathfrak{B}}{\frac{1}{2}(n+1)(n+2) \mathfrak{A}+(n+1) \mathfrak{B}+\mathfrak{5}} p
\]
如果 $p$ 不是最大根, 那么它的计算更难. 可见, 用递推级数求根, 含重根的方程比不含重 根的方程要难得多.

\section{$\S 348$}

现在我们来看看分式分母有虚因式时,递推级数的情形. 设分式
\[
\frac{a+b z+c z^{2}+d z^{3}+\cdots}{1-\alpha z-\beta z^{2}-\gamma z^{3}-\delta z^{4}-\cdots}
\]
在实因式 $1-q z, 1-r z, \cdots$ 之外, 还有一个三项因式 $1-2 p z \cos \varphi+p^{2} z^{2}$, 它含有两个线性 虚因式. 记分式的递推级数为
\[
A+B z+C z^{2}+D z^{3}+\cdots+P z^{n}+Q z^{n+1}+\cdots
\]
从 $\S 218$ 我们知道, 系数
\[
P=\frac{\mathfrak{I} \sin (n+1) \varphi+\mathfrak{B} \sin n \varphi_{p}}{\sin \varphi}+\mathfrak{S} q^{n}+\mathfrak{D} r^{n}+\cdots
\]
如果 $p$ 小于 $q, r, \cdots$ 中的一个,则方程
\[
x^{m}-\alpha x^{m-1}-\beta x^{m-2}-\gamma x^{m-3}-\cdots=0
\]
的最大根是实的. 用递推级数求这个最大根时,跟方程不含虚根没有什么不同.

\section{$\$ 349$}

只要共轭虚根的积小于最大实根的平方, 虚根的存在对最大实根的求法就不产生干 扰. 如果共轭虚根的积等于或大于最大实根的平方,那么从递推级数就得不到最大实根, 原因是, 在这种情况下, 即使级数继续到无穷, 幂 $p^{n}$ 与最大实根的同次幂相比较也不消 失. 下面举几个例子,证实这里所讲.

例 8 求方程
\[
x^{3}-2 x-4=0
\]
的最大实根.

该方程的因式是
\[
(x-2)\left(x^{2}+2 x+2\right)
\]
我们看到实根为 2 , 虚根积为 2 , 小于实根平方, 实根可用递推级数求出. 从递推尺度 0,2 , 4 得递推级数的系数为
\[
1,0,2,4,4,16,24,48,112,192,416,832, \cdots
\]
由此得实根为 2 .

例 9 方程
\[
x^{3}-4 x^{2}+8 x-8=0
\]
实根为 2 , 两虚根的积为 4 , 等于实根的平方.

我们用递推级数来求这个实根, 看结果怎样. 为简化结果, 令 $x=2 y$, 得
\[
y^{3}-2 y^{2}+2 y-1=0
\]
由此得递推级数的系数为
\[
1,2,2,1,0,0,1,2,2,1,0,0,1,2,2,1, \cdots
\]
是循环的. 我们的结论只能是: 或者最大根不是实的, 或者最大实根的平方不大于虚根的 积.

例 10 方程
\[
x^{3}-3 x^{2}+4 x-2=0
\]
实根为 1 , 虚根的积为 2 .

从递推尺度 $3,-4,2$ 得递推级数的系数
\[
1,3,5,5,1,-7,-15,-15,1,33,65,65,1, \cdots
\]
有正有负, 从它我们得不到实根 1 .

\section{$\S 350$}

假定两个虚根的积 $p^{2}$ 大于任何实根的平方, 那么 $n$ 趋向无穷时, 与幂 $p^{n}$ 相比,幂 $q^{n}$, $r^{n}, \cdots$ 都可以忽略. 这样我们有
\[
\begin{aligned}
& P=\frac{\mathscr{M} \sin (n+1) \varphi+\mathfrak{B} \sin n \varphi_{p^{n}}}{\sin \varphi} \\
& Q=\frac{\mathfrak{A} \sin (n+2) \varphi+\mathfrak{B} \sin (n+1) \varphi_{p^{n+1}}}{\sin \varphi}
\end{aligned}
\]
从而
\[
\frac{Q}{P}=\frac{\mathfrak{U} \sin (n+2) \varphi+\mathfrak{B} \sin (n+1) \varphi_{p}}{\mathfrak{A} \sin (n+1) \varphi+\mathfrak{B}_{\sin } n \varphi}
\]
即使 $n$ 为无穷, 这个表达式也不取常值, 因为正弦的值时正时负, 是摆动的.

\section{$\S 351$}

类似地, 可以得到分式 $\frac{R}{Q}$ 和 $\frac{S}{R}$, 从这两个分式消去 $\mathfrak{U}$ 和 $\mathfrak{B}$, 并使数 $n$ 不出现, 得
\[
P p^{2}+R=2 Q p \cos \varphi
\]
由此得
\[
\cos \varphi=\frac{P p^{2}+R}{2 Q p}
\]
类似地,得
\[
\cos \varphi=\frac{Q p^{2}+S}{2 R p}
\]
由这两式得
\[
p=\sqrt{\frac{R^{2}-Q S}{Q^{2}-P R}}
\]
和
\[
\cos \varphi=\frac{Q R-P S}{2 \sqrt{\left(Q^{2}-P R\right)\left(R^{2}-Q S\right)}}
\]
因此, 如果将递推级数延续到, 同幂 $p^{n}$ 相比其他根的幂都可忽略时, 那么用这里的方法 就可求出三项式因式 $1-2 p z \cos \varphi+p^{2} z^{2}$.

\section{$\S 352$}

上节结果的推导, 经验不够丰富的读者可能感到困难. 这里我们详细做一下, 从 $\frac{Q}{P}$ 我们得到
\[
\mathfrak{A} P p \sin (n+2) \varphi+\mathfrak{B P p} \sin (n+1) \varphi=\mathfrak{A} Q \sin (n+1) \varphi+\mathfrak{B} Q \sin n \varphi
\]
从而
\[
\frac{\mathfrak{U}}{\mathfrak{B}}=\frac{Q \sin n \varphi-P p \sin (n+1) \varphi}{P p \sin (n+2) \varphi-Q \sin (n+1) \varphi}
\]
类似地,得
\[
\frac{\mathfrak{H}}{\mathfrak{B}}=\frac{R \sin (n+1) \varphi-Q p \sin (n+2) \varphi}{Q p \sin (n+3) \varphi-R \sin (n+2) \varphi}
\]
由两式右端相等得
\[
0=\left\{\begin{array}{l}
Q^{2} p \sin n \varphi \sin (n+3) \varphi-Q^{2} p \sin (n+1) \varphi \sin (n+2) \varphi \\
-Q R \sin n \varphi \sin (n+2) \varphi+Q R \sin (n+1) \varphi \sin (n+1) \varphi \\
-P Q p^{2} \sin (n+1) \varphi \sin (n+3) \varphi+P Q p^{2} \sin (n+2) \varphi \sin (n+2) \varphi
\end{array}\right.
\]
利用
\[
\sin a \sin b=\frac{1}{2} \cos (a-b)-\frac{1}{2} \cos (a+b)
\]
得
\[
0=\frac{1}{2} Q^{2} p(\cos 3 \varphi-\cos \varphi)+\frac{1}{2} Q R(1-\cos 2 \varphi)+\frac{1}{2} P Q p^{2}(1-\cos 2 \varphi)
\]
除以 $\frac{1}{2} Q$, 得
\[
\left(P p^{2}+R\right)(1-\cos 2 \varphi)=Q p(\cos \varphi-\cos 3 \varphi)
\]
由
\[
\cos \varphi=\cos 2 \varphi \cos \varphi+\sin 2 \varphi \sin \varphi
\]
和
\[
\cos 3 \varphi=\cos 2 \varphi \cos \varphi-\sin 2 \varphi \sin \varphi
\]
得
\[
\cos \varphi-\cos 3 \varphi=2 \sin 2 \varphi \sin \varphi=4 \sin ^{2} \varphi \cos \varphi
\]
又
\[
1-\cos 2 \varphi=2 \sin ^{2} \varphi
\]
从而
\[
P p^{2}+R=2 Q p \cos \varphi
\]
进而
\[
\cos \varphi=\frac{P p^{2}+R}{2 Q p}
\]
类似地,得
\[
\cos \varphi=\frac{Q p^{2}+S}{2 R p}
\]
由这两式得到前节结果
\[
p=\sqrt{\frac{R^{2}-Q S}{Q^{2}-P R}}
\]
和
\[
\cos \varphi=\frac{Q R-P S}{2 \sqrt{\left(Q^{2}-P R\right)\left(R^{2}-Q S\right)}}
\]
\section{$\S 353$}

如果产生递推级数的分式, 其分母含有几个不相等的三项式因式, 那么看看从 $\S 219$ 开始的几节所给出的通项, 我们就会明白, 这求根会怎样地不准确. 我们指出一 点, 如果求到了比较靠近一个实真根值, 那么通过对方程做变换, 可以求出更靠近真根的 值. 事实上, 置 $x$ 等于比较靠近真根的值与 $y$ 的和, 求新方程的最小根 $y$, 求得的这个 $y$, 加上比较靠近真根的值, 就是 $x$ 的更靠近真根的值.

例 11 方程
\[
x^{3}-3 x^{2}+5 x-4=0
\]
有一个靠近 1 的根,因为 $x=1$ 时
\[
x^{3}-3 x^{2}+5 x-4=-1
\]
令 $x=1+y$, 则
\[
1-2 y-y^{3}=0
\]
我们求这个方程的最小根, 以 $2,0,1$ 为递推尺度的级数, 其系数为
\[
1,2,4,9,20,44,97,214,472,1041,2296, \cdots
\]
由此得最小根 $y$ 近似地等于
\[
\frac{1041}{2296}=0.453397
\]
从而
\[
x=1.453397
\]
这个值与真值靠得很近. 用别的办法很难这么容易地求得这个值.

\section{$\S 354$}

如果一个级数延续到某项以后, 它接近于一个几何级数, 这时用一项除后项, 商就是 对应方程的根. 设
\[
P, Q, R, S, T, \cdots
\]
是递推级数延续到了极远处的项,已经成几何级数,这时
\[
T=\alpha S+\beta R+\gamma Q+\delta P
\]
即递推尺度为 $\alpha,+\beta,+\gamma,+\delta$, 令 $\frac{Q}{P}=x$, 则
\[
\frac{R}{P}=x^{2}, \frac{S}{P}=x^{3}, \frac{T}{P}=x^{4}
\]
代它们人上面的方程, 得
\[
x^{4}=\alpha x^{3}+\beta x^{2}+\gamma x+\delta
\]
可见 $\frac{Q}{P}$ 确实是方程的一个根, 这一点和前面的方法告诉我们, $\frac{Q}{P}$ 是方程的最大根.

\section{$\S 355$}

这种求根方法常常也可以用于项数无穷的方程. 作为例子, 我们考虑方程
\[
\frac{1}{2}=z-\frac{z^{3}}{6}+\frac{z^{5}}{120}-\frac{z^{7}}{5040}+\cdots
\]
其最小根是 $30^{\circ}$ 或 $\frac{1}{6}$ 半圆的弧度数, 改写方程为 
\[
 1-2 z+\frac{z^{3}}{3}-\frac{z^{5}}{60}+\frac{z^{7}}{2520}-\cdots=0
\]
递推尺度为
\[
2,0,-\frac{1}{3}, 0,+\frac{1}{60}, 0,-\frac{1}{2520}, 0, \cdots
\]
级数的系数为
\[
1,2,4, \frac{23}{3}, \frac{44}{3}, \frac{1681}{60}, \frac{2408}{45}, \cdots
\]
从而我们近似地有
\[
z=\frac{1681 \cdot 45}{2408 \cdot 60}=\frac{1681 \cdot 3}{2408 \cdot 4}=\frac{5043}{9632}=0.52356
\]
圆周与直径的比我们知道, $z$ 的真值应为 $0.523598$, 我们求得的值的误差为 $\frac{3}{100} 000$. 我们 的方法在这里的情况下是有用的, 因为根全是实的, 且其余的根都离最小根够远. 在无穷 方程中这样的情况不多见,所以我们的方法在求解无穷方程时很少使用. 

