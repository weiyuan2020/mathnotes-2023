\chapter{第二十二章 关于圆的几个问题的解}

\section{$\S 529$}

前面我们看到了, 半径为 1 时半圆周 $\pi$ 或 180 度的弧等于
\[
\text { 3. } 14159265358979323846264338
\]
这个数以 10 为底的对数, 即常用对数等于
\[
0.497149872694133854351268288
\]
用 $2.30258$ 乘这个常数对数,得自然对数等于
\[
\text { 1. } 1447298858494001741434237
\]
180 度角的弧为已知, 从它我们可以求出任何度数的角的弧. 记所求弧长为 $z$, 设 $z$ 所对角 为 $n$ 度, 则 $180: n=\pi: z$, 也即 $z=\frac{\pi n}{180}, z$ 的对数等于 $n$ 的对数减去对数
\[
\text { 1.758122632409172215452526413 }
\]
如果弧所对角的单位为分, 即这弧为 $n^{\prime}$, 则弧长 $z$ 的对数等于 $n$ 的对数减去对数
\[
\text { 3. } 536273882792815847961293211
\]
如果弧所对角的单位为秒, 即这弧为 $n^{\prime \prime}$, 则弧长 $z$ 的对数等于 $n$ 的对数减去
\[
\text { 5. } 314425133176459480470060009
\]
或者等于 $n$ 的对数加上对数
\[
\text { 4. } 685574866823540519529939990
\]
再从和的首数中减去 10 .

\section{$\S 530$}

反之, 从弧长可求出角度, 也就可以求出正弦、正切和正割. 相应地弧可以和角一样 用度分秒表示. 记半径为 1 的圆的以半径为度量单位的弧为 $z, z$ 用十进制数表示. 取 $z$ 的 对数, 先照正弦、正切、正割对数表那样, 把 10 加到所取对数的首数上去, 再减去
\[
4. 685574866823540519529939990
\]
或者把

$314425133176459480470060009$

加到所取数上去, 得到的都是用秒表示的弧 $z$ 的对数. 当然在后一种情况下首数应该减 去 10. 长度等于半径的弧所对的角, 可以不用对数而用黄金规则得到. 由 $\pi$ 比 $180^{\circ}$ 等于 1 比长度等于半径的弧的度数. 我们得到长度等于半径的弧, 用度表示, 为 用分表示,为
\[
3737.74677078493925260788
\]
用秒表示, 为
\[
206264.8062470963551564728
\]
用通常的方法表示, 为
\[
57^{\circ} 17^{\prime} 44^{\prime \prime} 48^{\prime \prime \prime} 22^{\prime \prime \prime \prime} 29^{\prime \prime \prime \prime \prime} 22^{\prime \prime \prime \prime \prime}
\]
借助上册讲的级数, 得这段弧的
\[
\begin{aligned}
& \text { 正弦 }=0.84147098480789 \\
& \text { 余弦 }=0.54030230586814
\end{aligned}
\]
这两个数的前数除以后数, 就得到这段弧的正切.

\section{$\S 531$}

有了从弧长求正弦和正切的准备, 对关于圆的许多问题我们就可以进行求解了. 首 先, 非零弧都比自己的正弦长, 弧同余弦的关系则不然, 角度为零时余弦为 1 , 弧小于余 弦; 直角的余弦为零, 弧大于余弦. 可见, $0^{\circ}$ 与 $90^{\circ}$ 之间必定有一个弧, 它等于自己的余弦, 我们就来求这个弧.

题 I

求等于自己余弦的弧.

解: 记所求弧为 $s$, 则 $s=\cos s$. 从这个方程求 $s$, 恐怕没有什么方法比试位法更合用, 但这方法要先知道 $s$ 的一个近似值. 这近似值可以是猜出来的, 也可以是将三个或更多 个 $s$ 值代入方程算出来的. 计算时 $s$ 与 $\cos s$ 的位数应取得相同. 我们来计算, 取 $s=30^{\circ}$, 依 $\S 329$ 所讲
\[
\begin{gathered}
\log 30=1.4771213 \\
\text { 减 } \quad 1.7581226
\end{gathered}
\]
$\log 30^{\circ}$ 弧 $=9.7189987$

但
\[
\log \cos 30^{\circ}=9.9375306
\]
可见, $30^{\circ}$ 时弧小于余弦, 因而所求弧大于 $30^{\circ}$. 取 $s=40^{\circ}$, 则
\[
\begin{aligned}
& \log 40=1.6020600 \\
& \text { 减 } 1.7581226 \\
& \log 40^{\circ} \text { 弧 }=9.8439374
\end{aligned}
\]
但
\[
\log \cos 40^{\circ}=9.8842540
\]
可见, 所求弧稍大于 $40^{\circ}$, 取 $s=45^{\circ}$, 则
\[
\begin{aligned}
& \log 45=1.6532125
\end{aligned}
\]
减 $1.7581226$

$\log 45^{\circ}$ 弧 $=9.8950899$

但
\[
\log \cos 45^{\circ}=9.8494850
\]
可见所求角在 $40^{\circ}$ 与 $45^{\circ}$ 之间, 可进一步求一个更好的近似值, 事实上, 对数的误差
\[
\begin{aligned}
& 40^{\circ} \text { 时为 }+403166 \\
& 45^{\circ} \text { 时为 }-456044 \\
& \text { 误差的差为 } 859215
\end{aligned}
\]
取值比 $40^{\circ}$ 增加 $5^{\circ}$, 误差比 403166 增加 859215 . 由此得所求弧大于 $42^{\circ}$. 但这范围仍嫌 大,为得到更小的范围,取 $s=42^{\circ}$ 和 $s=43^{\circ}$,计算得
\[
\begin{array}{cc}
s=42^{\circ} & s=43^{\circ} \\
\log s=1.6232493 & 1.6334685
\end{array}
\]
\begin{tabular}{rll} 
减 & $1.7581226$ & $1.7581226$ \\
\hline $\log s$ & $9.8651267$ & $9.8753459$
\end{tabular}

而
\[
\begin{aligned}
& \log \cos s=9.8710735 \quad 9.8641275 \\
& +59468-112184 \\
& 112184 \\
& 171652 \text { : } 59468=1^{\circ} \text { : } 20^{\prime} 47^{\prime \prime}
\end{aligned}
\]
这样我们得到 $s$ 的真值所在的一个极小范围, $42^{\circ} 20^{\prime}$ 至 $42^{\circ} 21^{\prime}$. 化这两个角度为分, 进行 计算得

\begin{tabular}{cc}
$s=2540^{\prime}$ & $s=2541^{\prime}$ \\
$\log s=3.4048337$ & $3.4050047$ \\
减 $3.5362739$ & $3.5362739$ \\
\hline $\log s=9.8685598$ & $9.8687308$ \\
$\log \cos s=9.8687851$ & $9.8686700$ \\
\hline$+2253$ & \\
208 & \\
\hline \multicolumn{2}{c}{$2861: 2253=1: 47^{\prime \prime} 15^{\prime \prime \prime}$}
\end{tabular}

由此得所求等于自己余弦的弧为 $42^{\circ} 20^{\prime} 47^{\prime \prime} 15^{\prime \prime \prime}$, 弧和它的余弦都为 $0.7390850$.

\section{$\S 532$}

参见图 112,弦 $A B$ 分扇形 $A C B$ 为弓形 $A E B$ 和三角形 $A C B$ 两 部分. $\angle A C B$ 小时弓形小于三角形, $\angle A C B$ 为够大的针角时弓形 大于三角形,因而必定有使这两部分相等的三角形.

题 II

求被弦分为相等两部分的扇形, 即求扇形 $A C B$, 使三角形 $A C B$ 等于弓形 $A E B$.

解: 取半径 $A C=1$, 记所求弧 $A E B=2 s$, 因而它的一半 $A E=E B=s$. 画半径 $C E$, 则 $A F=\sin s, C F=\cos s$. 从而三角形 $A C B=$ $\sin s \cos s=\frac{1}{2} \sin 2 s$, 而扇形 $A C B$ 的面积等于 $s$. 我们的问题要求扇形的面积为三角形的 两倍, 即要求 $s=\sin 2 s$. 这样, 问题就成了求等于自己倍角正弦的弧. 首先, $\angle A C B$ 应大于 直角, 因而 $s$ 大于 $45^{\circ}$, 我们取三个近似值进行如下的计算


【图,待补】
%%![](https://cdn.mathpix.com/cropped/2023_02_05_c02c95ceeae1a896f812g-06.jpg?height=316&width=284&top_left_y=519&top_left_x=1208)

图 112 

\begin{tabular}{ccc}
$s=50^{\circ}$ & $s=55^{\circ}$ & $s=54^{\circ}$ \\
$\log s=1.6989700$ & $1.7403627$ & $1.7323938$ \\
减 $1.7581226$ & $1.7581226$ & $1.7581226$ \\
\hline $9.9408474$ & $9.9822401$ & $9.9742712$ \\
$\log \sin 2 s=9.9933515$ & $9.9729858$ & $9.9782063$ \\
\hline$+525041$ & $-92543$ & $+39351$ \\
92543 & \\
\hline $517584: 525041=5^{\circ}: 4^{\circ} 15^{\prime}$ &
\end{tabular}

将误差修正加到近似值 $54^{\circ}$ 上去, 得 $s=54^{\circ} 17^{\prime} 54^{\prime \prime}$, 这个值与真值的差不超过 1 秒, 下面对 相差 1 秒的近似值进行计算
\[
\begin{array}{ccc}
s=54^{\circ} 17^{\prime} & s=54^{\circ} 18^{\prime} & s=54^{\circ} 19^{\prime} \\
\text { 也即 } & \text { 也即 } & \text { 也即 } \\
s=3257^{\prime} & s=3258^{\prime} & s=3259^{\prime} \\
\text { 又 } & \text { 又 } & \text { 又 } \\
2 s=108^{\circ} 34^{\prime} & 2 s=108^{\circ} 36^{\prime} & 2 s=108^{\circ} 38^{\prime}
\end{array}
\]
\begin{tabular}{ccc} 
补角 $=71^{\circ} 26^{\prime}$ & 补角 $=71^{\circ} 24^{\prime}$ & 补角 $=71^{\circ} 22^{\prime}$ \\
$\log s=3.5128178$ & $=3.5129511$ & $=3.5130844$ \\
减 $3.5362739$ & $3.5362739$ & $3.5362739$ \\
\hline $\log s=9.9765439$ & $=9.9766772$ & $=9.9768105$ \\
$\log \sin 2 s=9.9767872$ & $=9.9767022$ & $=9.9766171$ \\
\hline$+2433$ & $+250$ & $-1934$ \\
\hline & 1934 & \\
\hline & 2184
\end{tabular}

这样我们得到 $2184: 250=1^{\prime}: 6^{\prime \prime} 52^{\prime \prime \prime}$. 由此得 $s=54^{\circ} 18^{\prime} 6^{\prime \prime} 52^{\prime \prime \prime}$. 要得到更准确的解, 须使用 位数更多的对数表, 下面对相差 $10^{\prime \prime}$ 的两个近似值进行计算
\[
\begin{array}{cc}
s=54^{\circ} 18^{\prime} 0^{\prime \prime} & s=54^{\circ} 18^{\prime} 10^{\prime \prime} \\
\text { 也即 } & \text { 也即 } \\
s=195480^{\prime \prime} & s=195490^{\prime \prime} \\
2 s=108^{\circ} 36^{\prime} 0^{\prime \prime} & 2 s=108^{\circ} 36^{\prime} 20^{\prime \prime} \\
\text { 补角 }=71^{\circ} 24^{\prime} 0^{\prime \prime} & \text { 补角 }=71^{\circ} 23^{\prime} 40^{\prime \prime} \\
\log s=5.2911023304 & =5.2911245466 \\
\text { 减 } 5.3144251332 & =5.3144251332 \\
\hline 9.9766771972 & 9.9766994134 \\
\log \sin 2 s=9.9767022291 & 9.9766880552 \\
+250319 & -113582
\end{array}
\]
113582
\[
363901: 250319=10^{\prime \prime \prime}: 6^{\prime \prime} 52^{\prime \prime \prime} 43^{\prime \prime \prime \prime} 33^{\prime \prime \prime \prime \prime}
\]
其补角 $=71^{\circ} 23^{\prime} 46^{\prime \prime \prime} 14^{\prime \prime \prime} 32^{\prime \prime \prime \prime} 54^{\prime \prime \prime \prime \prime}, \angle A C B$ 的正弦的对数, 也即
\[
\begin{aligned}
\log \sin 2 s & =9.9766924791 \\
\sin 2 s & =0.9477470
\end{aligned}
\]
又
\[
\sin s=A F=B F=0.8121029
\]
它的两倍, 也即 
\[
\text { 弦 } A B=1.6242058
\]
此外
\[
\cos s=C F=0.5835143
\]
这样就可以准确地画出所求扇形,给出问题的解.

\section{$\S 533$}

类似地,可求出等分四分之一圆的正弦.

题 III

参见图 113 , 求等分四分之一圆 $A C B$ 的正弦 $D E$.

解: 设弧 $A E=s$, 则由弧 $A E B=\frac{\pi}{2}$, 得弧 $B E=\frac{\pi}{2}-s$, 四分

之一圆的面积为 $\frac{1}{4} \pi$. 扇形 $A C E$ 的面积 $=\frac{1}{2} s$, 用它减去 $\triangle C D E$ 的面积得
\[
\triangle C D E \text { 的面积 }=\frac{1}{2} \sin s \cos s
\]

【图,待补】
%%![](https://cdn.mathpix.com/cropped/2023_02_05_c02c95ceeae1a896f812g-08.jpg?height=344&width=328&top_left_y=673&top_left_x=1162)

图 113

得
\[
\triangle A D E \text { 的面积 }=\frac{1}{2} s-\frac{1}{2} \sin s \cos s
\]
题目要它的两倍等于四分之一圆,由此得
\[
\frac{1}{4} \pi=s-\frac{1}{2} \sin 2 s
\]
从而
\[
s-\frac{1}{4} \pi=\frac{1}{2} \sin 2 s
\]
令
\[
s-\frac{1}{4} \pi=s-45^{\circ}=u
\]
则 $2 s=90^{\circ}+2 u$, 从而
\[
u=\frac{1}{2} \cos 2 u, \quad 2 u=\cos 2 u
\]
这样就把题 II 化成了题 I, 求等于自己余弦的弧, 题 I 已经求得
\[
2 u=42^{\circ} 20^{\prime} 47^{\prime \prime} 15^{\prime \prime \prime}, \quad u=20^{\circ} 10^{\prime} 23^{\prime \prime} 37^{\prime \prime \prime}
\]
得
\[
\text { 弧 } A E=s=66^{\circ} 10^{\prime} 23^{\prime \prime} 37^{\prime \prime \prime}, \quad \text { 弧 } B E=23^{\circ} 49^{\prime} 36^{\prime \prime \prime} 23^{\prime \prime \prime}
\]
由此得半径被分成的两部分
\[
\begin{gathered}
C D=0.4039718, A D=0.5960281 \\
\text { 正弦 } D E=0.9147711
\end{gathered}
\]
此法等分四分之一圆,当然也就把整个圆等分为八等分.

\section{$\S 534$}

凡过圆心的直线都将圆等分为二. 类似地, 从圆周上任一点都可画直线将圆等分为 三,或等分为更多等份. 我们来等分为四.

题 IV

参见图 $114, A C B D$ 为半圆, 试作等分 $A C B D$ 为两份的 弦 $A D$.

解: 记所求弧 $A D=s$, 画半径 $C D$, 则

扇形 $A C D$ 的面积 $=\frac{1}{2} s$

从扇形 $A C D$ 减去


【图,待补】
%%![](https://cdn.mathpix.com/cropped/2023_02_05_c02c95ceeae1a896f812g-09.jpg?height=225&width=407&top_left_y=651&top_left_x=1106)

图 114

三角形 $A C D$ 的面积 $=\frac{1}{2} A C \cdot D E=\frac{1}{2} \sin s$

得
\[
\text { 弓形 } A D=\frac{1}{2} s-\frac{1}{2} \sin s
\]
它应该等于半圆 $A D B$ 的一半, 半圆 $A D B$ 的面积为 $\frac{1}{2} \pi$, 因而
\[
s-\sin s=\frac{1}{2} \pi=90^{\circ} \text { 或 } s-90^{\circ}=\sin s
\]
令 $s-90^{\circ}=u$, 则 $\sin s=\cos u, u=\cos u$. 根据题 $\mathrm{I}$, 得
\[
u=42^{\circ} 20^{\prime} 47^{\prime \prime} 14^{\prime \prime \prime}
\]
从而
\[
\angle A C D=s=132^{\circ} 20^{\prime} 47^{\prime \prime} 14^{\prime \prime \prime}, \quad \angle B C D=47^{\circ} 39^{\prime} 12^{\prime \prime} 46^{\prime \prime \prime}
\]
弦 $A D=1.8295422$ 即为所求.

\section{$\S 535$}

题 III 求出了面积等于 $\frac{1}{4}$ 个圆的弓形, 半圆是面积等于 $\frac{1}{2}$ 个圆的弓形. 现在我们来求 面积等于 $\frac{1}{3}$ 个圆的弓形.

题 V

参见图 115, 求过圆周上一点 $A$ 分圆为三等分的弦 $A B$ 和 $A C$.

解: 取圆的半径为 1 , 则半圆周的长等于 $\pi$, 记弧 $A B=A C=s$, 则扇形 $A E B$ 和 $A F C$ 的 面积都为
\[
\frac{1}{2} s-\frac{1}{2} \sin s
\]
题目要求这面积等于圆面积的 $\frac{1}{3}$, 圆面积为 $\pi$, 因而要求
\[
\frac{1}{2} s-\frac{1}{2} \sin s=\frac{\pi}{3}=60^{\circ} \text { 或 } s-\sin s=120^{\circ}
\]
从而
\[
s-120^{\circ}=\sin s
\]
令 $s-120^{\circ}=u$, 则 $u=\sin \left(u+120^{\circ}\right)=\sin \left(60^{\circ}-u\right)$. 这样就把问 题化成了求弧 $u$, 使等于 $60^{\circ}-u$ 的正弦. 因而 $u$ 小于 $60^{\circ}$. 我们取 近似值进行下面的计算


【图,待补】
%%![](https://cdn.mathpix.com/cropped/2023_02_05_c02c95ceeae1a896f812g-10.jpg?height=374&width=343&top_left_y=306&top_left_x=1169)

图 115

$u=40^{\circ}$
$60^{\circ}-u=20^{\circ}$
$1.6020600$
$1.7581226$
$9.8439374$
$9.5340517$
$-3098857$

可见 $u$ 小于 $30^{\circ}$, 下面的计算表明它应该大于 $29^{\circ}$. 取 $u=29^{\circ}$ 进行计算
\[
\begin{gathered}
u=29^{\circ} \\
60^{\circ}-u=31^{\circ} \\
\log u=1.4623980 \\
=1.7581226 \\
\text { 减 } \quad \begin{array}{c}
\log u=9.7042754 \\
\log \sin \left(60^{\circ}-u\right)=9.7118393 \\
+75639 \\
-200287 \\
275926: 75639=1^{\circ}: 16^{\prime} 26^{\prime \prime}
\end{array}
\end{gathered}
\]
得 $u$ 的近似值 $29^{\circ} 16^{\prime} 26^{\prime \prime}$. 为了得到该角更准确的值, 下面我们对相差 1 秒的两个近似值 进行计算
\[
\begin{array}{cc}
u=29^{\circ} 16^{\prime} & u=29^{\circ} 17^{\prime} \\
\text { 也即 } & \text { 也即 } \\
u=1756^{\prime} & u=1757^{\prime}
\end{array}
\]
\begin{tabular}{cc}
$60^{\circ}-u=30^{\circ} 44^{\prime}$ & $60^{\circ}-u=30^{\circ}-43^{\prime}$ \\
$\log u=3.2445245$ & $3.2447718$ \\
减 $3.5362739$ & $3.5362739$ \\
\hline $\log u=9.7082506$ & $9.7084979$ \\
$\log \sin \left(60^{\circ}-u\right)=9.7084575$ & $9.7082450$ \\
\hline$+2069$ & $-2529$ \\
2529 &
\end{tabular}

$4598: 2069=1^{\prime}: 27^{\prime \prime} 0^{\prime \prime \prime}$

这样 $u=29^{\circ} 16^{\prime} 27^{\prime \prime} 0^{\prime \prime \prime}$, 从而弧
\[
s=A E B=149^{\circ} 16^{\prime} 27^{\prime \prime} 0^{\prime \prime \prime}=A F C
\]
由此得弧
\[
B C=61^{\circ} 27^{\prime} 6^{\prime \prime} 0^{\prime \prime \prime}
\]
弦 $A B=A C=1.9285340$ 即为所求.

\section{$\S 536$}

前几题都归结为求弧, 使等于给定的正弦或余弦. 下一题方法类似, 但更难.

题 VI

参见图 116, $A E B$ 为半圆, $E D$ 为正弦,求弧 $A E$, 使其长 等于直线 $A D$ 与 $D E$ 之和.

解: 显然, 这段弧必大于圆周的 $\frac{1}{4}$, 我们求它的补 $B E$. 记 $B E$ 为 $s$, 则弧 $A E=180^{\circ}-s$. 又 $A C=1, C D=\cos s, D E=$ $\sin s$, 这样从 $A E$ 应满足的条件得


【图,待补】
%%![](https://cdn.mathpix.com/cropped/2023_02_05_c02c95ceeae1a896f812g-11.jpg?height=244&width=388&top_left_y=1436&top_left_x=1089)

图 116
\[
180^{\circ}-s=1+\cos s+\sin s
\]
将
\[
\begin{gathered}
\sin s=2 \sin \frac{1}{2} s \cos \frac{1}{2} s \\
1+\cos s=2 \cos \frac{1}{2} s \cos \frac{1}{2} s
\end{gathered}
\]
代入, 得
\[
180^{\circ}-s=2 \cos \frac{1}{2} s\left(\sin \frac{1}{2} s+\cos \frac{1}{2} s\right)
\]
由 
\[
\cos \left(45^{\circ}-\frac{1}{2} s\right)=\frac{1}{\sqrt{2}} \cos \frac{s}{2}+\frac{1}{\sqrt{2}} \sin \frac{s}{2}
\]
得
\[
\sin \frac{1}{2} s+\cos \frac{1}{2} s=\sqrt{2} \cos \left(45^{\circ}-\frac{1}{2} s\right)
\]
代入, 得
\[
180^{\circ}-s=2 \sqrt{2} \cos \frac{s}{2} \cos \left(45^{\circ}-\frac{1}{2} s\right)
\]
对所得等式, 我们取近似值 $20^{\circ}, 21^{\circ}$ 进行计算
\[
\begin{aligned}
& \frac{1}{2} s=20^{\circ} \quad \frac{1}{2} s=21^{\circ} \\
& 45^{\circ}-\frac{1}{2} s=25^{\circ} \quad 45^{\circ}-\frac{1}{2} s=24^{\circ} \\
& 180^{\circ}-s=140^{\circ} \quad 180^{\circ}-s=138^{\circ} \\
& \log \left(180^{\circ}-s\right)=2.1461280 \quad 2.1398791 \\
& \text { 减 } 1.7581226 \quad 1.7251226 \\
& \log \left(180^{\circ}-s\right)=0.3880054 \quad 0.3817565 \\
& \log \cos \frac{1}{2} s=9.9729858 \quad 9.9701517 \\
& \log \cos \left(45^{\circ}-\frac{1}{2} s\right)=9.9572757 \quad 9.9607302 \\
& \log 2 \sqrt{2}=0.4515450 \quad 0.4515450 \\
& \begin{array}{ll}0.3818065 & 0.3824269\end{array} \\
& \text { 误差 }+61989 \quad-6704 \\
& 6704 \\
& 68693: 61989=1^{\circ}: 54^{\prime \prime}
\end{aligned}
\]
得 $\frac{1}{2} s$ 在 $20^{\circ} 54^{\prime}$ 与 $20^{\circ} 55^{\prime}$ 之间, 对这两个新近似值再作计算 
\[
\begin{aligned}
& \text { Infinite analysio (无穷分析引论 Fulraduclian } \\
& \frac{1}{2} s=20^{\circ} 54^{\prime} \quad \frac{1}{2} s=20^{\circ} 55^{\prime} \\
& 45^{\circ}-\frac{1}{2} s=24^{\circ} 6^{\prime} \quad 45^{\circ}-\frac{1}{2} s=24^{\circ} 5^{\prime} \\
& s=41^{\circ} 48^{\prime} \quad 180^{\circ}-s=138^{\circ} 10^{\prime} \\
& 180^{\circ}-s=138^{\circ} 12^{\prime} \quad s=41^{\circ} 50^{\prime} \\
& \text { 也即 } \\
& \text { 也即 } \\
& 180^{\circ}-s=8292^{\prime} \quad 180^{\circ}-s=8290^{\prime} \\
& \log \left(180^{\circ}-s\right)=3.9186593 \quad \log \left(180^{\circ}-s\right)=3.9185545 \\
& \text { 减 } 3.5362739 \quad 3.5362739 \\
& 0.3823854 \quad 0.3822806 \\
& \log \cos \frac{1}{2} s=9.9704419 \quad 9.9703937 \\
& \log \cos \left(45^{\circ}-\frac{1}{2} s\right)=9.9603919 \quad 9.9604484 \\
& \log 2 \sqrt{2}=0.4515450 \quad 0.4515450 \\
& \begin{array}{ll}0.3823788 & 0.3823871\end{array} \\
& \text { 误差 }+66 \quad-1065 \\
& 1065 \\
& 1131: 66=1^{\prime}: 3^{\prime \prime} 30^{\prime \prime \prime}
\end{aligned}
\]
可见 $\frac{1}{2} s=20^{\circ} 54^{\prime} 3^{\prime \prime} 30^{\prime \prime \prime}$
\[
s=41^{\circ} 48^{\prime} 7^{\prime \prime} 0^{\prime \prime \prime}=B E
\]
得所求弧
\[
A E=138^{\circ} 11^{\prime} 53^{\prime \prime} 0^{\prime \prime \prime}
\]
此时直线
\[
D E=0.6665578, \quad A D=1.7454535
\]
满足题目要求.

\section{$\S 537$}

现在我们把弧与它的正切相比较, 由于在第一象限中弧小于它的正切, 我们求等于 自己正切一半的弧.

题 VII

参见图 117, 求扇形 $A C D$, 使等于三角形 $A C E$ 的一半, $A C E$ 由半径 $A C$, 切线 $A E$ 和 割线 $C E$ 构成.


【图,待补】
%%![](https://cdn.mathpix.com/cropped/2023_02_05_c02c95ceeae1a896f812g-14.jpg?height=395&width=256&top_left_y=492&top_left_x=691)

图 117

解: 记弧 $A D$ 为 $s$, 则扇形 $A C D=\frac{1}{2} s$. 三角形 $A C E=\frac{1}{2} \tan s$. 依题意应该有 $s=\frac{1}{2} \tan s$ 或 $2 s=\tan s$. 取近似值进行如下计算

\begin{tabular}{cccc}
$s=60^{\circ}$ & $s=70^{\circ}$ & $s=66^{\circ}$ & $s=67^{\circ}$ \\
$\log 2 s=2.0791812$ & $2.1461280$ & $2.1205739$ & $2.1271048$ \\
$1.7581226$ & $1.7581226$ & $1.7581226$ & $1.7581226$ \\
\hline $\log 2 s=0.3210586$ & $0.3880054$ & $0.3624513$ & $0.3689822$ \\
\hline $\log \tan s=0.2385606$ & $0.4389341$ & $0.3514169$ & $0.3721481$ \\
\hline$+824980$ & $-509287$ & $+110344$ & $-31659$
\end{tabular}

得 $s$ 在 $66^{\circ} 46^{\prime}$ 与 $66^{\circ} 47^{\prime}$ 之间, 为得到更准确的值, 我们再作如下计算


\[
\begin{array}{c|c}
 s=66^{\circ} 46^{\prime} & s=66^{\circ} 47^{\prime} \\
 \text { 也即 } & \text { 也即 } \\
 s=4006^{\prime} & s=4007^{\prime} \\
 2 s=8012 & 2 s=8014 \\
 \log 2 s=3.9037409 & 3.9038493 \\

3.5362739 & 3.5362739 \\
\hline \log 2 s=0.3674670 & 0.3675754 \\
 \log \tan s=0.3672499 & 0.3675985 \\
\hline
 \text { 误差 }+2171& - 231 \\
\hline
 231 
\end{array}
\]

$2402: 2171=1^{\prime}: 54^{\prime \prime} 14^{\prime \prime \prime}$ 由此得弧
\[
s=A D=66^{\circ} 46^{\prime} 54^{\prime \prime} 14^{\prime \prime \prime}
\]
因而正切 $A E=2.3311220$ 即为所求.

\section{$\S 538$}

现在我们提出

题 VIII

参见图 $118, A B C$ 为圆的四分之一. 求弧 $A E$, 使等于弦 $A E$ 延 长到交点 $F$.

解: 设弧 $A E=s$, 则它的弦 $A E=2 \sin \frac{1}{2} s$, 正矢 $A D=1-\cos s=$ $2 \sin \frac{s}{2} \sin \frac{s}{2}$,由 $\triangle A D E$ 和 $\triangle A C F$ 相似得
\[
2 \sin \frac{1}{2} s \cdot \sin \frac{1}{2} s: 2 \sin \frac{1}{2} s=1: s
\]

【图,待补】
%%![](https://cdn.mathpix.com/cropped/2023_02_05_c02c95ceeae1a896f812g-15.jpg?height=350&width=304&top_left_y=591&top_left_x=1188)

图 118

由此得 $s \cdot \sin \frac{1}{2} s=1$. 取近似值作如下计算

\begin{tabular}{cccc}
$s=70^{\circ}$ & $s=80^{\circ}$ & $s=84^{\circ}$ & $s=85^{\circ}$ \\
$\log s=1.8450980$ & $1.9030900$ & $1.9242793$ & $1.9294189$ \\
减 $1.7581226$ & $1.7581226$ & $1.7581226$ & $1.7581226$ \\
\hline $0.0869754$ & $0.1449674$ & $0.1661567$ & $0.1712963$ \\
$\log \sin \frac{1}{2} s=9.7585913$ & $9.8080675$ & $9.8255109$ & $9.8296833$ \\
\hline $9.8455667$ & $9.9530349$ & $9.9916676$ & $0.0009796$ \\
误差 $+0.1544332$ & $0.0469650$ & $+83223$ & $-9796$
\end{tabular}

得 $s$ 在 $84^{\circ} 53^{\prime}$ 与 $84^{\circ} 54^{\prime}$ 之间, 再作下列计算
\[
\begin{array}{cc}
s=84^{\circ} 53^{\prime} & s=84^{\circ} 54^{\prime} \\
\text { 也即 } & \text { 也即 } \\
s=5093^{\prime} & s=5094^{\prime} \\
\frac{1}{2} s=42^{\circ} 26 \frac{1}{2}{ }^{\prime} & \frac{1}{2} s=40^{\circ} 27^{\prime} \\
\log s=3.7069737 & 3.7070589
\end{array}
\]
\begin{tabular}{rc}
$3.5362739$ & $3.5362739$ \\
\hline $0.1706998$ & $0.1707850$ \\
$\log \sin \frac{1}{2} s=9.8292003$ & $9.8292694$ \\
$0.9999001$ & $0.0000544$ \\
误差 $+998$ & $-544$
\end{tabular}

得所求
\[
\begin{gathered}
\text { 弧 } s=A E=84^{\circ} 53^{\prime} 38^{\prime \prime} 51^{\prime \prime \prime} \\
\text { 弧 } B E=5^{\circ} 6^{\prime} 21^{\prime \prime} 9^{\prime \prime \prime}
\end{gathered}
\]
\section{$\S 539$}

虽然第一象限中的弧都小于自己的正切,但随后的象限中有等于自己正切的弧. 下一个题目我们利用级数来求这样的弧.

题 IX

求出一切等于自己正切的弧.

解: 具有这种性质的第一个弧是无穷小弧. 第二象限中正切为负, 因而没有这样的 弧. 第三象限中有一个略小于 $270^{\circ}$ 的这样的弧. 再下去, 在第五、第七等象限中有这样的 弧. 记四分之一圆周为 $q$, 则所求的表示式为 $(2 n+1) q-s$, 有
\[
(2 n-1) q-s=c \tan s=\frac{1}{\tan s}
\]
设 $\tan s=x$, 则
\[
s=x-\frac{1}{3} x^{3}+\frac{1}{5} x^{5}-\frac{1}{7} x^{7}+\cdots
\]
从而
\[
(2 n+1) q=\frac{1}{x}+x-\frac{1}{3} x^{3}+\frac{1}{5} x^{5}-\frac{1}{7} x^{7}+\cdots
\]
由数 $n$ 越大弧 $s$ 越小, 显见 $x$ 是一个很小的量, $x$ 的一个近似值为
\[
x=\frac{1}{(2 n+1) q}
\]
也即
\[
\frac{1}{x}=(2 n+1) q
\]
更准确的近似值可从下式得到
\[
\begin{aligned}
\frac{1}{x}= & (2 n+1) q-s=(2 n+1) q-\frac{1}{(2 n+1) q}-\frac{2}{3(2 n+1)^{3} q^{3}}- \\
& \frac{13}{15(2 n+1)^{5} q^{5}}-\frac{146}{105(2 n+1)^{7} q^{7}}-\frac{2343}{945(2 n+1)^{9} q^{9}}-\cdots
\end{aligned}
\]
将 $q=\frac{\pi}{2}=1.5707963267948$ 代入得所求弧等于
\[
\begin{aligned}
& (2 n+1) \cdot 1.57079632679-\frac{1}{2 n+1} \cdot 0.63661977- \\
& \frac{0.17200818}{(2 n+1)^{3}}-\frac{0.09062598}{(2 n+1)^{5}}-\frac{0.05892837}{(2 n+1)^{7}}-\frac{0.04258548}{(2 n+1)^{9}}-\cdots
\end{aligned}
\]
化以半径为单位的表示为度分秒表示, 则所求弧等于
\[
(2 n+1) \cdot 90^{\circ}-\frac{131313^{\prime \prime}}{2 n+1}-\frac{35479^{\prime \prime}}{(2 n+1)^{3}}-\frac{18693^{\prime \prime}}{(2 n+1)^{5}}-\frac{12155^{\prime \prime}}{(2 n+1)^{7}}-\frac{8784^{\prime \prime}}{(2 n+1)^{9}}
\]
这样,满足要求的弧依次为
\[
\begin{array}{ll}
\text { I } & 1 \cdot 90^{\circ}-90^{\circ} \\
\text { II } & 3 \cdot 90^{\circ}-12^{\circ} 32^{\prime} 48^{\prime \prime} \\
\text { III } & 5 \cdot 90^{\circ}-7^{\circ} 22^{\prime} 32^{\prime \prime} \\
\text { IV } & 7 \cdot 90^{\circ}-5^{\circ} 14^{\prime} 22^{\prime \prime} \\
\text { V } & 9 \cdot 90^{\circ}-4^{\circ} 3^{\prime} 59^{\prime \prime} \\
\text { VI } & 11 \cdot 90^{\circ}-3^{\circ} 19^{\prime} 24^{\prime \prime} \\
\text { VII } & 13 \cdot 90^{\circ}-2^{\circ} 48^{\prime} 37^{\prime \prime} \\
\text { VII } & 15 \cdot 90^{\circ}-2^{\circ} 26^{\prime} 5^{\prime \prime} \\
\text { IX } & 17 \cdot 90^{\circ}-2^{\circ} 8^{\prime} 51^{\prime \prime} \\
\text { X } & 19 \cdot 90^{\circ}-1^{\circ} 55^{\prime} 16^{\prime \prime}
\end{array}
\]
\section{$\S 540$}

解这类问题的方法, 不外以上例子中所用的几种, 因而我们不再列题目. 举出前面题 目的主要目的是为加深对圆的性质的认识. 这些求积的题目, 原有的老方法对它们都无 能为力. 如果在解某个问题时,得到的弧与整个圆可公度,或者得到的弧的正弦(或正切) 可用半径画出, 那就等于做到了化圆为方. 例如, 解题 VI ( §536 图 116) 时, 如果得到的正 弦 $D E$ 不是 $0.6665578$, 而是 $0.6666666=\frac{2}{3}$, 那就等于发现了一条美妙的性质: 可以画 出等于直线
\[
A D+D E=1+\frac{2}{3}+\sqrt{\frac{5}{9}}
\]
的弧 $A E$. 到现在为止尚无人证明化圆为方为不可能, 倘可能, 显然, 不会有什么方法能比 本章的方法更合用. 

