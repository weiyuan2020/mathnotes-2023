\chapter{第十三章 递推级数}

\section{$\S 211$}

按分数函数表示的那样进行除法, 我们得到一种级数. 棣莫弗称这种级数为递推级 数. 递推级数的任何一项,都依某个固定的公式,由前几项推出.

固定公式由分数函数的分母决定. 这我们前面讲过. 现在, 我们能够把分数函数分解 为更简单的部分分式. 因而分数函数的递推级数, 可用更简单的递推级数之和表示. 本章 我们就讲构成递推级数的更简递推级数.

\section{$\S 212$}

设用除法将真分数函数
\[
\frac{a+b z+c z^{2}+d z^{3}+\cdots}{1-\alpha z-\beta z^{2}-\gamma z^{3}-\delta z^{4}-\cdots}
\]
表示成了递推级数
\[
A+B z+C z^{2}+D z^{3}+E z^{4}+F z^{5}+\cdots
\]
这真分数函数的部分分式我们会求, 将部分分式表示成递推级数, 做起来简单, 其性质易 于考查. 部分分式的递推级数加起来就是这真分数函数的递推级数
\[
A+B z+C z^{2}+D z^{3}+E z^{4}+F z^{5}+\cdots
\]
\section{$\S 213$}

设部分分式产生的递推级数为
\[
\begin{gathered}
a+b z+c z^{2}+d z^{3}+e z^{4}+\cdots \\
a^{\prime}+b^{\prime} z+c^{\prime} z^{2}+d^{\prime} z^{3}+e^{\prime} z^{4}+\cdots \\
a^{\prime \prime}+b^{\prime \prime} z+c^{\prime \prime} z^{2}+d^{\prime \prime} z^{3}+e^{\prime \prime} z^{4}+\cdots \\
a^{\prime \prime \prime}+b^{\prime \prime \prime} z+c^{\prime \prime \prime} z^{2}+d^{\prime \prime \prime} z^{3}+e^{\prime \prime \prime} z^{4}+\cdots
\end{gathered}
\]
这些级数加起来应该等于 
\[
A+B z+C z^{2}+D z^{3}+E z^{4}+\cdots
\]
由此我们得到
\[
\begin{gathered}
A=a+a^{\prime}+a^{\prime \prime}+a^{\prime \prime \prime}+\cdots \\
B=b+b^{\prime}+b^{\prime \prime}+b^{\prime \prime \prime}+\cdots \\
C=c+c^{\prime}+c^{\prime \prime}+c^{\prime \prime \prime}+\cdots \\
D=d+d^{\prime}+d^{\prime \prime}+d^{\prime \prime \prime}+\cdots
\end{gathered}
\]
这样一来, 如果能够求出部分分式产生的各个级数中 $z^{n}$ 的系数,那么这些系数的和就是 递推级数 $A+B z+C z^{2}+D z^{3}+\cdots$ 中 $z^{n}$ 的系数.

\section{$\S 214$}

这里可能产生一个疑问, 两个级数相等, 它们同次幂的系数一定相等吗, 即
\[
A+B z+C z^{2}+D z^{3}+\cdots=\mathfrak{A}+\mathfrak{B} z+\mathfrak{S}^{2}+\mathfrak{D} z^{3}+\cdots
\]
时,一定有 $A=\mathfrak{A}, B=\mathfrak{B}, C=\mathfrak{5}, D=\mathfrak{D}, \cdots$ 吗?我们利用等式对 $z$ 的任何值都成立这一点,就可消除这一疑问.如果$z=0,$则显然$A=\mathfrak{A},$两边去掉这相等的项,除剩下的方程以$z,$我们得到
\[
B+C z+D z^{2}+\cdots=\mathfrak{B}+\left(5 z+D z^{2}+\cdots\right.
\]
由此我们得到 $B=\mathfrak{B}$. 类似地, 我们得到 $C=\mathfrak{5}, D=\mathfrak{D}, \cdots$

\section{$\S 215$}

现在我们考察由分数函数的各种部分分式所产生的级数. 首先,分式
\[
\frac{\mathfrak{A}}{1-p z}
\]
产生的级数为
\[
\frac{\mathfrak{A}}{1-p z}=\mathfrak{U}+\mathfrak{U} p z+\mathfrak{U} p^{2} z^{2}+\mathfrak{A} p^{3} z^{3}+\cdots
\]
其通项为
\[
\mathfrak{A} p^{n} z^{n}
\]
称这个表达式为通项,是因为将其中的 $n$ 依次换为所有的整数, 我们就得到级数的所有 的项.

其次,分式
\[
\frac{\mathfrak{A}}{(1-p z)^{2}}
\]
产生的级数为
\[
\frac{\mathfrak{A}}{(1-p z)^{2}}=\mathfrak{U}+2 \mathfrak{U} p z+3 \mathfrak{X} p^{2} z^{2}+4 \mathfrak{X} p^{3} z^{3}+\cdots
\]
其通项为
\[
(n+1) \mathfrak{A} p^{n} z^{n}
\]
再次
\[
\frac{\mathfrak{X}}{(1-p z)^{3}}=\mathfrak{U}+3 \mathfrak{A} p z+6 \mathfrak{X} p^{2} z^{2}+10 \mathfrak{A} p^{3} z^{3}+\cdots
\]
其通项为
\[
\frac{(n+1)(n+2)}{1 \cdot 2} \mathfrak{X} p^{n} z^{n}
\]
一般地, 分式
\[
\frac{\mathfrak{U}}{(1-p z)^{k}}
\]
产生的级数
\[
\frac{\mathfrak{A}}{(1-p z)^{k}}=\mathfrak{U}+k \mathfrak{X} p z+\frac{k(k+1)}{1 \cdot 2} \mathfrak{X} p^{2} z^{2}+\frac{k(k+1)(k+2)}{1 \cdot 2 \cdot 3} \mathfrak{X} p^{3} z^{3}+\cdots
\]
其通项为
\[
\frac{(n+1)(n+2)(n+3) \cdots(n+k-1)}{1 \cdot 2 \cdot 3 \cdot \cdots \cdot(k-1)} \mathfrak{A} p^{n} z^{n}
\]
从级数的构成本身, 得到这通项为
\[
\frac{k(k+1)(k+2) \cdots(k+n-1)}{1 \cdot 2 \cdot 3 \cdot \cdots \cdot n} \mathfrak{X} p^{n} z^{n}
\]
这两个通项的相等, 可由交叉相乘得出. 事实上, 交叉相乘得
\[
\begin{aligned}
& 1 \cdot 2 \cdot 3 \cdot \cdots \cdot n(n+1) \cdot \cdots \cdot(n+k-1)= \\
& 1 \cdot 2 \cdot 3 \cdot \cdots \cdot(k-1) \cdot k \cdot \cdots \cdot(k+n-1)
\end{aligned}
\]
这是一个等式.

\section{$\S 216$}

这样, 每给一个分数函数, 我们先将它分解成状如 $\frac{\mathfrak{U}}{(1-p z)^{k}}$ 的部分分式, 再求出每 个部分分式的递推级数的通项, 最后将求得的通项加起来, 就得到所给分数函数的递推 级数
\[
A+B z+C z^{2}+D z^{3}+\cdots
\]
的通项.

例 1 求分数函数
\[
\frac{1-z}{1-z-2 z^{2}}
\]
的递推级数的通项.

由该函数得到的级数是 
\[
\begin{aligned}
& 1+0 z+2 z^{2}+2 z^{3}+6 z^{4}+10 z^{5}+22 z^{6}+42 z^{7}+86 z^{8}+\cdots
\end{aligned}
\]
为了求得通项的系数, 我们先将
\[
\frac{1-z}{1-z-2 z^{2}}
\]
分解成部分分式, 得
\[
\frac{\frac{2}{3}}{1+z}+\frac{\frac{1}{3}}{1-2 z}
\]
由此得到所求通项为
\[
\left(\frac{2}{3}(-1)^{n}+\frac{1}{3} 2^{n}\right) z^{n}=\frac{2^{n} \pm 2}{3} z^{n}
\]
$n$ 为偶数时取正号, $n$ 为奇数时取负号.

例 2 求分式
\[
\frac{1-z}{1-5 z+6 z^{2}}
\]
产生的递推级数
\[
1+4 z+14 z^{2}+46 z^{3}+146 z^{4}+454 z^{5}+\cdots
\]
的通项.

分母等于 $(1-2 z)(1-3 z)$, 从而分式分解为
\[
\frac{-1}{1-2 z}+\frac{2}{1-3 z}
\]
由此得通项为
\[
2 \cdot 3^{n} z^{n}-2^{n} z^{n}=\left(2 \cdot 3^{n}-2^{n}\right) z^{n}
\]
例 3 求分式
\[
\frac{1+2 z}{1-z-z^{2}}
\]
展成的级数
\[
1+3 z+4 z^{2}+7 z^{3}+11 z^{4}+18 z^{5}+29 z^{6}+47 z^{7}+\cdots
\]
的通项

分母的因式为

\[
1-\frac{1+\sqrt{5}}{2} z \text { 和 } 1-\frac{1-\sqrt{5}}{2} z
\]

从而分式的分解式为
\[
\frac{\frac{1+\sqrt{5}}{2}}{1-\frac{1+\sqrt{5}}{2} z}+\frac{\frac{1-\sqrt{5}}{2}}{1-\frac{1-\sqrt{5}}{2} z}
\]
由此得通项为 
\[
\left(\frac{1+\sqrt{5}}{2}\right)^{n+1} z^n+\left(\frac{1-\sqrt{5}}{2}\right)^{n+1} z^n
\]

例 4 求分式
\[
\frac{a+b z}{1-\alpha z-\beta z^{2}}
\]
展成的级数
\[
a+(\alpha a+b) z+\left(\alpha^{2} a+\alpha b+\beta a\right) z^{2}+\left(\alpha^{3} a+a^{2} b+2 \alpha \beta a+\beta b\right) z^{3}+\cdots
\]
的通项.

所给分式的部分分式表示式为
\[
\frac{\left(a\left(\sqrt{\alpha^{2}+4 \beta}+\alpha\right)+2 b\right): 2 \sqrt{\alpha^{2}+4 \beta}}{1-\frac{\alpha+\sqrt{\alpha^{2}+4 \beta}}{2}}+\frac{\left(a\left(\sqrt{\alpha^{2}+4 \beta}-\alpha\right)-2 b\right): 2 \sqrt{\alpha^{2}+4 \beta}}{1-\frac{\alpha-\sqrt{\alpha^{2}+4 \beta}}{2}}
\]
由此得所求通项为
\[
\begin{aligned}
& \frac{a\left(\sqrt{\alpha^{2}+4 \beta}+\alpha\right)+2 b}{2 \sqrt{\alpha^{2}+4 \beta}}\left(\frac{\alpha+\sqrt{\alpha^{2}+4 \beta}}{2}\right)^{n} z^{n}+ \\
& \frac{a\left(\sqrt{\alpha^{2}+4 \beta}-\alpha\right)-2 b}{2 \sqrt{\alpha^{2}+4 \beta}}\left(\frac{\alpha-\sqrt{\alpha^{2}+4 \beta}}{2}\right)^{n} z^{n}
\end{aligned}
\]
这是通项的通用公式, 适用于每项都由前两项推出的递推级数.

例 5 求分式
\[
\frac{1}{1-z-z^{2}+z^{3}}=\frac{1}{(1-z)^{2}(1+z)}
\]
展成的级数
\[
1+z+2 z^{2}+2 z^{3}+3 z^{4}+3 z^{5}+4 z^{6}+4 z^{7}+\cdots
\]
的通项.

级数系数的规律性是显然的. 分式的部分分式表示成为
\[
\frac{\frac{1}{2}}{(1-z)^{2}}+\frac{\frac{1}{4}}{1-z}+\frac{\frac{1}{4}}{1+z}
\]
由此得通项为
\[
\frac{1}{2}(n+1) z^{n}+\frac{1}{4} z^{n}+\frac{1}{4}(-1)^{n} z^{n}=\frac{2 n+3 \pm 1}{4} z^{n}
\]
$n$ 为偶数时取正号, $n$ 为奇数时取负号.

\section{$\S 217$}

用上述方法我们可以求出一切递推级数的通项, 因为分式都可以分解成以线性因式 的幂为分母的部分分式. 但是如果我们想避开虚表达式, 那我们就得考虑状如 
\[
 \frac{\mathfrak{A}+\mathfrak{B} p z}{1-2 p z \cos \varphi+p^{2} z^{2}}, \frac{\mathfrak{A}+\mathfrak{B} p z}{\left(1-2 p z \cos \varphi+p^{2} z^{2}\right)^{2}}, \frac{9(1+\mathfrak{B} p z}{\left(1-2 p z \cos \varphi+p^{2} z^{2}\right)^{k}}
\]
的部分分式所展成的级数. 首先, 由于
\[
\cos n \varphi=2 \cos \varphi \cos (n-1) \varphi-\cos (n-2) \varphi
\]
我们得到
\[
\frac{9 \mathfrak{A}}{1-2 p z \cos \varphi+p^{2} z^{2}}
\]
展成的级数为
\[
\begin{aligned}
& \mathfrak{A}+2 \mathfrak{A} p z \cos \varphi+2 \mathfrak{A} p^{2} z^{2} \cos 2 \varphi+2 \mathfrak{A} p^{3} z^{3} \cos 3 \varphi+ \\
& 2 \mathfrak{A} p^{4} z^{4}+\cos 4 \varphi+\cdots+\mathfrak{A} p^{2} z^{2}+2 \mathfrak{A} p^{3} z^{3} \cos \varphi+ \\
& 2 \mathfrak{A} p^{4} z^{4} \cos 2 \varphi+\cdots+9 \mathfrak{A} p^{4} z^{4}+\cdots
\end{aligned}
\]
这个级数的通项求起来不那么容易.

\section{$\S 218$}

为了写出上节级数的通项, 我们考虑这样两个级数
\[
\begin{gathered}
P p z \sin \varphi+P p^{2} z^{2} \sin 2 \varphi+P p^{3} z^{3} \sin 3 \varphi+P z^{4} z^{4} \sin 4 \varphi+\cdots \\
Q+Q p z \cos \varphi+Q p^{2} z^{2} \cos 2 \varphi+Q p^{3} z^{3} \cos 3 \varphi+Q p^{4} z^{4} \cos 4 \varphi+\cdots
\end{gathered}
\]
它们分别是分母同为
\[
1-2 p z \cos \varphi+p^{2} z^{2}
\]
的两个分式的展开式. 前一个分式为
\[
\frac{P p z \sin \varphi}{1-2 p z \cos \varphi+p^{2} z^{2}}
\]
后一个分式为
\[
\frac{Q-Q p z \cos \varphi}{1-2 p z \cos \varphi+p^{2} z^{2}}
\]
这两个分式相加, 和为
\[
\frac{Q+P p z \sin \varphi-Q p z \cos \varphi}{1-2 p z \cos \varphi+p^{2} z^{2}}
\]
这个和展成的级数,其通项为
\[
(P \sin n \varphi+Q \cos n \varphi) p^{n} z^{n}
\]
将分式
\[
\frac{\mathfrak{U}+\mathfrak{B} p z}{1-2 p z \cos \varphi+p^{2} z^{2}}
\]
与刚才的和相比较, 得
\[
Q=\mathfrak{A}, P=\mathfrak{A} \cos \varphi+\mathfrak{B} \csc \varphi
\]
从而,展开 
\[
\frac{\mathfrak{A}+\mathfrak{B} p z}{1-2 p z \cos \varphi+p^{2} z^{2}}
\]
所成的级数, 其通项为
\[
\frac{\mathfrak{A} \cos \varphi \sin n \varphi+\mathfrak{B} \sin n \varphi+\mathscr{A} \sin \varphi \cos n \varphi}{\sin \varphi} p^{n} z^{n}=\frac{\mathscr{A} \sin (n+1) \varphi+\mathfrak{B} \sin n \varphi}{\sin \varphi} p^{n} z^{n} 
\]
\section{$\S 219$}

为了得到以幂
\[
\left(1-2 p z \cos \varphi+p^{2} z^{2}\right)^{K}
\]
为分母情况下的通项, 我们把这种分式表示成两个含有虚数的分式之和
\[
\frac{a}{(1-(\cos \varphi+\sqrt{-1} \sin \varphi) p z)^{K}}+\frac{b}{(1-(\cos \varphi-\sqrt{-1} \sin \varphi) p z)^{K}}
\]
这个和展成的级数,其通项为
\[
\begin{aligned}
& \frac{(n+1)(n+2)(n+3) \cdots(n+K-1)}{1 \cdot 2 \cdot 3 \cdot \cdots \cdot(K-1)}(\cos n \varphi+\sqrt{-1} \sin n \varphi) a p^{n} z^{n}+ \\
& \frac{(n+1)(n+2)(n+3) \cdots(n+K-1)}{1 \cdot 2 \cdot 3 \cdot \cdots \cdot(K-1)}(\cos n \varphi-\sqrt{-1} \sin n \varphi) b p^{n} z^{n}
\end{aligned}
\]
$\hat{亽}$
\[
a+b=f, a-b=\frac{g}{\sqrt{-1}}
\]
冊
\[
a=\frac{f \sqrt{-1}+g}{2 \sqrt{-1}}, b=\frac{f \sqrt{-1}-g}{2 \sqrt{-1}}
\]
那么,表达式
\[
\frac{(n+1)(n+2)(n+3) \cdots(n+K-1)}{1 \cdot 2 \cdot 3 \cdot \cdots \cdot(K-1)}(f \cos n \varphi+g \sin n \varphi) p^{n} z^{n}
\]
就是分式的和
\[
\frac{\frac{1}{2} f+\frac{1}{2 \sqrt{-1}} g}{(1-(\cos \varphi+\sqrt{-1} \sin \varphi) p z)^{K}}+\frac{\frac{1}{2} f-\frac{1}{2 \sqrt{-1}} g}{(1-(\cos \varphi-\sqrt{-1} \sin \varphi) p z)^{K}}
\]
或写成单个分式
\[
\frac{\left\{\begin{array}{l}
f-K f p z \cos \varphi+\frac{K(K-1)}{1 \cdot 2} f p^{2} z^{2} \cos 2 \varphi-\frac{K(K-1)(K-2)}{1 \cdot 2 \cdot 3} f p^{3} z^{3} \cos 3 \varphi+\cdots+ \\
K g p z \sin \varphi-\frac{K(K-1)}{1 \cdot 2} f p^{2} z^{2} \sin 2 \varphi+\frac{K(K-1)(K-2)}{1 \cdot 2 \cdot 3} f p^{3} z^{3} \sin 3 \varphi-\cdots
\end{array}\right.}{\left(1-2 p z \cos \varphi+p^{2} z^{2}\right)^{k}}
\]
展成的级数的通项. 

\section{$\S 220$}

如果 $k=2$, 那么分式
\[
\frac{f-2 p z(f \cos \varphi-g \sin \varphi)+p^{2} z^{2}(f \cos 2 \varphi-g \sin 2 \varphi)}{\left(1-2 p z \cos \varphi+p^{2} z^{2}\right)^{2}}
\]
产生的级数, 其通项为
\[
(n+1)(f \cos n \varphi+g \sin n \varphi) p^{n} z^{n}
\]
但是,由分式
\[
\frac{a}{1-2 p z \cos \varphi+p^{2} z^{2}}=\frac{a-2 a p z \cos \varphi+a p^{2} z^{2}}{\left(1-2 p z \cos \varphi+p^{2} z^{2}\right)^{2}}
\]
产生的级数, 其通项为
\[
\frac{a \sin (n+1) \varphi}{\sin \varphi} p^{n} z^{n}
\]
将这两个分式相加, 并令
\[
\begin{gathered}
a+f=\mathfrak{A} \\
2 a \cos \varphi+2 f \cos \varphi-2 g \sin \varphi=-\mathfrak{B} \\
a+f \cos 2 \varphi-g \sin 2 \varphi=0
\end{gathered}
\]
则
\[
\begin{gathered}
g=\frac{\mathfrak{B}+2 \mathfrak{A} \cos \varphi}{2 \sin \varphi}=\frac{\mathfrak{B} \sin \varphi+\mathfrak{U} \sin 2 \varphi}{2 \sin ^{2} \varphi} \\
a=\frac{\mathfrak{H}+\mathfrak{B} \cos \varphi}{1-\cos 2 \varphi}=\frac{\mathfrak{H}+\mathfrak{B} \cos \varphi}{2 \sin ^{2} \varphi} \\
f=-\frac{\mathfrak{H} \cos 2 \varphi+\mathfrak{B} \cos \varphi}{2 \sin ^{2} \varphi}
\end{gathered}
\]
由此, 分式
\[
\frac{\mathfrak{U}+\mathfrak{B} p z}{\left(1-2 p z \cos \varphi+p^{2} z^{2}\right)^{2}}
\]
产生的级数,其通项为
\[
\begin{aligned}
& \frac{\mathfrak{H}+\mathfrak{B} \cos \varphi \sin (n+1) \varphi p^{n} z^{n}+(n+1) \cdot}{2 \sin ^{3} \varphi} \\
& \mathfrak{B} \sin \varphi \sin n \varphi+\mathfrak{U} \sin 2 \varphi \sin n \varphi-\mathfrak{B} \cos \varphi \cos n \varphi-\mathscr{U} \cos 2 \varphi \cos n \varphi p_{p^{n} z^{n}}= \\
& 2 \sin ^{2} \varphi \\
& -\frac{(n+1)(\mathfrak{X} \cos (n+2) \varphi+\mathfrak{B} \sin (n+1) \varphi)}{2 \sin ^{2} \varphi} p^{n} z^{n}+ \\
& \frac{(\mathfrak{U}+\mathfrak{B} \cos \varphi) \sin (n+1) \varphi}{2 \sin ^{3} \varphi} p^{n}= \\
& \frac{\frac{1}{2}(n+3) \sin (n+1) \varphi-\frac{1}{2}(n+1) \sin (n+3) \varphi}{2 \sin ^{3} \varphi}
\end{aligned}
\]
\[
 \frac{\frac{1}{2}(n+2) \sin n \varphi-\frac{1}{2} n \sin (n+2) \varphi}{2 \sin ^{3} \varphi} \mathfrak{B}_{p^{n} z^{n}}
\]
也即, 分式
\[
\frac{\mathfrak{U}+\mathfrak{B p z}}{\left(1-2 p z \cos \varphi+p^{2} z^{2}\right)^{2}}
\]
产生的级数, 其通项为
\[
\begin{gathered}
\frac{(n+3) \sin (n+1) \varphi-(n+1) \sin (n+3) \varphi}{4 \sin ^{3} \varphi} \mathfrak{X} p^{n} z^{n}+ \\
\frac{(n+2) \sin n \varphi-n \sin (n+2) \varphi_{\mathfrak{B}} p^{n} z^{n}}{4 \sin ^{3} \varphi} 
\end{gathered}
\]
\section{$\S 221$}

$k=3 $ 时, 分式 
\[\frac{f-3 p z(f \cos \varphi-g \sin \varphi)+3 p^{2} z^{2}(f \cos 2 \varphi-g \sin 2 \varphi)-p^{3} z^{3}(f \cos 3-g \sin 3 \varphi)}{\left(1-2 p z \cos \varphi+p^{2} z^{2}\right)^{3}\]
产生的级数,其通项为
\[
\frac{(n+1)(n+2)}{1 \cdot 2}(f \cos n \varphi+g \sin n \varphi) p^{n} z^{n}
\]
而分式
\[
\frac{a+b p z}{\left(1-2 p z \cos \varphi+p^{2} z^{2}\right)^{2}}=\frac{a-p z(2 a \cos \varphi-b)+p^{2} z^{2}(a-2 b \cos \varphi)+b q^{3} z^{3}}{\left(1-2 p z \cos \varphi+p^{2} z^{2}\right)^{3}}
\]
产生的级数,其通项为
\[
\begin{aligned}
& \frac{(n+3) \sin (n+1) \varphi-(n+1) \sin (n+3) \varphi}{4 \sin ^{3} \varphi} a p^{n} z^{n}+ \\
& \frac{(n+2) \sin n \varphi-n \sin (n+2) \varphi}{4 \sin ^{3} \varphi}
\end{aligned}
\]
这两个分式相加, 并令分子等于 $\mathfrak{U}$, 则
\[
\begin{gathered}
a+f=\mathfrak{A} \\
3 f \cos \varphi-3 g \sin \varphi+2 a \cos \varphi-b=0 \\
3 f \cos 2 \varphi-3 g \sin 2 \varphi+a-2 b \cos \varphi=0 \\
f \cos 3 \varphi-g \sin 3 \varphi-b=0
\end{gathered}
\]
从而得
\[
\begin{gathered}
a=\frac{f \cos 3 \varphi-g \sin 3 \varphi-3 f \cos \varphi+3 g \sin \varphi}{2 \cos \varphi}= \\
2 g \sin ^{2} \varphi \tan \varphi-f-2 \sin ^{2} \varphi
\end{gathered}
\]
又得

%%09p161-180
\[
\begin{aligned}
& \text { Infinite analysies } \\
& \qquad \frac{f}{g}=\frac{\sin 5 \varphi-2 \sin 3 \varphi+\sin \varphi}{\cos 5 \varphi-2 \cos 3 \varphi+\cos \varphi}
\end{aligned}
\]
和
\[
a+f=\mathfrak{U}=2 g \sin ^{2} \varphi \tan \varphi-2 f \sin ^{2} \varphi
\]
也即
\[
\frac{\mathfrak{X}}{2 \sin ^{2} \varphi}=\frac{g \sin \varphi-f \cos \varphi}{\cos \varphi}
\]
由此最后得到
\[
\begin{aligned}
& f=\frac{\mathfrak{A}(\sin \varphi-2 \sin 3 \varphi+\sin 5 \varphi)}{16 \sin ^{5} \varphi} \\
& g=\frac{\mathfrak{A}(\cos \varphi-2 \cos 3 \varphi+\cos 5 \varphi)}{16 \sin ^{5} \varphi}
\end{aligned}
\]
由于
\[
16 \sin ^{5} \varphi=\sin 5 \varphi-5 \sin 3 \varphi+10 \sin \varphi
\]
得到
\[
\begin{gathered}
a=\frac{\mathfrak{U}(9 \sin \varphi-3 \sin 3 \varphi)}{16 \sin ^{5} \varphi} \\
b=\frac{\mathfrak{U}(-\sin 2 \varphi+\sin 2 \varphi)}{16 \sin ^{5} \varphi}=0
\end{gathered}
\]
又由于
\[
3 \sin \varphi-\sin 3 \varphi=4 \sin ^{3} \varphi
\]
得到
\[
a=\frac{3 \mathfrak{A}}{4 \sin ^{2} \varphi}
\]
这样一来, 通项为
\[
\begin{aligned}
& \frac{(n+1)(n+2)}{1 \cdot 2} \mathscr{X} p^{n} z^{n} \frac{\sin (n+1) \varphi-2 \sin (n+3) \varphi+\sin (n+5) \varphi}{16 \sin ^{5} \varphi}+ \\
& 3 \mathscr{M} p^{n} z^{n} \cdot \frac{(n+3) \sin (n+1) \varphi-(n+1) \sin (n+3) \varphi}{16 \sin ^{5} \varphi}= \\
& \frac{9\left(p^{n} z^{n}\right.}{16 \sin ^{5} \varphi}\left\{\frac{(n+4)(n+5)}{1 \cdot 2} \sin (n+1) \varphi-\right. \\
& \frac{2(n+1)(n+5)}{1 \cdot 2} \sin (n+3) \varphi+ \\
& \left.\frac{(n+1)(n+2)}{1 \cdot 2} \sin (n+5) \varphi\right\}
\end{aligned}
\]
\section{$\S 222$}

这样,产生于 
\[
\frac{\mathfrak{A}+\mathfrak{B}_{p z}}{\left(1-2 p z \cos \varphi+p^{2} z^{2}\right)^{3}}
\]
的级数,其通项为
\[
\begin{aligned}
& \frac{\mathfrak{3} p^{n} z^{n}}{16 \sin ^{5} \varphi}\left\{\frac{(n+4)(n+5)}{1 \cdot 2} \sin (n+1) \varphi-\right. \\
& \frac{2(n+1)(n+5)}{1 \cdot 2} \sin (n+3) \varphi+ \\
& \left.\frac{(n+1)(n+2)}{1 \cdot 2} \sin (n+5) \varphi\right\}+ \\
& \frac{\mathfrak{B} p^{n} z^{n}}{16 \sin ^{5} \varphi}\left\{\frac{(n+3)(n+4)}{1 \cdot 2} \sin n \varphi-\right. \\
& \frac{2 n(n+4)}{1 \cdot 2} \sin (n+2) \varphi+ \\
& \left.\frac{n(n+1)}{1 \cdot 2} \sin (n+4) \varphi\right\}
\end{aligned}
\]
进一步,让 $k$ 的值加 1 , 我们有, 产生于
\[
\frac{\mathfrak{U}+\mathfrak{B} p z}{\left(1-2 p z \cos \varphi+p^{2} z^{2}\right)^{4}}
\]
的级数,其通项为
\[
\begin{aligned}
& \frac{\mathfrak{A} p^{n} z^{n}}{64 \sin ^{7} \varphi}\left\{\frac{(n+7)(n+6)(n+5)}{1 \cdot 2 \cdot 3} \sin (n+1) \varphi-\right. \\
& \frac{3(n+1)(n+7)(n+6)}{1 \cdot 2 \cdot 3} \sin (n+3) \varphi+ \\
& \frac{3(n+1)(n+2)(n+7)}{1 \cdot 2 \cdot 3} \sin (n+5) \varphi- \\
& \left.\frac{(n+1)(n+2)(n+3)}{1 \cdot 2 \cdot 3} \sin (n+7) \varphi\right\}+ \\
& \frac{\mathfrak{B} p^{n} z^{n}}{64 \sin ^{7} \varphi}\left\{\frac{(n+6)(n+5)(n+4)}{1 \cdot 2 \cdot 3} \sin n \varphi-\right. \\
& \frac{3 n(n+6)(n+5)}{1 \cdot 2 \cdot 3} \sin (n+2) \varphi+ \\
& \frac{3 n(n+1)(n+6)}{1 \cdot 2 \cdot 3} \sin (n+4) \varphi- \\
& \left.\frac{n(n+1)(n+2)}{1 \cdot 2 \cdot 3} \sin (n+6) \varphi\right\}
\end{aligned}
\]
分母幂次更高时, 通项的求法, 不难从上面的讨论弄清. 我们列出下面的等式 ${ }^{(1}$, 供进一 步讨论时使用.
\[
\sin \varphi=\sin \varphi
\]
(1) 参见 $\S 262 .-$ 译者 
\[
\begin{aligned}
& 4 \sin ^{3} \varphi=3 \sin \varphi-\sin 3 \varphi \\
& 16 \sin ^{5} \varphi=10 \sin \varphi-5 \sin 3 \varphi+\sin 5 \varphi \\
& 64 \sin ^{7} \varphi=35 \sin \varphi-21 \sin 3 \varphi+7 \sin 5 \varphi-\sin 7 \varphi \\
& 256 \sin ^{9} \varphi=125 \sin \varphi-84 \sin 3 \varphi+36 \sin 5 \varphi-9 \sin 7 \varphi+\sin 9 \varphi-\cdots 
\]
\section{$\S 223$}

部分分式级数的通项, 我们会求了. 部分分式级数通项之和, 就是分式级数的通项. 下面举例,具体说明.

从分式
\[
\frac{1}{(1-z)\left(1-z^{2}\right)\left(1-z^{3}\right)}=\frac{1}{1-z-z^{2}+z^{4}+z^{5}-z^{6}}
\]
得递推级数
\[
1+z+2 z^{2}+3 z^{3}+4 z^{4}+5 z^{5}+7 z^{6}+8 z^{7}+10 z^{8}+12 z^{9}+\cdots
\]
求通项.

分解分母为因式, 得
\[
\frac{1}{(1-z)^{3}(1+z)\left(1+z+z^{2}\right)}
\]
分解分式为部分分式, 得
\[
\begin{gathered}
\frac{1}{6(1-z)^{3}}+\frac{1}{4(1-z)^{2}}+\frac{17}{72(1-z)}+\frac{1}{8(1+z)}+\frac{2+z}{9\left(1+z+z^{2}\right)} \\
\text { 第一个部分分式 } \frac{1}{6(1-z)^{3}} \text { 给出通项 } \\
\frac{(n+1)(n+2) 1}{1 \cdot 2} \frac{1}{6} z^{n}=\frac{n^{2}+3 n+2}{12} z^{n}
\end{gathered}
\]
第二、三、四个部分分式
\[
\frac{1}{4(1-z)^{2}}, \frac{17}{72(1-z)}, \frac{1}{8(1+z)}
\]
给出的通项依次为
\[
\frac{n+1}{4} z^{n}, \frac{17}{72} z^{n}, \frac{1}{8}(-1)^{n} z^{n}
\]
第五个部分分式
\[
\frac{2+z}{9\left(1+z+z^{2}\right)}
\]
$\mapsto$
\[
\frac{\mathfrak{A}+\mathfrak{B} p z}{1-2 p z \cos \varphi+p^{2} z^{2}}(\S 218)
\]
相比较, 我们有 
\[
\begin{aligned}
& \text { Infinite analysis } \text { 无爸分析与讷) Slxaductian } \\
& \qquad p=-1, \varphi=\frac{\pi}{3}=60^{\circ}, \mathfrak{X}=\frac{2}{9}, \mathfrak{B}=-\frac{1}{9}
\end{aligned}
\]
从而第五个部分分式给出的通项为
\[
\begin{aligned}
\frac{2 \sin (n+1) \varphi-\sin n \varphi}{9 \sin \varphi}(-1)^{n} z^{n}= & \frac{4 \sin (n+1) \varphi-2 \sin n \varphi}{9 \sqrt{3}}(-1)^{n} z^{n}= \\
& \frac{4 \sin (n+1) \frac{\pi}{3}-2 \sin n \frac{\pi}{3}}{9 \sqrt{3}}(-1)^{n} z^{n}
\end{aligned}
\]
这五个通项的和
\[
\left(\frac{n^{2}}{12}+\frac{n}{2}+\frac{47}{72}\right) z^{n} \pm \frac{1}{8} z^{n} \pm \frac{4 \sin (n+1) \frac{\pi}{3}-2 \sin n \frac{\pi}{3}}{9 \sqrt{3}} z^{n}
\]
就是我们所求的通项. $n$ 为偶数时取正号, $n$ 为奇数时取负号, 我们指出, $n=3 m$ 时
\[
\frac{4 \sin \frac{1}{3}(n+1) \pi-2 \sin \frac{1}{3} n \pi}{9 \sqrt{3}}=\pm \frac{2}{9}
\]
$n=3 m+1$ 或 $n=3 m+2$ 时, 该表达式等于 $\mp \frac{1}{9}(n$ 为偶数时取负号, $n$ 为奇数时取正号 $)$. 由此我们得到:

\begin{tabular}{l|l}
\hline 如果 & 则通项为 \\
\hline$n=6 m+0$ & $\left(\frac{n^{2}}{12}+\frac{n}{2}+1\right) z^{n}$ \\
\hline$n=6 m+1$ & $\left(\frac{n^{2}}{12}+\frac{n}{2}+\frac{5}{12}\right) z^{n}$ \\
\hline$n=6 m+2$ & $\left(\frac{n^{2}}{12}+\frac{n}{2}+\frac{2}{3}\right) z^{n}$ \\
\hline$n=6 m+3$ & $\left(\frac{n^{2}}{12}+\frac{n}{2}+\frac{3}{4}\right) z^{n}$ \\
\hline$n=6 m+4$ & $\left(\frac{n^{2}}{12}+\frac{n}{2}+\frac{2}{3}\right) z^{n}$ \\
\hline$n=6 m+5$ & $\left(\frac{n^{2}}{12}+\frac{n}{2}+\frac{5}{12}\right) z^{n}$ \\
\hline
\end{tabular}

例如 $n=50$, 此时 $n=6 m+2$, 级数的项为 $234 z^{50}$.

例 6 从分式
\[
\frac{1+z+z^{2}}{1-z-z^{4}+z^{5}}
\]
得到递推级数
\[
1+2 z+3 z^{2}+3 z^{3}+4 z^{4}+5 z^{5}+6 z^{6}+6 z^{7}+7 z^{8}+\cdots
\]
求通项.

该分式可化成
\[
\frac{1+z+z^{2}}{(1-z)^{2}(1+z)\left(1+z^{2}\right)}
\]
从而可分解成
\[
\frac{3}{4(1-z)^{2}}+\frac{3}{8(1-z)}+\frac{1}{8(1+z)}+\frac{-1+z}{4\left(1+z^{2}\right)}
\]
前三个部分分式
\[
\frac{3}{4(1-z)^{2}}, \frac{3}{8(1-z)}, \frac{1}{8(1+z)}
\]
给出的通项依次为
\[
\frac{3(n+1)}{4} z^{n}, \frac{3}{8} z^{n}, \frac{1}{8}(-1)^{n} z^{n}
\]
将第四个部分分式 $\frac{-1+z}{4\left(1+z^{2}\right)}$ 与表达式
\[
\frac{\mathfrak{U}+\mathfrak{B} p z}{1-2 p z \cos \varphi+p^{2} z^{2}}
\]
相比较,我们有
\[
p=1, \cos \varphi=0, \varphi=\frac{\pi}{2}, \mathfrak{U}=-\frac{1}{4}, \mathfrak{B}=+\frac{1}{4}
\]
从而,第四个部分分式对应的通项为
\[
\left(-\frac{1}{4} \sin \frac{n+1}{2} \pi+\frac{1}{4} \sin \frac{n}{2} \pi\right) z^{n}
\]
我们所求的通项就等于这四个通项的和
\[
\left(\frac{3}{4} n+\frac{9}{8}\right) z^{n} \pm \frac{1}{8} z^{n}-\frac{1}{4}\left(\sin \frac{n+1}{2} \pi-\sin \frac{n}{2} \pi\right) z^{n}
\]
由此我们得到:

\begin{tabular}{l|l}
\hline 如果 & 则通项为 \\
\hline$n=4 m+0$ & $\left(\frac{3}{4} n+1\right) z^{n}$ \\
\hline$n=4 m+1$ & $\left(\frac{3}{4} n+\frac{5}{4}\right) z^{n}$ \\
\hline$n=4 m+2$ & $\left(\frac{3}{4} n+\frac{3}{2}\right) z^{n}$ \\
\hline$n=4 m+3$ & $\left(\frac{3}{4} n+\frac{3}{4}\right) z^{n}$ \\
\hline
\end{tabular}

例如, $n=50$, 此时 $n=4 m+2$, 级数的项为 $39 z^{50}$.

\section{$\S 224$}

由递推级数, 易于得到产生它的分式, 有了这分式, 就可以按刚讨论过的规则求出级数的通项. 从递推级数的递推性, 即每一项都可由它的前几项推出, 我们可直接写出产生 它的分式的分母. 这分母的因式决定通项的形状, 分子决定通项的系数. 设递推级数为
\[
A+B z+C z^{2}+D z^{3}+E z^{4}+F z^{5}+\cdots
\]
假定递推规律为
\[
D=\alpha C+\beta B+\gamma A, E=\alpha D+\beta C+\gamma B, F=\alpha E+\beta D+\gamma C, \cdots
\]
则分母为
\[
1-\alpha z-\beta z^{2}-\gamma z^{3}
\]
棣莫弗称
\[
+\alpha,+\beta,+\gamma
\]
为递推尺度, 递推尺度决定递推规律, 并给出由递推级数所产生的分式的分母.

\section{$\S 225$}

为了求出通项, 也即求出任何幂 $z^{n}$ 的系数, 我们先求出 $1-\alpha z-\beta z^{2}-\gamma z^{3}$ 的线性因 式, 或者有时为避免虚数, 求出它的二次因式. 如果因式都是实的, 且相异, 为
\[
(1-p z)(1-q z)(1-r z)
\]
那么产生级数的分式可分解为
\[
\frac{\mathfrak{U}}{1-p z}+\frac{\mathfrak{B}}{1-q z}+\frac{\mathfrak{S}}{1-r z}
\]
因而级数的通项为
\[
\left(\mathfrak{A} p^{n}+\mathfrak{B} q^{n}+\mathfrak{C}^{n}\right) z^{n}
\]
如果因式中有两个相等, 例如 $q=p$, 则通项为
\[
\left((\mathfrak{U}(n+1)+\mathfrak{B}) p^{n}+\left(\mathfrak{r ^ { n }}\right) z^{n}\right.
\]
如果 $p=q=r$, 则通项为
\[
\left(\mathfrak{U} \frac{(n+1)(n+2)}{1 \cdot 2}+\mathfrak{B}(n+1)+\widetilde{5}\right) p^{n} z^{n}
\]
如果分母 $1-\alpha z-\beta z^{2}-\gamma z^{3}$ 有二次因式,即它等于
\[
(1-p z)\left(1-2 q z \cos \varphi+q^{2} z^{2}\right)
\]
则通项为
\[
\left(\mathfrak{A} p^{n}+\frac{\mathfrak{B} \sin (n+1) \varphi+\mathfrak{S} \sin n \varphi}{\sin \varphi} q^{n}\right) z^{n}
\]
令 $n$ 等于 $1,2,3$, 我们就应该得到 $A, B z, C z^{2}$, 由此即可求出 $\mathfrak{I}, \mathfrak{B}, \mathfrak{S}$ 的值.

\section{$\S 226$}

如果递推尺度只有两个数, 即级数的每一项都可由其前两项推出, 关系式为
\[
C=\alpha B-\beta A, D=\alpha C-\beta B, E=\alpha D-\beta C, \cdots
\]
则产生级数 

\[
A+B z+C z^2+D z^3+E z^4+\cdots+P z^n+Q z^{n+1}+\cdots
\]

的分式的分母为
\[
1-\alpha z+\beta z^{2}
\]
如果这分母的因式为
\[
(1-p z)(1-q z)
\]
则
\[
p+q=\alpha, p q=\beta
\]
且级数的通项为
\[
\left(\mathfrak{U} p^{n}+\mathfrak{B} q^{n}\right) z^{n}
\]
由此, $n=0$, 得
\[
A=\mathfrak{A}+\mathfrak{B}
\]
$n=1$ 得
\[
B=\mathfrak{X}_{p}+\mathfrak{B} q
\]
从而
\[
\begin{gathered}
A q-B=\mathfrak{U}(q-p) \\
\mathfrak{U}=\frac{A q-B}{q-p}, \mathfrak{B}=\frac{A p-B}{p-q}
\end{gathered}
\]
有了 $\mathfrak{U}$ 和 $\mathfrak{B}$, 就可以得到
\[
\begin{gathered}
P=\mathfrak{U} \mathfrak{p}^{\mathfrak{n}}+\mathfrak{B} q^{n}, Q=\mathfrak{U} \mathfrak{p}^{\mathfrak{n}+1}+\mathfrak{B} q^{n+1} \\
\mathfrak{X} \mathfrak{B}=\frac{B^{2}-\alpha A B+\beta A^{2}}{4 \beta-\alpha^{2}} 
\end{gathered}
\]
\section{$\S 227$}

有了上节的准备, 我们可以找到方法, 使得每项都可以由它的前一项, 而不是前两项 推出,由
\[
P=\mathfrak{A} p^{n}+\mathfrak{B} q^{n}, Q=\mathfrak{A} p \cdot p^{n}+\mathfrak{B} q \cdot q^{n}
\]
得
\[
P q-Q=\mathfrak{U}(q-p) p^{n}, P q-Q=\mathfrak{B}(p-q) q^{n}
\]
两式相乘,得
\[
P^{2} p q-(p+q) P Q+Q^{2}+\mathfrak{X B}(p-q)^{2} p^{n} q^{n}=0
\]
但
\[
p+q=\alpha, p q=\beta,(p-q)^{2}=(p+q)^{2}-4 p q=\alpha^{2}-4 \beta, p^{n} q^{n}=\beta^{n}
\]
代入前式, 得
\[
\beta P^{2}-\alpha P Q+Q^{2}=\left(\beta A^{2}-\alpha A B+B^{2}\right) \beta^{n}
\]
或
\[
\frac{\beta P^{2}-\alpha P Q+Q^{2}}{\beta A^{2}-\alpha A B+\beta^{2}}=\beta^{n}
\]
每一项都是由前两项推出的, 这是递推级数的一条重要性质. 根据这条性质, 从任何一项 $P$ 都可推出其下一项 $Q$
\[
Q=\frac{1}{2} \alpha P+\sqrt{\left(\frac{1}{4} \alpha^{2}-\beta\right) P^{2}+\left(B^{2}-\alpha A B+\beta A^{2}\right) \beta^{n}}
\]
虽然式中有根号, 但不会得到无理数, 因为递推级数的项都是有理的.

\section{$\S 228$}

进一步, 从任给的相邻两项 $P z^{n}$ 和 $Q z^{n+1}$, 我们可以求出远离它们的项 $X z^{2 n}$. 令
\[
X=f P^{2}+g P Q-h \mathscr{\mathcal { X }} \mathfrak{B} \beta^{n}
\]
由
\[
\begin{gathered}
P=\mathfrak{U} p^{n}+\mathfrak{B} q^{n} \\
Q=\mathfrak{X} p^{n+1}+\mathfrak{B} q^{n+1} \\
X=\mathfrak{A} p^{2 n}+\mathfrak{B} q^{2 n}
\end{gathered}
\]
得
\[
\begin{gathered}
f P^{2}=f \mathfrak{U}^{2} p^{2 n}+f \mathfrak{B}^{2} q^{2 n}+2 f \mathfrak{U} \mathfrak{B} \beta^{n} \\
g P Q=g \mathfrak{U}^{2} p \cdot p^{2 n}+g \mathfrak{B}^{2} q \cdot q^{2 n}+g \mathfrak{U} \mathfrak{B} \alpha \beta^{n} \\
-h \mathfrak{U} \mathfrak{B} \beta^{n}=-h \mathfrak{A} \mathfrak{B} \beta^{n} \\
X=\mathfrak{U} P^{2 n}+\mathfrak{B} Q^{2 n}
\end{gathered}
\]
比较右端, 得
\[
\begin{aligned}
& f+g p=\frac{1}{\mathfrak{A}} \\
& f+g p=\frac{1}{\mathfrak{B}} \\
& h=2 f+g \alpha
\end{aligned}
\]
从而
\[
g=\frac{\mathfrak{B}-\mathfrak{U}}{\mathfrak{H} \mathfrak{B}(p-q)}, f=\frac{\mathfrak{A} p-\mathfrak{B} q}{\mathfrak{U} \mathfrak{B}(p-q)}
\]
但
\[
\mathfrak{B}-\mathfrak{U}=\frac{\alpha A-2 B}{p-q}, \mathfrak{A} p-\mathfrak{B} q=\frac{\alpha B-2 \beta A}{p-q}
\]
代入上式,得
\[
f=\frac{\alpha B-2 \beta A}{\mathfrak{H} \mathfrak{B}\left(\alpha^{2}-4 \beta\right)}, g=\frac{\alpha A-2 B}{\mathfrak{H} \mathfrak{B}\left(\alpha^{2}-4 \beta\right)}
\]
或者
\[
\begin{aligned}
& f=\frac{2 \beta A-\alpha B}{B^{2}-\alpha A B+\beta A^{2}} \\
& g=\frac{2 B-\alpha A}{B^{2}-\alpha A B+\beta A^{2}}
\end{aligned}
\]
\[
h=\frac{\left(4 \beta-\alpha^{2}\right) A}{B^{2}-\alpha A B+\beta A^{2}}
\]
最后得到
\[
X=\frac{(2 \beta A-\alpha B) P^{2}+(2 B-\alpha A) P Q}{B^{2}-\alpha A B+\beta A^{2}}-A \beta^{n}
\]
用类似地方法我们得到
\[
X=\frac{\left(\alpha \beta A-\left(\alpha^{2}-2 \beta\right) B\right) P^{2}+(2 B-\alpha A) Q^{2}}{\alpha\left(B^{2}-\alpha A B+\beta A^{2}\right)}-\frac{2 B \beta^{n}}{\alpha}
\]
从 $X$ 的两个表达式中消去含 $\beta^{n}$ 的项, 得
\[
X=\frac{(\beta A-\alpha B) P^{2}+2 B P Q-A Q^{2}}{B^{2}-\alpha A B+\beta A^{2}} 
\]
\section{$\S 229$}

用上节方法, 我们来确定更远的项,记级数为
\[
A+B z+C z^{2}+\cdots+P z^{n}+Q z^{n+1}+R z^{n+2}+\cdots+X z^{2 n}+Y z^{2 n+1}+Z z^{2 n+2}+\cdots
\]
那么由上节结果我们有
\[
Z=\frac{(\beta A-\alpha B) Q^{2}+2 B Q R-A R^{2}}{B^{2}-\alpha A B+\beta A^{2}}
\]
将
\[
R=\alpha Q-\beta P
\]
代入,得
\[
Z=\frac{-\beta^{2} A P^{2}+2 \beta(\alpha A-B) P Q+\left(\alpha B-\left(\alpha^{2}-\beta\right) A\right) Q^{2}}{B^{2}-\alpha A B+\beta A^{2}}
\]
由 $Z=\alpha Y-\beta X$, 从而 $Y=\frac{Z+\beta X}{\alpha}$; 由此得
\[
Y=\frac{-\beta B P^{2}+2 \beta A P Q+(B-\alpha A) Q^{2}}{B^{2}-\alpha A B+\beta A^{2}}
\]
这样, 用上节方法从 $X, Y$ 我们可以确定 $z^{4 n}$ 和 $z^{4 n+1}$ 的系数, 再进一步, 可以确定 $z^{8 n}$ 和 $z^{8 n+1}$ 的系数,类推.

例 递推级数
\[
1+3 z+4 z^{2}+7 z^{3}+11 z^{4}+18 z^{5}+\cdots+P z^{n}+Q z^{n+1}+\cdots
\]
每一项的系数都等于其前两项系数的和, 从而产生这个级数的分式的分母为
\[
1-z-z^{2}
\]
继而
\[
\alpha=1, \beta=-1 \text {, 又 } A=1, B=3
\]
进而
\[
B^{2}-\alpha A B+\beta A^{2}=5
\]
这样我们得到
\[
Q=\frac{P+\sqrt{5 P^{2}+20(-1)^{n}}}{2}=\frac{P+\sqrt{5 P^{2} \pm 20}}{2}
\]
其中的双重符号, $n$ 为偶数时取正, $n$ 为奇数时取负. 取 $n=4$, 此时 $P=11$, 计算得
\[
Q=\frac{11+\sqrt{5 \cdot 121+20}}{2}=\frac{11+25}{2}=18
\]
记 $z^{2 n}$ 的系数为 $X$, 则
\[
X=\frac{-4 P^{2}+6 P Q-Q^{2}}{5}
\]
由此得到 $z^{8}$ 的系数为
\[
\frac{-4 \cdot 121+6 \cdot 198-324}{5}=76
\]
由
\[
Q=\frac{P+\sqrt{5 P^{2} \pm 20}}{2}
\]
得
\[
Q^{2}=\frac{3 P^{2} \pm 10+P \sqrt{5 P^{2} \pm 20}}{2}
\]
从而
\[
X=\frac{-P^{2} \mp 2+P \sqrt{5 P^{2} \pm 20}}{2}
\]
我们看到, 在本例中, 对任何的 $n$, 由 $P z^{n}$ 我们都得到
\[
\frac{P+\sqrt{5 P^{2} \pm 20}}{2} z^{n+1}, \frac{-P^{2} \mp 2+P \sqrt{5 P^{2} \pm 20}}{2} z^{2 n}
\]
\section{$\S 230$}

类似地, 对每项都由其前三项推出的递推级数, 也可求出每项只由其前两项推出的 公式. 设级数
\[
A+B z+C z^{2}+D z^{3}+\cdots+P z^{n}+Q z^{n+1}+R z^{n+2}+\cdots
\]
每项都由其前三项推出, 递推尺度为 $\alpha,-\beta,+\gamma$, 也即产生该级数的那个分式, 其分母为
\[
1-\alpha z+\beta z^{2}-\gamma z^{3}
\]
借助分母的因式
\[
(1-p z)(1-q z)(1-r z)
\]
表示 $P, Q, R$ 时, 我们有
\[
\begin{gathered}
P=\mathfrak{X} p^{n}+\mathfrak{B} q^{n}+\mathfrak{C}^{n} \\
Q=\mathfrak{U} p p^{n}+\mathfrak{B} q q^{n}+\mathfrak{5} r r^{n}
\end{gathered}
\]
\[
 R=\mathfrak{A} p^{2} p^{n}+\mathfrak{B} q^{2} q^{n}+\mathscr{C} r^{2} r^{n}
\]
由于
\[
p+q+r=\alpha, p q+p r+q r=\beta, p q r=\gamma
\]
我们得到方程
\[
\begin{aligned}
& R^{3}-(2 \alpha Q-\beta P) R^{2}+\left(\left(\alpha^{2}+\beta\right) Q^{2}-(\alpha \beta+3 \gamma) P Q+\alpha \gamma P^{2}\right) R- \\
& \left((\alpha \beta-\gamma) Q^{3}-\left(\alpha \gamma+\beta^{2}\right) Q^{2} P+2 \beta \gamma P^{2} Q-\gamma^{2} P^{3}\right)= \\
& \left(C^{3}-(2 \alpha B-\beta A) C^{2}+\left(\left(\alpha^{2}+\beta\right) B^{2}-(\alpha \beta+3 \gamma) A B+\alpha \gamma A^{2}\right) C-\right. \\
& \left.\left((\alpha \beta-\gamma) B^{3}-\left(\alpha \gamma+\beta^{2}\right) A B^{2}+2 \beta \gamma A^{2} B-\gamma^{2} A^{3}\right)\right) \cdot \gamma^{n}
\end{aligned}
\]
这是 $R$ 的三次方程,这个方程的解,就是由前两项 $P, Q$ 推出 $R$ 的公式.

\section{$\S 231$}

上面讨论了递推级数的通项. 现在我们来考察它的和. 先指出一点, 递推级数的和等 于产生它的那个分式. 这分式的分母可根据递推规律写出. 所以就只剩下求分子了. 设级 数为
\[
A+B z+C z^{2}+D z^{3}+E z^{4}+F z^{5}+G z^{6}+\cdots
\]
递推规律给出分母为
\[
1-\alpha z+\beta z^{2}-\gamma z^{3}+\delta z^{4}
\]
记级数的和, 也即产生级数的分式为
\[
\frac{a+b z+c z^{2}+d z^{3}}{1-\alpha z+\beta z^{2}-\gamma z^{3}+\delta z^{4}}
\]
相比较,我们得到
\[
\begin{gathered}
a=A \\
b=B-\alpha A \\
c=C-\alpha B+\beta A \\
d=D-\alpha C+\beta B-\gamma A
\end{gathered}
\]
代入上式, 得级数的和为
\[
\frac{A+(B-\alpha A) z+(C-\alpha B+\beta A) z^{2}+(D-\alpha C+\beta B-\gamma A) z^{3}}{1-\alpha z+\beta z^{2}-\gamma z^{3}+\delta z^{4}} 
\]
\section{$\S 232$}

有了级数和,求出到某项为止的部分和就不难了. 记部分和为
\[
s=A+B z+C z^{2}+D z^{3}+E z^{4}+\cdots+P z^{n}
\]
级数和已知, 现在求 $P z^{n}$ 后面那一部分的和,记它为
\[
t=Q z^{n+1}+R z^{n+2}+S z^{n+3}+T z^{n+4}+\cdots
\]
用 $z^{n+1}$ 除这个 $t$, 我们得到一个类似于整个级数的级数, 因而它的和为 
\[
\begin{aligned}
& t=\frac{Q z^{n+1}+(R-\alpha Q) z^{n+2}+(S-\alpha R+\beta Q) z^{n+3}+(T-\alpha S+\beta R-\gamma Q) z^{n+4}}{1-\alpha z+\beta z^{2}-\gamma z^{3}+\delta z^{4}}
\end{aligned}
\]
从整个级数的和减去后面那一部分的和, 就得到我们所要的部分和
\[
\begin{gathered}
s=\frac{A+(B-\alpha A) z+(C-\alpha B+\beta A) z^{2}+(D-\alpha C+\beta B-\gamma A) z^{3}}{1-\alpha z+\beta z^{2}-\gamma z^{3}+\delta z^{4}}- \\
\frac{Q z^{n+1}+(R-\alpha Q) z^{n+2}+(S-\alpha R+\beta Q) z^{n+3}+(T-\alpha S+\beta R-\gamma Q) z^{n+4}}{1-\alpha z+\beta z^{2}-\gamma z^{3}+\delta z^{4}}
\end{gathered}
\]
\section{$\S 233$}

如果递推尺度为两个数: $\alpha,-\beta$, 那么从分式
\[
\frac{A+(B-\alpha A) z}{1-\alpha z+\beta z^{2}}
\]
得到的
\[
A+B z+C z^{2}+D z^{3}+\cdots+P z^{n}
\]
的和为
\[
\frac{A+(B-\alpha A) z-Q z^{n+1}-(R-\alpha Q) z^{n+2}}{1-\alpha z+\beta z^{2}}
\]
由级数的性质我们有

代入上式, 得
\[
R=\alpha Q-\beta P
\]
\[
\frac{A+(B-\alpha A) z-Q z^{n+1}+\beta P z^{n+2}}{1-\alpha z+\beta z^{2}}
\]
例 7 设级数截止到 $P z^{n}$ 的这一部分为
\[
1+3 z+4 z^{2}+7 z^{3}+\cdots+P z^{n}
\]
这里
\[
\alpha=1, \beta=-1, A=1, B=3
\]
则这一部分的和为
\[
\frac{1+2 z-Q z^{n+1}-P z^{n+2}}{1-z-z^{2}}
\]
令 $z=1$, 则这和等于
\[
1+3+4+7+11+\cdots+P=P+Q-3
\]
将
\[
Q=\frac{P+\sqrt{5 P^{2} \pm 20}}{2}
\]
代入, 得
\[
1+3+4+7+11+\cdots+P=\frac{3 P-6+\sqrt{5 P^{2} \pm 20}}{2}
\]
也即这时的部分和可从最后一项求出. 

