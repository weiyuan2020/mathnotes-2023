\chapter{第十一章 四阶线}

\section{$\S 260$}

四阶线的通用方程为
\[
\begin{aligned}
& \alpha y^{4}+\beta y^{3} x+\gamma y^{2} x^{2}+\delta y x^{3}+\varepsilon x^{4}+\zeta y^{3}+\eta y^{2} x+ \\
& \theta y x^{2}+\omega x^{3}+\kappa y^{2}+\lambda y x+\mu x^{2}+\nu y+\xi x+o=0
\end{aligned}
\]
但是根据不同情况采用不同方式, 利用改变坐标角、改换轴和原点的位置, 可以把它化得 更简单, 为了用讲过的方法对四阶线进行分类, 我们根据最高次部分的因式将方程分为 下列八种情况:

I. 四个线性因式全是虚的;

II. 只有两个线性因式是实的, 且不相等;

III. 只有两个线性因式是实的, 且相等;

IV.四个线性因式全实,且相异;

$\mathrm{V}$. 两个相等,另外两个不等;

VI. 不同的两对,每对相等;

VII.三个相等;

VIII. 四个全相等.

情况 I

\section{$\S 261$}

最高次部分的四个线性因式全虚时,曲线没有伸向无穷的分支. 我们是根据伸向无 穷的分支分类的,所以本情况下的四阶线属同一类,于是我们有

第一类

曲线没有伸向无穷的分支,其方程的最简形状为
\[
\begin{aligned}
& \left(y^{2}+m^{2} x^{2}\right)\left(y^{2}-2 p x y+q^{2} x^{2}\right)+a y^{2} x+b y x^{2}+ \\
& c y^{2}+d y x+e x^{2}+f y+g x+h=0
\end{aligned}
\]
其中 $p^{2}<q^{2}$. 由于四次部分必定包含项 $y^{4}$ 和 $x^{4}$, 因而用所给坐标 $x, y$ 加上或减去某个数 的方法,可以消去三次部分中的 $y^{3}$ 和 $x^{3}$. 

情况 II

\section{$\S 262$}

只有两个因式是实的, 且不相等时, 用改变倾角和轴的位置的方法, 可使这两个因式 一个为 $y$, 一个为 $x$. 即方程的形状为
\[
y x\left(y^{2}-2 m y x+n^{2} x^{2}\right)+a y^{2} x+b y x^{2}+c y^{2}+d y x+e x^{2}+f y+g x+h=0
\]
其中 $m^{2}<n^{2}$.

四次部分中必定含有 $y^{3} x$ 和 $y x^{3}$, 因而可以消去三次部分中的 $y^{3}$ 和 $x^{3}$. 从而曲线有 两条渐近直线, 其方程分别为 $y=0$ 和 $x=0$. 其性质依次由方程
\[
\begin{gathered}
n^{2} y x^{3}+e x^{2}+g x+h=0 \\
x y^{3}+c y^{2}+f y+h=0
\end{gathered}
\]
表示,这样我们得到

第二类

有两条渐近直线, 状如 $u=\frac{A}{t}$, 其中 $c \neq 0, e \neq 0$.

第三类

有两条渐近直线, 形状分别为 $u=\frac{A}{t}$ 和 $u=\frac{A}{t^{2}}$, 方程为
\[
y x\left(y^{2}-2 m y x+n^{2} x^{2}\right)+a y^{2} x+b y x^{2}+c y^{2}+d y x+f y+g x+h=0
\]
其中 $c \neq 0, g \neq 0$.

第四类

有两条渐近直线, 形状分别为 $u=\frac{A}{t}$ 和 $u=\frac{A}{t^{3}}$, 包含于方程
\[
y x\left(y^{2}-2 m y x-n^{2} x^{2}\right)+a y^{2} x+b y x^{2}+c y^{2}+d y x+f y+h=0
\]
其中 $c \neq 0$.

第五类

有两条渐近直线,都属 $u=\frac{A}{t^{2}}$ 型, 包含于方程
\[
y x\left(y^{2}-2 m y x+n^{2} x^{2}\right)+a y^{2} x+b y x^{2}+d y x+f y+g x+h=0
\]
其中 $f \neq 0, g \neq 0$.

第六类

有两条渐近直线, 形状分别为 $u=\frac{A}{t^{2}}$ 和 $u=\frac{A}{t^{3}}$, 包含于方程
\[
y x\left(y^{2}-2 m y x+n^{2} x^{2}\right)+a y^{2} x+b y x^{2}+d y x+f y+h=0
\]
其中 $f \neq 0$. 

第七类

有两条渐近直线, 形状都为 $u=\frac{A}{t^{3}}$, 含于方程
\[
y x\left(y^{2}-2 m y x+n^{2} x^{2}\right)+a y^{2} x+b y x^{2}+d y x+h=0
\]
以上各类 $n^{2}$ 都大于 $m^{2}$.

情况 III

\section{$\S 263$}

只有两个因式是实的, 且相等, 此时方程为
\[
y^{2}\left(y^{2}-2 m y x+n^{2} x^{2}\right)+a y x^{2}-b x^{3}+c y^{2}+d y x+e x^{2}+f y+g x+h=0
\]
这里又是 $n^{2}>m^{2}$. 只要 $b \neq 0$, 该方程就给出

第八类

有一条 $u^{2}=A t$ 状的渐近抛物线.

如果 $b=0$, 那么 $x=\infty$ 时得
\[
y^{2}+\frac{a y}{n^{2}}+\frac{e}{n^{2}}+\frac{g}{n^{2} x}+\frac{h}{n^{2} x^{2}}=0
\]
从而 $a^{2}<4 n^{2} e$ 时得

第九类

没有伸向无穷的分支.

如果 $b=0, a^{2}<4 n^{2} e$, 且 $g \neq 0$, 我们得到

第十类

有两条平行的状如 $u=\frac{A}{t}$ 的渐近线.

第十 a 类

有两条相重合的状如 $u=\frac{A}{t}$ 的渐近线.

如果 $b=0, g=0$, 且 $a^{2}>4 n^{2} e$, 我们有

第十一类

有两条平行的状如 $u=\frac{A}{t^{2}}$ 的渐近线.

如果 $b=0, a^{2}=4 n^{2} e$, 但 $g \neq 0$, 我们有

第十二类

有一条 $u^{2}=\frac{A}{t}$ 状的渐近线.

如果 $b=0, g=0, a^{2}=4 n^{2} e$, 且 $h<0$, 我们有 

第十三类

有一条 $u^{2}=\frac{A}{t^{2}}$ 状的双曲渐近线.

$b=0, g=0, a^{2}=4 n^{2} e$, 而 $h>0$ 时, 得

第十四类

没有伸向无穷的分支.

情况 V

\section{$\S 264$}

四次部分的四个因式全实, 且相异时, 方程的形状为
\[
y x(y-m x)(y-n x)+a y^{2} x+b y x^{2}+c y^{2}+d y x+e x^{2}+f y+g x+h=0
\]
曲线有四条渐近直线, 其形状为 $u=\frac{A}{t}$, 或 $u=\frac{A}{t^{2}}$, 或 $u=\frac{A}{t^{3}}$. 根据 $\S 251$ 所讲得

第十五类

有四条双曲渐近线, 形状全为 $u=\frac{A}{t}$.

第十六类

有四条双曲渐近线,三条状如 $u=\frac{A}{t}$,一条状如 $u=\frac{A}{t^{2}}$.

第十七类

有四条双曲渐近线, 三条状如 $u=\frac{A}{t}$,一条状如 $u=\frac{A}{t^{3}}$.

第十八类

有四条双曲渐近线, $u=\frac{A}{t}$ 状, $u=\frac{A}{t^{2}}$ 状各两条.

第十九类

有四条双曲渐近线, $u=\frac{A}{t}$ 状两条, $u=\frac{A}{t^{2}}$ 状, $u=\frac{A}{t^{3}}$ 状各一条.

第二十类

有四条双曲渐近线, $u=\frac{A}{t}$ 状, $u=\frac{A}{t^{3}}$ 状各两条.

第二十一类

有四条双曲渐近线, 形状全为 $u=\frac{A}{t^{2}}$.

第二十二类

有四条双曲渐近线, $u=\frac{A}{t^{2}}$ 状三条, $u=\frac{A}{t^{3}}$ 状一条.

%%07p121-140
第二十三类

有四条双曲渐近线, $u=\frac{A}{t^{2}}$ 状, $u=\frac{A}{t^{3}}$ 状各两条.

第二十四类

有四条双曲渐近线, 形状全为 $u=\frac{A}{t^{3}}$.

情况 V

\section{$\S 265$}

四个因式, 两个相等, 另外两个不等时, 方程为
\[
y^{2} x(y+n x)+a y x^{2}+b x^{3}+c y^{2}+d y x+e x^{2}+f y+g x+h=0
\]
首先相等因式部分产生同于情况 III 的各类,不相等部分产生同于情况 II 的各类,两部分 组合产生 $7 \times 6=42$ 类. 但这里面有两种是不可能的, 即两条 $u=\frac{A}{t^{2}}$ 状的平行渐近线, 第三 条为 $u=\frac{A}{t}$ 状的,第四条或者为 $u=\frac{A}{t^{2}}$ 状的, 或者为 $u=\frac{A}{t^{3}}$ 状的. 这样剩下的就是 40 类, 加 上前面的 24 类, 总共是 64 类. 逐条处理, 篇幅太长, 时间也不允许, 不逐条处理, 当然不能 断定它们都是实的, 有谁感到需要, 逐条进行处理, 可能会对这分类作出订正.

情况 VI

\section{$\S 266$}

不同的两对因子, 每对相等,此时方程为
\[
y^{2} x^{2}+a y^{3}+b x^{3}+c y^{2}+d y x+e x^{2}+f y+g x+h=0
\]
每对产生 7 类, 组合成 49 类, 但是由于 $b$ 不能同时既为正又为负, 所以有两类不可能, 这 样本情况共产生 47 类, 逐条处理,这数目也太大, 到现在为止, 我们共得到 111 类.

情况 VII

$y^{3} x+a y x^{2}+b x^{3}+c y^{2}+d y x+e x^{2}+f y+g x+h=0$

因式 $x: c \neq 0$ 时, 给出 $u=\frac{A}{t}$ 状渐近线; $c=0, f \neq 0$ 时, 给出 $u=\frac{A}{t^{2}}$ 状渐近线; $c=0, f=0$ 时,给出 $u=\frac{A}{t^{3}}$ 状渐近线. 因式 $y^{3}$, 除非 $b=0$, 都给出 $u^{3}=A t^{2}$ 状抛物渐近线, 如果 $b=0$, 那 么置 $x=\infty$ 得
\[
y^{3}+a y x+d y+e x+g+\frac{c y^{2}+f y+h}{x}=0
\]
这里, 如果 $e \neq 0$, 得 $y^{3}+a y x+e x=0$, 由此, 如果 $a \neq 0$, 得 $y^{2}+a x=0$ 和 $a y+e=0$. 从而, 在 $u^{2}=A t$ 状的抛物渐近线之外,还有一条双曲渐近线,其方程为
\[
(a y+e) x-\frac{e^{3}}{a^{3}}-\frac{d e}{a}+g+\frac{c e^{2}-a f e+a^{2} h}{a^{2} x}=0
\]
$e^{3}+a^{2} d e-a^{3} g \neq 0$ 时, 该渐近线为 $u=\frac{A}{t}$ 状; 否则, 为 $u=\frac{A}{t^{2}}$ 状. $a=0, e \neq 0$ 时, 得 $y^{2}+$ $e x=0$, 给出 $u^{3}=A t$ 状的抛物渐近线. $e=0, a=0$ 时, 得 $y^{3}+d y+g=0$, 它给出的渐近线 是: $u=\frac{A}{t}$ 状的一条或三条,或者一条 $u=\frac{A}{t}$ 状的和一条 $u^{2}=\frac{A}{t}$ 状的;或者一条 $u^{3}=\frac{A}{t}$ 状 的. 总共是 8 类,与因式 $x$ 产生的 3 类相组合, 得 24 类, 加到前六种情况上去, 共计 135 类.

情况 VIII

\section{$\S 268$}

四个因式全相等,此时因式为
\[
y^{4}+a y^{2} x+b y x^{2}+k x^{3}+c y^{2}+d y x+e x^{2}+f y+g x+h=0
\]
$k \neq 0$ 时 , 得

第一百三十六类

有唯一的 $u^{4}=A t^{3}$ 状抛物渐近线.

$k=0, b \neq 0$, 得 $y^{4}+b y x^{2}+e x^{2}=0$. 由此得 $y^{3}+b x^{2}=0$ 和 $b y+e=0$. 由渐近直线 $b y+e=0$ 得
\[
(b y+e) x^{2}+\frac{e^{4}}{b^{4}}+\frac{a e^{2} x}{b^{2}}+\frac{c e^{2}}{b^{2}}-\frac{d e x}{b}-\frac{e f}{b}+g x+h=0
\]
从而 $a e^{2}-b d e+b^{2} g \neq 0$ 时, 得 $u=\frac{A}{t}$ 状的渐近线, 否则, 得 $u=\frac{A}{t^{2}}$ 状的, 由此得

第一百三十七类

有一条 $u^{3}=A t^{2}$ 状的渐近拋物线和一条 $u=\frac{A}{t}$ 状的渐近双曲线,也得到 

第一百三十八类

有一条 $u^{3}=A t^{2}$ 状的渐近抛物线和一条 $u=\frac{A}{t^{2}}$ 状的渐近双曲线.

\section{$\S 269$}

若 $k=0, b=0$, 则
\[
y^{4}+a y^{2} x+c y^{2}+d y x+e x^{2}+f y+g x+h=0
\]
$e \neq 0$ 时得 $y^{4}+a y^{2} x+e x^{2}=0$. 该方程, $a^{2}<4 e$ 时不可能, $a^{2}>4 e$ 时给出两条 $u^{2}=A t$ 状的 同轴抛物渐近线, $a^{2}=4 e$ 时这两条抛物线重合,这样我们得到第一百三十九,一百四十, 一百四十一类.

$e=0$ 时得方程
\[
y^{4}+a y^{2} x+c y^{2}+d y x+f y+g x+h=0
\]
$a \neq 0$ 时其形状为
\[
y^{4}+a y^{2} x+c y^{2}+d y x+g x=0
\]
由此得方程 $y^{2}+a x=0$ 和 $a y^{2}+d y+g=0$. 这后一个方程, 或者给出两个相同的实 $y$ 值, 或者给出两个不同的实 $y$ 值, 或者不给出实 $y$ 值, 第一种情形在一条渐近抛物线之外, 还 有两条 $u=\frac{A}{t}$ 状的平行渐近线;第二种情形一条 $u^{2}=\frac{A}{t}$ 状的渐近线;第三种情形没有渐 近线,这里我们又得到三类,这是第一百四十二,一百四十三和一百四十四类.

\section{$\S 270$}

$a=0$ 时方程的形状为
\[
y^{4}+c y^{2}+d y x+f y+g x+h=0
\]
此时若 $d \neq 0$, 则曲线有一条 $u^{3}=A t$ 状的渐近抛物线和一条 $u=\frac{A}{t}$ 状的渐近直线,含于方 程 $d y+g=0$ 中. 最后, 如果 $d=0$, 则曲线有一条 $u^{4}=A t$ 状的渐近抛物线. 这样四阶线共 有 146 类. 但最后的几类, 多数都可以进一步分成不同的类.

\section{$\S 271$}

从以上的讨论, 我们清楚地看到, 五阶和更高阶线,其类别的个数会是多么大的一个 数. 如果要像三阶线那样逐一列举分析的话,那不写成一大本书是办不到的, 至于四阶线 和更高阶线的基本性质, 可以从通用方程, 照三阶线那样推出, 这里我们不讲. 

