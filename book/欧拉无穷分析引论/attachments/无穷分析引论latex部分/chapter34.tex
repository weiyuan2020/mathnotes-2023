\chapter{第十六章 依据纵标性质求曲线}

\section{$\S 364$}

设 $P$ 和 $Q$ 是横标 $x$ 的两个有理函数, 又设曲线方程为 $y^{2}-P y+Q=0$. 从而每个横标 $x$ 或者不对应任何纵标, 或者对应两个纵标. 对应的这两个纵标, 和为 $P$, 积为 $Q$. 因而, 如 果 $P$ 为常数, 则对应于同一个横标的两个纵标之和恒为常数, 即曲线有直径. $P=a+n x$ 时情况同于 $P$ 为常数. 此时直径的方程为 $z=\frac{1}{2} a+\frac{1}{2} n x$, 这里的直径是广泛意义下的, 不排除纵标为倾斜的. 如果 $Q$ 为常量, 则曲线与轴不相交. 如果 $Q=\alpha+\beta x+\gamma x^{2}$, 又如果 这右端表达式有两个实因式, 则曲线与轴有两个交点, $Q$ 是轴上两个区间的积,纵标的积 与这两个区间的积的比为常数.

\section{$\S 365$}

圆雉曲线的性质, 别的很多曲线也具有. 例如, 又曲线取渐近线为轴时, 对应于同一 个横标的两个纵标之积为常数, 方程 $y^{2}-P y \pm a^{2}=0$ 表示的曲线都具有这种性质. 又如, $\S 87$ 图 19 , 圆雉曲线, 取交曲线于 $E, F$ 的直线 $E F$ 为轴时, 积 $P M \cdot P N$ 与积 $P E \cdot P F$ 之比 为常数,这条性质,方程
\[
y^{2}-P y+a x-n x^{2}=0
\]
表示的每一条曲线也都具有. $y^{2}-P y=a x-x^{2}$ 时我们有 $P M \cdot P N=P E \cdot P F$ 或 $p m$. $p n=E p \cdot p F$. 圆的这条性质, 不仅无数的高阶曲线具有, 别的圆雉曲线也具有. 事实上, 令 $P=b+n x$, 则方程为
\[
y^{2}-n x y+x^{2}=a x+b y
\]
该方程表示: $n=0$ 时为圆, $\angle E P M$ 是直角; $n^{2}<4$ 时, 是椭圆; $n^{2}>4$ 时, 是双曲线; $n^{2}=4$ 时, 是抛物线.

\section{$\S 366$}
由此我们可以得到, 对以 $A B, E F$ 为轴或主直径的任一圆雉曲线 $A E B F$, 参见图 77 , 任画两条与主轴成半直角、交于 $h$ 的直线 $p q$ 和 $m n$, 则 $m h \cdot n h=p h \cdot q h$. 这结论可从一条 有名的性质推出. 这性质为: 过中心 $C$ 引与主轴成半直角的直线 $P Q$ 和 $M N$, 则 $M C \cdot$ $N C=P C \cdot Q C$, 且平行于这两条直线的直线都满足这条规律. 因而我们有 $m h \cdot n h=p h$. $q h$. 应该指出, 只要求直线 $M N$ 和 $P Q$ 对同一根主轴的倾角相同,即 $\angle P C A=\angle N C A$, 那 么由 $C P=C N$, 任何平行于这两条直线的直线被交点分成的两段之积都相等, 即 $m h$ 。 $h n=p h \cdot h q$.


【图,待补】
%%![](https://cdn.mathpix.com/cropped/2023_02_05_d70135c5d5f1afcc0634g-20.jpg?height=393&width=309&top_left_y=457&top_left_x=664)

图 77

\section{$\S 367$}

方程 $y^{2}-P y+Q=0$ 使每一个横标对应两个纵标. 现在 我们就来考虑有关这两个纵标的问题. 设横标 $A P=x$ 对应的 两个纵标为 $P M$ 和 $P N$. 参见图 78. 首先, 我们寻求具有这样 一种性质的所有曲线
\[
P M^{2}+P N^{2}=a^{2}
\]
$a$ 为常数. 由
\[
P M+P N=P \text { 和 } P M \cdot P N=Q
\]

【图,待补】
%%![](https://cdn.mathpix.com/cropped/2023_02_05_d70135c5d5f1afcc0634g-20.jpg?height=290&width=405&top_left_y=1090&top_left_x=1090)

图 78

得
\[
P M^{2}+P N^{2}=P^{2}-2 Q
\]
从而,令
\[
P^{2}-2 Q=a^{2} \text { 或 } Q=\frac{P^{2}-a^{2}}{2}
\]
曲线就具有所要的性质,也即所求曲线的方程为
\[
y^{2}-P y+\frac{P^{2}-a^{2}}{2}=0
\]
令 $P=2 n x$, 得到的是具有所要性质的圆雉曲线
\[
y^{2}-2 n x y+2 n^{2} x^{2}-\frac{1}{2} a^{2}=0
\]
这是以中心为原点的椭圆方程.

%%09p161-180
\section{$\S 368$}

由此得到椭圆的一条相当漂亮的性质,参见 图 $79, I G H K$ 为椭圆的外切平行四边形,边平行于椭 圆的共轭直径 $A B$ 和 $E F$, 切点为 $A, B, E, F . M N$ 是平 行于直径的任何一根弦. 平行四边形的对角线 $G K$, $H I$ 与弦 $M N$ 的交点分别为 $P$ 和 $p$. 我们有平方和 $P M^{2}+P N^{2}$ 和 $p M^{2}+p N^{2}$ 都恒为常数, 等于 $2 C E^{2}$. 类 似地, 引平行于另一直径 $A B$ 的弦 $R S$, 则
\[
P R^{2}+P S^{2}=\pi R^{2}+\pi S^{2}=2 C A^{2}
\]
我们进行证明, 令 $C A=C B=a, C E=C F=b, C Q=t$,$Q M=u$, 则


【图,待补】
%%![](https://cdn.mathpix.com/cropped/2023_02_05_94c61b3fb31a70215b66g-01.jpg?height=375&width=486&top_left_y=454&top_left_x=1011)

图 79 
\[
a^{2} u^{2}+b^{2} t^{2}=a^{2} b^{2}
\]
但 $a: b=C Q(t): P Q$, 且 $C P: C Q$ 为常数, 记为 $m: 1$. 从而, 令 $C P=x, P M=y$, 则
\[
x=m t, \quad y=u+\frac{b t}{a}
\]
或
\[
t=\frac{x}{m}, \quad u=y-\frac{b x}{m a}
\]
代入,得
\[
a^{2} y^{2}-\frac{2 a b x y}{m}+\frac{2 b^{2} x^{2}}{m^{2}}=a^{2} b^{2}
\]
令 $\frac{b}{m a}=n$, 得
\[
y^{2}-2 n x y+2 n^{2} x^{2}=b^{2}
\]
这是我们前面求出的证明 $P M^{2}+P N^{2}$ 为常数的那个方程.

\section{$\S 369$}

现在我们求立方和 $P M^{3}+P N^{3}$ 恒为常数的方程. 参见 $\S 367$ 图 78, 由 $P M+P N=$ $P$, 得
\[
P M^{3}+P N^{3}=P^{3}-3 P Q
\]
由此, 令 $P M^{3}+P N^{3}=a^{3}$, 则 $Q=\frac{P^{3}-a^{3}}{3 P}$. 因而所求曲线的通用方程为
\[
y^{2}-P y+\frac{1}{3} P^{2}-\frac{a^{3}}{3 P}=0
\]
这里的 $P$ 可以是 $x$ 的任何有理函数, 具有这种性质的最简单的曲线, 是取 $P=3 n x, a=$ $3 n b$ 时的三阶线 

$x y^{2}-3 n x^{2} y+3 n^{2} x^{3}-3 n^{2} b^{3}=0$

它属于我们前面划分的第二类.

\section{$\S 370$}

类似地, 我们求 $P M^{4}+P N^{4}$ 为常数的曲线, 由
\[
P M^{4}+P N^{4}=P^{4}-4 P^{2} Q+2 Q^{2}
\]
得
\[
P^{4}-4 P^{2} Q+2 Q^{2}=a^{4}
\]
从而
\[
Q=P^{2}+\sqrt{\frac{1}{2} p^{4}+\frac{1}{2} a^{4}}
\]
因为 $P$ 和 $Q$ 都应该是 $x$ 的有理, 也即单值函数, 所以要保证 $y$ 对任何横标 $x$ 都最多有两个 值, 量 $\sqrt{\frac{1}{2} P^{4}+\frac{1}{2} a^{4}}$ 必须有理. 但这不可能, 所以函数 $Q$ 恒为二值的. 从而纵标 $y$ 是四值 函数. 由方程 $y^{2}-P y+Q=0$ 得
\[
y=\frac{1}{2} P \pm \sqrt{-\frac{3}{4} P^{2} \pm \sqrt{\frac{1}{2} P^{4}+\frac{1}{2} a^{4}}}
\]
可见, $\sqrt{\frac{1}{2} P^{4}+\frac{1}{2} a^{4}}$ 不取正号, $y$ 就不可能是实的. 这样一来,虽然函数 $Q$ 是二值的,但 纵标 $y$ 的值恒不多于二个, 且其四次幕的和,如问题所要求的,为常数.

\section{$\S 371$}

如果要求曲线满足对应于同一个横标的两个纵标的五次幂之和为常数, 即 $P M^{5}+$ $P N^{5}=a^{5}$, 则应该有
\[
P^{5}-5 P^{3} Q+P Q^{2}=a^{5}
\]
从方程 $y^{2}-P y+Q=0$ 得 $Q=-y^{2}+P y$, 从而
\[
P^{5}-5 P^{4} y+10 P^{3} y^{2}-10 P^{2} y^{3}+5 P y^{4}=a^{5}
\]
或
\[
(P-y)^{5}+y^{5}+a^{5}
\]
如果要求 $P M^{6}+P N^{6}=a^{6}$,那么用同样的方法我们求得方程
\[
(P-y)^{6}+y^{6}=a^{6}
\]
一般地,要曲线满足 $P M^{n}+P N^{n}=a^{n}$, 则得到的方程为
\[
(P-y)^{n}+y^{n}=a^{n}
\]
$P$ 可以为 $x$ 的任何单值函数. 这方程的含义是显然的. 因为两个纵标的和为 $P$, 所以如果 一个纵标为 $y$, 则另一个为 $P-y$, 由此直接得到 
\[
(P-y)^{n}+y^{n}=a^{n}
\]
\section{$\S 372$}
如果改消去 $Q$ 为消去 $P$, 为此将 $P=\frac{y^{2}+Q}{y}$ 代入表示 $P, Q$ 关系的方程, 那么当 $P M^{n}+P N^{n}=a^{n}$ 时, 得方程
\[
y^{n}+\frac{Q^{n}}{y^{n}}=a^{n}
\]
由于纵标的积为 $Q$, 所以如果一个纵标为 $y$, 则另一个为 $\frac{Q}{y}$. 由此直接得到所求方程. 这 样, 对 $P M^{n}+P N^{n}=a^{n}$ 的曲线,我们求得了两个方程
\[
(P-y)^{n}+y^{n}=a^{n} \text { 和 } y^{n}+\frac{Q^{n}}{y^{n}}=a^{n}
\]
从后一个方程得
\[
y^{2 n}=a^{n} y^{n}-Q^{n} \text { 和 } y^{n}=\frac{1}{2} a^{n} \pm \sqrt{\frac{1}{4} a^{2 n}-Q^{n}}
\]
从而
\[
y=\sqrt[n]{\frac{1}{2} a^{n} \pm \sqrt{\frac{1}{4} a^{2 n}-Q^{n}}}
\]
该式是二值函数, 并且只要 $Q^{n}$ 是有理的, 也即只要 $Q^{n}$ 是单值函数, 对于每一个横标 $x$, 它 给出的纵标个数都不多于 2 . 前一个方程 $y^{n}+(P-y)^{n}=a^{n}$ 的优点是次数低.

\section{$\S 373$}

这两个方程不仅适用于 $n$ 为正整数的情形, 也适用于负整数和分数的情形, 对负整 数指数

\begin{tabular}{c|c}
\hline 要求 & 结果方程 \\
\hline$\frac{1}{P M}+\frac{1}{P N}=\frac{1}{a}$ & $a P=P y-y^{2}$ \\
\hline$\frac{1}{P M^{2}}+\frac{1}{P N^{2}}=\frac{1}{a^{2}}$ & 或 $a Q+a y^{2}=Q y$ \\
\hline$\frac{1}{P M^{3}}+\frac{1}{P N^{3}}=\frac{1}{a^{3}}$ & $a^{2} y^{2}+a^{2}(P-y)^{2}=y^{2}(P-y)^{2}$ \\
或 $a^{2} Q^{2}+a^{2} y^{4}=Q^{2} y^{2}$ \\
\hline
\end{tabular}

等. 

对分数指数

\begin{tabular}{c|c}
\hline 要求 & 结果方程 \\
\hline$\sqrt{P M}+\sqrt{P N}=\sqrt{a}$ & $\sqrt{y}+\sqrt{P-y}=\sqrt{a}$ 或 $y=\sqrt{a y}+\sqrt{Q}$ \\
化为有理形式, 为 \\
$y^{2}-P y+\frac{1}{4}(a-P)^{2}=0$ \\
或 $y^{2}-(a-2 \sqrt{Q}) y+Q=0$ \\
$\sqrt[3]{P M}+\sqrt[3]{P N}=\sqrt[3]{a}$ & $\sqrt[3]{y}+\sqrt[3]{P-y}=\sqrt[3]{a}$ \\
或 $y^{2}-P y+\frac{1}{27 a}(a-P)^{3}=0$ \\
或 $\sqrt[3]{y}+\sqrt[3]{\frac{Q}{y}}=\sqrt[3]{a}$ \\
或 $y^{2}-(a-3 \sqrt[3]{a Q}) y+Q=0$ \\
\hline
\end{tabular}

等.

可见,照上面的方法,用一个通用方程可以包含所有满足
\[
P M^{n}+P N^{n}=a^{n}
\]
的代数曲线. $n$ 可为正整数,可为负整数,可为分数.

\section{$\S 374$}

前面考虑的是一个横标值对应两个纵标值的曲线. 现在我们考虑一个横标值对应三 个纵标值的曲线. 这种曲线的通用方程为
\[
y^{3}-P y^{3}+Q y-R=0
\]
其中 $P, Q$ 和 $R$ 都是 $x$ 的某个单值函数. 设对应于同一个横标 $x$ 的三纵标值为 $p, q, r$. 当然 其中必定有一个是实的, 但我们这里主要考虑曲线上这三个纵标都为实数的部分, 从方 程的性质我们有
\[
P=p+q+r, \quad Q=p q+p r+q r, \quad R=p q r
\]
因而如果要曲线满足 $p+q+r$, 或 $p q+p r+q r$, 或 $p q r$ 为常数, 那么要做的就是取 $P$, 或 $Q$, 或 $R$ 为常数, 而保留其余两个为任意.

\section{$\S 375$}

我们也可以求 $p^{n}+q^{n}+r^{n}$ 处处为常数的曲线, 上册给了
\[
\begin{gathered}
p+q+r=P \\
p^{2}+q^{2}+r^{2}=P^{2}-2 Q \\
p^{3}+q^{3}+r^{3}=P^{3}-3 P Q+3 R \\
P^{4}+Q^{4}+r^{4}=P^{4}-4 P^{2} Q+2 Q^{2}+4 P R
\end{gathered}
\]
\[
P^{5}+q^{5}+r^{5}=P^{5}-5 P^{3} Q+5 P Q+5 P^{2} R-5 Q R
\]
等. 如果 $n$ 为负数, 我们置 $z=\frac{1}{y}$, 则方程化为
\[
z^{3}-\frac{Q z^{2}}{R}+\frac{P z}{R}-\frac{1}{R}=0
\]
它的三个根为 $\frac{1}{p}, \frac{1}{q}, \frac{1}{r}$, 此时我们有
\[
\begin{gathered}
\frac{1}{p}+\frac{1}{q}+\frac{1}{r}=\frac{Q}{R} \\
\frac{1}{p^{2}}+\frac{1}{q^{2}}+\frac{1}{r^{2}}=\frac{Q^{2}-2 P R}{R^{2}} \\
\frac{1}{P^{3}}+\frac{1}{q^{3}}+\frac{1}{r^{3}}=\frac{Q^{3}-3 P Q R+3 R^{2}}{R^{3}} \\
\frac{1}{P^{4}}+\frac{1}{q^{4}}+\frac{1}{r^{4}}=\frac{Q^{4}-4 P Q^{2} R+4 Q R^{2}+2 P^{2} R^{2}}{R^{4}}
\end{gathered}
\]
等.

每令一个这种表达式为常数, 我们就得到函数 $P, Q, R$ 之间的一个关系. 利用这个关 系, 我们可以消去方程 $y^{3}-P y^{2}+Q y-R=0$ 里函数 $P, Q, R$ 中的一个, 得到所求曲线的 方程. 例如, 求满足 $p^{3}+q^{3}+r^{3}=a^{3}$ 的曲线, 得关系 $P^{3}-3 P Q+3 R=a^{3}$, 将 $R=y^{3}-$ $P y^{2}+Q y$ 代入, 得所求曲线的方程为
\[
3 y^{3}-3 P y^{2}+3 Q y+P^{3}-3 P Q=a^{3}
\]
\section{$\S 376$}
当 $n$ 为正整数和负整数时, 利用所给公式, 问题都容易解决. 主要困难在 $n$ 为分数的 时候,假定我们求满足
\[
\sqrt{p}+\sqrt{q}+\sqrt{r}=\sqrt{a}
\]
的曲线. 该等式两边平方, 利用 $p+q+r=P$, 得
\[
P+2 \sqrt{p q}+2 \sqrt{p r}+2 \sqrt{q r}=a
\]
或
\[
\frac{a-p}{2}=\sqrt{p q}+\sqrt{p r}+\sqrt{q r}
\]
该等式两边再平方,利用 $p q+p r+q r=Q$,得
\[
\begin{aligned}
\frac{(a-p)^{2}}{4} & =Q+2 \sqrt{p^{2} q r}+2 \sqrt{p q^{2} r}+2 \sqrt{p q r^{2}} \\
& =Q+2(\sqrt{p}+\sqrt{q}+\sqrt{r}) \sqrt{p q r}=2 \sqrt{a R}+Q
\end{aligned}
\]
由此得
\[
(a-P)^{2}=4 Q+8 \sqrt{a R}
\]
也即
 $Q=\frac{(a-P)^{2}}{4}-2 \sqrt{a R}$

因而所求曲线的方程为
\[
y^{3}-P y^{2}+\left(\frac{1}{4}(a-P)^{2}-2 \sqrt{a R}\right) y-R=0
\]
有理化,并利用
\[
R=\frac{\left(a^{2}-2 a P+P^{2}-4 Q\right)^{2}}{64 a}
\]
则所得方程成为
\[
y^{3}-P y^{2}+Q y-\frac{\left(a^{2}-2 a P+P^{2}-4 Q\right)^{2}}{64 a}=0 
\]
\section{$\S 377$}

但对更高次的根, 这过程极为繁难. 我们来寻求另外的方法, 这可以从下面的例子中 得到启发,假定我们求满足
\[
\sqrt[3]{p}+\sqrt[3]{q}+\sqrt[3]{r}=\sqrt[3]{a}
\]
的曲线, 令
\[
\sqrt[3]{p q}+\sqrt[3]{p r}+\sqrt[3]{q r}=v
\]
利用 $\sqrt[3]{p q r}=\sqrt[3]{R}$, 得
\[
\sqrt[3]{p^{2}}+\sqrt[3]{q^{2}}+\sqrt[3]{r^{2}}=\sqrt[3]{a^{2}}-2 v
\]
和
\[
p+q+r=a-3 v \sqrt[3]{a}+3 \sqrt[3]{R}=P
\]
因而
\[
\sqrt[3]{p^{2} q^{2}}+\sqrt[3]{p^{2} r^{2}}+\sqrt[3]{q^{2} r^{2}}=v^{2}-2 \sqrt[3]{a R}
\]
和
\[
p q+p r+q r=Q=v^{3}-3 v \sqrt[3]{a R}+3 \sqrt{R^{2}}
\]
求出了 $P$ 和 $Q$ 的表达式之后, 取 $v$ 为 $x$ 的某个函数, 我们就得到所求曲线的方程
\[
y^{3}-(a-3 v \sqrt{a}+3 \sqrt[3]{R}) y^{2} & +\left(v^{3}-3 v \sqrt[3]{a R}+3 \sqrt[3]{R^{2}}\right) y-R=0 
\]
\section{$\S 378$}

尽管繁难, 但是可以得到问题的通用解. 事实上, 方程 $y^{3}-P y^{2}+Q y-R=0$ 中的函 数 $y$ 表示三个纵标 $p, q$ 和 $r$. 令 $p=y$, 则
\[
P=y+q+r, \quad Q=q y+r y+q r
\]
或
\[
q+r=P-y, \quad q r=Q-y(q+r)=Q-P y+y^{2}
\]
由此得
\[
q-r=\sqrt{P^{2}+2 P y-3 y^{2}-4 Q}
\]
从而
\[
\begin{aligned}
& q=\frac{1}{2}(P-y)+\frac{1}{2} \sqrt{P^{2}+2 P y-3 y^{2}-4 Q} \\
& r=\frac{1}{2}(P-y)-\frac{1}{2} \sqrt{P^{2}+2 P y-3 y^{2}-4 Q}
\end{aligned}
\]
因而满足 $p^{n}+q^{n}+r^{n}=a^{n}$ 的曲线,其方程为
\[
\begin{aligned}
& y^{n}+\left[\frac{1}{2}(P-y)+\frac{1}{2} \sqrt{P^{2}+2 P y-3 y^{2}-4 Q}\right]^{n}+ \\
& {\left[\frac{1}{2}(P-y)-\frac{1}{2} \sqrt{P^{2}+2 P y-3 y^{2}-4 Q}\right]^{n}=a^{n}}
\end{aligned}
\]
该方程把 $n$ 为整数和分数的问题都解决了.

\section{$\S 379$}

利用这同一种方法可以解决有关三个纵标的大量的其他问题. 例如, 可以换 $a^{n}$ 为 $x$ 的某个函数, 也可以换 $p, q, r$ 的幂的和为 $p, q, r$ 的别的函数, 但这里要求 $p, q, r$ 的地位相 同, 即函数不因 $p, q, r$ 的位置交换而改变. 比如, 可以要求对应于同一个纵标 $x$ 的这三个 纵标 $p, q, r$ 构成的三角形面积为常数. 该三角形的面积为
\[
\frac{1}{4} \sqrt{2 p^{2} q^{2}+2 p^{2} r^{2}+2 q^{2} r^{2}-p^{4}-q^{4}-r^{4}}
\]
我们令它等于 $a^{2}$,由于
\[
\begin{gathered}
p^{4}+q^{4}+r^{4}=P^{4}-4 P^{2} Q+4 P R+2 Q^{2} \\
p^{2} q^{2}+p^{2} r^{2}+q^{2} r^{2}=Q^{2}-2 P R
\end{gathered}
\]
我们有
\[
16 a^{4}=4 P^{2} Q-8 P R-P^{4}
\]
从而
\[
R=\frac{1}{2} P Q-\frac{1}{8} P^{3}-\frac{2 a^{4}}{p}
\]
得所求方程为
\[
y^{3}-P y^{2}+Q y-\frac{1}{2} P Q+\frac{1}{8} P^{3}+\frac{2 a^{4}}{p}=0
\]
如果取 $P$ 为常数 $2 b$, 我们就还得到这些三角形的周长为常数. 再取 $Q=m x^{2}+n b x+k a^{2}$, 得表示三阶线的方程
\[
y^{3}+m x^{2} y-2 b y^{2}+n b x y-m b x^{2}+k a^{2} y-n b^{2} x+\frac{a^{4}}{b}-k a^{2} b+b^{3}=0
\]
这三阶线具有这样两条性质,一是对应于同一个横标 $x$ 的三个纵标 $p, q, r$ 和为常数,等 于 $2 b$,二是以这三个纵标为边的三角形,面积为常数, 等于 $a^{2}$. 

\section{$\S 380$}

对应于同一横标的四个或更多个纵标的类似问题,也可以 用同样的方法处理, 不产生新的困难, 因此我们转向另外的问 题. 转向考虑对应于不同横标的纵标之间的关系. 参见图 80 , 设 纵标 $P M$ 和 $Q N$ 有着某种关系, $P M$ 和 $Q N$ 对应的横标分别为 $A P=x$ 和 $A Q=-x$. 又设 $y=X$ 是这条曲线的方程, $X$ 是 $x$ 的一 个函数, 对 $x$ 它给出纵标 $P M$, 对 $-x$ 它给出纵标 $Q N$. 如果 $X$ 是 $x$ 的偶函数, 记为 $P$, 则 $Q N=P M$; 如果 $X$ 是 $x$ 的奇函数, 记为 $Q$, 则 $Q N=-P M$; 如果 $P$ 和 $R$ 是 $x$ 的偶函数, $Q$ 和 $S$ 是 $x$ 的奇函 数, 又如果曲线的方程为 $y=\frac{P+Q}{R+S}$, 则


【图,待补】
%%![](https://cdn.mathpix.com/cropped/2023_02_05_94c61b3fb31a70215b66g-08.jpg?height=378&width=328&top_left_y=460&top_left_x=1167)

图 80
\[
P M=\frac{P+Q}{R+S}, \quad Q N=\frac{R-Q}{R-S}
\]
\section{$\S 381$}

我们求一条曲线, 具有性质 $P M+Q N$ 等于常数, 等于 $2 A B=2 a$. 显然 $Q$ 为奇函数时, 方程 $y=a+Q$ 满足要求. 此时我们有 $P M=a+Q, Q N=a-Q, P M+Q N=2 a$. 令 $y-a=$ $u$, 即 $u=Q$. 这是曲线 $y=a+Q$ 在直线 $B p$ 为轴, $B$ 为原点之下对坐标 $B p=x, p M=u$ 的方 程. 该曲线以点 $B$ 为中心, 位于对顶象限中的两部分相等. 照下面这样画出的曲线 $M B N$ 都满足我们的要求. 取一条直线 $P Q$ 作轴, 从中心 $B$ 向轴引垂线 $B A$, 对 $A$ 两边长度相等 的横标 $A P=A Q$, 使 $P M+Q N$ 恒为常数,等于 $2 A B$.

\section{$\S 382$}

以点 $B$ 为中心对称的曲线, 前面我们求出了它在坐标 $x, u$ 之下的两个方程
\[
\begin{aligned}
& \text { I } \\
& 0=\alpha x+\beta u+\gamma x^{3}+\delta x^{2} u+\varepsilon x u^{2}+\zeta u^{3}+\eta x^{5}+\theta x^{4} u+\cdots \\
& \text { II } \\
& 0=\alpha+\beta x^{2}+\gamma x u+\delta u^{2}+\varepsilon x^{4}+\zeta x^{3} u+\eta x^{2} u^{2}+\theta x u^{3}+\cdots
\end{aligned}
\]
令这两个方程中的 $u=y-a$, 我们得到 $x, y$ 间的两个通用方程, 它们表示的代数曲线都 满足我们的要求. 首先过点 $B$ 的任何一条直线都满足要求. 其次, 以点 $B$ 为中心的圆雉曲 线也都合乎条件, 此时横标 $A P, A Q$ 都对应两个纵标 (曲线为双曲线时取纵标线平行于 另一条渐近线), 我们有和相等的两对纵标. 

\section{$\S 383$}
将纵标 $P M, Q N$ 之和换为同次幂的和, 求法也类似. 记这条件为 $P M^{n}+Q M^{n}=2 a^{n}$. 显然, 只要 $Q$ 为 $x$ 的一个奇函数, 方程 $y^{n}=a^{n}+Q$ 就满足要求, 因为 $P M^{n}=a^{n}+Q$ 时, $Q N^{n}=a^{n}-Q$. 从而 $P M^{n}+Q N^{n}=2 a^{n}$. 令 $y^{n}-a^{n}=u$, 令 $u=Q$, 就把以点 $B$ 为中心对称的 曲线用坐标 $x$ 和 $u$ 表示了出来. 换上节方程中的 $u$ 为 $y^{n}-a^{n}$ 就得到满足所提条件的曲线 的通用方程.

\section{$\S 384$}

这类问题不产生什么困难, 我们转向求这样的曲线 $M B N$, 使轴上与点 $A$ 等距处的 横标 $A P$ 和 $A Q$ 所对应的纵标 $P M$ 和 $Q N$ 的乘积 $P M \cdot Q N$ 为常数, 设为 $a^{2}$. 这个问题有很 多特殊的解, 讨论通用解之前, 先给出一个最重要的特殊解. 设 $P$ 和 $Q$ 分别为横标 $A P=x$ 的偶函数和奇函数, 令纵标 $P M=y=P+Q$, 则 $x$ 为负时得 $Q N=P-Q$, 依要求应该有
\[
P M \cdot Q N=P^{2}-Q^{2}=a^{2} \text { 或 } P=\sqrt{a^{2}+Q^{2}}
\]
$Q^{2}$ 是 $x$ 的偶函数, 因而 $\sqrt{a^{2}+Q^{2}}$ 是偶函数, 可作为 $P$, 因而 $y=Q+\sqrt{a^{2}+Q^{2}}$ 是所求曲线 的方程, $Q$ 是一个奇函数.

\section{$\S 385$}

根号有两值,因而每个横标 $x$ 都对应两个纵标.一正一负,横标 $A P$ 对应纵标
\[
Q+\sqrt{a^{2}+Q^{2}} \text { 和 } Q-\sqrt{a^{2}+Q^{2}}
\]
横标 $A Q$ 对应纵标
\[
-Q+\sqrt{a^{2}+Q^{2}} \text { 和 }-Q-\sqrt{a^{2}+Q^{2}}
\]
可见曲线以点 $A$ 为中心对称. 根号不能只取一种符号. 例如, 取 $\frac{a^{2}}{4 x}-x$ 作奇函数 $Q$, 则 $a^{2}+Q^{2}$ 是完全平方, 得到 $\sqrt{a^{2}+Q^{2}}=\frac{a^{2}}{4 x}+x$, 是奇函数, 不能作 $P, P$ 应为偶函数. 我们取 的奇函数 $Q$, 必须使 $a^{2}+Q^{2}$ 不是完全平方.

\section{$\S 386$}

类似地, 如果令 $y=(P+Q)^{n}$, 则 $Q N=(P-Q)^{n}$, 应该 $\left(P^{2}-Q^{2}\right)^{n}=a^{2}$. 从而
\[
P^{2}=a^{\frac{2}{n}}+Q^{2}, \quad P=\sqrt{a^{\frac{2}{n}}+Q^{2}}
\]
只要它是无理的, 就为偶函数, 此时满足要求的曲线, 其方程为 
\[
y=\left(Q+\sqrt{a^{\frac{2}{n}}+Q^2}\right)^n
\]易于列出这样的曲线的方程, 画一条曲线, 它有以点 $A$ 为中心对称, 也即有相似且相等的 两段. 记该曲线对应于横标 $A P=x$ 的纵标为 $z$, 则 $z$ 是 $x$ 的奇函数,因而可以作为我们的 $Q$, 从求得的方程得
\[
y^{\frac{1}{n}}-Q=\sqrt{a^{\frac{2}{n}}+Q^{2}}
\]
从而
\[
Q=z=\frac{y^{\frac{2}{n}}-a^{\frac{2}{n}}}{2 y^{\frac{1}{n}}}
\]
令 $\frac{1}{n}=m$, 将 $z=\frac{y^{2 m}-a^{2 m}}{2 y^{m}}$ 代入 $z, x$ 间的给定方程, 就得到所求曲线的 $x, y$ 间方程. 我们 求出了 $z, x$ 间的两个方程
\[
0=\alpha+\beta x^{2}+\gamma x z+\delta z^{2}+\varepsilon x^{4}+\zeta x^{3} z+\eta x^{2} z^{2}+\theta x z^{3}+\cdots
\]
和
\[
0=\alpha x+\beta z+\gamma x^{3}+\delta x^{2} z+\varepsilon x z^{2}+\zeta z^{3}+\eta x^{5}+\theta x^{4} z+\cdots
\]
由 $z$ 的任何倍数都可作 $Q$, 我们略去除数 2 , 将 $z=y^{m}-\frac{a^{2 m}}{y^{m}}$ 代入这两个方程, 就得到符合 条件的曲线的两个通用方程.

\section{$\S 387$}

设 $R$ 同于 $P$ 也是偶函数, $S$ 同于 $Q$ 也是奇函数, 又设所求曲线的方程为
\[
y=\frac{P+Q}{R+S}=P M
\]
则 $Q N=\frac{P-Q}{R-S}$, 应 $\frac{P^{2}-Q^{2}}{R^{2}-S^{2}}=a^{2}$. 取 $y=\frac{P+Q_{a}}{R-S}$, 或者取 $y=\left(\frac{P+Q}{P-Q}\right)^{n} a$ 都易于满足这条 件. 这取法可避免原来的每个横标对应两个或更多个纵标的不便, 求出每个横标只对应 一个纵标的曲线. 满足这条件的最简单的线是方程 $y=\frac{b+x}{b-x}$ 表示的二阶线, 这是双曲 线. 又令 $Q=n x$, 代入前面求得的方程 $y=Q+\sqrt{a^{2}+Q^{2}}$, 得 $y^{2}-2 n x y=a^{2}$, 是双曲线, 两 种方法都得到双曲线是我们的问题的解.

\section{$\S 388$}

从前面所讲可见, 换 $x$ 为 $-x$, 换 $y$ 为 $\frac{a^{2}}{y}$ 时所求曲线的方程应该不变, 表达式
\[
\left(y^{n}+\frac{a^{2 n}}{y^{n}}\right) P \text { 和 }\left(y^{n}-\frac{a^{2 n}}{y^{n}}\right) Q
\]
都具有这种性质, 其中 $P$ 和 $Q$ 分别为 $x$ 的偶函数和奇函数. 用任何多个这种表达式构成的方程, 它们表示的曲线都满足要求. 因而, 如果 $M, P, R, T, \cdots$ 为 $x$ 的任何偶函数, $N$, $Q, S, V, \cdots$ 为 $x$ 的任何奇函数, 那么我们有通用方程
\[
\begin{aligned}
0= & M+\left(\frac{y}{a}+\frac{a}{y}\right) P+\left(\frac{y^{2}}{a^{2}}+\frac{a^{2}}{y^{2}}\right) R+\left(\frac{y^{3}}{a^{3}}+\frac{a^{3}}{y^{3}}\right) T+\cdots+ \\
& \left(\frac{y}{a}-\frac{a}{y}\right) Q+\left(\frac{y^{2}}{a^{2}}-\frac{a^{2}}{y^{2}}\right) S+\left(\frac{y^{3}}{a^{3}}-\frac{a^{3}}{y^{3}}\right) V+\cdots
\end{aligned}
\]
如果以 $x$ 乘这通用方程的奇函数, 则偶函数变奇函数, 奇函数变偶函数, 因而下面这样的 方程也满足要求
\[
\begin{aligned}
0= & N+\left(\frac{y}{a}+\frac{a}{y}\right) Q+\left(\frac{y^{2}}{a^{2}}+\frac{a^{2}}{y^{2}}\right) S+\left(\frac{y^{3}}{a^{3}}+\frac{a^{3}}{y^{3}}\right) V+\cdots+ \\
& \left(\frac{y}{a}-\frac{a}{y}\right) p+\left(\frac{y^{2}}{a^{2}}-\frac{a^{2}}{y^{2}}\right) R+\left(\frac{y^{3}}{a^{3}}-\frac{a^{3}}{y^{3}}\right) T+\cdots
\end{aligned}
\]
去分母,得下面两个 $n$ 阶有理方程, $n$ 不确定
\[
\begin{gathered}
\text { I } \\
0=a^{n} y^{n} M+a^{n-1} y^{n+1}(P+Q)+a^{n-2} y^{n+2}(R+S)+a^{n-3} y^{n+3}(T+V)+\cdots+ \\
a^{n+1} y^{n-1}(P-Q)+a^{n+2} y^{n-2}(R-S)+a^{n+3} y^{n-3}(T-V)+\cdots \\
\text { II } \\
0=a^{n} y^{n} N+a^{n-1} y^{n+1}(P+Q)+a^{n-2} y^{n+2}(R+S)+a^{n-3} y^{n+3}(T+V)+\cdots- \\
a^{n+1} y^{n-1}(P-Q)-a^{n+2} y^{n-2}(R-S)-a^{n+3} y^{n-3}(T-V)-\cdots \\
\end{gathered}
\]
\section{$\S 389$}

表达式
\[
\left(y^{n}+\frac{a^{2 n}}{y^{n}}\right) P \text { 和 }\left(y^{n}-\frac{a^{2 n}}{y^{n}}\right) Q
\]
中的 $n$ 可以为分数. 特别地将前节末去分母公式中的指数 $1,2,3,4, \cdots$ 依次换为 $\frac{1}{2}, \frac{3}{2}$, $\frac{5}{2}, \frac{7}{2}, \cdots$ 时得到的通用方程, 其无理性可消去. 事实上, 此时我们有方程
\[
\begin{aligned}
0= & \frac{y+a}{\sqrt{a y}} P+\frac{y^{3}+a^{3}}{a y \sqrt{a y}} R+\frac{y^{5}+a^{5}}{a^{2} y^{2} \sqrt{a y}} T+\cdots+ \\
& \frac{y-a}{\sqrt{a y}} Q+\frac{y^{3}-a^{3}}{a y \sqrt{a y}} S+\frac{y^{5}-a^{5}}{a^{2} y^{2} \sqrt{a y}} V+\cdots
\end{aligned}
\]
或
\[
\begin{aligned}
0= & +\frac{y+a}{\sqrt{a y}} Q+\frac{y^{3}+a^{3}}{a y \sqrt{a y}} S+\frac{y^{5}+a^{5}}{a^{2} y^{2} \sqrt{a y}} V+\cdots+ \\
& \frac{y-a}{\sqrt{a y}} P+\frac{y^{3}-a^{3}}{a y \sqrt{a y}} R+\frac{y^{5}-a^{5}}{a^{2} y^{2} \sqrt{a y}} T+\cdots
\end{aligned}
\]
去分母,得 
\[
\begin{aligned}
& \text { III } \\
& 0=+a^{n} y^{n+1}(P+Q)+a^{n-1} y^{n+2}(R+S)+a^{n-2} y^{n+3}(T+V)+\cdots+ \\
& a^{n+1} y^{n}(P-Q)+a^{n+2} y^{n-1}(R-S)+a^{n+3} y^{n-2}(T-V)+\cdots \\
& \text { IV } \\
& 0=+a^{n} y^{n+1}(P+Q)+a^{n-1} y^{n+2}(R+S)+a^{n-2} y^{n+3}(T+V)+\cdots- \\
& a^{n+1} y^{n}(P-Q)-a^{n+2} y^{n-1}(R-S)-a^{n+3} y^{n-2}(T-V)-\cdots
\end{aligned}
\]
\section{$\S 390$}

前面导出了四个方程, 从每阶方程都可求出满足要求的线. 首先, 从一阶方程得过点 $B$ 平行于轴 $A P$ 的直线. 对二阶, $n=1$ 时从前两个方程得 $\alpha a x y+y^{2}-a^{2}=0$, 这是令 $N=$ $\alpha x, P=1, Q=0$ 时从第二个方程得到的,第一个方程不给出任何曲线; $n=0$ 时从后两个方 程得到
\[
y=(\alpha+\beta x) \pm a(\alpha-\beta x)=0
\]
对三阶, 从前两个方程, $n=1$ 时得
\[
0=a y\left(\alpha+\beta x^{2}\right)+y^{2}(\gamma+\delta x)+a^{2}(\gamma-\delta x)
\]
和
\[
0=\alpha a y x+y^{2}(\gamma+\delta x)-a^{2}(\gamma-\delta x)
\]
从后两个方程, $n=0$ 和 $n=1$ 时得
\[
0=y\left(\alpha+\beta x+\gamma x^{2}\right) \pm\left(\alpha-\delta x+\gamma x^{2}\right)
\]
和
\[
0=a y^{2}(\alpha+\beta x)+y^{3} \pm a^{2} y(\alpha-\beta x) \pm a^{3}
\]
类似地, 可以求出满足要求的各阶线.

