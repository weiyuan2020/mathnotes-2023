\chapter{第十五章 有一条或几条直径的曲线}

\section{$\S 336$}

我们讨论过, 二阶线至少有一条正交直径, 它分整个曲线为相似且相等的两部分. 抛 物线只有一条这样的直径, 因而它由相似且相等的两部分组成. 椭圆和双曲线各有两条 这样的直径, 都相交于中心成直角. 因而这两种曲线都被分成四个相等且相似的弧或分 支. 过圆心的直线都分圆为相等且相似的两部分, 又等弦所对的弧相等且相似, 因而圆有 无穷多个相等且相似的部分.

\section{$\S 337$}

现在我们考察一个曲线的两个或更多个部分的相似, 导出这种具有两个或更多个相 似部分的曲线的通用方程. 先考虑直角坐标 $x, y$ 间的方程, 垂直相交于 $C$ 的直线 $A B, E F$ 将整个平面分为如图 68 所示的 $Q, R, S, T$ 四部分. $x, y$ 都取正值时得曲线的 $Q$ 中部分; $x$ 正 $y$ 负, 得曲线的 $R$ 中部分; $x$ 负 $y$ 正, 得曲线的 $S$ 中部分; 最后 $x, y$ 都负时, 得曲线的 $T$ 中部分.


【图,待补】
%%![](https://cdn.mathpix.com/cropped/2023_02_05_d70135c5d5f1afcc0634g-09.jpg?height=321&width=575&top_left_y=1407&top_left_x=536)

图 68

\section{$\S 338$}

如果换 $y$ 为 $-y$ 方程不变,则曲线的 $Q, R$ 中部分相等且相似. $y$ 的偶次幂具有这种性 质. 可见, 如果方程不含 $y$ 的奇次幂, 则曲线的 $Q, R$ 中部分相等且相似. 横标 $C P=x$ 所在 的直线 $A B$ 是这种曲线的直径, 具有这种性质的代数曲线, 其通用方程为
\[
0=\alpha+\beta x+\gamma x^{2}+\delta y^{2}+\varepsilon x^{3}+\zeta x y^{2}+\eta x^{4}+\theta x^{2} y^{2}+\iota y^{4}+\cdots
\]
可称该表达式为 $x$ 和 $y^{2}$ 的有理函数. 如果 $Z$ 为 $x$ 和 $y^{2}$ 的有理函数, 则方程 $Z=0$ 表示曲线 被直线 $A B$ 分成相似且相等的两部分. 这种曲线的 $S, T$ 中部分也相等且相似. 

\section{$\S 339$}

换 $x$ 为一 $x$ 方程不变, 则该方程表示的曲线的 $Q, S$ 中部分相等且相似. 因而, 如果 $Z$ 是 $x^{2}$ 和 $y$ 的有理函数, 则方程 $Z=0$ 表示的曲线被直线 $E F$ 分成相等且相似的两部分. 这 种曲线,其方程的形状为
\[
0=\alpha+\beta y+\gamma x^{2}+\delta y^{2}+\varepsilon x^{2} y+\zeta y^{3}+\eta x^{4}+\theta x^{2} y^{2}+\iota y^{4}+\cdots
\]
该方程表示的曲线, $S$ 中部分与 $Q$ 中部分相似且相等, $T$ 中部分与 $R$ 中部分也相似且 相等.

\section{$\S 340$}

如果同时换 $x$ 为 $-x$, 换 $y$ 为 $-y$ 方程不变, 则曲线的 $Q, R$ 中部分分别与对顶的 $T, S$ 中部分相似且相等. 设 $Z=0$ 是这样的曲线的方程. 首先, 如果 $Z$ 对 $x, y$ 都是偶函数, 或者 $Z$ 是 $x, y$ 的偶次齐次函数的和, 则方程 $Z=0$ 具有上述性质. 其次, 如果 $Z$ 是 $x, y$ 的奇次齐 次函数的和, 则 $x, y$ 都换为负时, $Z$ 变为 $-Z$. 于是由 $Z=0$ 得 $-Z=0$. 从而 $Q$ 中部分与 $T$ 中部分, $R$ 中部分与 $S$ 中部分相等且相似. 这样, 本节曲线的通用方程有两个.一个为
\[
0=\alpha+\beta x^{2}+\gamma x y+\delta y^{2}+\varepsilon x^{4}+\zeta x^{3} y+\eta x^{2} y^{2}+\theta x y^{3}+c y^{4}+\kappa x^{6}+\cdots
\]
另一个为
\[
0=\alpha x+\beta y+\gamma x^{3}+\delta x^{2} y+\varepsilon x y^{2}+\zeta y^{3}+\eta x^{5}+\theta x^{4} y+\omega x^{3} y^{2}+\cdots
\]
\section{$\S 341$}

因而, 有着相似且相等两部分的这种曲线分为两类.一类是这两部分位于一条直线 的两侧, 正交于该直线的弦都被该直线平分. 该直线是曲线的正交直径, $\S 338$ 和 $\$ 339$ 讨论的是这种曲线. 另一类是曲线在对顶的 $Q$ 和 $T$ 或 $R$ 和 $S$ 中的部分相似且相等, 过点 $C$ 的每一条直线都将曲线分成错位相等的两部分. 上节讨论的是这种曲线. 为区别这两类 相等起见, 称第一类为直径相等, 称第二类为错位相等. 错位相等时, 有这样一个点 $C$, 过它 的弦被它平分. 称这种点 $C$ 为中心. 称错位相等曲线为有中心, 称直径相等曲线为有直径.

\section{$\S 342$}

前面讲了, 函数 $Z$ 中纵标 $y$ 的次数都为偶数时, 方程 $Z=0$ 表示的曲线以直线 $A B$ 为 直径; 横标 $x$ 的次数都为偶数时, 则曲线以 $E F$ 为直径. 如果函数 $Z$ 中, $x, y$ 两者的次数都 为偶数, 则直线 $A B, E F$ 都是正交直径. 从而曲线位于 $Q, R, S, T$ 中的这四部分将相等且 相似,具有此种性质的曲线, 其通用方程为
\[
0=\alpha+\beta x^{2}+\gamma y^{2}+\delta x^{4}+\varepsilon x^{2} y^{2}+\zeta y^{4}+\eta x^{6}+\theta x^{2} y^{2}+\cdots
\]
\section{$\S 343$}

含于该方程的曲线有交于 $C$ 相垂直的两条直径, $A B$ 和 $E F$. 这些直线的阶数, 必定或 为 2 , 或为 4 , 或为 $6, \cdots \cdots$. 奇阶线不能有相垂直的两条直径. 再则, 由于该方程含于 $\S 340$ 的第一个方程之中, 所以这些曲线也都有中心, 这中心为 $C$. 因而过点 $C$ 向两边延伸到曲 线的线段被点 $C$ 平分. 这样, $Z$ 为 $x^{2}$ 和 $y^{2}$ 的有理函数时, 方程 $Z=0$ 给出的曲线有两条 直径.

\section{$\S 344$}

前面讨论了有两条直径的曲线的方程. 现在我们考察有多条直径的曲线的方程. 首 先, 容易证明, 如果曲线只有两条直径, 则这两条直径互相垂直. 这是因为具有两条直径 的曲线, 其方程都含于我们刚求得的方程之中. 假定某条曲 线有两条直径 $A B$ 和 $E F$, 交于 $C$, 但不垂直, 如图 69 所示. 因 为 $E C$ 是直径, 所以其两侧的曲线相同, $E C$ 的一侧有直径 $A C$, 另一侧必有直径 $G C$, 满足 $\angle G C E=\angle A C E$. 类似地, 由 $G C$ 为直径, 在必有性质如 $E C$ 的直径 $I C$, 满足 $\angle G C I=$ $\angle G C E$. 继而, $\angle I C L=\angle I C G$ 时, 直线 $L C$ 为直径. 这样地继 续下去, 每次我们都将得到新的直径, 直至新直径与第一直 径 $A C$ 重合. 当 $\angle A C E$ 与直角的比为有理数时, 这种重合必 定出现.


【图,待补】
%%![](https://cdn.mathpix.com/cropped/2023_02_05_d70135c5d5f1afcc0634g-11.jpg?height=390&width=422&top_left_y=973&top_left_x=1091)

图 69

\section{$\S 345$}

$\angle A C E$ 与直角的比不为有理数时, 直径的数目无穷, 此时曲线为圆, 圆的过中心的 每条直线都是正交直径. 正交直径, 指它分曲线为相似且相等的两部分. 由前面讲的可 知, 代数曲线不能有平行的两条直径. 否则, 照前面的作法将得到间距相等的无数条相平 行的直径. 这样一条直线与曲线有无穷多个交点. 这是代数曲线所不能具有的性质.

\section{$\S 346$}

如果一曲线有几条直径, 则它们必相交于同一点 $C$, 且被等角分开. 这几条直径分为 两类, 交替出现. 直径 $C G$ 的性质同于直径 $C A$. 曲线取 $C G$ 为轴的方程, 同于取 $C A$ 为轴的 方程, 隔一取一的直径 $C A, C G, C L, \cdots$ 对曲线的关系相同. 同样地, 直径 $C E, C I, \cdots$ 对曲 线的关系也相同. 因此, 如果直径条数有限, 则 $\angle A C G$ 除得尽四个直角, 也即 $\angle A C E$ 除 得尽 $180^{\circ}$, 也称 $180^{\circ}$ 为半圆, 记为 $\pi$. 

\section{$\S 347$}

参见图 $70, \angle A C E=90^{\circ}=\frac{1}{2} \pi$, 这是我们讨论过的有两条 相垂直的直径的情形. 我们再次考察这些曲线, 但改用新法. 新 法可用于考察有多条直径的曲线. 设曲线有两条直径 $A B$ 和 $E F$. 在曲线上任取一点 $M$, 画出点 $M$ 与中心 $C$ 的连线 $C M$, 令 $C M=z$, 令 $\angle A C M=s$. 我们求 $z, s$ 间的方程. 首先由直线 $A C$ 为 直径知, 换 $s$ 为 $-s$ 时, $s$ 的函数 $z$ 应该不变, 这是因为取对 $\angle A C M=s$ 的负角 $\angle A C m$, 我们有 $C m$ 等于 $C M$. 换 $s$ 为 $-s, s$ 的 函数 $\cos s$ 不变. 因此, 如果 $z$ 是 $\cos s$ 的某个有理函数, 则前面的要求将满足.


【图,待补】
%%![](https://cdn.mathpix.com/cropped/2023_02_05_d70135c5d5f1afcc0634g-12.jpg?height=400&width=328&top_left_y=458&top_left_x=1145)

图 70 

\section{$\S 348$}
置横标 $C P=x$, 纵标 $P M=y$, 则
\[
z=\sqrt{x^{2}+y^{2}}, \quad \cos s=\frac{x}{z}
\]
设以 $C A$ 为直径的曲线的方程为 $Z=0$, 则 $Z$ 应该是 $z$ 和 $\frac{x}{z}$, 或 $z$ 和 $x$ 的有理函数, 由有理 性, $Z$ 也应该是 $x^{2}+y^{2}$ 和 $x$ 的有理函数, 由有理性, $Z$ 也应该是 $x^{2}+y^{2}$ 和 $x$ 的有理函数. 但是, 如果 $Z$ 是 $x^{2}+y^{2}$ 和 $x$ 的函数, 则 $Z$ 也是 $y^{2}$ 和 $x$ 的函数. 事实上, 令 $x^{2}+y^{2}=u$, 由 $Z$ 应该是 $x$ 和 $u$ 的函数, 令 $u=t+x^{2}$, 则 $t=y^{2}$, 那么, $Z$ 就是 $t$ 和 $x$ 的函数, 也即 $y^{2}$ 和 $x$ 的函 数. 只要 $Z$ 是 $y^{2}$ 和 $x$ 的函数, 直线 $C A$ 就是曲线 $Z=0$ 的直径, 具有一条直径的曲线的这条 性质,是我们前面已经求了出来的.

\section{$\S 349$}

依假定, 我们的曲线有两条直径 $A B$ 和 $E F$, 因而 $C B$ 是性质同于 $C A$ 的直径. 如果直 线 $C M=z$ 以直径 $C B$ 为参数, 则由 $\angle B C M=\pi-s$, 换 $s$ 为 $\pi-s$ 时, $s$ 的函数 $z$ 应不变. $\sin s=\sin (\pi-s)$, 因而函数 $\sin s$ 具有这一性质, 但它不满足前面提出的条件. 要求我们 求出这样一个表达式, 对于 $s,-s, \pi-s$, 其值相同. $\cos 2 s$ 满足这一要求
\[
\cos 2 s=\cos (-2 s)=\cos 2(\pi-s)
\]
因而, 如果 $Z$ 是 $z$ 和 $\cos 2 s$ 的有理函数, $Z=0$ 就是具有两条直径 $A B$ 和 $E F$ 的曲线的方程. 但 $\cos 2 s=\frac{x^{2}-y^{2}}{z^{2}}$, 因而 $Z$ 应该是 $x^{2}+y$ 和 $x^{2}-y^{2}$, 或 $x^{2}$ 和 $y^{2}$ 的函数, 和前面求得的 一致.

\section{$\S 350$}

现在我们考察具有三条直径 $A B, E F$ 和 $G H$ 的曲线. 这三 条直径交于同一点 $C$, 交角 $\angle A C E, \angle E C G, \angle G C B$ 都为 $60^{\circ}=$ $\frac{\pi}{3}$, 如图 71 所示. 隔一取一的直径 $C A, C G, C F$ 有相同的性质.

因而, 如果令 $C M=z, \angle A C M=s$, 则由 $\angle G C M=\frac{2}{3} \pi-s$, 知曲 线方程 $Z=0$ 中的 $Z$ 应该是这样的: $Z$ 为 $z$ 和某个量 $\omega$ 的函数. $\omega$ 依赖于 $s$, 且换 $s$ 为 $-s$ 或 $\frac{2}{3} \pi-s$ 时 $\omega$ 不变, 我们取 $\omega=\cos 3 s$,


【图,待补】
%%![](https://cdn.mathpix.com/cropped/2023_02_05_d70135c5d5f1afcc0634g-13.jpg?height=328&width=366&top_left_y=439&top_left_x=1148)

图 71 

因为
\[
\cos 3 s=\cos (-3 s)=\cos (2 \pi-3 s)
\]
置坐标 $C P=x, P M=y$, 得
\[
\cos 3 s=\frac{x^{3}-3 x y^{2}}{z^{3}}
\]
因而 $Z$ 应该是 $x^{2}+y^{2}$ 和 $x^{3}-3 x y^{2}$ 的有理函数.

\section{$\S 351$}

如果令 $x^{2}+y^{2}=t, x^{3}-3 x y^{2}=u$, 则有三条直径的曲线, 其通用方程为
\[
0=\alpha+\beta t+\gamma u+\delta t^{2}+\varepsilon t u+\zeta u^{2}+\eta t^{3}+\cdots
\]
它给出 $x, y$ 的方程
\[
0=\alpha+\beta\left(x^{2}+y^{2}\right)+\gamma x\left(x^{2}-3 y^{2}\right)+\delta\left(x^{2}+y^{2}\right)^{2}+\cdots
\]
圆的方程为 $0=\alpha+\beta x^{2}+\beta y^{2}$, 圆有无穷多条直径, 圆也满足有三条直径的条件. 因而有三 条直径的最简单的曲线是方程
\[
x^{3}-3 x y^{2}=a x^{2}+a y^{2}+b^{3}
\]
表示的三阶线. 这种三阶线有三条渐近线, 这三条渐近线构成以 $C$ 为中心的等边三角形, 且都是 $u=\frac{A}{t^{2}}$ 状的,属于前面的第五类.

\section{$\S 352$}

如果曲线如图 72 所示, 有四条直径: $A B, E F, G H, I K$, 它们交于一点 $C$, 相邻两条的 交角都为半直角 $\frac{1}{4} \pi$, 则直径 $C A, C G, C B, C H$ 有相同的性质. 令 $C M=z, \angle A C M=s$, 我 们应该确定 $s$ 的一个这样的函数, 换 $s$ 为 $-s$ 或 $\frac{2}{4} \pi-s$, 它不变. $\cos 4 s$ 是这样的函数. 因而, 如果 $Z$ 是 $z$ 和 $\cos 4 s$ 或 $x^{2}+y^{2}$ 和 $x^{4}-6 x^{2} y^{2}+y^{4}$ 的函数, 则方程 $Z=0$ 就给出有四条 直径的曲线,令
\[
t=x^{2}+y^{2}, \quad u=x^{4}-6 x^{2} y^{2}+y^{4}
\]
则 $Z$ 是 $t$ 和 $u$ 的函数. 令 $v=t^{2}-u$, 则 $Z$ 是 $t$ 和 $v$, 也即 $x^{2}+y^{2}$ 和 $x^{2} y^{2}$ 的函数. 也可以确定 $Z$ 为 $x^{2}+y^{2}$ 和 $x^{4}+y^{4}$ 的函数.


【图,待补】
%%![](https://cdn.mathpix.com/cropped/2023_02_05_d70135c5d5f1afcc0634g-14.jpg?height=445&width=448&top_left_y=565&top_left_x=604)

图 72

\section{$\S 353$}

要方程 $Z=0$ 表示的曲线有五条直径,则 $Z$ 应该是 $z$ 和 $\cos 5 \mathrm{~s}$ 的函数, 取直角坐标 $x, y$ 时, 由
\[
\cos 5 s=\frac{x^{5}-10 x^{3} y^{2}+5 x y^{4}}{z^{5}}
\]
得知, $Z$ 应该是
\[
x^{2}+y^{2} \text { 和 } x^{5}-10 x^{3} y^{2}+5 x y^{4}
\]
的函数.从而,除掉圆,方程
\[
x^{5}-10 x^{3} y^{2}+5 x y^{4}=a\left(x^{2}+y^{2}\right)^{2}+b\left(x^{2}+y^{2}\right)+c
\]
表示的五阶线就是最简单的有五条直径的曲线. 最高次部分的因式都为实的, 因而该线 有五条渐近线,围成以 $C$ 为中心的正五边形.

\section{$\S 354$}

从以上的讨论可以看出,一般地,如果 $Z$ 是 $z$ 和 $\cos n s$ 的函数,或者在直角坐标下, $Z$ 是 $x^{2}+y^{2}$ 和
\[
x^{n}-\frac{n(n-1)}{1 \times 2} x^{n-2} y^{2}+\frac{n(n-1)(n-2)(n-3)}{1 \times 2 \times 3 \times 4} x^{n-4} y^{4}-\cdots
\]
的一个有理函数, 则方程 $Z=0$ 表示的曲线有 $n$ 条直径, 相邻直径的夹角都等于 $\frac{\pi}{n}$, 或 者令 
\[
\begin{aligned}
& \text { Infinile analyoio (无穷分析与论 . Fulraduelion } \\
& t=x^{2}+y^{2} \\
& u=x^{n}-\frac{n(n-1)}{1 \times 2} x^{n-2} y^{2}+\frac{n(n-1)(n-2)(n-3)}{1 \times 2 \times 3 \times 4} x^{n-4} y^{4}-\cdots
\end{aligned}
\]
则方程
\[
0=\alpha+\beta t+\gamma u+\delta t^{2}+\varepsilon t u+\zeta u^{2}+\eta t^{3}+\theta t^{2}+\cdots
\]
给出的曲线有 $n$ 条直径. 这样,我们可以求出有随便多少条直径的曲线,这些直径交于一 点, 且相邻两条的夹角都相等. 当然, 直径条数确定的代数全都包含在我们的方程之中.

\section{$\S 355$}

有多条直径的曲线, 必定有直径条数两倍那么多相似且相等的部分. 例如, $\$ 348$
图70,有两条直径的曲线,有4个相似且相等的部分:AE,BE,AF,BF  $\$ 350$图71,有三条直径的曲线, 有 6 个相似且相等的部分: $A E, G E, G B, F B, F H, A H ; \S 352$ 图 72 , 有四 条直径的曲线, 有 8 个相似且相等的部分: $A E, A K, G E, G I, B I, B F, H F, H K$. 相等部分 的个数都是直径条数的两倍. 但是 $\S 341$ 我们说过, 有这样的曲线, 它们有相似的两个部 分但无直径.下面我们来寻求这种有相似且相等部分,但无直径的曲线.

\section{$\S 356$}

我们从图 73 上的曲线开始, 它的位于对顶区域的两部 分 $A M E$ 和 $B K F$ 相等. 如果曲线只有两部分相等, 则这两部 分必对顶. 从相等部分的个数多于 2 的情形, 可进一步看清 这一点. 像做过的那样, 令 $C M=z, \angle A C M=s$. 显然, $s$ 和 $\pi+$ $s$ 对应相同的 $z$ 值. 因为 $\angle A C M=\pi+s$ 时 $z=C K$, 但 $C K$ 应等 于 $C M$. 我们要求出使 $s$ 和 $\pi+s$ 有同值的表达式. $\tan s$ 是这 样的表达式, 因为 $\tan s=\tan (\pi+s)$. 从而, 如果 $Z$ 是 $z$ 和 $\tan$ $s$, 也即 $x^{2}+y^{2}$ 和 $\frac{x}{y}$ 的函数,那么方程 $Z=0$ 表示的就是我们 所要的曲线. 令 $\frac{x}{y}=t$, 则 $x^{2}+y^{2}=y^{2}\left(1+t^{2}\right)$. 从而 $Z$ 是 $t$ 和$y^{2}\left(1+t^{2}\right)$, 也即 $t$ 和 $y^{2}$ 的函数. 结果与我们前面得到的相同.


【图,待补】
%%![](https://cdn.mathpix.com/cropped/2023_02_05_d70135c5d5f1afcc0634g-15.jpg?height=472&width=386&top_left_y=1267&top_left_x=1090)

图 73 



\section{$\S 357$}

正切是分数, 为避免处理分数的麻烦, 我们改用正弦和余弦. 由于
\[
\sin 2 s=\sin 2(\pi+s), \quad \cos 2 s=\cos 2(\pi+s)
\]
取 $Z$ 为 $z, \sin 2 s$ 和 $\cos 2 s$, 也即 $x^{2}+y^{2}, 2 x y$ 和 $x^{2}-y^{2}$ 的函数, 则 $Z=0$ 即为所求. 这里应 该指出, 如缺少 $\sin 2 s$ 和 $\cos 2 s$ 中的一个, 则曲线有直径. 我们得到 $Z$ 应该是 $x^{2}, y^{2}$ 和 $x y$ 的函数,方程为
\[
0=\alpha+\beta x^{2}+\gamma x y+\delta y^{2}+\varepsilon x^{4}+\zeta x^{3} y+\eta x^{2} y^{2}+\theta x y^{3}+\imath y^{4}+\cdots
\]
如果方程中不含 $x$ 的项, 那么以 $x$ 除整个方程, 得
\[
0=\beta x+\gamma y+\varepsilon x^{3}+\zeta x^{2} y+\eta x y^{2}+\theta y^{3}+\kappa x^{5}+\cdots
\]
这两个方程都是我们前面已经求了出来的.

\section{$\S 358$}

现在我们求如图 74 所示的, 只含三个相似且相等部分 $A M, B N$ 和 $D L$ 的曲线, 从点 $C$ 到直线作邻线夹角相等的三 条直线 $C M, C N, C L$, 则这三条直线相等. 因而, 令 $\angle A C M=$ $s$, 直线 $C M=z$, 那么由 $\angle M C N=\angle N C L=\frac{2}{3} \pi$, 我们求由 $s$ 确定的 $z$, 使
\[
s, \frac{2}{3} \pi+s \text { 和 } \frac{4}{3} \pi+s
\]
对应相同的 $z$ 值. $\sin 3 s$ 和 $\cos 3 s$ 对这三个角都有相同的值.


【图,待补】
%%![](https://cdn.mathpix.com/cropped/2023_02_05_d70135c5d5f1afcc0634g-16.jpg?height=362&width=403&top_left_y=714&top_left_x=1091)

图 74

因此, 如果 $Z$ 是 $x^{2}+y^{2}, 3 x^{2} y-y^{3}$ 和 $x^{3}-3 x y^{2}$ 这三个量的有理函数, 那么方程 $Z=0$ 就给 出所要的所有曲线, 从而这种曲线的通用方程为
\[
\begin{aligned}
0= & \alpha+\beta\left(x^{2}+y^{2}\right)+\gamma\left(3 x^{2} y-y^{3}\right)+\delta\left(x^{3}-3 x y^{2}\right)+\varepsilon\left(x^{2}+y^{2}\right)^{2}+ \\
& \zeta\left(x^{2}+y^{2}\right)\left(3 x^{2} y-y^{3}\right)+\eta\left(x^{2}+y^{2}\right)\left(x^{3}-3 x y^{2}\right)+\cdots
\end{aligned}
\]
这样,具有这种性质的三阶线含于下面的这个方程之中
\[
0=\alpha+\beta x^{2}+\beta y^{2}+\delta x^{3}+3 \gamma x^{2} y-3 \delta x y^{2}-\gamma y^{3}
\]
\section{$\S 359$}

参见图 73, 如果曲线有四个相等的部分 $A M, E N, B K$ 和 $F L$, 那么从中心 $C$ 到曲线引 邻线夹角相等的四条直线, 例如 $C M, C N, C K$ 和 $C L$, 则这四条直线相等. 令 $\angle A M C=s$, 直线 $C M=z$. 由于
\[
\angle M C N=\angle N C K=\angle K C L=90^{\circ}=\frac{1}{2} \pi
\]
用 $s$ 表示的 $z$ 应该使
\[
s, \frac{1}{2} \pi+s, \pi+s, \frac{3}{2} \pi+s
\]
对应的 $z$ 值相同. 表达式 $\sin 4 s$ 和 $\cos 4 s$ 具有这种性质. 因而, 如果 $Z$ 是三个量
\[
x^{2}+y^{2}, 4 x^{3} y-4 x y^{3} \text { 和 } x^{4}-6 x^{2} y^{2}+y^{4}
\]
的某个函数,则方程 $Z=0$ 就给出具有四个相等部分的曲线. 从而这类曲线的通用方程为
\[
0=\alpha+\beta x^{2}+\beta y^{2}+\gamma x^{4}+\delta x^{3} y+\varepsilon x^{2} y^{2}-\delta x y^{3}+\gamma y^{4}+\cdots
\]
\section{$\S 360$}

没有直径,但有五个相等且相似部分的曲线, 求法类似. 这时方程 $Z=0$ 中的 $Z$ 应该 是下面这三个量的有理函数
\[
x^{2} y^{2}, 5 x^{4} y-10 x^{2} y^{3}+y^{5}, x^{5}-10 x^{3} y^{2}+5 x y^{4}
\]
一般地, 如果相等部分的个数为 $n$, 那么 $Z$ 应该是下面这三个表达式的有理函数
\[
n x^{n-1} y-\frac{n(n-1)(n-2)}{1 \times 2 \times 3} x^{n-3} y^{3}+\frac{n(n-1)(n-2)(n-3)(n-4)}{1 \times 2 \times 3 \times 4 \times 5} x^{n-5} y^{5}-\cdots
\]
和
\[
x^{n}-\frac{n(n-1)}{1 \times 2} x^{n-2} y^{2}+\frac{n(n-1)(n-2)(n-3)}{1 \times 2 \times 3 \times 4} x^{n-4} y^{4}-\cdots
\]
如果后两个表达式中有一个不在方程中出现, 那么曲线将有 $n$ 条直径.

\section{$\S 361$}

有若干个相等部分的曲线分两类, 有直径的和没有直 径的. 这两类包含了有两个或更多个相似相等部分的一切 代数曲线. 为证实这一点, 我们假定一连续曲线有相似且相 等的两部分 $O A a$ 和 $O B b$, 如图 75 所示. 作点 $A, B$ 的连线 $A B$, 并以 $A B$ 为底作顶角 $C$ 等于角 $O$ 的等腰三角形 $A B C$. 由于 $\angle O A C$ 和 $\angle O B C$ 相等,知曲线的 $C A a, C B b$ 这两部分 相似且相等. 又由于连续规律, 如果取 $\angle B C D, \angle D C E, \cdots$ 都等于 $\angle A C B$, 且 $C D=C E=C A=C B$, 则曲线的 $D d$, $E e, \cdots$ 部分都相似且相等于 $A a, B b$. 这样, 只要 $\angle A C B$ 与$360^{\circ}$ 之比不是无理数, 相等部分的个数就是有限的, 否则是无限的, 从而曲线不是代数 的. 这是我们前面讨论过的没有直径的曲线.


【图,待补】
%%![](https://cdn.mathpix.com/cropped/2023_02_05_d70135c5d5f1afcc0634g-17.jpg?height=388&width=409&top_left_y=1161&top_left_x=1088)

图 75 

\section{$\S 362$}

如果相似且相等的两部分 $O A a$ 和 $O B b$ 在直线 $A O$ 和 $B O$ 的内侧, 如图 76 所示, 我们 引相平行的直线 $A R$ 和 $B S$, 使
\[
\angle O A R=\angle O B S=\frac{1}{2} \angle A O B
\]
连 $A, B$ 成直线 $A B$, 过 $A B$ 的中点 $C$ 引直线 $C V$, 使平行于 $A R$ 和 $B S$, 则 $a A$ 和 $b B$ 两部分关 于直线 $C V$ 相似且相等. 只要 $b a \neq 0$, 那么弧 $B b$ 从 $b$ 移到 $a$, 则 $B b$ 相似且相等于另一侧的 $a A ; a A$ 从 $a$ 移到 $e$, 移动距离 $a e=b a$, 则 $a A$ 相似且相等于另一侧的 $e E$; 继而 $e E$ 相似且相等于另一侧的 $d D$. 这样该曲线有无穷多个相似且相等的部分, 它们位于直线 $C V$ 的两 侧,也即这类曲线不是代数曲线.


【图,待补】
%%![](https://cdn.mathpix.com/cropped/2023_02_05_d70135c5d5f1afcc0634g-18.jpg?height=321&width=558&top_left_y=443&top_left_x=535)

图 76

\section{$\S 363$}

上节讨论的是直线 $A B$ 倾斜于平行直线 $A R$ 和 $B S$, 或者 $\triangle A O B$ 的边 $A O, B O$ 不相等 的情形. 如果 $A O=B O$, 则直线垂直于平行线 $A R, B S$, 也垂直于 $C V$. 这时的 $C V$ 通过 $O$, 因而此时点 $a, b$ 重合. 又由于 $a A$ 和 $b B$ 相等且相似, 知直线 $C V$ 是曲线的直径, 这属于我 们讨论过的有直径的曲线. 从而有两个或更多个相似且相等部分的代数曲线全都包含在 本章讨论的曲线之中. 

