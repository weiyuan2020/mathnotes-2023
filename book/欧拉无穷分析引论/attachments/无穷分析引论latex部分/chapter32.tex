\chapter{第十四章 曲线的曲率}

$\S 304$

前章我们考察了曲线各点处指示方向的直线. 进一步, 现在我们考察曲线, 它比原曲 线简单得多, 它的一部分与原曲线靠得很近, 至少有很少的一段与原曲线相合. 这样从简 单曲线的性质可以得到原曲线的性质. 这类似于对伸向无穷分支的考察方法. 先考虑切 线, 再进一步考虑比原曲线简单的曲线, 它更靠近原曲线, 不只是相切, 而是相合. 两条曲 线的这种关系叫密切.

\section{$\S 305$}

设给了直角坐标 $x, y$ 间的一个方程,曲线如图 55 所示. 我们来考虑靠近点 $M$ 的极小 的一部分 $M m$ 的性质, 确定了横标 $A P=x$ 和纵标 $P M=y$ 之后, 在新轴 $M R$ 上取极小的横 标 $M q=t$, 纵标 $q \mu=u$. 这样我们有 $x=p+t, y=q+u$. 将这两个值代入 $x, y$ 间方程, 得
\[
0=A t+B u+C t^{2}+D t u+E u^{2}+F t^{3}+G t^{2} u+\cdots
\]
它表示同一曲线关于轴 $M R$ 的性质. 由于新坐标 $t, u$ 极小,所以前两项以外的项, 与前两 项相比, 都可视为无穷小, 都可略去.

\section{$\S 306$}

系数 $A, B$ 不同时为零时, 去掉其余的项, 得方程 $0=A t+B u$, 它给出的直线在点 $M$ 处与曲线相切, 与曲线点 $M$ 处的方向相同. 这样我们有 $M q: q \mu=B:(-A), A, B$ 已给, 因 而曲线上点 $M$ 处切线 $M \mu$ 的位置已知. 现在我们来看看, 至少是在短距离之内, 曲线 $M m$ 对直线 $M \mu$ 的偏离. 为此我们取法线 $M N$ 为轴, 并从点 $m$ 引直线纵标 $m r$. 令 $M r=r, r m=$ $s$, 则
\[
\begin{array}{lr}
t=\frac{-A r+B s}{\sqrt{A^{2}+B^{2}}}, \quad u=\frac{-A s-B r}{\sqrt{A^{2}+B^{2}}} \\
r=\frac{-A t-B u}{\sqrt{A^{2}+B^{2}}}, \quad s=\frac{B t-A u}{\sqrt{A^{2}+B^{2}}}
\end{array}
\]
由
\[
-A t-B u=C t^{2}+D t u+E u^{2}+F t^{3}+G t^{2} u+\cdots
\]
知 $r$ 与 $t$ 及 $u$ 比较为无穷小. 且 $r$ 与 $s$ 比较也为无穷小, 这是因为 $s$ 为 $t, u$ 的一次函数, 而 $r$ 用 $t, u$ 的二次或更高次幂表示.

\section{$\S 307$}

如果我们不略去二次部分 $C t^{2}+D t u+E u^{2}$, 只略去它后面的项, 那么结果会更靠近 曲线 $M m$. 此时我们得到 $t, u$ 间的方程
\[
-A t-B u=C t^{2}+D t u+E u^{2}
\]
把前节求得的 $t$ 和 $u$ 值代入该方程, 得
\[
\begin{aligned}
r \sqrt{A^{2}+B^{2}}= & \frac{\left(A^{2} C+A B D+B^{2} E\right) r^{2}}{A^{2}+B^{2}}+ \\
& \frac{\left(A^{2} D-B^{2} D-2 A B C+2 A B E\right) r s}{A^{2}+B^{2}}+ \\
& \frac{\left(A^{2} E-A B D+B^{2} C\right) s^{2}}{A^{2}+B^{2}}
\end{aligned}
\]
由 $r$ 与 $s$ 比较为无穷小知, $r^{2}$ 和 $r s$ 与 $s^{2}$ 比较可略去. 这样我们得到
\[
s^{2}=\frac{\left(A^{2}+B^{2}\right) r \sqrt{A^{2}+B^{2}}}{A^{2} E-A B D+B^{2} C}
\]
该方程表示的曲线与原曲线在点 $M$ 处密切.

\section{$\S 308$}

也即所给曲线的极短的一段弧 $M m$, 与轴为 $M n$, 参数为
\[
\frac{\left(A^{2}+B^{2}\right) \sqrt{A^{2}+B^{2}}}{A^{2} E-A B D+B^{2} C}
\]
的抛物线的顶点重合. 我们称该抛物线顶点处的曲率为所给曲线点 $M$ 处的曲率. 曲线中 圆的曲率我们最清楚, 各点相同, 都与自己的半径成反比. 因而用与曲线有着相同曲率的 圆, 来定义曲线的曲率, 最为方便. 通常称这种圆为密切圆. 因此我们要求出这样的圆, 它 的曲率等于抛物线顶点处的曲率, 我们就用这个圆来代替密切抛物线.

\section{$\S 309$}

为了求出这样的圆, 我们视圆的曲率为末知数, 并照前面的方法用抛物线的曲率表 示它. 这样我们就可以反过来用密切圆代替密切抛物线. 假定所给曲线 $M m$ 是半径为 $a$ 的圆, 其性质由方程 $y^{2}=2 a x-x^{2}$ 表示, 取 $A P=p, P M=q$, 得 $q^{2}=2 a p-p^{2}$, 令
\[
x=p+t, \quad y=q+u
\]
得方程
\[
q^{2}+2 q u+u^{2}=2 a p+2 a t-p^{2}-2 p t-t^{2}
\]
由于 $q^{2}=2 a p-p^{2}$, 该方程化简为 
\[
\begin{aligned}
& 0=2 a t-2 p t-2 q u-t^{2}-u^{2}
\end{aligned}
\]
与前面的方程相比较,得
\[
A=2 a-2 p, \quad B=-2 q, \quad C=-1, \quad D=0, \quad E=-1
\]
从而
\[
\begin{gathered}
A^{2}+B^{2}=4\left(a^{2}-2 a p+p^{2}+q^{2}\right)=4 a^{2} \\
\left(A^{2}+B^{2}\right) \sqrt{A^{2}+B^{2}}=8 a^{3} \\
A^{2} E-A B D+B^{2} C=-A^{2}-B^{2}=-4 a^{2}
\end{gathered}
\]
由此得知半径为 $a$ 的圆密切抛物线 $s^{2}=2 a r$ 于顶点. 反之, 如果曲线有密切抛物线 $s^{2}=b r$, 则该曲线有半径为 $\frac{1}{2} b$ 的密切圆.

\section{$\S 310$}

前面我们已经求出了曲线 $M m$ 的一条密切抛物线
\[
s^{2}=\frac{\left(A^{2}+B^{2}\right) \sqrt{A^{2}+B^{2}}}{A^{2} E-A B D+B^{2} C}
\]
该曲线在点 $M$ 处的曲率等于半径为
\[
\frac{\left(A^{2}+B^{2}\right) \sqrt{A^{2}+B^{2}}}{2\left(A^{2} E-A B D+B^{2} C\right)}
\]
的圆的曲率. 也即, 该表达式给出密切圆的半径. 密切圆的半径, 通常称为密切半径, 也称 为曲率半径. 这也就是说, 从给定的 $x, y$ 间的方程, 我们可以推出 $t, u$ 间的方程. 从 $t, u$ 间 的方程, 我们立即可以求出曲线在点 $M$ 处的密切半径, 也即点 $M$ 处密切于曲线的圆的半 径. 去掉 $t, u$ 间方程中 $t, u$ 次数和大于 2 的项,得如下形状的方程
\[
0=A t+B u+C t^{2}+D t u+E u^{2}
\]
从该方程求得密切半径等于
\[
\frac{\left(A^{2}+B^{2}\right) \sqrt{A^{2}+B^{2}}}{2\left(A^{2} E-A B D+B^{2} C\right)}
\]
\section{$\S 311$}

这里有一点不确定, 即 $\sqrt{A^{2}+B^{2}}$ 可正可负, 因而密切半径表达式可正可负. 也即曲 线的向点 $N$ 的一面可凸可凹. 为了变这不确定为确定, 我们应该弄清曲线上点 $m$ 是在切 线 $M_{\mu}$ 的 $A N$ 一侧, 还是在切线的另一侧. 前一情况下曲线的向 $N$ 面为凹, 密切圆圆心在 直线 $M N$ 上, 后一种情况下, 圆心在直线 $N M$ 的延长线上, 在点 $M$ 之外. 可见, 弄清 $q m$ 是 短于还是长于 $q \mu$, 不确定性即完全排除. 短于, 则曲线的向 $N$ 面为凹, 长于, 则曲线的向 $N$ 面为凸. 

\section{$\S 312$}

但 $q \mu=-\frac{A t}{B}, q m=u$, 我们须弄清 $-\frac{A t}{B}$ 比 $u$ 大还是比 $u$ 小. $m \mu$ 很短, 令 $m \mu=\omega$, 则 $u=-\frac{A t}{B}-\omega$ 代入 $t, u$ 间方程, 得
\[
0=-B \omega+C t^{2}-\frac{A D t^{2}}{B}-D t \omega+\frac{A^{2} E t^{2}}{B^{2}}+\frac{2 A E t \omega}{B}+E \omega^{2}
\]
该表达式中 $\omega$ 比 $t$ 小很多,略去含 $t \omega$ 和含 $\omega^{2}$ 的项,得
\[
\omega=\frac{\left(B^{2} C-A B D+A^{2} E\right) t^{2}}{B^{3}}
\]
当
\[
\frac{B^{2} C-A B D+A^{2} E}{B^{3}} \text { 或 } \frac{A^{2} E-A B D+B^{2} C}{B}
\]
为正时, $\omega$ 为正, 曲线向 $N$ 的一面为凹. 如果 $\omega$ 为负,则曲线向 $N$ 的一面为凸.

\section{$\S 313$}

为了更清楚起见, 我们对可能遇到的几种情形分别进行讨论. 先设 $B=0$, 此时纵标 线 $P M$ 是曲线 $M m$ 的切线, 密切半径为 $\frac{A}{2 E}$. 曲线如图 57 所示, 凹面对 $R$, 还是相反, 凸面 对 $R$, 这可从方程 $0=A t+C t^{2}+D t u+E u^{2}$ 判定. 事实上, 由 $M q=t, q m=u$ 知, 与 $u$ 相比较 $t$ 为无穷小, 与 $u^{2}$ 相比较, 略去含 $t^{2}, t u$ 的项, 得 $A t+E u^{2}=0$. 由该方程我们看到, 系数 $A$, $E$ 反号, 即 $\frac{A}{E}$ 为负时, 曲线凹面向 $R$. 如果系数 $A, E$ 同号, $\frac{A}{E}$ 为正, 则曲线在切线的另一 侧. 此时应该横标 $t$ 为负, 对应的纵标 $q m$ 才能为实数.


【图,待补】
%%![](https://cdn.mathpix.com/cropped/2023_02_05_b169e65bf9064e07ce6cg-19.jpg?height=352&width=335&top_left_y=1629&top_left_x=682)

图 57

\section{$\S 314$}

设切线 $M_{\mu}$ 倾斜于轴 $A P$, 或倾斜于平行于轴的直线, 如 $\S 286$ 图 55 所示. 此时 $\angle R M \mu$ 为锐角, 法线 $M N$ 与轴的交点位于 $P$ 点右侧. 这种情况下横标 $t$ 对应正的纵标 $u$, 因此系数 $A, B$ 异号, 分数 $\frac{A}{B}$ 为负. 我们知道, 这样的情况下, 如果
\[
\frac{A^{2} E-A B D+B^{2} C}{B}
\]
为正,或 (因 $\frac{A}{B}$ 为负)
\[
\frac{A^{2} E-A B D+B^{2} C}{A}
\]
为负,则曲线凹面向 $N$. 反之,如果
\[
\frac{A^{2} E-A B D+B^{2} C}{B}
\]
为负,或
\[
\frac{A^{2} E-A B D+B^{2} C}{A}
\]
为正,则曲线凸面向 $N$. 两种情况下, 密切半径都等于
\[
\frac{\left(A^{2}+B^{2}\right) \sqrt{A^{2}+B^{2}}}{2\left(A^{2} E-A B D+B^{2} C\right)}
\]
\section{$\S 315$}

设 $A=0$, 则如图 58 所示, 直线 $M R$ 平行于轴, 并为曲线的切线. 与 $t$ 比较, $u$ 为无穷 小, 由此得 $0=B u+C t^{2}$. 可见 $B, C$ 同号, 即 $B C$ 为正时, $u$ 应该为负, 因而曲线凹面向点 $P$. 根据前面导出的规则, $A=0$ 时 $N$ 与 $P$ 重合. 此时密切半径等于 $\frac{B}{2 C}$. 切线与轴的交点位于 $P$ 的右侧时 (图 59), 前面所给出的规则依然适用. 此时曲线凹面还是凸面向 $N$, 由表达式
\[
\frac{A^{2} E-A B D+B^{2} C}{B}
\]
的正负决定.密切半径依然都等于
\[
\frac{\left(A^{2}+B^{2}\right) \sqrt{A^{2}+B^{2}}}{2\left(A^{2} E-A B D+B^{2} C\right)}
\]

【图,待补】
%%![](https://cdn.mathpix.com/cropped/2023_02_05_b169e65bf9064e07ce6cg-20.jpg?height=361&width=335&top_left_y=1859&top_left_x=280)

图 58


【图,待补】
%%![](https://cdn.mathpix.com/cropped/2023_02_05_b169e65bf9064e07ce6cg-20.jpg?height=383&width=639&top_left_y=1841&top_left_x=743)

图 59

%%08p141-160
\section{$\S 316$}

设给定的曲线为椭圆, 图 60 上是第一象限部分, 中心为 $A$, 横半轴 $A D=a$, 与它共轭 的半轴 $A C=b$. 置横标 $x$ 于轴 $A D$ 上, 并从中心 $A$ 计算起, 得椭圆方程
\[
a^{2} y^{2}+b^{2} x^{2}=a^{2} b^{2}
\]
任取横标 $A P=p$, 记对应纵标 $P M=q$, 得
\[
a^{2} q^{2}+b^{2} p^{2}=a^{2} b^{2}
\]
令 $x=p+t, y=q+u$, 得
\[
a^{2} q^{2}+2 a^{2} q u+a^{2} u^{2}+b^{2} p^{2}+2 b^{2} p t+b^{2} t^{2}=a^{2} b^{2}
\]
或
\[
2 b^{2} p t+2 a^{2} q u+b^{2} t^{2}+a^{2} u^{2}=0
\]
首先由于 $t, u$ 的系数为正, 法线 $M N$ 交轴于点 $P$ 之左, 又由于 $A=2 b^{2} p, B=2 a^{2} q$, 我们有
\[
P M: P N=B: A=a^{2} q: b^{2} p, \quad P N=\frac{b^{2} p}{a^{2}}
\]
再由于 $C=b^{2}, D=0, E=a^{2}$, 我们有
\[
\frac{A^{2} E-A B D+B^{2} C}{B}=\frac{4 a^{2} b^{2}\left(a^{2} q^{2}+b^{2} p^{2}\right)}{2 a^{2} q}=\frac{4 a^{4} b^{4}}{2 a^{2} q}
\]
该表达式为正,这表明曲线凹面向 $N$.


【图,待补】
%%![](https://cdn.mathpix.com/cropped/2023_02_05_d70135c5d5f1afcc0634g-01.jpg?height=363&width=462&top_left_y=1269&top_left_x=612)

图 60

\section{$\S 317$}

由
\[
A^{2}+B^{2}=4\left(a^{4} q^{2}+b^{4} p^{2}\right), \quad A^{2} E-A B D+B^{2} C=4 a^{4} b^{4}
\]
得密切半径等于 $\frac{\left(a^{4} q^{2}+b^{4} p^{2}\right)^{\frac{3}{2}}}{a^{4} b^{4}}$, 但
\[
M N=\sqrt{q^{2}+\frac{b^{4} p^{2}}{a^{4}}}
\]
从而
\[
\sqrt{a^{4} q^{2}+b^{4} p^{2}}=a^{2} \cdot M N
\]
进而密切半径等于 $\frac{a^{2} \cdot M N^{3}}{b^{4}}$, 如果从中心 $A$ 向法线 $M N$ 引垂线 $A O$, 那么由于 $A N=p-$ $\frac{b^{2} p}{a^{2}}$ 且. $\triangle M N P$ 和 $\triangle A N O$ 相似, 得
\[
\begin{gathered}
N O=\frac{a^{2} b^{2} p^{2}-b^{4} p^{2}}{a^{4} M N} \\
M O=N O+M N=\frac{a^{2} q^{2}+b^{2} p^{2}}{a^{2} M N}=\frac{b^{2}}{M N}
\end{gathered}
\]
从而 $M N=\frac{b^{2}}{M O}$, 继而密切半径等于 $\frac{a^{2} b^{2}}{M O^{3}}$. 该表达式对轴 $A D$ 和 $A C$ 都适用.

\section{$\S 318$}

知道了曲线每点处的密切半径,曲线的性质也就完全清楚了.事实上,如果曲线被分 成尽可能小的部分, 每一部分就都可以看成以本部分的密切半径为半径的圆弧. 这样我 们就可以相当精确地画出曲线. 方法是, 得到了曲线通过的许多点之后, 对每一点依次求 出切线、法线和密切半径. 那么位于两点之间的这些很小的部分, 就可以用圆规画出. 点 取得越多, 靠得越近, 用这种方法画出的曲线就越精确.

\section{$\S 319$}

因为 $\S 286$ 图 55 上过点 $M$ 的一小段曲线, 与密切圆的一段弧重合, 所以不只整个 $M m$, 而且连上 $M n$ 的这一段, 曲率相同. 事实上, 由于描述曲线很小部分 $M m$ 的方程, 是 坐标 $M r=r, r m=s$ 间的 $s^{2}=\alpha r$, 所以每个极小横标 $M r=r$ 都对应由方程决定的两个纵

标, 一正一负. 因而曲线同时向 $n$ 和 $m$ 两个方向延伸, 在密切半径 $\frac{1}{2} \alpha$ 为有限值处, 在小范 围内, 每点两侧曲率都相同. 此时曲线不会急剧改变, 不会出现尖点, 也不会 $M n$ 部分凸 面向 $N$, 而 $M m$ 部分凹面向 $N$. 曲线上两侧一凸一凹的点叫拐点, 或反向弯曲点. 密切半 径为有限值的点, 既不能是尖点, 也不能是拐点.

\section{$\S 320$}

从 $t, u$ 间方程
\[
0=A t+B u+C t^{2}+D t u+E u^{2}+F t^{3}+G t^{2} u+H t u^{2}+\cdots
\]
我们求得密切半径等于
\[
\frac{\left(A^{2}+B^{2}\right) \sqrt{A^{2}+B^{2}}}{2\left(A^{2} E-A B D+B^{2} C\right)}
\]
显然, 如果 $A^{2} E-A B D+B^{2} C=0$, 则密切半径为无穷, 即密切圆成为直线, 曲率为零, 曲 线在一点两侧的两小部分在同一直线上. 为对此情况下的直线进行详细考察, 我们对 $F t^{3}+G t^{2} u+H t u^{2}+I u^{3}$ 也作代换
\[
t=\frac{-A r+B s}{\sqrt{A^{2}+B^{2}}}, \quad u=\frac{-A s-B r}{\sqrt{A^{2}+B^{2}}}
\]
与 $r \sqrt{A^{2}+B^{2}}$ 比较,含 $r$ 的后继项都可忽略,去掉它们,得方程
\[
\begin{aligned}
& r \sqrt{A^{2}+B^{2}}=\alpha s^{2}+\beta s^{3}+\gamma s^{4}+\delta s^{5}+\cdots
\end{aligned}
\]
\section{$\S 321$}

跟前面一样, 从该方程我们立即得到密切半径等于 $\frac{\sqrt{A^{2}+B^{2}}}{2 \alpha}$. 但 $\alpha=0$ 时这密切半 径为无穷, 为更准确地考察曲线性质, 取后继项 $\beta s^{3}$, 得 $r \sqrt{A^{2}+B^{2}}=\beta s^{3} \cdot \beta \neq 0$ 时, 与 $\beta s^{3}$ 比较, $\gamma s^{4}, \delta s^{5}, \cdots$ 都可去. 这样点 $M$ 处密切曲线的方程为
\[
r \sqrt{A^{2}+B^{2}}=\beta s^{3}
\]
从这个方程, 我们可以确定曲线在点 $M$ 附近的形状. $r$ 取负值时, 对应的纵标 $s$ 也为负, 因 而在点 $M$ 附近曲线 $m M_{\mu}$ 为蛇形, 如图 61 所示, 也即点 $M$ 为拐点.


【图,待补】
%%![](https://cdn.mathpix.com/cropped/2023_02_05_d70135c5d5f1afcc0634g-03.jpg?height=359&width=539&top_left_y=1218&top_left_x=554)

图 61

\section{$\S 322$}

如果 $\alpha$ 为零, $\beta$ 也为零, 则曲线在点 $M$ 附近的性质由方程
\[
r \sqrt{A^{2}+B^{2}}=\gamma s^{4}
\]
描述. 由该方程使每个横标 $r$ 对应两个纵标 $s$, 一正一负, 知曲线的两部分 $M m$ 和 $M \mu$ 在切 线的同一侧, 如图 62 所示. 如果 $\alpha, \beta, \gamma$ 都为零, 则曲线在点 $M$ 附近的性质由方程
\[
r \sqrt{A^{2}+B^{2}}=\delta s^{5}
\]
描述, 曲线又以点 $M$ 为拐点, 如图 61 所示. 如果 $\delta$ 也等于零, 则方程为
\[
r \sqrt{A^{2}+B^{2}}=\varepsilon s^{6}
\]
曲线又如图 62 所示, 没有拐点. 一般地, 如果 $s$ 的指数为奇数, 则曲线在 $M$ 点处有一个拐点; 如果 $s$ 的指数为偶数, 则曲线在点 $M$ 处没有拐点, 如图 62 所示.


【图,待补】
%%![](https://cdn.mathpix.com/cropped/2023_02_05_d70135c5d5f1afcc0634g-04.jpg?height=366&width=486&top_left_y=348&top_left_x=585)

图 62

\section{$\S 323$}

我们讨论了点 $M$ 为单重点的情形, 即方程
\[
0=A t+B u+C t^{2}+D t u+E u^{2}+F t^{3}+\cdots
\]
中系数 $A, B$ 不同时为零的情形. 如果 $A, B$ 都为零, 则曲线有两个或更多个分支在点 $M$ 处 相交, 如图 56 所示. 这时应该像前面做过的那样, 分别对点 $M$ 处每个分支的曲率和性质 进行考察. 设某个分支的切线方程为 $m t+n u=0$. 我们来求该分支的坐标 $r, s$ 间方程. $r$ 取 在法线 $M N$ 上,与 $s$ 比较, $r$ 为无穷小,因而我们令
\[
t=\frac{-m r+n s}{\sqrt{m^{2}+n^{2}}}, \quad u=\frac{-m s-n r}{\sqrt{m^{2}+n^{2}}}
\]
代入方程, 并去掉与保留项相比较为无穷小的项. 那么, $M$ 为二重点时得方程
\[
r s=\alpha s^{3}+\beta s^{4}+\gamma s^{5}+\delta s^{6}+\cdots
\]
$M$ 为三重点时得方程
\[
r s^{2}=\alpha s^{4}+\beta s^{5}+\gamma s^{6}+\cdots
\]
类推. 这些方程都可以化为
\[
r=\alpha s^{2}+\beta s^{3}+\gamma s^{4}+\delta s^{5}+\cdots
\]
\section{$\S 324$}

从该方程我们看到, 所考虑的分支在点 $M$ 处的密切半径为 $\frac{1}{2 \alpha}, \alpha$ 为零时它为无穷大. 此时表示曲线性质的方程或为 $r=\beta s^{3}$, 或为 $r=\gamma s^{4}$, 或为 $r=\delta s^{5}, \cdots$ 跟前面一样, 从这些方 程可以断定点 $M$ 处有无拐点, $s$ 的指数为奇数时有拐点, 为偶数时没有. 各分支切线不同 时, 应该对过点 $M$ 的每一个分支进行这样的判断. 

\section{$\S 325$}
如果点$M$处两条或更多条切线重合,则应换个方式进行讨论.设$A,B$都为零,即方程为
\[
0=C t^{2}+D t u+E u^{2}+F t^{3}+G t^{2} u+\cdots
\]
且表达式 $C t^{2}+D t u+E u^{2}$ 的两个线性因式相等, 即点 $M$ 处相交的两个分支有共同的切 线,设
\[
C t^{2}+D t u+E u^{2}=(m t+n u)^{2}
\]
参见图 55 , 我们变 $t, u$ 间方程为 $M r=r, r m=s$ 间方程. 为此, 令
\[
t=\frac{-m r+n s}{\sqrt{m^{2}+n^{2}}}, \quad u=\frac{-m s-n r}{\sqrt{m^{2}+n^{2}}}
\]
得方程
\[
r^{2}=\alpha r s^{2}+\beta s^{3}+\gamma r s^{3}+\delta s^{4}+\varepsilon r s^{4}+\zeta s^{5}+\cdots
\]
与第一项 $r^{2}$ 相比较,略去了 $r$ 的次数大于 1 的项.

\section{$\S 326$}

与 $s$ 相比较 $r$ 为无穷小, 因而与项 $\beta s^{3}$ 相比较, 别的项都可略去. 这样, 只要 $\beta \neq 0$, 曲 线在点 $M$ 附近的性质就由方程 $r^{2}=\beta s^{3}$ 描述. 由 $r=s \sqrt{\beta s}=s^{2} \sqrt{\frac{\beta}{s}}$, 知点 $M$ 处密切半径为 $\frac{1}{2} \sqrt{\frac{s}{\beta}}$, 进而由 $M$ 处 $s$ 可忽略, 知密切半径为零. 因而点 $M$ 处曲率为无穷大, 也即点 $M$ 处 的这段曲线是无穷小圆的一部分. 由 $r$ 反号时 $s$ 不变, 知曲线以点 $M$ 为尖点, 且被公切线 $M t$ 分为 $M m, M \mu$ 两部分, 两部分都凸面向 $M t$, 见图 63.


【图,待补】
%%![](https://cdn.mathpix.com/cropped/2023_02_05_d70135c5d5f1afcc0634g-05.jpg?height=467&width=560&top_left_y=1389&top_left_x=567)

图 63

\section{$\S 327$}

如果 $\beta=0$, 并 $\delta s^{4}$ 在方程中出现, 且与 $\delta s^{4}$ 相比较, $\gamma r s^{3}$ 略去,则曲线在点 $M$ 附近的性 质, 由方程 $r^{2}=\alpha r s^{2}+\delta s^{4}$ 表示. $\alpha^{2}<-4 \delta$ 时, 有复因式, 点 $M$ 为共轭点. $\alpha^{2}>-4 \delta$ 时, 方程 分解为 $r=f s^{2}, r=g s^{2}$. 此时点 $M$ 处两个分支相切, 密切半径分别为 $\frac{1}{2 f}$ 和 $\frac{1}{2 g}$. 如果这两个 分支凹面朝向相同, 则曲线形状如图 64 所示, 为两条内切圆弧; 如果凹面朝向相反, 则曲 线形状如图 65 所示, 为两条外切圆弧.


【图,待补】
%%![](https://cdn.mathpix.com/cropped/2023_02_05_d70135c5d5f1afcc0634g-06.jpg?height=462&width=366&top_left_y=516&top_left_x=360)

图 64


【图,待补】
%%![](https://cdn.mathpix.com/cropped/2023_02_05_d70135c5d5f1afcc0634g-06.jpg?height=510&width=407&top_left_y=482&top_left_x=876)

图 65

\section{$\S 328$}

如果 $\delta$ 也可略去,则方程或者能够或者不能够分解成另外两个方程. 能够时, 得到在 点 $M$ 处相切的两个分支, 其性质都由状如 $r=\alpha s^{m}$ 的方程表示. 种类数等于以 $M$ 为单重点 的分支的两两组合数. 称包含在方程 $r=\alpha s^{m}$ 中的这种分支为一阶分支. 不能够时, 则表示 曲线性质的方程, 或者为 $r^{2}=\alpha s^{5}$, 或者为 $r^{2}=\alpha s^{7}$, 或者为 $r^{2}=\alpha s^{9}, \cdots$ 这类分支, 包括前面 讨论过的 $r^{2}=\alpha s^{3}$, 都叫二阶分支. 它们中的每一个都相当于在点 $M$ 处有公切线的两个分 支. 二阶分支的形状都如图 63 所示, 即都同于方程 $r^{2}=\alpha s^{3}$ 所表示的, 在点 $M$ 处有尖点. 但有一点不同, 点 $M$ 处的密切半径, 方程 $r^{2}=\alpha s^{3}$ 的为无穷小, 而其余的方程的都为无穷 大. 事实上, 从方程 $r^{2}=\alpha s^{5}$ 得 $r=s^{2} \sqrt{\alpha s}$, 从而点 $M$ 处的密切半径为 $\frac{1}{2 \sqrt{\alpha s}}$, 因 $s=0$, 这密切 半径为无穷大.

\section{$\S 329$}

如果交于点 $M$ 的三个分支的切线重合, 那么在点 $M$, 或者三个一阶分支相切, 或者 一个二阶分支与一个一阶分支相切, 或者只有一个三阶分支通过. 表示三阶分支的方程 为 $r^{3}=\alpha s^{4}, r^{3}=\alpha s^{5}, r^{3}=\alpha s^{7}, r^{3}=\alpha s^{8}, \cdots$ 通用方程为 $r^{3}=\alpha s^{n}, n$ 为大于 3 但不被 3 整除的整 数. 这些分支的形状, $n$ 为奇数时, $M$ 为拐点; $n$ 为偶数时, 如图 62 所示, $M$ 不是拐点. 此 外, 这些曲线在点 $M$ 处的密切半径, $n<6$ 时为无穷小, $n>6$ 时为无穷大. 

\section{$\S 330$}
类似地,如果交于点 $M$ 的四个分支的切线相重合, 那么在点 $M$, 或者四个一阶分支, 或者两个一阶分支和一个二阶分支, 或者两个二阶分支, 或者一个一阶分支和一个三阶 分支相切. 最后过点 $M$ 的或者只是一个四阶分支. 四阶分支的通用方程为 $r^{4}=\alpha s^{n}, n$ 是大 于 4 的奇整数. 这些方程都像图 63 上二阶分支那样, 给出一个尖点. $M$ 处的密切半径, $n<8$ 时为无穷小, $n>8$ 时为无穷大.

\section{$\S 331$}

可以类似地对五阶和更高阶分支进行考察. 5 阶, 7 阶, 9 阶和一切奇阶分支, 其形状 都类似一阶分支, 或者有一个拐点, 或者没有拐点. 6 阶, 8 阶和一切偶阶分支, 其形状都 跟 2 阶和 4 阶分支一样, 如图 63 所示, 在 $M$ 处有一个尖点. 至于密切半径, 因为这些弧的 方程都为 $r^{m}=\alpha s^{n}$, 其中 $n>m$, 显然 $n<2 m$ 时为无穷小, $n>2 m$ 时为无穷大.

\section{$\S 332$}

这样我们可以把曲线分为三种类型. 一、有连续曲率, 既无拐点, 也无尖点. 首先, 密 切半径处处为有限值的曲线, 属此类型. 其次, 密切半径为无穷大或无穷小, 但不影响曲 率连续, 这样的曲线也属此类型. 例如, 方程 $\alpha r^{m}=s^{n}$ 表示的曲线就是, 这里 $m$ 为奇数, $n$ 为 大于 $m$ 的偶数. 二、有一个拐点, 密切半径或为无穷大, 或为无穷小, 方程为 $\alpha r^{m}=s^{n}, m, n$ 都是奇数, 且 $n>m$. 密切半径, $n>2 m$ 时为无穷大, $n<2 m$ 时为无穷小. 三、有一个尖点. 尖点处两分支凸面相向、会合、相切并终止. 方程为 $\alpha r^{m}=s^{n}$, 其中 $m$ 为偶数, $n$ 为奇数. 密 切半径或为无穷小, 或为无穷大.

\section{$\S 333$}

曲线连续延伸, 其可能的形状不外这三种类型. 首先, 连续曲线的一个分支不能构成 图 66 上点 $C$ 处的有限角 $\angle A C B$. 其次, 由于在尖点处两个分支凸面相向, 所以不能有图 67 样的尖点. 该尖点处, 分支 $A C$ 和 $B C$ 有公切线, 但 $A C$ 的凹面迎着 $B C$ 的凸面, 凹面迎 凸面表明曲线的这部分没画全. 如果根据方程把它画全了, 它的形状应该如图 64 所示. L'Hospital 称图 67 上那样的尖点为第二类尖点, 有方法画出具有这种尖点的曲线. 应该 指出, 一个方程所描述的曲线, 用工具所能画出的常常只是它的一部分, 而不是整个曲 线. 指出的这一点可避开关于第二类尖点的争论. 


【图,待补】
%%![](https://cdn.mathpix.com/cropped/2023_02_05_d70135c5d5f1afcc0634g-08.jpg?height=199&width=948&top_left_y=25&top_left_x=337)


【图,待补】
%%![](https://cdn.mathpix.com/cropped/2023_02_05_d70135c5d5f1afcc0634g-08.jpg?height=225&width=352&top_left_y=402&top_left_x=379)

图 66


【图,待补】
%%![](https://cdn.mathpix.com/cropped/2023_02_05_d70135c5d5f1afcc0634g-08.jpg?height=323&width=295&top_left_y=305&top_left_x=987)

图 67

尽管像是有理由说第二类尖点不存在, 但有第二类尖点的代数曲线, 我们可以举出 无穷多,其中甚至有四阶线,例如方程
\[
y^{4}-2 y^{2} x-4 y x^{2}-x^{3}=0
\]
表示的就是. 该方程产生于 $y=\sqrt{x} \pm \sqrt[4]{x^{3}}$. 虽然第一项为 $\sqrt{x}$, 但它的符号是确定的, 必须 为正, 否则第二项 $\sqrt[4]{x^{3}}=\sqrt{x \sqrt{x}}$ 为虚数. 这个例子启示我们, 应该对前面举过的一些例子 作认真的考察.

\section{$\S 334$}

如果两个分支于点 $M$ 处有公切线, 则如 $\S 327$ 图 64 所示, 从点 $M$ 伸出四段弧, 即 $M m, M_{\mu}, M n, M_{\nu}$. 两分支由不同的方程刻画时, 显然, 属同一方程的两段, 一段为另一段 的延伸; 两分支由同一方程刻画时, 弧 $M m$ 的延伸, 既可为 $\nu M$, 亦可为 $n M$. 由于弧 $M n$ 和 $M_{\nu}$ 的延伸都可以为 $M m$, 所以 $M n$ 与 $M_{\nu}$ 互为对方的延伸. 因而, 与任何另外两对弧一 样, $m M, M_{\mu}$ 也可视为一条连续曲线. 这时点 $M$ 处有两个第二类尖点, $m M_{\mu}$ 和 $n M_{\nu}$.

\section{$\S 335$}

前节所讨论的分支, 个数为 2 , 它们于点 $M$ 处相切, 无拐点, 不构成尖点, 且由同一方 程刻画. 这分支个数可以更多, 只要它们相切于点 $M$, 且由同一方程刻画, 它们中的每一 个就都可视为其他任何一个的延伸. 当 $r, s$ 间方程为
\[
\alpha^{2} \gamma^{2 m}-2 \alpha \beta \gamma^{m} s^{n}+\beta^{2} s^{2 n}=0
\]
时, 情况就是这样. 此时分支由 $\alpha \gamma^{m}=\beta s^{n}$ 表示, 从点 $M$ 出发的四段弧中, 任何两段都可视为 连续曲线. 这样我们就得到许多第二类尖点. 也即对延伸的讨论, 有时联系到第二类尖点, 但考虑的不是方程表示的整个曲线,而只是一个或几个分支. 

