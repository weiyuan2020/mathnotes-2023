\documentclass[openany,twoside,scheme=chinese,fontset=none]{ctexrep}
\usepackage{geometry}
\geometry{
    paperheight=260mm,
    paperwidth=185mm,
    top=25mm,
    bottom=15mm,
    left=25mm, % 左侧留 5mm 装订线距离
    right=15mm
}

\setmainfont{XITS}  % 英文字体, Times 风格

\setCJKmainfont{Source Han Serif SC}[         % 方正书宋_GBK
    BoldFont=Source Han Serif SC Bold,  % 思源宋体粗体
    ItalicFont=FZKai-Z03                % 方正楷体_GBK
    ]
\setCJKsansfont{Source Han Sans SC}[             % 方正黑体_GBK
    BoldFont=Source Han Sans SC Bold    % 思源黑体粗体
    ]
\setCJKmonofont{FZFangSong-Z02}         % 方正仿宋_GBK

\setCJKfamilyfont{zhsong}{FZShuSong-Z01}
\setCJKfamilyfont{zhxbs}{Source Han Serif SC Bold}
\setCJKfamilyfont{zhdbs}{Source Han Serif SC Heavy}
\setCJKfamilyfont{zhhei}{FZHei-B01}
\setCJKfamilyfont{zhdh}{Source Han Sans SC Bold}
\setCJKfamilyfont{zhfs}{FZFangSong-Z02}
\setCJKfamilyfont{zhkai}{FZKai-Z03}

\newcommand{\songti}{\CJKfamily{zhsong}}
\newcommand{\xbsong}{\CJKfamily{zhxbs}}
\newcommand{\dbsong}{\CJKfamily{zhdbs}}
\newcommand{\heiti}{\CJKfamily{zhhei}}
\newcommand{\dahei}{\CJKfamily{zhdh}}
\newcommand{\fangsong}{\CJKfamily{zhfs}}
\newcommand{\kaishu}{\CJKfamily{zhkai}}


\usepackage{amsmath}
\usepackage{amsthm} % amsthm 与 ntheorem 冲突
\usepackage{amssymb}
\usepackage{hyperref} % \url
\usepackage{graphicx} % \includegraphics


\usepackage{enumerate} % 罗列专用宏包
\usepackage{fancybox} % 使用盒子
 


\usepackage{xcolor} % 使用颜色 color
\usepackage{booktabs} % 三线表宏包 toprule midrule bottomrule

% \usepackage{ntheorem} % 调整定理格式
% \theorembodyfont{\upshape} %正体
% \theorembodyfont{\itshape} %斜体
% 定理类环境默认的字体为斜体,可以通过 \theorembodyfont 来设置字体,其中 \upshape 为正体,\itshape 为斜体。\theorembodyfont 必须放在 newthorem 的上方。



\theoremstyle{plain} % default
\newtheorem{thm}{Theorem}[chapter] % 如果不采用章节号做前缀, 则不用[section]
\newtheorem{myLemma}[thm]{Lemma}% 这句定义使得 lem 环境和 thm 共享编号

\theoremstyle{definition} % definition
\newtheorem{myDefinition}[thm]{Definition} % 这句定义使得 defn 环境和 thm 共享编号
\newtheorem{myExample}[thm]{Example}% 这句定义使得 example 环境和thm 共享编号
\newtheorem{myRemark}[thm]{Remark}
\newtheorem{myProposition}[thm]{Proposition}
\newtheorem{myNotation}[thm]{Notation}
\newtheorem{exercise}{练习}
\newtheorem{answer}{题目解答}
\newtheorem{remark}{Remark}
\newtheorem{case}{\normalfont\bfseries 案例} %
\newtheorem{solve}{解}
\newtheorem{Example}{例}
\newtheorem{example}{例}
\newtheorem{theorem}{定理}
\newtheorem{definition}{定义}




% 我的笔记环境
\newcommand{\mybox}[1]{
    % \fbox{mynotes:\\#1}
    \vskip 2.5mm
    \fbox{\parbox{120mm}{mynotes:\\#1}}
    \vskip 2.5mm
}



\usepackage{mathtools} % xrightarrow

\usepackage{tikz}       % tikz
% \usepackage{colortbl}   % tikznode color 块上色



% falling factorial power
\newcommand{\fallingfactorial}[1]{%
  ^{\underline{#1}}%
}
% rising factorial power
\newcommand{\risingfactorial}[1]{%
  ^{\overline{#1}}%
}
% tikz pic draw
% \newcommand\y{\cellcolor{clight2}}
\definecolor{clight2}{RGB}{212, 237, 244}%
\newcommand\tikznode[3][]%
{\tikz[remember picture,baseline=(#2.base)]
	\node[minimum size=0pt,inner sep=0pt,#1](#2){#3};%
}
\tikzset{>=stealth}

\title{具体数学阅读笔记-chap1 repertoire method 参考}
\author{weiyuan}
\date{2022-07-03}
\begin{document}
\maketitle
\section{Solve}
\begin{equation*}
    \left\{
        \begin{array}{ll}
            r_0 &= 1\\
            r_n &= r_{n-1}+3n+5\\ 
        \end{array}
    \right.
\end{equation*}

First, get some cases
\begin{align*}
    r_0 &= 1 & \\
    r_1 &= 1+3\times 1 + 5 &= 9  \\
    r_2 &= 9+3\times 2 + 5 &= 20 \\
    r_3 &= 20+3\times 3 + 5 &= 34 
\end{align*}

Unsimplified cases
\begin{align*}
    r_0 &= 1 & \\
    r_1 &= r_0+3\times 1 + 5 &= 9  \\
    r_2 &= r_1+3\times 2 + 5 &= 20 \\
    r_3 &= r_2+3\times 3 + 5 &= 34 
\end{align*}

A pattern in unsimplified cases
\begin{equation*}
    r_n = 1 A(n) + 3 B(n) +5 C(n)
\end{equation*}
where $ A(n), B(n), C(n) $ are simple functions of n
\begin{equation*}
    \left\{
        \begin{array}{ll}
            A(n) &= 1\\
            B(n) &= \frac{n(n+1)}{2}\\
            C(n) &= n\\
        \end{array}
    \right.
\end{equation*}
\begin{align*}
    r_n &= 1 \times 3 \times \frac{n(n+1)}{2} +5 \times n\\
        &= \frac{3}{2}n^2 +\frac{13}{2}+1
\end{align*}

Summarizing 
\begin{equation*}
    \left\{
        \begin{array}{ll}
            r_0 &= 1\\
            r_n &= r_{n-1}+3n+5\\
        \end{array}
    \right.
\end{equation*}
is $ r_n=\frac{3}{2}n^2+\frac{13}{2}n+1 $ .

Testing
\begin{table}[htbp]
	\centering
	\small
	\caption{r(n) 与 n 之间的关系}
	\begin{tabular}{c|ccc ccc}
		\toprule
		n & 0 & 1 & 2 & 3 & 4 & 5 \\  
		\midrule
		$ r_n $ & 1 & 9 & 20 & 34 & 51 & 71\\
        $ \frac{3}{2}n^2+\frac{13}{2}n+1 $ & 1 & 9 & 20 & 34 & 51 & 71\\
		\bottomrule
	\end{tabular}%
	\label{tab:rnCompare}%
\end{table}%

Prove it by induction.\\
First we generalize:
\begin{equation*}
    \left\{
        \begin{array}{ll}
            r_0 &= 1\\
            r_n &= r_{n-1}+3n+5\\
        \end{array}
    \right.
\end{equation*}
replace constants by variables $ \alpha, \beta,\gamma $ 
\begin{equation*}
    \left\{
        \begin{array}{ll}
            r_0 &= \alpha\\
            r_n &= r_{n-1}+\beta n+\gamma\\
        \end{array}
    \right.
\end{equation*}
Cases of our generalized version
\begin{table}[htbp]
	\centering
	\small
	\caption{Cases of our generalized version}
	\begin{tabular}{cl}
		\toprule
		n & $ r_n $ \\  
		\midrule
        0   & $ \alpha $     \\
        1   & $ \alpha + \beta + \gamma $    \\
        2   & $ \alpha + \beta + \gamma + 2\times \beta + \gamma  $   \\
            & $ = \alpha + 3\beta + 2\gamma $ \\
        3   & $ \alpha + 3\beta + 2\gamma + 3\times \beta + \gamma $ \\
            & $ = \alpha + 6\beta + 3\gamma $   \\
        4   & $ \alpha + 10\beta + 4\gamma $ \\
		\bottomrule
	\end{tabular}%
	\label{tab:casesofgene}%
\end{table}%

Wild assumption: \\
Let's assume that there are three-fixed
functions $ A, B, C $  such that the solution to 
the above always has this form:
\begin{equation*}
    r_n = \alpha A(n) + \beta B(n) + \gamma C(n)
\end{equation*}

We don't know this is true but the evidence suggests it

Can we figure out what A, B, and C are? Yes!

Is this easier tha the original problem? Yes!

Here's How\\
We assume that any recurrence defined by:
\begin{equation*}
    \left\{
        \begin{array}{ll}
            r_0 &= \alpha\\
            r_n &= r_{n-1}+\beta n+\gamma\\
        \end{array}
    \right.
\end{equation*}
has a soluntion that looks like:
\begin{equation*}
    r_n = \alpha A(n) + \beta B(n) + \gamma C(n)
\end{equation*}
no matter what $ \alpha, \beta $ and $ \gamma $ are.

Different $ \alpha, \beta $ and $ \gamma $ will define Different $ r_n $. But $ A(n), B(n) $ and $ C(n) $ are the same of all of them!

What does this buy us?

For any $ \alpha, \beta $ and $ \gamma $ , the equations
\begin{equation*}
    \left\{
        \begin{array}{ll}
            r_0 &= \alpha\\
            r_n &= r_{n-1}+\beta n+\gamma\\
        \end{array}
    \right.
\end{equation*}
are always solved by
\begin{equation*}
    r_n = \alpha A(n) + \beta B(n) + \gamma C(n)
\end{equation*}

If we pick up really simple functions (with really easy values for  $ \alpha, \beta $ and $ \gamma $) we can solve for A, B and C.

And once we have A, B,and C, we have a solution to the general recurrence.


\section{Easy Solutions}
\begin{equation*}
    \left\{
        \begin{array}{ll}
            r_0 &= \alpha\\
            r_n &= r_{n-1}+\beta n+\gamma\\
            r_n &= \alpha A(n) + \beta B(n) + \gamma C(n)\\
        \end{array}
    \right.
\end{equation*}

First easy solution.\\
Let's try $ r_n = 1 $ 
\begin{equation*}
    \left\{
        \begin{array}{ll}
            1 &= \alpha\\
            1 &= 1+\beta n+\gamma\\
            1 &= \alpha A(n) + \beta B(n) + \gamma C(n)\\
        \end{array}
    \right.
\end{equation*}
\begin{equation*}
    \left\{
        \begin{array}{l}
            \alpha = 1\\
            \beta = 0\\
            \gamma = 0\\
        \end{array}
    \right.
    \rightarrow A(n) = 1
\end{equation*}
$ r_n=1 $ has consequences
\begin{equation*}
    \left\{
        \begin{array}{ll}
            r_0 &= \alpha\\
            r_n &= r_{n-1}+\beta n+\gamma\\
            r_n &= \alpha A(n) + \beta B(n) + \gamma C(n)\\
        \end{array}
    \right.
    \rightarrow
    \left\{
        \begin{array}{ll}
            r_0 &= \alpha\\
            r_n &= r_{n-1}+\beta n+\gamma\\
            r_n &= \alpha + \beta B(n) + \gamma C(n)\\
        \end{array}
    \right.
\end{equation*}

Let's try $ r_n = n $ 
\begin{equation*}
    \left\{
        \begin{array}{ll}
            0 &= \alpha\\
            n &= n-1+\beta n+\gamma\\
            n &= \alpha + \beta B(n) + \gamma C(n)\\
        \end{array}
    \right.
\end{equation*}
\begin{equation*}
    \left\{
        \begin{array}{l}
            \alpha = 0\\
            \beta = 0\\
            \gamma = 1\\
        \end{array}
    \right.
    \rightarrow C(n) = n
\end{equation*}
$ r_n=1 $ has consequences
\begin{equation*}
    \left\{
        \begin{array}{ll}
            r_0 &= \alpha\\
            r_n &= r_{n-1}+\beta n+\gamma\\
            r_n &= \alpha + \beta B(n) + \gamma C(n)\\
        \end{array}
    \right.
    \rightarrow
    \left\{
        \begin{array}{ll}
            r_0 &= \alpha\\
            r_n &= r_{n-1}+\beta n+\gamma\\
            r_n &= \alpha + \beta B(n) + \gamma n\\
        \end{array}
    \right.
\end{equation*}

Let's try $ r_n = n^2 $ 
\begin{equation*}
    \left\{
        \begin{array}{ll}
            0 &= \alpha \\
            n^2 &= (n-1)^2+\beta n+\gamma \\
            n^2 &= \alpha + \beta B(n) + \gamma n \\
        \end{array}
    \right.
\end{equation*}
\begin{equation*}
    \left\{
        \begin{array}{l}
            \alpha = 0\\
            \beta = 2\\
            \gamma = -1\\
        \end{array}
    \right.
    \rightarrow B(n) = \frac{n(n+1)}{2}
\end{equation*}
$ r_n=1 $ has consequences
\begin{equation*}
    \left\{
        \begin{array}{ll}
            r_0 &= \alpha\\
            r_n &= r_{n-1}+\beta n+\gamma\\
            r_n &= \alpha + \beta B(n) + \gamma n\\
        \end{array}
    \right.
    \rightarrow
    \left\{
        \begin{array}{ll}
            r_0 &= \alpha\\
            r_n &= r_{n-1}+\beta n+\gamma\\
            r_n &= \alpha + \beta \frac{n(n+1)}{2} + \gamma n\\
        \end{array}
    \right.
\end{equation*}

Let's try it out
\begin{equation*}
    \left\{
        \begin{array}{ll}
            r_0 &= \alpha\\
            r_n &= r_{n-1}+\beta n+\gamma\\
            r_n &= \alpha + \beta \frac{n(n+1)}{2} + \gamma n\\
        \end{array}
    \right.
\end{equation*}

Testing
\begin{equation*}
    \left\{
        \begin{array}{ll}
            r_0 &= 1\\
            r_n &= r_{n-1}+3 n+5\\
        \end{array}
    \right.
\end{equation*}
$ \alpha = 1, \beta = 3, \gamma = 5 $ 
\begin{equation*}
    r_n = 1+3 \frac{n^2+n}{2}5n = \frac{3}{2}n^2+\frac{13}{2}n+1
\end{equation*}

Summations
Recurrence like these $ \left\{
    \begin{array}{ll}
        r_0 &= \alpha \\
        r_n &= r_{n-1}+\beta n+\gamma \\
    \end{array}
\right. $ \\
with solution: $ r_n = \alpha + \beta \frac{n^2+n}{2}+ \gamma n $. \\
Can be used to solve summations like these 
\begin{equation*}
    S_n = \sum_{i=0}^{n}(3i+2)
\end{equation*}
\begin{equation*}
    \left\{
        \begin{array}{ll}
            S_0 &= 2 \\
            S_n &= S_{n-1}+3n+2 \\
        \end{array}
    \right.
\end{equation*}
\begin{equation*}
    \left\{
        \begin{array}{l}
            \alpha = 2\\
            \beta=3\\
            \gamma=2\\
        \end{array}
    \right.
\end{equation*}
\begin{align*}
    S_n &= 2+3\frac{n^2+n}{2}+2n\\
        &= \frac{3}{2}n^2+\frac{7}{2}n+2
\end{align*}

\section{Let's try something harder.}

\begin{equation*}
    \left\{
        \begin{array}{l}
            r_0 = 1\\
            r_n = 2r_{n-1}+n\\
        \end{array}
    \right.
\end{equation*}

Case 
\begin{table}[htbp]
	\centering
	\small
	\caption{harder cases}
	\begin{tabular}{c|cccccc}
		\toprule
		n & 0 & 1 & 2 & 3 & 4 & 5 \\
		\midrule
        $ r_n $  & 1 & 3 & 8 & 19 & 42 & 89 \\
		\bottomrule
	\end{tabular}%
	\label{tab:casesofharder}%
\end{table}%

First generalize
\begin{equation*}
    \left\{
        \begin{array}{l}
            r_0 = 1\\
            r_n = 2r_{n-1}+n\\
        \end{array}
    \right.
\end{equation*}
\begin{equation*}
    \left\{
        \begin{array}{l}
            r_0 = \alpha\\
            r_n = \beta r_{n-1}+\gamma \\
        \end{array}
    \right.
\end{equation*}

\begin{table}[htbp]
	\centering
	\small
	\caption{generalize the harder cases}
	\begin{tabular}{cl}
		\toprule
		n & $ r_n $ \\
		\midrule
        0 & $ \alpha $  \\
        1 & $ \beta\alpha+\gamma $  \\
        2 & $ \beta(\beta\alpha+\gamma)+2\gamma $  \\
        3 & $ \beta(\beta(\beta\alpha+\gamma)+2\gamma)+3\gamma $  \\
		\bottomrule
	\end{tabular}%
	\label{tab:casesofharderChange003}%
\end{table}%
In this case, $ \alpha, \beta, \gamma $ mixed up
\begin{equation*}
    \left\{
        \begin{array}{l}
            r_0 = 1\\
            r_n = 2r_{n-1}+n \\
        \end{array}
    \right. \rightarrow
    \left\{
        \begin{array}{l}
            r_0 = \alpha \\
            r_n = 2 r_{n-1} + \beta n + \gamma \\
        \end{array}
    \right.
\end{equation*}
\begin{table}[htbp]
	\centering
	\small
	\caption{generalize the harder cases(change)}
	\begin{tabular}{cl}
		\toprule
		n & $ r_n $ \\
		\midrule
        0 & $ \alpha $  \\
        1 & $ 2\alpha+\beta+\gamma $  \\
        2 & $ 2(2\alpha+\beta+\gamma)+2\beta+\gamma $  \\
          & $ 4\alpha + 4\beta + 3\gamma $ \\
        3 & $ 2(4\alpha + 4\beta + 3\gamma)+3\beta+\gamma $  \\
          & $ 8\alpha + 11\beta + 7\gamma $ \\
		\bottomrule
	\end{tabular}%
	\label{tab:casesofharderChange}%
\end{table}%
没有$ \gamma $ 的情况
\begin{equation*}
    \left\{
        \begin{array}{l}
            r_0 = \alpha\\
            r_n = 2r_{n-1}+\beta n \\
        \end{array}
    \right.
\end{equation*}
\begin{table}[htbp]
	\centering
	\small
	\caption{generalize the harder cases($ \gamma = 0 $ )}
	\begin{tabular}{cl}
		\toprule
		n & $ r_n $ \\
		\midrule
        0 & $ \alpha $  \\
        1 & $ 2\alpha+\beta $  \\
        2 & $ 2(2\alpha+\beta)+2\beta $  \\
          & $ 4\alpha + 4\beta $ \\
        3 & $ 2(4\alpha + 4\beta)+3\beta $  \\
          & $ 8\alpha + 11\beta $ \\
		\bottomrule
	\end{tabular}%
	\label{tab:casesofharderChange002}%
\end{table}%
\begin{equation*}
    r_n = \alpha A(n)+\beta B(n)
\end{equation*}
$ r_n=1 $ , $ \left\{\begin{array}{l}
    1=\alpha\\
    1=2\cdot 1+\beta n\\
\end{array}\right. $ 
这是不可能的。因此$ \gamma \neq 0 $ 

$ r_n=1 $ 
\begin{equation*}
    \left\{
        \begin{array}{l}
            1=\alpha\\
            1=2\cdot 1+\beta n + \gamma\\
        \end{array}
    \right.
\end{equation*}
$ (\alpha, \beta, \gamma) = (1,0,-1) $ 
\begin{equation*}
    A(n)-C(n)=1
\end{equation*}
\begin{equation*}
    C(n)=A(n)-1    
\end{equation*}

$ r_n = n $ 
\begin{equation*}
    \left\{
        \begin{array}{l}
            0=\alpha\\
            n=2(n-1) + \beta n + \gamma\\
        \end{array}
    \right.
\end{equation*}
$ (\alpha, \beta, \gamma) = (0,-1,2) $ 
\begin{equation*}
    -B(n)+2C(n)=n
\end{equation*}
\begin{equation*}
    B(n)=2C(n)-n = 2A(n)-n-2
\end{equation*}

$ r_n = n^2 $ 不能推出有效信息, $ n^2 = 2(n-1)^2+\beta n + \gamma $ , 推不出合理的解.

$ r_n = 2^n $ 
\begin{equation*}
    \left\{
        \begin{array}{l}
            2^0=\alpha\\
            2^n=2\times 2^{n-1} + \beta n + \gamma\\
        \end{array}
    \right.
\end{equation*}
$ (\alpha, \beta, \gamma) = (1,0,0) $ 
\begin{equation*}
    A(n)=2^n
\end{equation*}
\begin{equation*}
    \left\{
        \begin{array}{ll}
            A(n) &= 2^n \\
            B(n) &= 2^{n+1}-2-n \\
            C(n) &= 2^n-1 \\
        \end{array}
    \right.
\end{equation*}
\begin{equation*}
    r_n = \alpha 2^n + \beta(2^{n+1}-n-2)+\gamma(2^n-1)
\end{equation*}

Example: 
$ (\alpha,\beta,\gamma)=(1,1,0) $ 
\begin{equation*}
    r_n = 1\times 2^n + 1\times (2^{n+1}-n-2) = 3\cdot 2^n-n-2
\end{equation*}

Reprise
\begin{equation*}
    \left\{
        \begin{array}{l}
            r_0 = 1\\
            r_n=2r_{n-1}+n
        \end{array}
    \right. \rightarrow(\text{general})
    \left\{
        \begin{array}{l}
            r_0 = \alpha\\
            r_n = 2r_{n-1}+\beta n+ \gamma\\
        \end{array}
    \right.
\end{equation*}
\begin{equation*}
    r_n = \alpha 2^n + \beta(2^{n+1}-n-2)+\gamma(2^n-1)
\end{equation*}
\begin{equation*}
    \left\{
        \begin{array}{llll}
            1   &= A(n)     &       &-C(n)  \\
            n   &=          &-B(n)  &+2C(n) \\
            2^n &= A(n)     &       &       \\
        \end{array}
    \right.
\end{equation*}
\footnote{这份 ppt 非常详细, 
我以后总结知识也应如此.
将来回顾可以快速看懂.}
\end{document}