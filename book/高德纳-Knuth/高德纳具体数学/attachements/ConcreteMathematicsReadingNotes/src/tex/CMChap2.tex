\documentclass[openany,twoside,scheme=chinese,fontset=none]{ctexrep}
\usepackage{geometry}
\geometry{
    paperheight=260mm,
    paperwidth=185mm,
    top=25mm,
    bottom=15mm,
    left=25mm, % 左侧留 5mm 装订线距离
    right=15mm
}

\setmainfont{XITS}  % 英文字体, Times 风格

\setCJKmainfont{Source Han Serif SC}[         % 方正书宋_GBK
    BoldFont=Source Han Serif SC Bold,  % 思源宋体粗体
    ItalicFont=FZKai-Z03                % 方正楷体_GBK
    ]
\setCJKsansfont{Source Han Sans SC}[             % 方正黑体_GBK
    BoldFont=Source Han Sans SC Bold    % 思源黑体粗体
    ]
\setCJKmonofont{FZFangSong-Z02}         % 方正仿宋_GBK

\setCJKfamilyfont{zhsong}{FZShuSong-Z01}
\setCJKfamilyfont{zhxbs}{Source Han Serif SC Bold}
\setCJKfamilyfont{zhdbs}{Source Han Serif SC Heavy}
\setCJKfamilyfont{zhhei}{FZHei-B01}
\setCJKfamilyfont{zhdh}{Source Han Sans SC Bold}
\setCJKfamilyfont{zhfs}{FZFangSong-Z02}
\setCJKfamilyfont{zhkai}{FZKai-Z03}

\newcommand{\songti}{\CJKfamily{zhsong}}
\newcommand{\xbsong}{\CJKfamily{zhxbs}}
\newcommand{\dbsong}{\CJKfamily{zhdbs}}
\newcommand{\heiti}{\CJKfamily{zhhei}}
\newcommand{\dahei}{\CJKfamily{zhdh}}
\newcommand{\fangsong}{\CJKfamily{zhfs}}
\newcommand{\kaishu}{\CJKfamily{zhkai}}


\usepackage{amsmath}
\usepackage{amsthm} % amsthm 与 ntheorem 冲突
\usepackage{amssymb}
\usepackage{hyperref} % \url
\usepackage{graphicx} % \includegraphics


\usepackage{enumerate} % 罗列专用宏包
\usepackage{fancybox} % 使用盒子
 


\usepackage{xcolor} % 使用颜色 color
\usepackage{booktabs} % 三线表宏包 toprule midrule bottomrule

% \usepackage{ntheorem} % 调整定理格式
% \theorembodyfont{\upshape} %正体
% \theorembodyfont{\itshape} %斜体
% 定理类环境默认的字体为斜体,可以通过 \theorembodyfont 来设置字体,其中 \upshape 为正体,\itshape 为斜体。\theorembodyfont 必须放在 newthorem 的上方。



\theoremstyle{plain} % default
\newtheorem{thm}{Theorem}[chapter] % 如果不采用章节号做前缀, 则不用[section]
\newtheorem{myLemma}[thm]{Lemma}% 这句定义使得 lem 环境和 thm 共享编号

\theoremstyle{definition} % definition
\newtheorem{myDefinition}[thm]{Definition} % 这句定义使得 defn 环境和 thm 共享编号
\newtheorem{myExample}[thm]{Example}% 这句定义使得 example 环境和thm 共享编号
\newtheorem{myRemark}[thm]{Remark}
\newtheorem{myProposition}[thm]{Proposition}
\newtheorem{myNotation}[thm]{Notation}
\newtheorem{exercise}{练习}
\newtheorem{answer}{题目解答}
\newtheorem{remark}{Remark}
\newtheorem{case}{\normalfont\bfseries 案例} %
\newtheorem{solve}{解}
\newtheorem{Example}{例}
\newtheorem{example}{例}
\newtheorem{theorem}{定理}
\newtheorem{definition}{定义}




% 我的笔记环境
\newcommand{\mybox}[1]{
    % \fbox{mynotes:\\#1}
    \vskip 2.5mm
    \fbox{\parbox{120mm}{mynotes:\\#1}}
    \vskip 2.5mm
}



\usepackage{mathtools} % xrightarrow

\usepackage{tikz}       % tikz
% \usepackage{colortbl}   % tikznode color 块上色



% falling factorial power
\newcommand{\fallingfactorial}[1]{%
  ^{\underline{#1}}%
}
% rising factorial power
\newcommand{\risingfactorial}[1]{%
  ^{\overline{#1}}%
}
% tikz pic draw
% \newcommand\y{\cellcolor{clight2}}
\definecolor{clight2}{RGB}{212, 237, 244}%
\newcommand\tikznode[3][]%
{\tikz[remember picture,baseline=(#2.base)]
	\node[minimum size=0pt,inner sep=0pt,#1](#2){#3};%
}
\tikzset{>=stealth}

\title{具体数学阅读笔记-chap2}
\author{weiyuan}
\date{2022-07-04}
\begin{document}
\maketitle
\section{求和}
\subsection{求和符号}
\begin{equation}
    \sum_{k=1}^{n}a_k = \sum_{1\leqslant k\leqslant n}a_k
\end{equation}
\begin{align*}
    \sum_{1\leqslant k<100}k^2
    &= 1^2+3^2+\dots +99^2\\
    &= 1^2+2^2+\dots +100^2-(2^2+4^2+\dots 100^2)\\
    &= 1^2+2^2+\dots +100^2-4\times(1^2+2^2+\dots 50^2)\\
    &= \frac{100\times 101\times 201}{6}-4 \frac{50\times 51\times 101}{6}\\
    &= \frac{100\times 101}{6}(201-102)\\
    &= \frac{99\times 100\times 101}{6}\\
    &= 166650
\end{align*}
⼀般形式 ⽐有确定界限形式 更好处理

$ k\rightarrow k+1$ \\
⼀般形式, 
\begin{equation}
    \sum_{1\leqslant k\leqslant n}a_k =
    \sum_{1\leqslant k+1\leqslant n}a_{k+1} = 
    \sum_{0\leqslant k\leqslant n-1}a_{k+1}
\end{equation}
有确定界限形式
\begin{equation}
    \sum_{k=1}^{n}a_k = \sum_{k=0}^{n-1}a_{k-1}
\end{equation}
{\color{red}不要轻易省略 0项
计算有效性不等同于理解的有效性}
\begin{equation*}
    \sum_{k=0}^{n}k(k-1)(n-k)\quad
    \sum_{k=2}^{n-1}k(k-1)(n-k)
\end{equation*}
保持求和界限简单大有裨益

\begin{equation}
    \sum_{\begin{array}{l}p\leqslant N \\ p\text{是素数}\\ \end{array}}\frac{1}{p} = \sum_{p}[p\text{是素数}][p\leqslant N]/p
\end{equation}

$ [p(k)] $ 约定$ \rightarrow p(k)$ 为假 $ \left\{\begin{array}{l}
    [p(k)]\text{必定是}0 \\
    a_k[p(k)],a_k\text{无意义},\; a_k[p(k)]\text{仍是}0
\end{array}\right. $ 

由约定$ \frac{[0\text{是素数}][0\leqslant N]}{0} = 0 $ 

\subsection{求和与递归式 Sums and recurrences}
和式
\begin{equation}
    S_n = \sum_{k=0}^{n}a_k
\end{equation}
等价于递归式

封闭形式解递归式的值 $ \Rightarrow $ 封闭形式解和式
\begin{equation}
    \left\{
        \begin{array}{l}
            S_0 = a_0 \\
            S_n = S_{n-1}+a_n,\quad n>0.
        \end{array}
    \right.
\end{equation}
若$ a_n = const.+ k\cdot n $ , 则有
\begin{equation}\label{ReceEasy}
    \left\{
    \begin{aligned}
        R_0 &= \alpha \\
        R_n &= R_{n-1} + \beta + \gamma n, \quad n>0
    \end{aligned}
    \right.
\end{equation}
\begin{align*}
    R_1 &= R_0 + \beta + \gamma \\
    R_2 &= R_0 + 2\beta + 3\gamma \\
        &\vdots \\
    R_n &= A(n)\alpha + B(n)\beta + C(n)\gamma
\end{align*}
\begin{equation}\label{RecuEasyABC}
    R_n = A(n)\alpha + B(n)\beta + C(n)\gamma
\end{equation}
repertoire method
令$ R_n=1 $ , 则$ \alpha=1, \beta=0, \gamma=0 $ , 
\begin{equation*}
    A(n)=1
\end{equation*}
令$ R_n=n $ , 则$ \alpha=0, \beta=1, \gamma=0 $ , 
\begin{equation*}
    B(n)=n
\end{equation*}
令$ R_n=n^2 $ , 则$ \alpha=0, \beta=-1, \gamma=2 $ , 
\begin{equation*}
    C(n)= \frac{n(n+1)}{2}
\end{equation*}

\begin{example}
    \begin{equation*}
        \sum_{k=0}^{n}(a+b k)
    \end{equation*}
\end{example}
\begin{solve}
    \begin{equation*}
        \left\{
            \begin{array}{l}
                R_0 = a \\
                R_n = R_{n-1} + a+bn\\
            \end{array}
        \right.
    \end{equation*}
    \begin{equation*}
        \left\{
            \begin{array}{l}
                R_0 = \alpha \\
                R_n = R_{n-1} + \beta + \gamma n\\
            \end{array}
        \right.
    \end{equation*}
    $ \alpha = \beta =a, \gamma = b $ 
    \begin{align*}
        A(n)\alpha + B(n)\beta + C(n)\gamma 
        &= a A(n) + a B(n) + b C(n)\\
        &= a + a n + b \frac{n(n+1)}{2}\\
        &= a(n+1)+\frac{bn(n+1)}{2}
    \end{align*}
\end{solve}

对上述递归情况进行推广
\begin{equation}\label{RecuHard}
    \left\{
    \begin{aligned}
        R_0 &= \alpha \\
        R_n &= R_{n-1} + \beta + \gamma n + \delta n^2, \quad n>0
    \end{aligned}
    \right.
\end{equation}
\begin{equation}\label{RecuHardABC}
    R_n = A(n)\alpha + B(n)\beta + C(n)\gamma +D(n) \delta
\end{equation}
$ \delta=0 $ 时 (\ref{RecuHard})与(\ref{ReceEasy})一致, 说明
$ A(n), B(n), C(n) $ 不变\\
$ R_n = n^3 $ 
\begin{align*}
    R_n = R_{n-1} &= n^3 - (n-1)^3 \\
    &= 3n^2 - 3n + 1
\end{align*}
解得$ \alpha=0, \beta=1, \gamma=-3, \delta=3 $ 
\begin{align*}
    n^3 &= B(n)-3C(n)+3D(n)\\
    &= n-3 \frac{n(n+1)}{2}+3D(n)
\end{align*}
\begin{align*}
    3D(n) &= n^3 - n + 3 \frac{n(n+1)}{2} \\
    &= n(n+1) \left[(n-1)+\frac{3}{2}\right] \\
    &= n(n+1)(n+\frac{1}{2})\\
    D(n) &= \frac{1}{3}\left((n+1)(n+\frac{1}{2})n\right)
\end{align*}

1. 和式用递归式求解\\
2. 递归式用和式求解\\
\begin{equation*}
    \left\{
        \begin{array}{lll}
            T_0 &= 0 & \\
            T_n &= 2 T_{n-1}+1, & n>0 \\
fallings        \end{array}
    \right.
\end{equation*}
\begin{equation*}
    \left\{
        \begin{array}{lll}
            \frac{T_0}{2^0} &= 0 & \\
            \frac{T_n}{2^n} &= \frac{T_{n-1}}{2^{n-1}}+\frac{1}{2^n}, & n>0 \\
        \end{array}
    \right.
\end{equation*}
记$ S_n = \frac{T_n}{2^n} $ 

{\color{red}求和因子} $ s_n $ 
\begin{align*}
    a_n T_n     &= b_n T_{n-1}+c_n \\
    s_n a_n T_n &= s_n b_n T_{n-1}+s_n c_n
\end{align*}
选取求和因子使得 $ \quad s_n b_n = s_{n-1} a_{n-1}  $ 
这样一来,如果记 $ S_n = S_{n-1}+s_n c_n $ ,我们就得到一个和式-递归式
\begin{equation*}
    S_n = S_{n-1}+s_n c_n
\end{equation*}
\begin{equation*}
    S_n = s_0 a_0 T_0 +\sum_{k=1}^{n} s_k c_k = s_1b_1T_0 +\sum_{k=1}^{n} s_k c_k 
\end{equation*}
\begin{equation}
    T_n = \frac{1}{s_n a_n}\left(s_0 a_0 T_0+\sum_{k=1}^{n}s_k c_k\right)
\end{equation}
例如,当 $ n=1 $ 时得到 $ T_1 = \cfrac{(s_1 b_1 T_0+s_1c_1)}{s_1a_1} = \cfrac{(b_1 T_0+c_1)}{a_1} $ \footnote{$ s_1 $ 的值消去了,所以它可以是除零以外的任何数.)}

但是,我们怎样才能有足够的智慧求出正确的 $ s_n $ 呢?没有问题:关系
式$ s_n=\cfrac{s_{n-1}a_{-1}}{b_n} $  可以被展开,从而我们发现,分式
\begin{equation}
    s_n = \frac{a_{n-1}a_{n-2}\dots a_1}{b_{n}b_{n-1}\dots b_2}
\end{equation}
或者这个值的任何适当的常数倍,会是一个合适的求和因子.例如,
河内塔递归式有 $ a_n=1 $ 和 $ b_n = 2 $ ,由刚刚推导出来的一般方法可知,如
果要把递归式转化为和式,那么 $ s_n = 2^{-n} $ 就是一个用来相乘的好东西.
发现这个乘数并不需要闪光的思想灵感.

我们必须小心谨慎,永远不用0做除数.只要所有的 a 和所有的 b 都不为
零,那么求和因子方法就能奏效.

我们来把这些想法应用到“快速排序”研究中所出现的递归式,快速
排序是计算机内部数据排序的一种最重要的方法.当把它应用到有 n 个随机排列的项目时,用典型的快速排序方法所做的比较步骤的平均
次数满足递归式
\begin{equation}
    \begin{array}{l}
        C_0 = C_1 = 0;
        C_n = n+1+\frac{2}{n}\sum_{k=0}^{n-1}C_k, \quad n>1.
    \end{array}
\end{equation}

$ C_2 = 3 $ , $ C_3 = 6 $ , $ C_4 = \frac{19}{2} $ 

\begin{equation*}
    C_n = n+1+\frac{2}{n}\sum_{k=0}^{n-1}C_k, \quad n>1.
\end{equation*}
消去分母中的$ n $ 
\begin{equation}\label{nCn}
    nC_n = n(n+1)+2\sum_{k=0}^{n-1}C_k, \quad n>1.
\end{equation}
$ n\rightarrow (n-1) $ 
\begin{equation}\label{n-1Cn}
    (n-1)C_{n-1} = (n-1)n+2\sum_{k=0}^{n-2}C_k, \quad n-1>1.
\end{equation}
% (\ref{nCn})-(\ref{n-1Cn})可以消去和式
\begin{equation*}
    nC_n - (n-1)C_{n-1} = 2n+2C_{n-1}, \quad n>2.
\end{equation*}
递归式化为
\begin{equation}
    \begin{array}{l}
        C_0 = C_1 = 0;C_2 = 3
        nC_n = (n+1)C_{n-1} + 2n, \quad n>2.
    \end{array}
\end{equation}

使用求和因子方法处理 $ a_n = n, b_n = n+1 $ 且
\begin{equation*}
    {\color{red}c_n = 2n-2[n=1]+2[n=2].}
\end{equation*}
故而我们现在能用求和因子方法.前面描述的一般方法告诉我们,要
用
\begin{equation*}
    s_n = \frac{a_{n-1}a_{n-2}\dots a_1}{b_{n}b_{n-1}\dots b_2} = \frac{(n-1)\times (n-2)\dots \times 1}{(n+1)\times n\times \dots \times 3} = \frac{2}{(n+1)n}
\end{equation*}
的某个倍数来遍乘该递归式
\begin{equation*}
    T_n = \frac{1}{s_n a_n}\left(s_0 a_0 T_0+\sum_{k=1}^{n}s_k c_k\right)
\end{equation*}
\begin{align*}
    T_n 
    &= \frac{n+1}{2}\left(0+\sum_{k=1}^{n}s_k c_k\right)\\
    &= \frac{n+1}{2}\left(\sum_{k=1}^{n}\frac{2}{k(k+1)}2k\right)\\
    &= 2(n+1)\sum_{k=1}^{n}\frac{1}{k+1}\\
    &= 2(n+1)\sum_{k=1}^{n}\frac{1}{k+1}-\frac{2}{3}(n+1), \quad n>1
\end{align*}
mynote({\color{blue}!这里的推导有问题,$ s_0=0 $ 不能直接代入})

调和数(harmonic number)
\begin{equation}
    H_n= 1+\frac{1}{2}+\frac{1}{3}+\dots+\frac{1}{n}=\sum_{k=1}^{n}\frac{1}{k}.
\end{equation}
之
所以这样命名,是因为小提琴弦所产生的第k个泛音(harmonic)是弦
长 $ \frac{1}{k} $ 处所产生的基音.
\begin{align*}
    \sum_{q\leqslant k\leqslant n}\frac{1}{k+1}
    &= \sum_{1\leqslant k-1 \leqslant n}\frac{1}{k} \\
    &= \sum_{2\leqslant k \leqslant n+1}\frac{1}{k} \\
    &= \sum_{1\leqslant k \leqslant n}\frac{1}{k}- \frac{1}{1}+\frac{1}{n+1} \\
    &= H_n - \frac{n}{n+1}.
\end{align*}
\begin{equation}
    C_n = 2(n+1)H_n-\frac{8}{3}n-\frac{2}{3}, \; n>1.
\end{equation}
\subsection{求和式处理}
分配律  distributive law \\
结合律  associative law \\
交换律  communicative law \\
等差级数 arithmetic progression \\
扰动法 perturbation method 把单独⼀次从和式中分出去 \\
⼏何级数 geometric progression \\

\subsection{多重求和}

\subsubsection{Exercise 1}
\begin{equation}
    A = \begin{bmatrix}
        a_1 a_1 & a_1 a_2 & \cdots & a_1 a_n \\
        a_2 a_1 & a_2 a_2 & \cdots & a_2 a_n \\
        \vdots  & \vdots  &        & \vdots  \\
        a_n a_1 & a_n a_2 & \cdots & a_n a_n \\
    \end{bmatrix}
\end{equation}
求 $ S_{\triangleleft} = \sum_{1\leqslant j \leqslant k \leqslant n} a_j a_k $ \footnote{下三角形矩阵的符号是一个右上部分的直角三角形, 目前我还不会输入}

\begin{solve}
    $ \because a_j a_k = a_k a_j $ , $ \therefore  $ 矩阵A沿主对角线对称, $ S_{\triangleleft} = S_{\triangleright} $ .
    \begin{equation*}
        [1\leqslant j \leqslant k \leqslant n] + [1\leqslant k \leqslant j \leqslant n] =[1\leqslant j , k \leqslant n] + [1\leqslant j = k \leqslant n]
    \end{equation*}
    \begin{align*}
        2S_{\triangleleft} &= S_{\triangleleft}+S_{\triangleright} = S_{A}+S_{diag(A)} \\
        &= \sum_{1\leqslant j , k \leqslant n} a_j a_k + \sum_{1\leqslant j = k \leqslant n} a_j a_k\\
        &= \left(\sum_{j=1}^{n} a_j\right)\left(\sum_{k=1}^{n} a_k\right) + \sum_{k=1}^{n} a_k^2\\
        &= \left(\sum_{k=1}^{n} a_k\right)^2 + \sum_{k=1}^{n} a_k^2
    \end{align*}
    $ \therefore S_{\triangleleft} = \frac{1}{2}[\left(\sum_{k=1}^{n} a_k\right)^2 + \sum_{k=1}^{n} a_k^2] $ 
\end{solve}

\subsubsection{Exercise 2}

\begin{equation}
    S = \sum_{1\leqslant j < k \leqslant n} (a_k-a_j)(b_k-b_j)
\end{equation}

\begin{solve}
    交换$ j,k $ 仍有对称性.\\
    \begin{equation*}
        S = \sum_{1\leqslant j < k \leqslant n} (a_k-a_j)(b_k-b_j)
          = \sum_{1\leqslant j < k \leqslant n} (a_j-a_k)(b_j-b_k)
    \end{equation*}
    \begin{equation*}
        [1\leqslant j < k \leqslant n] + [1\leqslant k < j \leqslant n] =[1\leqslant j , k \leqslant n] - [1\leqslant j = k \leqslant n]
    \end{equation*}
    \begin{align*}
        2S  &= 2\sum_{1\leqslant j < k \leqslant n} (a_k-a_j)(b_k-b_j)\\
            &= \sum_{1\leqslant j < k \leqslant n} (a_k-a_j)(b_k-b_j) + \sum_{1\leqslant k < j \leqslant n} (a_k-a_j)(b_k-b_j)\\
            &= \sum_{1\leqslant j , k \leqslant n} (a_k-a_j)(b_k-b_j) - \sum_{1\leqslant j = k \leqslant n} (a_k-a_j)(b_k-b_j)\\
            & (a_j-a_k=0, b_j-b_k=0, [j=k])
            &= \sum_{1\leqslant j , k \leqslant n} (a_k b_k - a_j b_k - a_k b_j + a_j b_j) \\
            &= \sum_{j=1}^{n} \sum_{k=1}^{n} a_k b_k - \sum_{j=1}^{n} \sum_{k=1}^{n} a_j b_k - \sum_{j=1}^{n} \sum_{k=1}^{n} a_k b_j + \sum_{j=1}^{n} \sum_{k=1}^{n} a_j b_j \\
            &= n\sum_{k=1}^{n} a_k b_k - \sum_{j=1}^{n} \sum_{k=1}^{n} a_j b_k - \sum_{j=1}^{n} \sum_{k=1}^{n} a_k b_j + n \sum_{j=1}^{n}  a_j b_j \\
            &= 2 n \sum_{k=1}^{n} a_k b_k - 2 \sum_{j=1}^{n} a_j  \sum_{k=1}^{n} b_k 
    \end{align*}
    \begin{equation*}
        S = n \sum_{k=1}^{n} a_k b_k - \left(\sum_{k=1}^{n} a_k \right) \left(\sum_{k=1}^{n} b_k \right)
    \end{equation*}
\end{solve}
对上式结果重新排序得
\begin{equation*}
    \left(\sum_{k=1}^{n} a_k \right) \left(\sum_{k=1}^{n} b_k \right) = n \sum_{k=1}^{n} a_k b_k - \sum_{1\leqslant j < k \leqslant n} (a_k-a_j)(b_k-b_j)
\end{equation*}

\begin{theorem}
    切比雪夫单调不等式 (Chebyshec's monotonic inequality)

\begin{equation*}
    \begin{array}{lll}
        \left(\sum_{k=1}^{n} a_k \right) \left(\sum_{k=1}^{n} b_k \right)   &\leqslant n \sum_{k=1}^{n} a_k b_k & \quad a_1\leqslant a_2\leqslant \dots\leqslant a_n, \text{and } b_1\leqslant b_2\leqslant \dots\leqslant b_n \\        
        & & \quad a_1\geqslant a_2\geqslant \dots\geqslant a_n, \text{and } b_1\geqslant b_2\geqslant \dots\geqslant b_n\\
        \left(\sum_{k=1}^{n} a_k \right) \left(\sum_{k=1}^{n} b_k \right)   &\geqslant n \sum_{k=1}^{n} a_k b_k & \quad a_1\leqslant a_2\leqslant \dots\leqslant a_n, \text{and } 
        b_1\geqslant b_2\geqslant \dots\geqslant b_n\\
        & & \quad a_1\geqslant a_2\geqslant \dots\geqslant a_n, \text{and }
        b_1\leqslant b_2\leqslant \dots\leqslant b_n \\        
    \end{array}
\end{equation*}
\end{theorem}
    一般来说,如果$ a_1 \leqslant a_2 \leqslant \dots \leqslant a_n $ 且 $ p $ 是$ \{1,\dots,n \} $ 的一个排列。\\
    那么不难证明:\\
    当$ b_{p(1)}\leqslant \dots \leqslant b_{p(n)} $ 时$ \sum_{k=1}^{n} a_k b_{P(k)} $ 最大.\\
    当$ b_{p(1)}\geqslant \dots \geqslant b_{p(n)} $ 时$ \sum_{k=1}^{n} a_k b_{P(k)} $ 最小.

    \begin{equation*}
        \sum_{k\in K} a_k = \sum_{P(k)\in K} a_{P(k)} 
    \end{equation*}
    $ P(k) $ 为这些整数的任意一个排列。
    \begin{equation*}
        f: J\Rightarrow K, \quad j\in J \quad f(j) \in K
    \end{equation*}
    \begin{equation*}
        \sum_{j\in J}a_{f(j)} = \sum_{k\in K} a_k \quad \#f^-(k)
    \end{equation*}
    式中 $ \#f^-(k) $ 表示集合$ f^-(k) = \{j | f(j)=k\} $ 中元素的个数
    \begin{equation*}
        \sum_{j\in J}[f(j)=k] = \#f^-(k)
    \end{equation*}
    \begin{equation*}
        \sum_{j\in J}a_{f(j)} = \sum_{\begin{array}{l}j\in J\\ k\in K\\ \end{array}}a_k[f(j)=k] = \sum_{k\in K}a_k \sum_{j\in J} [f(j)=k]
    \end{equation*}
    若有 $ \#f^-(k) = 1 $ (一一对应)\footnote{这里还不太理解}
    \begin{equation*}
        \sum_{j\in J} a_{f(j)} = \sum_{f(j)\in K}a_{f(j)} = \sum_{k\in K} a_k
    \end{equation*}

\subsubsection{Exercise 3}
\begin{equation*}
    S_n = \sum_{1\leqslant j < k\leqslant n}\frac{1}{k-j}
\end{equation*}
首先写出前几项,尝试寻找规律:
\begin{align*}
    S_1 &= 0\\
    S_2 &= \frac{1}{2-1} = 1\\
    S_3 &= \frac{1}{2-1}+\frac{1}{3-1}+\frac{1}{3-2} = \frac{5}{2}\\
    S_4 &= \frac{1}{2-1}+\frac{1}{3-1}+\frac{1}{4-1}+\frac{1}{3-2}+\frac{1}{4-2}+\frac{1}{4-3} = \frac{13}{3}
\end{align*}
\begin{solve}
    1. 先对$ j $ 求和
    \begin{align*}
        S_n 
        &= \sum_{1\leqslant k \leqslant n} \sum_{1\leqslant j < k}\frac{1}{k-j}\\
        &= \sum_{1\leqslant k \leqslant n} \sum_{1\leqslant (k-j) < k}\frac{1}{k-(k-j)} \quad j\Rightarrow (k-j) \\
        &= \sum_{1\leqslant k \leqslant n} \sum_{0<j\leqslant k-1}\frac{1}{j} \\
        &= \sum_{1\leqslant k \leqslant n} H_{k-1} \quad(H_k\text{为调和级数})\\
        &= \sum_{1\leqslant k+1 \leqslant n} H_{k}\quad k\Rightarrow k+1 \\
        &= \sum_{0\leqslant k < n} H_{k}
    \end{align*}
    2. 先对$ k $ 求和
    \begin{align*}
        S_n 
        &= \sum_{1\leqslant j \leqslant n} \sum_{j < k\leqslant n}\frac{1}{k-j}\\
        &= \sum_{1\leqslant j \leqslant n} \sum_{j < (k+j)\leqslant n}\frac{1}{(k+j)-j} \quad k\Rightarrow (k+j) \\
        &= \sum_{1\leqslant j \leqslant n} \sum_{0<k\leqslant n-j}\frac{1}{k} \\
        &= \sum_{1\leqslant j \leqslant n} H_{n-j} \quad(H_k\text{为调和级数})\\
        &= \sum_{1\leqslant n-j \leqslant n} H_{k}\quad j\Rightarrow n-j \\
        &= \sum_{0\leqslant j < n} H_{j}
    \end{align*}
    以上两种常用的求和顺序都无法得到这个多重求和的结果,我们需要转换思路.

    3. 先用$ k+j $ 替换$ k $ (先换元,再求和)
    \begin{align*}
        S_n 
        &= \sum_{1\leqslant j < (k+j)\leqslant n}\frac{1}{(k+j)-j} \quad k\Rightarrow k+j \\
        &= \sum_{1\leqslant j < (k+j)\leqslant n}\frac{1}{k} \\
        &= \sum_{1\leqslant k \leqslant n} \sum_{1\leqslant j \leqslant n-k}\frac{1}{k} \quad\text{首先对}j\text{求和} \\
        &= \sum_{1\leqslant k \leqslant n} \frac{n-k}{k}\\
        &= \sum_{1\leqslant k \leqslant n} \left( \frac{n}{k}-1 \right) = n H_n - n
    \end{align*}
    综上可得$ \sum_{1\leqslant k\leqslant n}H_k = n H_n - n $ 
\end{solve}
代数:
$ k+f(j) $ , $ f $ 为任意函数.\\
用$ k-f(j) $ 替换$ k $ ,并对$ j $ 先求和较好。\\
几何:
$ S_n \;(n=4) $ 
\begin{equation*}
    \begin{array}{ccccc}
            & k=1   & k=2   & k=3   & k=4   \\
        j=1 & & \frac{1}{1} & +\frac{1}{2} & +\frac{1}{3} \\
        j=2 & &             & +\frac{1}{1} & +\frac{1}{2} \\
        j=3 & &             &              & +\frac{1}{1} \\
        j=4 & &             &              &              \\
    \end{array}
\end{equation*}
先对$ j $ 求和(按列) $ H_1 + H_2 + H_3 $ 
先对$ k $ 求和(按行) $ H_3 + H_2 + H_1 $ 
$ k\Rightarrow k+j $ 按对角线求和 
% \begin{align*}
%     \frac{3}{1}+\frac{2}{2}+\frac{1}{3} &= \sum_{k=1}^{3}\frac{3-k}{k} \\ 
%     &= 3\sum_{k=1}^{3}\frac{1}{k}-\sum_{k=1}^{3} 1 \\
%     &= 3H_3 - 3
% \end{align*}
\begin{equation*}
    \sum_{k=1}^{n}\frac{n-k}{k} = n \sum_{k=1}^{n}\frac{1}{k}-\sum_{k=1}^{n} 1
\end{equation*}
$ nH_n-n $ ,$ n=4 $ 
\begin{align*}
    \frac{4}{1}+\frac{3}{2}+\frac{2}{3}+\frac{1}{4} &= \sum_{k=1}^{4}\frac{4-k}{k} \\ 
    &= 4\sum_{k=1}^{4}\frac{1}{k}-\sum_{k=1}^{4} 1 \\
    &= 4H_4 - 4
\end{align*}
\begin{equation*}
    4\left(1+\frac{1}{2}+\frac{1}{3}+\frac{1}{4}\right)-4 = \frac{4}{2}+\frac{4}{3}+\frac{4}{4}
\end{equation*}
\begin{equation*}
    \begin{array}{ccccc}
            & k-j=0   & k-j=1   & k-j=2   & k-j=3   \\
        j=1 & & \frac{1}{1} & +\frac{1}{2} & +\frac{1}{3} \\
        j=2 & & \frac{1}{1} & +\frac{1}{2} &              \\
        j=3 & & \frac{1}{1} &              &              \\
        j=4 & &             &              &              \\
    \end{array}
\end{equation*}

\subsection{General methods}
\subsubsection{Exercise 4}
求 $ \square_n = \sum_{0\leqslant k \leqslant n}k^2 $ , $ n\geqslant 0 $ 的封闭形式
\begin{align*}
    \sum_{k=0}^{n}k^2 
    &= \sum_{k=0}^{n}[(k+1)^2-2k-1] \\
    &= \sum_{k=1}^{n+1}k^2 - 2 \sum_{k=0}^{n}k - \sum_{k=0}^{n}1
\end{align*}
\begin{align*}
    0^2-(n+1)^2 &= -2 \sum_{k=0}^{n}k - (n+1) \\
    2\sum_{k=0}^{n} k &= (n+1)^2 - (n+1)\\
    \sum_{k=0}^{n} k &= \frac{(n+1)n}{2}\\
\end{align*}
上述运算没有告诉我们$ \square_n $ 的值,但却能推导出$ \sum_{k=0}^{n}k $ 的值。我们可以利用这种思路求解$ \square_n $ 。
\begin{align*}
    \sum_{k=0}^{n}\left[(k+1)^3-k^3\right] &= \sum_{k=0}^{n}\left[ 3k^2+3k+1 \right] \\
    (n+1)^3 - 0^3 &= 3 \sum_{k=0}^{n}k^2 + 3 \sum_{k=0}^{n}k +  \sum_{k=0}^{n}1 \\
    (n+1)^3 &= 3 \sum_{k=0}^{n}k^2 + 3 \frac{n(n+1)}{2} + (n+1)
\end{align*}
\begin{align*}
    3\sum_{k=0}^{n}k^2 &= (n+1)^3 - 3\frac{n(n+1)}{2}-(n+1) \\
    \sum_{k=0}^{n}k^2 &= \frac{1}{3}(n+1)\left((n+1)^2-\frac{3}{2}n-1\right)\\
    \sum_{k=0}^{n}k^2 &= \frac{1}{3}(n+1)(n+\frac{1}{2})n
\end{align*}

reference book list:\\
1. (CRC Tables) CRC Standard Mathematical Tables\\
2. Handbook of Mathematical Functions\\
3. Sloane. Handbook of Integer Sequences\\
software: \\
Axiom MACSYMA Maple Mathematica\\
my: Octave maxima
熟悉标准的信息源

方法3:建立成套方法

参考第二节的内容

方法4:用积分替换和式 $ \sum \Rightarrow \int $ 
\begin{equation*}
    \square_n = 1\times1+1\times4+1\times9 +\dots+1\times n^2
\end{equation*}
该式近似等于0到$ n $ 之间曲线$ f(x)=x^2 $ 下的面积
\begin{align*}
    S &= \int_{0}^{n}x^2 dx \\
    &=\frac{n^3}{3}
\end{align*}
$ \square_n $ 近似等于 $ \frac{n^3}{3} $ 。
近似的误差$ E_n = \square_n - \frac{n^3}{3} $ 

1. 近似误差项递归式
\begin{align*}
    \square_n &= \square_{n-1}+n^2 \\
    E_n &= \square_n - \frac{n^3}{3} = \square_{n-1} + n^2 - \frac{n^3}{3} \\
    E_{n-1} &= \square_{n-1} - \frac{(n-1)^3}{3}\\
    E_n &= E_{n-1} + \frac{(n-1)^3}{3} + n^2 - \frac{n^3}{3} \\
        &= E_{n-1} + \frac{-3n^2+3n-1}{3} + n^2 \\
        &= E_{n-1} + n - \frac{1}{3}
\end{align*}

2. 对楔形误差项面积求和
\begin{align*}
    \square_n - \int_{0}^{n}x^2 dx 
    &= \sum_{k=1}^{n}\left( k^2 - \int_{k-1}^{k}x^2 dx \right) \\
    &= \sum_{k=1}^{n}\left( k^2 - \frac{k^3-(k-1)^3}{3} \right) \\
    &= \sum_{k=1}^{n}\left( k - \frac{1}{3} \right)
\end{align*}
\begin{equation*}
    E_n = \sum_{k=1}^{n}\left( k - \frac{1}{3} \right) = \frac{n(n+1)}{2}-\frac{n}{3} = \frac{n(3n+1)}{6}
\end{equation*}
\begin{align*}
    \square_n &= \frac{n^3}{3}+E_n \\
    &= \frac{n^3}{3}+\frac{n(3n+1)}{6} \\
    &= \frac{n(2n^2+3n-1)}{6} \\
    &= \frac{n(n+\frac{1}{2})(n+1)}{3}
\end{align*}

方法5:展开和收缩
\begin{align*}
    \square_n
    &= \sum_{1\leqslant k \leqslant n} k^2 = \sum_{1\leqslant j\leqslant k \leqslant n} k\\
    &= \sum_{1\leqslant j\leqslant n} \sum_{j\leqslant k\leqslant n} k \\
    &= \sum_{1\leqslant j\leqslant n} \left(\frac{(j+n)(n-j+1)}{2}\right) \\
    &= \frac{1}{2} \sum_{1\leqslant j\leqslant n} (n(n+1)+j-j^2) \\
    &= \frac{1}{2} \left[n^2(n+1) + \frac{n(n+1)}{2} - \square_n \right]\\
    &= \frac{1}{2} n(n+\frac{1}{2})(n+1)-\frac{1}{2}\square_n
\end{align*}
\begin{align*}
    \frac{3}{2}\square_n &= \frac{1}{2} n(n+\frac{1}{2})(n+1) \\
    \square_n &= \frac{1}{3} n(n+\frac{1}{2})(n+1) \\
\end{align*}

方法6:使用有限微积分

方法7:用生成函数

\subsection{有限微积分和无限微积分 Finite and infinite calculus}
\begin{table}[htbp]
	\centering
	\small
	\caption{有限微积分和无限微积分中的运算对比}
	\begin{tabular}{l|l}
		\toprule
        无限微积分 & 有限微积分 \\
		\midrule
        微分算子D & 差分算子$ \Delta $ \\
		\bottomrule
	\end{tabular}%
	\label{tab:ContiAndDis001}%
\end{table}%

\begin{equation*}
    \begin{array}{ll}
        Df(x) = \lim_{h \to 0}\frac{f(x+h)-f(x)}{h}  &  \Delta f(x) = f(x+1)-f(x) \\
        D(x^m)=mx^{m-1} & \Delta (x^3) = 3x^2+3x+1\\
    \end{array}
\end{equation*}

为使差分运算在形式上与微分运算类似,引入下降阶乘幂和上升阶乘幂。

下降阶乘幂 (failing factorial power), $ x\fallingfactorial{m} $ , 读作$ x $ 直降$ m $ 次.\\
\begin{equation*}
    x\fallingfactorial{m} = \underbrace{x(x-1)\dots(x-m+1)}_{\textit{m个因子}}, \quad (m\leqslant 0, m\in\mathbb{N}) 
\end{equation*}
上升阶乘幂 (rising factorial power), $ x\risingfactorial{m} $ , 读作$ x $ 直升$ m $ 次.\\
\begin{equation*}
    x\risingfactorial{m} = \underbrace{x(x+1)\dots(x+m-1)}_{\textit{m个因子}}, \quad (m\leqslant 0, m\in\mathbb{N}) 
\end{equation*}
\begin{align*}
    \Delta(x\fallingfactorial{m}) &= (x+1)\fallingfactorial{m}-x\fallingfactorial{m} \\
    &= (x+1)x\dots(x+1-m+1) - x(x-1)\dots(x-m+1) \\
    &= (x+1-(x-m+1))x(x-1)\dots(x-m+2) \\
    &= mx(x-1)\dots(x-m+2) \\
    &= mx\fallingfactorial{m-1}
\end{align*}

\begin{table}[htbp]
	\centering
	\small
	\caption{有限微积分和无限微积分中的运算对比}
	\begin{tabular}{l|l}
		\toprule
        无限微积分 & 有限微积分 \\
		\midrule
        D逆运算$ \int $ (积分算子,逆微分算子) & $ \Delta $逆运算 $ \sum $ (求和算子,逆差分算子) \\
        微积分基本定理 & \\
        $ g(x)=Df(x)\iff\int g(x)dx=f(x)+C $ & $ g(x)=\Delta f(x)\iff\sum g(x)\delta(x)=f(x)+C $ \\
        定积分 & 和式 \\
        若$ g(x)=Df(x) $ 那么 & 若$ g(x)=\Delta f(x) $ 那么 \\ 
        $ \int_{a}^{b}g(x)dx=f(x)|_{a}^b=f(b)-f(a) $ & 
        $ \sum_{a}^{b}g(x)\delta x=f(x)|_{a}^b=f(b)-f(a) $ \\
        $ \int_{b}^{a}g(x)dx = -\int_{a}^{b}g(x)dx $  &
        $ \sum_{b}^{a}g(x)\delta x = -\sum_{a}^{b}g(x)\delta x $ \\
        $ \int_{a}^{b}+\int_{b}^{c}=\int_{a}^{c} $ &
        $ \sum_{a}^{b}+\sum_{b}^{c}=\sum_{a}^{c} $ \\
        & \\
        $ \int_{0}^{n}x^m=\cfrac{x^{m+1}}{m+1}\Big|_0^n=\cfrac{n^{m+1}}{m+1} $, $ m\neq -1 $  &
        $ \sum_{0}^{n}k\fallingfactorial{m} = \cfrac{k\fallingfactorial{m+1}}{m+1}\Big|_0^n =\cfrac{n\fallingfactorial{m+1}}{m+1} $,$ m\neq-1 $ \\ 
        & $ \sum_{0}^{n}k\risingfactorial{m} = \cfrac{k\risingfactorial{m+1}}{m+1}\Big|_0^n =\cfrac{n\risingfactorial{m+1}}{m+1} $,$ m\neq-1 $ \\ 
        & \\
        $ (x+y)^2=x^2+2xy+y^2  $ 
        & $ (x+y)\fallingfactorial{2}=x\fallingfactorial{2}+2x\fallingfactorial{2}y\fallingfactorial{1}+y\fallingfactorial{2} $ \\
        & $ (x+y)\risingfactorial{2}=x\risingfactorial{2}+2x\risingfactorial{1}y\risingfactorial{1}+y\risingfactorial{2} $ \\
        & \\
        {\color{red}$ m=-1,\int_{a}^{b}x^{-1}=\ln{x}\Big|_a^b $ } &
        {\color{red}$ m=-1,\sum_{a}^{b}k\fallingfactorial{-1}=H_{k}\Big|_a^b $ }\\
        $ \int_{a}^{b}x^m=\begin{array}{l}
            \cfrac{x^{m+1}}{m+1}\Big|_a^b, \; m\neq -1 \\
            \ln{n}\Big|_a^b, \; m= -1 \\
        \end{array} $ &
        $ \sum_a^b k\fallingfactorial{m} = \begin{array}{l}
            \cfrac{k\fallingfactorial{m+1}}{m+1}\Big|_a^b,\; m\neq-1 \\
            H_k\Big|_a^b,\; m=-1 \\
        \end{array}$ \\ 
        & $ \sum_a^b k\risingfactorial{m} = \begin{array}{l}
            \cfrac{k\risingfactorial{m+1}}{m+1}\Big|_a^b ,\;m\neq-1\\
            H_{(k+1)}\Big|_a^b ,\;m\neq-1 \\ % {\color{red}(??)}\\
        \end{array} $ \\ 
        连续性问题的解中会出现自然对数 & 快速排序这样的问题中会出现调和数的原因 \\
        & \\
        $ e^{x} $, 性质 $ De^x = e^x $  & $ \Delta f(x)=f(x), f(x)=2^x $ 离散指数函数 \\
		\bottomrule
	\end{tabular}%
	\label{tab:ContiAndDis002}%
\end{table}%

\begin{equation}
    g(x)=Df(x),\iff \underbrace{\int g(x)dx}_{\textit{不定积分}} = f(x)+\underbrace{C}_{\textit{任意常数}}
\end{equation} 

\begin{equation}
    g(x)=\Delta f(x),\iff \underbrace{\sum g(x)\delta(x)}_{\textit{不定和式}} = f(x)+\underbrace{C}_{\textit{满足}p(x+1)=p(x)\textit{的任意函数}} 
\end{equation}

\begin{equation}
    \sum_{a}^{b}g(x)\delta x=f(x)|_{a}^b=f(b)-f(a) 
\end{equation}
设 $ g(x)=\Delta f(x)=f(x+1)-f(x) $ \\
如果$ b=a $ , 我们就有
\begin{equation}
    \sum_{a}^{a}g(x) \delta x = f(a)-f(a) = 0
\end{equation}
如果$ b=a+1 $ , 我们就有
\begin{equation}
    \sum_{a}^{a+1}g(x) \delta x = f(a+1)-f(a) = g(a)
\end{equation}

\begin{align*}
    \sum_{a}^{b+1}g(x) \delta x - \sum_{a}^{b}g(x) \delta x 
    &= \bigl[f(b+1)-f(a)\bigr] - \bigl[f(b)-f(a)\bigr] \\
    &= f(b+1)-f(b) = g(b)
\end{align*}
由数学归纳法$ a, b \in\mathbb{N} $ 且$ b\leqslant a $ 时,$ \sum_{a}^{b}g(x)\delta x $ 的确切含义是
\begin{equation}
    \sum_{a}^{b}g(x) \delta x = \sum_{k=1}^{b-1}g(k) = \sum_{a\leqslant k < b}g(k),\quad (b\geqslant a, a,b\in \mathbb{N})
\end{equation}
若有$ g(x)=f(x+1)-f(x) $ 
\begin{equation*}
    \sum_{a\leqslant k <b}g(l) = \sum_{a\leqslant k<b}(f(k+1)-f(k))=f(b)-f(a)
\end{equation*}

\begin{align*}
    \sum_{a}^{b}g(x)\delta x &= f(b)-f(a)\qquad (b<a) \\
    &= -(f(a)-f(b)) = -\sum_{a}^{b}g(x)\delta x
\end{align*}
\begin{equation*}
    \sum_{a}^{b}+\sum_{b}^{c} = \sum_{a}^{c}
\end{equation*}
应用:计算下降幂和式的简单方法
\begin{equation}
    \sum_{0\leqslant k<n}k\fallingfactorial{m} = \frac{k\fallingfactorial{m+1}}{m+1}\Big|_0^n = \frac{n\fallingfactorial{m+1}}{m+1},\quad \left(m,n\geqslant 0\quad m,n\in\mathbb{N}^+\right)
\end{equation}
$ m=1 $ 时,$ k\fallingfactorial{1}=k $ 
\begin{equation*}
    \sum_{0\leqslant k<n}k = \frac{n\fallingfactorial{2}}{2}=\frac{n(n-1)}{2}
\end{equation*}

$ k^2 = k(k-1)+k=k\fallingfactorial{2}+k\fallingfactorial{1} $ 
\begin{align*}
    \sum_{0\leqslant k<n} k^2
    % \sum_{0\leqslant k<n}
    % k\fallingfactorial{2}+
    % \sum_{0\leqslant k<n}
    % k\fallingfactorial{1} 
    &=\frac{n\fallingfactorial{3}}{3} + \frac{n\fallingfactorial{2}}{2} \\
    &=\frac{n(n-1)(n-2)}{3}+\frac{n(n-1)}{2} \\
    &=\frac{1}{3}n(n-\frac{1}{2})(n-1)
\end{align*}

$ k^3 = k(k-1)(k-2) +3k(k-1) +k=k\fallingfactorial{3}+3k\fallingfactorial{2}+k\fallingfactorial{1} $ 
\begin{align*}
    \sum_{0\leqslant k<n} k^3
    &=\frac{n\fallingfactorial{4}}{4}
    +3\frac{n\fallingfactorial{3}}{3}
    + \frac{n\fallingfactorial{2}}{2}\\
    &=\frac{n(n-1)(n-2)(n-3)}{4}+3\frac{n(n-1)(n-2)}{3}+\frac{n(n-1)}{2}\\
    &=\left(\frac{1}{2}n(n-1)\right)^2 = \left(\sum_{0\leqslant k<n} k\right)^2
\end{align*}

负指数下降幂
\begin{align*}
    x\fallingfactorial{3}  &= x(x-1)(x-2) \\
    x\fallingfactorial{2}  &= x(x-1) \\
    x\fallingfactorial{1}  &= x \\
    x\fallingfactorial{0}  &= 1 \\
    x\fallingfactorial{-1} &= \frac{1}{x+1} \\
    x\fallingfactorial{-2} &= \frac{1}{(x+1)(x+2)} \\
    &\vdots \\
    x\fallingfactorial{-m} &= \frac{1}{(x+1)(x+2)\dots(x+m)} \\
\end{align*}
\begin{equation*}
    x\fallingfactorial{3}\cdot \frac{1}{x-3+1} = x\fallingfactorial{2} \Rightarrow 
    x\fallingfactorial{0}\cdot \frac{1}{x-0+1} = x\fallingfactorial{-1}
\end{equation*}
为什么选用$ x\fallingfactorial{-1}=\frac{1}{x+1} $ 
而不是 $ x\fallingfactorial{-1}=\frac{1}{x+1} $ 作为下降阶乘幂的拓展定义?\footnote{当一个原有的记号被拓展包含更多种情形时,以一
种使得一般性法则继续成立的方式来表述它的定
义,这永远是最佳选择}

通常幂法则$ x^{m+n}=x^m x^n $, 
推广:\\
{\color{red}$ x\fallingfactorial{m+n} = x\fallingfactorial{m}(x-m)\fallingfactorial{n},\;(m,n\in\mathbb{N}^+) $ }\\
{\color{red}$ x\risingfactorial{m+n} = x\risingfactorial{m}(x+m)\risingfactorial{n},\;(m,n\in\mathbb{N}^+) $ }

{my}推广至正指数下降幂
\begin{align*}
    x\risingfactorial{3}  &= x(x+1)(x+2) \\
    x\risingfactorial{2}  &= x(x+1) \\
    x\risingfactorial{1}  &= x \\
    x\risingfactorial{0}  &= 1 \\
    x\risingfactorial{-1} &= \frac{1}{x-1} \\
    x\risingfactorial{-2} &= \frac{1}{(x-1)(x-2)} \\
    &\vdots \\
    x\risingfactorial{-m} &= \frac{1}{(x-1)(x-2)\dots(x-m)} \\
\end{align*}
\begin{equation*}
    x\risingfactorial{3}\cdot \frac{1}{x+3} = x\risingfactorial{2} \Rightarrow 
    x\risingfactorial{0}\cdot \frac{1}{x+0-1} = x\risingfactorial{-1}
\end{equation*}


例如
\begin{align*}
    x\fallingfactorial{2+3} 
    &= x\fallingfactorial{2}(x-2)\fallingfactorial{3}
    =x\fallingfactorial{3}(x-3)\fallingfactorial{2} \\
    x\fallingfactorial{2-3}
    &= x\fallingfactorial{2}(x-2)\fallingfactorial{-3}\\
    &= x(x-1)\frac{1}{(x-2+1)(x-2+2)(x-2+3)}\\
    &= x(x-1)\frac{1}{(x-1)x(x+1)}\\
    &= \frac{1}{x+1}
\end{align*}

$ m<0 $ 时,$ \Delta x\fallingfactorial{m} = m x\fallingfactorial{m-1} $ 是否仍成立?
\begin{align*}
    \Delta x\fallingfactorial{-2} 
    &= \frac{1}{(x+2)(x+3)} - \frac{1}{(x+1)(x+2)} \\
    &= \frac{(x+1)-(x+3)}{(x+1)(x+2)(x+3)} \\
    &= -2x\fallingfactorial{-3}
\end{align*}
通常幂法则对负指数下降阶乘幂仍然成立。

离散指数函数 $ 2^x $ 
\begin{align*}
    \Delta (c^x) &= c^{x+1}-c^x = (c-1)c^x \\
    c \neq 1 & \frac{c^x}{c-1}\xRightarrow{\Delta}c^x\\
    \sum_{a\leqslant k <b}c^k &= \sum_{a}^{b}c^x \delta x = \frac{c^x}{c-1}\Big|_a^b =\frac{c^b-c^a}{c-1}, \; c\neq 1
\end{align*}

\begin{table}[htbp]
	\centering
	\small
	\caption{Table 55(1994), What's difference}
	\begin{tabular}{l|l}
		\toprule
        $ f=\sum g $ & $ \Delta f= g $ \\
		\midrule
        $ x\fallingfactorial{0}=1 $ & $ 0 $ \\
        $ x\fallingfactorial{1}=x $ & $ 1 $ \\
        $ x\fallingfactorial{2}=x(x-1) $ & $ 2x $ \\
        $ x\fallingfactorial{m}   $ & $ m x\fallingfactorial{m-1} $ \\
        $ x\fallingfactorial{m+1} $ & $ (m+1) x\fallingfactorial{m} $ \\
        $ H_x $ & $ x\fallingfactorial{-1}=\frac{1}{x+1} $ \\
        $ 2^x $ & $ 2^x $ \\
        $ c^x $ & $ (c-1) c^x $ \\
        $ \frac{c^x}{c-1} $ & $  c^x $ \\
        $ cu(x) $, $ c $ is constant  & $ c\Delta u(x) $ \\
        $ u+v $ & $ \Delta u+\Delta v $ \\
        $ uv $ & $ u\Delta v+Ev \Delta u $, $ Ev=v(x+1) $  \\
		\bottomrule
	\end{tabular}%
	\label{tab:55rightside}%
\end{table}%
\begin{equation}
    D(uv)=uDv+vDu
\end{equation}
\begin{equation}
    \int uDv = uv - \int vDu
\end{equation}

\begin{equation}
    \begin{array}{rl}
        \Delta (u(x)v(x))
        &=u(x+1)v(x+1)-u(x)v(x)\\
        &=u(x+1)v(x+1)-u(x)v(x+1)+u(x)v(x+1) -u(x)v(x)\\
        &=\Delta u(x)v(x+1)+u(x)\Delta v(x)\\
        &=u\Delta v+Ev \Delta u\\
    \end{array}
\end{equation}
其中$ E $ 被称为移位算子.\\
在无限微积分中,令$ x+1 \rightarrow x $ 无限细分,避开了$ E $ 
\begin{equation}
    \sum u\Delta v = uv - \sum Ev \Delta u
\end{equation} 

\begin{example}
    $ \int x e^x dx \xrightarrow[]{\text{离散模拟}} \sum x 2^x \delta x \qquad(\sum_{k=0}^{n}k2^k) $ 
\end{example}
\begin{solve}
    令$ u(x)=x, \delta v(x)=2^x $,\\
    可得$ \delta u(x)=1, v(x)=2^x,Ev = 2^{x+1} $  
    \begin{align*}
        \sum x2^x\delta x
        &= x\cdot 2^x - \sum 2^{x+1}\cdot 1\delta x\\
        &= x\cdot 2^x - 2^{x+1}+C
    \end{align*}
    \begin{align*}
        \sum_{0}^{n} k2^k
        &= \sum_{0}^{n+1}x2x\delta x\\
        &= x\cdot 2^x-2^{x+1}\Big|_0^{n+1}
    \end{align*}
    关于第二组等式的推导,我一开始没有完全掌握,主要是对求和符号$ \sum $ 的上下标范围存在误解.\\
    记$ \sum_{0\leqslant k\leqslant n} k 2^k = S_n $ 
    \begin{align*}
        \sum_{0\leqslant k\leqslant n} k 2^k + (n+1)2^{n+1}
        &= \sum_{0\leqslant k\leqslant n+1} k 2^k \\
        &= \sum_{1\leqslant k\leqslant n+1} (k-1) 2^k + \sum_{1\leqslant k\leqslant n+1} 2^k \\
        &= \sum_{0\leqslant k\leqslant n} k 2^{k+1} + \sum_{1\leqslant k\leqslant n+1} 2^k \\
        &= 2S_n + \sum_{1\leqslant k\leqslant n+1} 2^k 
    \end{align*}
    \begin{align*}
        S_n + (n+1)2^{n+1}
        &= 2S_n + \frac{2^{n+1}-2^1}{2-1}\\
        S_n &= (n+1)2^{n+1} - (2^{n+1}-2)\\
        &=(n-1)2^{n+1}+2
    \end{align*}
\end{solve}

\begin{equation}
    \sum_{0\leqslant k< n} H_k =nH_n-n
\end{equation}

求解看起来更困难的和式$ \sum_{0\leqslant k< n} k H_k $ 

类比$ \int x\ln{x} dx $ 
\begin{align*}
    I &= \int x\ln{x} dx \\
    &= x^2 \ln{x} - \int x(\ln{x}+x\cdot \frac{1}{x})dx\\
    &= x^2 \ln{x} - I - \int xdx\\
    I &= \frac{1}{2}x^2 \ln{x} - \frac{1}{2}x^2 + C
\end{align*}
对$ \sum_{0\leqslant k< n} k H_k $,取 $ u(x)=H_x \Delta $ , $ v(x)=x = x\fallingfactorial{1}$\\
$ \Delta u(x) = x\fallingfactorial{-1} $ , $ v(x) = \frac{1}{2}x\fallingfactorial{2} $ ,$ Ev(x) = v(x+1)=\frac{1}{2}(x+1)\fallingfactorial{2} $ 
\begin{align*}
    \sum x H_x \delta x
    &= \frac{1}{2}x\fallingfactorial{2}H_x - \sum \frac{1}{2}(x+1)\fallingfactorial{2}x\fallingfactorial{-1} \delta x\\
    &= \frac{x\fallingfactorial{2}}{2}H_x - \sum \frac{(x+1)x}{2} \frac{1}{x+1} \delta x\\
    &= \frac{x\fallingfactorial{2}}{2}H_x - \sum \frac{x\fallingfactorial{1}}{2}\delta x\\
    &= \frac{x\fallingfactorial{2}}{2}H_x - \frac{1}{4}x\fallingfactorial{2}+C
\end{align*}
\begin{equation*}
    \sum_{0\leqslant k< n} k H_k 
    = \sum_{x=0}^{n-1}x H_x \delta x = \frac{(n-1)\fallingfactorial{2}}{2}(H_{n-1}-\frac{1}{2})
\end{equation*}
教材上是
\begin{equation*}
    \sum_{0\leqslant k< n} k H_k 
    = {\color{red}\sum_{x=0}^{n}x H_x \delta x = \frac{n\fallingfactorial{2}}{2}(H_{n}-\frac{1}{2})}
\end{equation*}
% 这是由求和符号的定义决定的,在本书中,求和符号定义是
% \begin{equation}
    % \sum_{k=1}^{n}a_k = \sum_{1\leqslant k \leqslant n} = a_1 + a_2 + \dots +a_n
% \end{equation}
{\color{blue}借助有限微积分的原理,我们
很容易地记住}
\begin{equation}
    \sum_{0\leqslant k<n}k = \frac{n\fallingfactorial{2}}{2} = n(n-1)/2
\end{equation}
\footnote{数学的终极目标是不需要聪明的想法}
myex $ \sum_{0\leqslant k < n}H_k $ 
\begin{equation*}
    \begin{array}{ll}
        u(x) = H_x & \Delta v(x) = x\fallingfactorial{0}=1 \\
        \Delta u(x) = x\fallingfactorial{-1} & v(x) = x\fallingfactorial{1} \\
        & E v(x) = v(x+1) = (x+1)\fallingfactorial{1}\\
    \end{array}
\end{equation*}
\begin{align*}
    \sum H_x\cdot 1 \delta x
    &= x\fallingfactorial{1}H_x - \sum x\fallingfactorial{-1}(x+1)\fallingfactorial{1} \delta x\\
    &= x\fallingfactorial{1}H_x - \sum x\fallingfactorial{0}\delta x\\
    &= x\fallingfactorial{1}H_x - x\fallingfactorial{1} + C
\end{align*}
\begin{equation*}
    \sum_{0\leqslant k<n}H_k = \sum_{0}^{n}H_x\delta x = n H_n - n - (0-0) = n H_n - n
\end{equation*}

\subsection{无限和式 Infinite sums}
$ a_k $ 非负,$ \sum_{k\in K} a_k$ 
\begin{definition}
    如果有 $ A= const $ . s.t. $ \forall $ 有限⼦集 $ F\subset K $, 均有
    \begin{equation*}
        \sum_{k\in F}a_k \leqslant A
    \end{equation*}
    那么我们定义 $ \sum_{k\in K} a_k$ 是最小的这样的A
    (所有这样的A总包含⼀个最小元素)。
    若没有这样的常数A,
    我们就说 $ \sum_{k\in K} a_k = \infty $
    即
    $ \forall A\in \mathbb{R} $ , $ \exists $ 
    有限多项 $ a_k $ 组成的
    ⼀个集合, 它的和超过A
\end{definition}
该定义与指标集K 中可能存在的任何次序⽆关

特殊情形:K为非负整数集合
$ a_k \leqslant 0 $ 意味着
\begin{equation*}
    \sum_{k\geqslant 0} a_k = \lim_{n\rightarrow \infty}\sum_{k=0}^{n}a_k
\end{equation*}
理由: 实数任意非减序列均有极限

$  F \subset \mathbb{N} $ , $ \forall i\in F, i\leqslant n $ , $ \exists \sum_{k\in F}a_k \leqslant \sum_{k=0}^{n}a_k \leqslant A $ . \;
 $ \therefore $  $  \left\{\begin{array}{l}
    A=\infty\\
    A\text{为有界常数}\\
\end{array}\right. $ 

又$ \forall A' < A$ . $ \exists n $ . s.t $ \sum_{k=0}^{n}a_k > A' $ , $ F = \{0,1,\dots,n\} $. 证明$ A' $ 不是有界常数。

\begin{exercise}
    $ a_k = x^k $ 有
    \begin{equation*}
        \sum_{k\geqslant 0}x^k = \lim_{n\rightarrow\infty}\frac{1-x^{n+1}}{1-x} = \left\{\begin{array}{ll}
            \frac{1}{1-x}, & 0\leqslant x<1 \\
            \infty, & x\geqslant 1 \\
        \end{array}\right.
    \end{equation*}
\end{exercise}

\begin{exercise}
    \begin{align*}
        S &= 1+\frac{1}{2}+\frac{1}{4}+\frac{1}{8}+\dots\\
        \frac{1}{2} S &= \frac{1}{2}+\frac{1}{4}+\frac{1}{8}+\dots = S-1,\; S=2\\
        T &= 1+2+4+8+\dots\\
        2T &= 2+4+8+\dots = T-1, \; T=-1({\color{red}\times})\\
        T &= \infty.\;(\text{另一个解})
    \end{align*}
\end{exercise}

\begin{exercise}
    \begin{align*}
        \sum_{k\geqslant0} \frac{1}{(k+1)(k+2)}
        &= \sum_{k\geqslant0} k\fallingfactorial{-2} \\
        &= \lim_{n\rightarrow\infty} \sum_{k=0}^{n-1} k\fallingfactorial{-2} \delta k \\
        &= \lim_{n\rightarrow\infty} \frac{k\fallingfactorial{-1}}{-1}\Big|_0^n \\
        &= \lim_{n\rightarrow\infty} (-1)\left(\frac{1}{n+1}-\frac{1}{0+1}\right) \\
        &= 1
    \end{align*}
\end{exercise}

⾮负和式$ \rightarrow $ 和式中有⾮负项与负项
1. 
\begin{align*}
    \sum_{k\geqslant 0}(-1)^k
    &= 1-1+1-1+1-1+\dots \\
    &= (1-1)+(1-1)+(1-1)+\dots = 0\\
    &= 1+(-1+1)+(-1+1)+(-1+1)+\dots = 1
\end{align*}
\begin{equation*}
    \sum_{k\geqslant 0} x^k = \frac{1}{1-x},\; x\in \left[0,1\right)
\end{equation*}
$ x=-1 $ 代入上式,$ I=\frac{1}{1-(-1)}=\frac{1}{2} $ 

2. 双向⽆限 $ \sum_k a_k $ 
\begin{equation*}
    \begin{array}{ll}
        k\geqslant 0 & a_k = \frac{1}{k+1} \\
        k< 0 & a_k = \frac{1}{k-1} \\
    \end{array}
\end{equation*}
\begin{equation*}
    \begin{array}{cl}
        \dots + (-\frac{1}{4})+(-\frac{1}{3})+(-\frac{1}{2})+1+\frac{1}{2}+\frac{1}{3}+\frac{1}{4}+\dots & \\
        \dots + 
        \Biggl((-\frac{1}{4})+
        \biggl((-\frac{1}{3})+
        \Bigl((-\frac{1}{2})+
        \bigl(1\bigr)+
        \frac{1}{2}\Bigr)+
        \frac{1}{3}\biggr)+
        \frac{1}{4}\Biggr) +\dots & = 1 \\
        \dots + 
        \biggl((-\frac{1}{4})+
        \Bigl((-\frac{1}{3})+
        \bigl((-\frac{1}{2})\bigr)+
        1\Bigr)+
        \frac{1}{2}\biggr)+
        \frac{1}{3}+
        \frac{1}{4} +\dots & = 1 \\
        {\color{red}(\;)=1-\frac{1}{n}-\frac{1}{n+1}} & \\
        \dots + 
        \Biggl((-\frac{1}{4})+
        \biggl((-\frac{1}{3})+
        \Bigl((-\frac{1}{2})+
        1+
        \frac{1}{2}\Bigr)+
        \frac{1}{3}+
        \frac{1}{4}\biggr) +
        \frac{1}{5}+
        \frac{1}{6}+
        \Biggr)
        \dots & = 1+\ln{2} \\
    \end{array}
\end{equation*}
从内往外第n对括号包含数
\begin{equation*}
    -\frac{1}{n+1}-\frac{1}{n}-\dots-\frac{1}{2}+1 +\frac{1}{2}+\dots +\frac{1}{2n-1} +\frac{1}{2n}=1+H_{2n}-H_{n+1}
\end{equation*}

$ \lim_{n\rightarrow\infty}\left(H_{2n}-H_{n+1}\right) = \ln{2} $ 

按照不同方式对其项相加而得出不同值的和式,有某些不同寻常之
处.关于分析学的高等教材中有五花八门的定义,它们对这样自相矛
盾的和式赋予了有意义的值,但是,如果我们采用那些定义,就不能
像一直在做的那样自由地对记号$  \sum $ 进行操作.就本书的目的而言,不
需要“条件收敛”这种精巧的改进,因此我们会坚持使用无限和的一
种定义,以保证在这一章里所做的所有运算都是正确的.

事实上,我们关于无限和式的定义相当简单.设 $ K $  是任意一个集合,
而$ a_k $  是对每一个 $ k\in K $ 定义的实值项(这里“ $ k $  ”实际上可以代表若干
个指标$ k_1,k_2,\dots $  ,因而 $ K $ 可以是多维的).任何实数 $ x $ 都可以写成其正
的部分减去负的部分
\begin{equation*}
    x=x^+-x^-,\; x^+:= x\times [x>0],\; x^- := -x\times [x<0]
\end{equation*}
对$ \{a_k\} $ 中的每一项这样操作,得到无限和式 $ \sum_{k\in K}a_k^+ $ 和 $ \sum_{k\in K}a_k^- $ (mynote: 将任意和式拆成两个非负和式)
\begin{equation}
     \sum_{k\in K}a_k =  \sum_{k\in K}a_k^+ -  \sum_{k\in K}a_k^-
\end{equation}
除非右边的两个和式都等于$ \infty $  .在后面这种情形,我们不定义 $ \sum_{k\in K}a_k  $ \\
设 
$ A^+ = \sum_{k\in K}a_k^+  $ , 
$ A^- = \sum_{k\in K}a_k^-  $ . 
如果 $ A^+ $ 和$ A^- $ 都是有限的,就说和式 $ \sum_{k\in K}a_k  $ 绝对收敛(converge absolutely) 于值 $ A= A^+-A^- $ .
如果 $ A^+=\infty $ 而 $ A^- $ 是有限的,就说和式 $ \sum_{k\in K}a_k  $ 发散 (diverge) 于 $ +\infty $ .
类似地,如果 $ A^+ $ 是有限的 而 $ A^-=\infty $ ,就说和式 $ \sum_{k\in K}a_k  $ 发散 (diverge) 于 $ -\infty $ .
如果 $ A^+ = A^-=\infty $ ,结果还很难说.
\footnote{换句话说,绝对收敛就意味着绝对值的和式收敛.}

mynote:\\
$ \forall $ 集合$ K $ (可以是多维的), $ a_k\in \mathbb{R} \;(k\in K) $ . $ \forall x\in\mathbb{R} $ , $ x = x^+-x^- $ 
\begin{equation*}
    \sum_{k\in K}a_k = \sum_{k\in K}\tikznode{ak1}{$ a_k^+ $ } - \sum_{k\in K}\tikznode{ak2}{$ a_k^- $ }
\end{equation*}

\begin{tikzpicture}[remember picture,overlay,cyan,rounded corners]
    % explicit coordinates are relative to the end of arrow,
	% they do not accumulate (note the single + preceding the coords)
    % "nonnegative"
    \draw[<-,shorten <=1pt] (ak1)
    % -- +(0.4,0)% short line to the right
    |- +(4,0.4)% short up and long right
    coordinate (pp)% remember position for other arrows
    node[right] {$ a_k^+,a_k^- $ 均非负};
    \draw(ak2)
    |- (pp);
\end{tikzpicture}

$ \sum_{k\in K} a_k^+ $ ,
$ \sum_{k\in K} a_k^- $ 不同时为0
\begin{equation*}
    \begin{array}{llll}
        A &= A^+ &- A^- & A =  \sum_{k\in K} a_k \\ 
          &A^+\text{有限} & A^-\text{有限}& \sum_{k\in K} a_k \text{绝对收敛(converge absolutely)} \\
          &A^+=\infty & A^-\text{有限}& \sum_{k\in K} a_k \text{发散(diverge)于}\infty \\
          &A^+\text{有限} & A^-=\infty& \sum_{k\in K} a_k \text{发散(diverge)于}-\infty \\
    \end{array}
\end{equation*}
$ a_k\in\mathbb{R} \Rightarrow a_k\in\mathbb{C} $ 
\begin{equation*}
    \sum_{k\in K} a_k =
    \tikznode{akRe}{$ \sum_{k\in K} \Re{(a_k)} $ } +
    i\tikznode{akIm}{$ \sum_{k\in K} \Im{(a_k)} $ }    
\end{equation*}

\begin{tikzpicture}[remember picture,overlay,cyan,rounded corners]
    % explicit coordinates are relative to the end of arrow,
	% they do not accumulate (note the single + preceding the coords)
    % "nonnegative"
    \draw[<-,shorten <=1pt] (akRe)
    % -- +(0.4,0)% short line to the right
    |- +(4,0.4)% short up and long right
    coordinate (pp2)% remember position for other arrows
    node[right] {$\begin{array}{l}\text{两式均有定义时,}\\ \sum_{k\in K}a_k \text{有定义}\\ \end{array}$};
    \draw(akIm)
    |- (pp2);
\end{tikzpicture}

$\begin{array}{l}
    \text{分配律}\\
    \text{结合律}\\
    \text{交换律}\\
\end{array}$
$ \Rightarrow $ 多个指标集绝对收敛的和式,永远可以对指标中的任何一个首先求和。(结果与求和顺序无关)

$ \forall $ 指标集$  J $  且 $ \{K_j|j\in J\} $  的元素是任意的指标集. s.t.
\begin{equation*}
    \sum_{
        \begin{array}{l}
            j\in J \\
            k\in K_j\\
        \end{array}}a_{j,k}
    \text{绝对收敛于}A
\end{equation*} 
那么对每一
个 $ j\in J $ , $ \exists A_j \in \mathbb{C} $ .s.t.
\begin{equation*}
    \sum_{k\in K_j}a_{j,k}
    \text{绝对收敛于}A_j. \;
    \text{且}\sum_{j\in J}A_j
    \text{绝对收敛于}A
\end{equation*}

对所有项非负证明这一结记即可.\\
每项分解成实部与虚部,正的和负的部分,证明一般情形.\\
设对所有指标 $ (j,k)\in M  $ 都有 $ a_{j,k}\geqslant 0 $.\\ 
其中$ M $ 是主指标集 $ \{(j,k)| j\in J, k\in K_j \} $ \\
给定 $ \sum_{(j,k)\in M}a_{j,k} $  是有限的 即 对所有有限子集 $ F\subset M $  有\\
\begin{equation*}
    \sum_{(j,k)\in M}a_{j,k} \leqslant A
\end{equation*}
而 A 是这样的最小上界.

$ \forall j\in J  $  形如 $ \sum_{k\in F_j}a_{j,k}  $ 的每一个和都以 A 为上界. 
其中 $ F_j $ 是 $ k_j $  的一
个有限子集, 从而这些和式有一
个最小上界 $ A_j \geqslant 0 $. 
且根据定义有 $ \sum_{k\in K_j} =A_j$.


需证明对所有有限子集 $ G\subset J $. A 是 $ \sum_{j\in G} A_j $  的最小上界.\\


假设 $ G $  是 $ J $  满足 $ \sum_{j\in G}A_j = A' > A $  的有限子集.\\
我们可以求出一个有限子集 $ F_j \subseteq K_j $ .\\
使得对每个满足 $ A_j >0 $ 的 $ j\in G $  均有$ \sum_{k\in F_j}a_{j,k}>\left(\frac{A}{A'}\right) A_j$ .\\
至少存在一个这样的$ j $ .\\
但此时有$ \sum_{j\in G, k\in F_j} a_{j,k}>\left(\frac{A}{A'}\right) \sum_{j\in G}A_j = A $ .\\
这与如下事实矛盾. 对有限子集 $ F\subseteq M $  有 
$ \sum_{(j,k)\in F} a_{j,k} \leqslant A $.\\ 
从而对所有有限子集 $ G\subset J $  都有 $ \sum_{j\in G}A_j \leqslant A $ .

最后, 设 $ A' $  是小于$ A $ 的任何一
个实数如果我们能找到一个有限集合
$ G\subseteq J $ s.t $ \sum_{j\in G} A_j > A' $. 
证明就完成了.

已知存在有限集合 $ F\subseteq M $ s.t. $ \sum_{(j,k)a_{j,k}> A'} $.\\
设 $ G $ 是 $ F $ 中 $ j $ 组成的集合, 又设
$ F_j = \{k|(j,k)\in F\} $ 
那么有
\begin{equation*}
    \sum_{j\in G}A_j \geqslant \sum_{j\in G}\sum_{k\in F_j} a_{j,k} = \sum_{(j,k)\in F}a_{j,k}>A'
\end{equation*}
证完.

\end{document}