% chap04sec01
\section{Limits of functions}

\begin{mydef}
    \label{mydef:4.1}
    Let $X$ and $Y$ be metric spaces; suppose $E \subset X$, $f$ maps $E$ into $Y$, and $p$ is a limit point of $E$. We write $f(x) \rightarrow q$ as $x \rightarrow p$, or
    \begin{equation}
        \label{eq:4.1}
        \lim_{x \to p} f(x) = q
    \end{equation}
    if there is a point $q \in Y$ with the following property: For every $\varepsilon > 0$ there exists a $\delta > 0$ such that
    \begin{equation}
        \label{eq:4.2}
        d_Y (f(x), q) < \varepsilon
    \end{equation}
    for all points $x \in E$ for which
    \begin{equation}
        \label{eq:4.3}
        0 < d_X (x, p) < \delta.
    \end{equation}
    The symbols $d_X$ and $d_Y$ refer to the distances in $X$ and $Y$,  respectively.
\end{mydef}
\mybox{
    逐点连续 \\
    逐点连续 是 局部性质.\\
    一致连续 是 整体性质.
}
If $X$ and/or $Y$ are replaced by the real line, the complex plane, or by some euclidean space $\R^{k}$, the distances $d_X$, $d_Y$ are of course replaced by absolute values, or by norms of differences (see Sec. 2.16).

It should be noted that $p \in X$, but that $p$ need not be a point of $E$ in the above definition. Moreover, even if $p \in E$, we may very well have $f(p) \neq \lim_{x \to p} f(x)$ ➔ .

We can recast this definition in terms of limits of sequences:

\begin{thm}
    \label{thm:4.2}
    Let $X,Y,E,f$ , and $p$ be as in Definition 4.1. Then
    \begin{equation}
        \label{eq:4.4}
        \lim_{x \to p} f(x) = q
    \end{equation}
    if and only if 
    \begin{equation}
        \label{eq:4.5}
        \lim_{n \to \infty} f(p_n) = q
    \end{equation}
    for every sequence $\sequence{p_n}$ in $E$ such that
    \begin{equation}
        \label{eq:4.6}
        p_n \neq p, \quad
        \lim_{n \to \infty} p_n = p.
    \end{equation}
\end{thm}

\begin{proof}
    Suppose (\ref{eq:4.4}) holds. Choose $\sequence{p_n}$ in $E$ satisfying (\ref{eq:4.6}). Let $\varepsilon > 0$ be given. Then there exists $\delta > 0$ such that $d_Y(f(x), q) < \varepsilon$ if $x \in E$ and $0 < d_X (x, p) < \delta$. Also, there exists $N$ such that $n > N$ implies $0 < d_X(p_n ,p) < \delta$. Thus, for $n > N$, we have $d_Y(f(p_n), q) < \delta$, which shows that (\ref{eq:4.5}) holds. 
    
    Conversely, suppose (\ref{eq:4.4}) is false. Then there exists some $\varepsilon > 0$ such that for every $\delta > 0$ there exists a point $x \in E$ (depending on $\delta$), for which $d_Y(f(x), q) \geq \varepsilon$ but $0 < d_X(x, p) < \delta$. Taking $\delta_n = 1/n (n = 1, 2, 3, ... )$, we thus find a sequence in $E$ satisfying (\ref{eq:4.6}) for which (\ref{eq:4.5}) is false.
\end{proof}

\begin{myCorollary*}
    If $f$ has a limit at $p$, this limit is unique.
\end{myCorollary*}

\begin{mydef}
    \label{mydef:4.3}
    Suppose we have two complex functions, $f$ and $g$, both defined on $E$. By $f + g$ we mean the function which assigns to each point $x$ of $E$ the number $f(x) + g(x)$. 
    Similarly we define the difference $f - g$, 
    the product $fg$, 
    and the quotient $f/g$ of the two functions, 
    with the understanding that the quotient is defined only at those points $x$ of $E$ at which $g(x) \neq 0$. 
    If $f$ assigns to each point $x$ of $E$ the same number $c$, 
    then $f$ is said to be a constant function, or simply a constant, 
    and we write $f = c$. 
    If $f$ and $g$ are real functions, and if $f(x) \geq g(x)$ for every $x \in E$, we shall sometimes write $f \geq g$, for brevity.

    Similarly, if $\mathbf{f}$ and $\mathbf{g}$ map $E$ into $\R^{k}$, we define $\mathbf{f} + \mathbf{g}$ and $\mathbf{f} \cdot \mathbf{g}$ by
    \begin{equation*}
        (\mathbf{f} + \mathbf{g})(x) 
        = \mathbf{f}(x)  
        + \mathbf{g}(x) , \quad
        (\mathbf{f} \cdot \mathbf{g})(x) 
        = \mathbf{f}(x)  
        \cdot \mathbf{g}(x) ;        
    \end{equation*}
    and if $\lambda$ is a real number, $(\lambda \mathbf{f})(x) = \lambda \mathbf{f}(x)$.
\end{mydef}

\begin{thm}
    \label{thm:4.4}
    Suppose $E \subset X$, a metric space, $p$ is a limit point of $E$, $f$ and $g$ are complex functions on $E$, and
    \begin{equation*}
        \lim_{x \to p} f(x) = A, \quad
        \lim_{x \to p} g(x) = B.
    \end{equation*}
    Then \\
    (a) $\lim_{x \to p} (f + g)(x) = A + B$; \\
    (b) $\lim_{x \to p} (f   g)(x) = A   B$; \\
    (b) $\lim_{x \to p} (\frac{f}{g})(x) = \frac{A}{B}$, if $B \neq 0$. \\
\end{thm}

\begin{proof}
    In view of Theorem \ref{thm:4.2}, these assertions follow immediately from the analogous properties of sequences (Theorem \ref{thm:3.3}).
\end{proof}

Remark:
    If $f$ and $g$ map $E$ into $\R^{k}$, 
    then (a) remains true, 
    and (b) becomes (b') 
    $\lim_{x \to p} (\mathbf{f} \cdot \mathbf{g})(x) = \mathbf{A \cdot B}$;

(Compare Theorem \ref{thm:3.4}.)