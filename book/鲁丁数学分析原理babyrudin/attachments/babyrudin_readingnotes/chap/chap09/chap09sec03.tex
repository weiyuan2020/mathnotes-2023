% chap09sec03
\section{The contraction principle}

We now interrupt our discussion of differentiation to insert a fixed point
theorem that is valid in arbitrary complete metric spaces. 
It will be used in the
proof of the inverse function theorem.

\begin{mydef}
    \label{mydef:9.22}
    Let $X$ be a metric space, with metric $d$. 
    If $\phi$ maps $X$ into $X$
    and if there is a number $c < 1$ such that
    \begin{equation}
        \label{eq:9.43}
        d(\phi(x), \phi(y)) \leq c d(x, y)
    \end{equation}
    for all $x, y \in X$, then $\phi$ is said to be a contraction of $X$ into $X$.
\end{mydef}

\begin{thm}
    \label{thm:9.23}
    If $X$ is a complete metric space, and if $\phi$ is a contraction of $X$ into $X$, 
    then there exists one and only one $x \in X$ such that $\phi(x) = x$.
\end{thm}

% todo add proof

