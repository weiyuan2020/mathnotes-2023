% chap11sec09

\section{Functions of Class $\mathscr{L}^2$}

As an application of the Lebesgue theory,
we shall now extend the Parseval theorem
(which we proved only for Riemann-integrable functions in Chap. \ref{chap:08})
and prove the Riesz-Fischer theorem for orthonormal sets of functions.

\begin{mydef}
    \label{mydef:11.34}
    Let $X$ be a measurable space.
    We say that a complex
    function $f \in \mathscr{L}^2(\mu)$ on $X$ if $f$ is measurable and if
    \begin{equation*}
        \int_X |f|^2 \d \mu < +\infty .
    \end{equation*}
    If $\mu$ Lebesgue measure,
    we say $f \in \mathscr{L}^2$.
    For $f \in \mathscr{L}^2(\mu)$
    (we shall omit the phrase ``on $X$'' from now on) we define
    \begin{equation*}
        \left\| f \right\| =
        \left\{ \int_X \left| f \right|^2 \d \mu \right\}^{1/2}
    \end{equation*}
    and call $\|f\|$ the $\mathscr{L}^2(\mu)$ norm of $f$.
\end{mydef}
\mybox{omit 省略}

\begin{thm}
    \label{thm:11.35}
    Suppose $ f \in \mathscr{L}^2(\mu)$
    and     $ g \in \mathscr{L}^2(\mu)$.
    Then    $fg \in \mathscr{L}  (\mu)$,
    and
    \begin{equation}
        \label{eq:11.98}
        \int_X \left| fg \right| \d \mu \leq
        \left\| f \right\| \left\| g \right\| .
    \end{equation}
\end{thm}

This is the Schwarz inequality,
which we have already encountered for series and for Riemann integrals.
It follows from the inequality
\begin{equation*}
    0 \leq
    \int_X \left( | f | + \lambda | g |  \right)^2 \d \mu =
    \left\| f \right\|^2 +
    2 \lambda \int_X | fg | \d \mu + \lambda^2 \left\| g \right\|^2 ,
\end{equation*}
which holds for every real $\lambda$.

\begin{thm}
    \label{thm:11.36}
    If   $f \in \mathscr{L}^2(\mu)$
    and  $f \in \mathscr{L}^2(\mu)$,
    then $f + g \in \mathscr{L}^2(\mu)$,
    and
    \begin{equation*}
        \left\| f + g \right\| \leq
        \left\| f \right\| + \left\| g \right\| .
    \end{equation*}
\end{thm}

\begin{proof}
    The Schwarz inequality shows that
    \begin{align*}
        \left\| f + g \right\|^2
         & = \int |f|^2 + \int f\bar{g} + \int \bar{f}g + \int |g|^2    \\
         & \leq \|f\|^2 + 2\|f\| \|g\| + \|g\|^2                        \\
         & = \left( \left\| f \right\| + \left\| g \right\| \right)^2 .
    \end{align*}
\end{proof}

\begin{myremark}
    \label{myremark:11.37}
    If we define  the distance between two functions $f$ and $g$ in
    $\mathscr{L}^2(\mu)$ to be $\left\| f-g \right\|$,
    we see that the conditions of Definition \ref{mydef:2.15} are satisfied,
    except for the fact that $\left\| f-g \right\| = 0$ does not imply that $f(x) = g(x)$ for all $x$,
    but only for almost all $x$.
    Thus, if we identify functions which differ only on a
    set of measure zero, $\mathscr{L}^2(\mu)$ is a metric space.

    We now consider $\mathscr{L}^2$ on an interval of the real line, with respect to
    Lebesgue measure.
\end{myremark}

\begin{thm}
    \label{thm:11.38}
    The continuous functions form a dense subset of $\mathscr{L}^2$ on $[a, b]$.
\end{thm}

More explicitly, this means that for any $f \in \mathscr{L}^2$ on $[a, b]$, and any $\varepsilon > 0$,
there is a function $g$, continuous on $[a, b]$, such that
\begin{equation*}
    \left\| f - g \right\| =
    \left\{ \int_{a}^{b} \left| f - g \right|^2 \d x \right\}^{1/2} <
    \varepsilon.
\end{equation*}


\begin{mydef}
    \label{mydef:11.39}
    We say that a sequence of complex functions $\sequence{\phi_n}$ is an
    orthonormal set of functions on a measurable space $X$ if
    \begin{equation*}
        \int_X \phi_n \bar{\phi}_m \d \mu =
        \left\{
        \begin{array}{ll}
            0 & (n \neq m), \\
            1 & (n =    m). \\
        \end{array}
        \right.
    \end{equation*}
    In particular, we must have $\phi_n \in \mathscr{L}^2(\mu)$.
    If $f \in \mathscr{L}^2(\mu)$ and if
    \begin{equation*}
        c_n = \int_X f \bar{\phi}_n \d \mu
        \quad (n = 1,2,3,\dots),
    \end{equation*}
    we write
    \begin{equation*}
        f \sim \sum_{n=1}^{\infty} c_n \phi_n ,
    \end{equation*}
    as in Definition \ref{mydef:8.10}.
\end{mydef}


Parseval theorem
\begin{thm}
    \label{thm:11.40}
    Suppose
    \begin{equation}
        \label{eq:11.99}
        f(x) \sim \sum_{-\infty}^{\infty} c_n e^{inx} ,
    \end{equation}
    where $f$ in $\mathscr{L}^2$ on $[-\pi, \pi]$.
    Let $s_n$ be the $n$th partial sum of (\ref{eq:11.99}).
    Then
    \begin{align}
        \label{eq:11.100}
        \lim_{n \to \infty} \left\| f - s_n \right\| & = 0, \\
        \label{eq:11.101}
        \sum_{-\infty}^{\infty} \left| c_n \right|^2 & =
        \frac{1}{2\pi} \int_{-\pi}^{\pi} \left| f \right|^2 \d x.
    \end{align}
\end{thm}

\begin{myCorollary*}
    If $f \in \mathscr{L}^2$ on $[-\pi, \pi]$, and if
    \begin{equation*}
        \int_{-\pi}^{\pi} f(x) e^{-inx} \d x = 0
        \quad (n = 0, \pm 1, \pm 2, \dots),
    \end{equation*}
    then $\left\| f \right\| = 0$.
\end{myCorollary*}

Thus if two functions in $\mathscr{L}^2$ have the same Fourier series,
they differ at most on a set of measure zero.

\begin{mydef}
    \label{mydef:11.41}
    Let $f$ and $f_n \in \mathscr{L}^2(\mu) (n = 1, 2, 3, ... )$.
    We say that $\sequence{f_n}$ converges to $f$ in $\mathscr{L}^2(\mu)$ if $\left\| f_n - f \right\| \rightarrow 0$.
    We say that $\sequence{f_n}$ is a Cauchy sequence in $\mathscr{L}^2(\mu)$ if for every $\varepsilon > 0$ there is an integer $N$ such that $n \geq N$, $m \geq N$ implies $\left\| f_n - f_m \right\| \leq \varepsilon$.
\end{mydef}

\begin{thm}
    \label{thm:11.42}
    If $\sequence{f_n}$ is a Cauchy sequence in $\mathscr{L}^2(\mu)$,
    then there exists a function $f \in \mathscr{L}^2(\mu)$
    such that $\sequence{f_n}$ converges to $f$ in $\mathscr{L}^2(\mu)$.
\end{thm}

This says, in other words, that $\mathscr{L}^2(\mu)$ is a \myKeywordblue{complete} metric space.

\begin{thm}
    \label{thm:11.43}
    \myKeyword{The Riesz-Fischer theorem}
    Let $\sequence{\phi_n}$ be orthonormal on $X$.
    Suppose $\sum \left| c_n \right|^2$ converges,
    and put $s_n = c_1 \phi_1 + \cdots + c_n\phi_n$.
    Then there exists a function $f \in \mathscr{L}^2(\mu)$
    such that $\sequence{s_n}$ converges to $f$ in $\mathscr{L}^2(\mu)$,
    and such that
    \begin{equation*}
        f \sim \sum_{n=1}^{\infty}c_n \phi_n .
    \end{equation*}
\end{thm}

\begin{mydef}
    \label{mydef:11.44}
    An orthonormal set $\sequence{\phi_n}$ is said to be \myKeywordblue{complete} if, for $f \in \mathscr{L}^2(\mu)$, the equations
    \begin{equation*}
        \int_X f \bar{\phi}_n \d \mu = 0
        \quad (n = 1,2,3,\dots)
    \end{equation*}
    implies that $\left\| f \right\| = 0$.
\end{mydef}


In the Corollary to Theorem \ref{thm:11.40}
we deduced the completeness of the
trigonometric system from the Parseval equation (\ref{eq:11.101}).
Conversely, the Parseval
equation holds for every complete orthonormal set:

\begin{thm}
    \label{thm:11.45}
    Let $\sequence{\phi_n}$ be a complete orthonormal set.
    If $f \in \mathscr{L}^2(\mu)$ and if
    \begin{equation}
        \label{eq:11.106}
        f \sim \sum_{n=1}^{\infty} c_n \phi_n,
    \end{equation}
    then
    \begin{equation}
        \label{eq:11.107}
        \int_X \left| f \right|^2 \d \mu =
        \sum_{n=1}^{\infty} \left| c_n \right|^2 .
    \end{equation}
\end{thm}


Combining Theorems \ref{thm:11.43} and \ref{thm:11.45}, we arrive at the very interesting
conclusion that every complete orthonormal set induces a 1-1 correspondence
between the functions $f \in \mathscr{L}^2(\mu)$
(identifying those which are equal almost everywhere)
on the one hand and the sequences $\sequence{c_n}$ for which $\sum \left| c_n \right|^2$ converges,
on the other. The representation
\begin{equation*}
    f \sim \sum_{n=1}^{\infty} c_n \phi_n ,
\end{equation*}
together with the Parseval equation, shows that $\mathscr{L}^2(\mu)$ may be regarded as an
infinite-dimensional euclidean space (the so-called ``Hilbert space''), in which
the point $f$ has coordinates $c_n$, and the functions $\phi_n$ are the coordinate vectors.