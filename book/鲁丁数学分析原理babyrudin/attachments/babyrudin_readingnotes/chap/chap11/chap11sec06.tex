% chap11sec06

\section{Integration}

\begin{mydef}
    \label{mydef:11.21}
    Suppose
    \begin{equation}
        \label{eq:11.51}
        s(x) = \sum_{i=1}^{n} c_i K_{E_i} (x)
        \quad (x \in X, x_i > 0)
    \end{equation}
    is measurable,
    and suppose $E \in \mathfrak{M}$.
    We define
    \begin{equation}
        \label{eq:11.52}
        I_E(s) =
        \sum_{i=1}^{n} c_i \mu \left( E \cap E_1 \right) .
    \end{equation}

    If $f$ is measurable and nonnegative,
    we define
    \begin{equation}
        \label{eq:11.53}
        \int_E f \d \mu =
        \sup I_E (s).
    \end{equation}
    where the sup is taken over all measurable simple functions $s$ such that $0 \leq s \leq f$

    The left member of (\ref{eq:11.53}) is called the Lebesgue integral of $f$, with respect to the measure $\mu$, over the set $E$.
    It should be noted that the integral may have the value $+ \infty$.

    It is easily verified that
    \begin{equation}
        \label{eq:11.54}
        \int_E s \d \mu =
        I_E (s)
    \end{equation}
    for every nonnegative simple measurable function $s$.
\end{mydef}

\begin{mydef}
    \label{mydef:11.22}
    Let $f$ be measurable, and consider the two integrals
    \begin{equation}
        \label{eq:11.55}
        \int_E f^+ \d \mu , \quad
        \int_E f^- \d \mu ,
    \end{equation}
    where $f^+$ and $f^-$ are defined as in (\ref{eq:11.47}).

    If at least one of the integrals in (\ref{eq:11.55}) is finite,
    we define
    \begin{equation}
        \label{eq:11.56}
        \int_E f \d \mu =
        \int_E f^+ \d \mu -
        \int_E f^- \d \mu
    \end{equation}

    If both integrals in (\ref{eq:11.55}) are finite, then (\ref{eq:11.56}) is finite,
    and we say that $f$ is \emph{integrable} (or \emph{summable}) on $E$ in the Lebesgue sense, with respect to $\mu$;
    we write $f \in \mathscr{L}(\mu)$ on $E$.
    If $\mu = m$, the usual notation is:
    $f \in \mathscr{L}$ on $E$.

    This terminology may be a little confusing:
    If (\ref{eq:11.56}) is $+\infty$ or $-\infty$,
    then the integral of $f$ over $E$ is defined,
    although $f$ is not integrable in the above sense of the word;
    $f$ is integrable on $E$ only if its integral over $E$ is finite.

    We shall be mainly interested in integrable functions,
    although in some cases it is desirable to deal with the more general situation.
\end{mydef}

\begin{myremark}
    \label{myremark:11.23}
    The following properties are evident:
    \begin{asparaenum}[(a)]
        \item If $f$ is measurable and bounded on $E$, and if $\mu(E) < + \infty$, then $f \in \mathscr{L}(\mu)$ on $E$.
        \item If $a \leq f(x) \leq b$ for $x \in E$, and $\mu(E) < + \infty$,
        then
        \begin{equation*}
            a\mu(E) \leq \int_E f \d \mu \leq b\mu(E) .
        \end{equation*}
        \item If $f$ and $g \in \mathsf{L}(\mu)$ on $E$, and if $f(x) \leq g(x)$ for $x \in E$, then
        \begin{equation*}
            \int_E f \d \mu \leq
            \int_E g \d \mu .
        \end{equation*}
        \item  If $f \in \mathscr{L}(\mu)$ on $E$, then $cf \in \mathscr{L}(\mu)$ on $E$, for every finite constant $c$, and
        \begin{equation*}
            \int_E cf \d \mu \leq
            c \int_E f \d \mu .
        \end{equation*}
        \item If $\mu(E) = 0$, and $f$ is measurable, then
        \begin{equation*}
            \int_E f \d \mu = 0.
        \end{equation*}
        \item If $f \in \mathscr{L}(\mu)$ on $E$, $A \in \mathfrak{M}$, and $A \subset E$, then $f \in \mathscr{L}(\mu)$ on $A$.
    \end{asparaenum}
\end{myremark}

\begin{thm}
    \label{thm:11.24}
    \begin{asparaenum}[(a)]
        \item Suppose $f$ is measurable and nonnegative on $X$. For $A \in \mathfrak{M}$, define
        \begin{equation}
            \label{eq:11.57}
            \phi(A) = \int_A f \d \mu .
        \end{equation}
        Then $\phi$ is countably additive on $\mathfrak{M}$.
        \item The same conclusion holds if $f \in \mathscr{L}(\mu)$ on $X$.
    \end{asparaenum}
\end{thm}

\begin{myCorollary*}
    If $A \in \mathfrak{M}$, $B \in \mathfrak{M}$, $B \subset A$, and $\mu(A-B)=0$, then
    \begin{equation*}
        \int_A f \d \mu =
        \int_B f \d \mu .
    \end{equation*}
    Since $A =B\cup (A - B)$, this follows from Remark \ref{myremark:11.23}(e).
\end{myCorollary*}

\begin{myremark}
    \label{myremark:11.25}
    The preceding corollary shows that sets of measure zero are negligible in integration.

    Let us write $f \sim g$ on $E$ if the set
    \begin{equation*}
        \int_A f \d \mu =
        \int_B f \d \mu .
    \end{equation*}
    has measure zero.

    Then $f \sim f$; $f \sim g$ implies $g \sim f$;
    and  $f \sim g$, $g \sim h$ implies $f \sim h$.
    That is, the relation $\sim$ is an equivalence relation.

    If $f \sim g$ on $E$, we clearly have
    \begin{equation*}
        \int_A f \d \mu =
        \int_A g \d \mu ,
    \end{equation*}
    provided the integrals exists, for every measurable subset $A$ of $E$.
    % todo
\end{myremark}

\begin{thm}
    \label{thm:11.26}
    If $f \in \mathscr{L}(\mu)$ on $E$, then $\left| f \right| \in \mathscr{L}(\mu)$ on $E$, and
    \begin{equation}
        \label{eq:11.63}
        \left| \int_E f \d \mu \right| \leq
        \int_E \left| f \right| \d \mu .
    \end{equation}
\end{thm}

\begin{thm}
    \label{thm:11.27}
    Suppose $f$ is measurable on $E$, $\left| f \right| \leq g$, and $g \in \mathscr{L}(\mu)$ on $E$.
    Then $f \in \mathscr{L}(\mu)$ on $E$.
\end{thm}

\begin{proof}
    We have $f^+ \leq g$ and $f^- \leq g$.
\end{proof}

\begin{thm}
    \label{thm:11.28}
    \myKeyword{Lebesgue's monotone convergence theorem}
    Suppose $E \in \mathfrak{M}$. Let $\sequence{f_n}$ be
    a sequence of measurable functions such that
    \begin{equation}
        \label{eq:11.64}
        0 \leq f_1(x) \leq f_2(x) \leq \cdots
        \quad (x \in E).
    \end{equation}

    Let $f$ be defined by
    \begin{equation}
        \label{eq:11.65}
        f_n(x) \rightarrow f(x)
        \quad (x \in E)
    \end{equation}
    as $n \rightarrow \infty$.
    Then
    \begin{equation}
        \label{eq:11.66}
        \int_E f_n \d \mu \rightarrow
        \int_E f \d \mu
        \quad (n \rightarrow \infty).
    \end{equation}
\end{thm}

\begin{thm}
    \label{thm:11.29}
    Suppose $f = f_1 + f_2$, where $f_i \in \mathsf{L}(\mu)$ on $E$ $(i = 1,2)$.
    Then $f \in \mathsf{L}(\mu)$ on $E$, and
    \begin{equation}
        \label{eq:11.73}
        \int_E f \d \mu =
        \int_E f_1 \d \mu +
        \int_E f_2 \d \mu .
    \end{equation}
\end{thm}

We are now in a position to reformulate Theorem \ref{thm:11.28} for series.

\begin{thm}
    \label{thm:11.30}
    Suppose $E \in \mathfrak{M}$.
    If $\sequence{f_n}$ is a sequence of nonnegative measurable functions and
    \begin{equation}
        \label{eq:11.76}
        f(x) = \sum_{n=1}^{\infty} f_n (x)
        \quad (x \in E),
    \end{equation}
    then
    \begin{equation*}
        \int_E f \d \mu =
        \sum_{n=1}^{\infty} \int_E f_n \d \mu .
    \end{equation*}
\end{thm}

\begin{proof}
    The partial sums of (\ref{eq:11.76}) form a monotonically increasing sequence.
\end{proof}

\begin{thm}
    \label{thm:11.31}
    \myKeyword{Fatou's theorem}
    Suppose $E \in \mathfrak{M}$.
    If $\sequence{f_n}$ is a sequence of nonnegative measurable functions and
    \begin{equation*}
        f(x) = \liminf_{n \rightarrow \infty} f_n (x)
        \quad (x \in E),
    \end{equation*}
    then
    \begin{equation}
        \label{eq:11.77}
        \int_E f \d \mu \leq
        \liminf _{n \rightarrow \infty} f_n \d \mu .
    \end{equation}
\end{thm}

Strict inequality may hold in (\ref{eq:11.77}).
An example is given in Exercise \ref{ex:11.5}.

\begin{thm}
    \label{thm:11.32}
    \myKeyword{Lebesgue's dominated convergence theorem}
    Suppose $E \in \mathfrak{M}$.
    Let $\sequence{f_n}$ be a sequence of measurable functions such that
    \begin{equation}
        \label{eq:11.82}
        f_n(x) \rightarrow f(x)
        \quad (x \in E).
    \end{equation}
    as $n \rightarrow \infty$.
    If there exists a functions such that
    \begin{equation}
        \label{eq:11.83}
        \left| f_n(x) \right| \leq g(x)
        \quad (n = 1,2,3,\dots,x \in E),
    \end{equation}
    then
    \begin{equation}
        \label{eq:11.84}
        \lim_{n \to \infty} \int_E f_n \d \mu =
        \int_E f \d \mu .
    \end{equation}
\end{thm}

\begin{myCorollary*}
    If $\mu(E) < +\infty$, $\sequence{f_n}$ is uniformly bounded on $E$, and $f_n(x) \rightarrow f(x)$ on $E$, then (\ref{eq:11.84}) holds.
\end{myCorollary*}

A uniformly bounded convergent sequence is often said to be boundedly
convergent.