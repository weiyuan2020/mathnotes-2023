% chap07sec05
\section{Equicontinuous families of functions}
In Theorem 3.6 we saw that every bounded sequence of complex numbers
contains a convergent subsequence,
and the question arises whether something similar is true for sequences of functions.
To make the question more precise,
we shall define two kinds of boundedness.
\mybox{两种有界的区别}

\begin{mydef}
    \label{mydef:7.19}
    Let $\sequence{f_n}$ be a sequence of functions defined on a set $E$.
    We say that $\sequence{f_n}$ is \emph{pointwise bounded} on $E$
    if the sequence $\sequence{f_n(x)}$ is bounded for every $x \in E$,
    that is, if there exists a finite-valued function $\phi$ defined on $E$
    such that
    \begin{equation*}
        \left| f_n(x) \right| < \phi(x)
        \quad (x \in E, n = 1,2,3,...).
    \end{equation*}

    We say that $\sequence{f_n}$ is \emph{uniformly bounded} on $E$ if
    there exits a number $M$ such that
    \begin{equation*}
        \left| f_n(x) \right| < M
        \quad (x \in E, n = 1,2,3,...).
    \end{equation*}
\end{mydef}


Now if $\sequence{f_n}$ is pointwise bounded on $E$
and $E_1$ is a countable subset of $E$,
it is always possible to find a subsequence $\sequence{f_{n_k}}$
such that $\sequence{f_{n_k}(x)}$ converges for every $x \in E_1$.
This can be done by the diagonal process which is used in the
proof of Theorem \ref{thm:7.23}.

However, even if $\sequence{f_n}$ is a uniformly bounded sequence of continuous functions on a compact set $E$,
there need not exist a subsquence which converges pointwise on E.
In the following example, this would be quite troublesome to prove with the equipment which we have at hand so far,
but the proof is quite simple if we appeal to a theorem from Chap. \ref{chap:11}.

\begin{newexample}
    \label{newexample:7.21}
    Let
    \begin{equation*}
        f_n (x) = \sin n x
        \quad (0 \leq x \leq 2\pi, n = 1,2,3,...).
    \end{equation*}
    Suppose there exists a sequence $\sequence{n_k}$ such that $\sequence{\sin n_k x}$ converges, for every $x \in [0, 2\pi]$.
    In that case we must have
    \begin{equation*}
        \lim_{k \to \infty} \left(
        \sin n_{k} x -
        \sin n_{k+1} x
        \right) = 0
        \quad (0 \leq x \leq 2\pi)
    \end{equation*}
    hence
    \begin{equation}
        \label{eq:7.40}
        \lim_{k \to \infty} \left(
        \sin n_{k} x -
        \sin n_{k+1} x
        \right)^2 = 0
        \quad (0 \leq x \leq 2\pi)
    \end{equation}
    By Lebesgue's theorem concerning integration of boundedly convergent
    sequences (Theorem \ref{thm:11.32}), (\ref{eq:7.40}) implies
    \begin{equation}
        \label{eq:7.41}
        \lim_{k \to \infty} \int_{0}^{2\pi}
        \left(
        \sin n_{k} x -
        \sin n_{k+1} x
        \right)^2 \d x = 0.
    \end{equation}
    But a simple calculation shows that
    \begin{equation*}
        \int_{0}^{2\pi}
        \left(
        \sin n_{k} x -
        \sin n_{k+1} x
        \right)^2 \d x = 2\pi.
    \end{equation*}
    which contradicts (\ref{eq:7.41}).
\end{newexample}

Another question is whether every convergent sequence contains a
uniformly convergent subsequence.
Our next example will show that this
need not be so, even if the sequence is uniformly bounded on a compact set.
(Example 7.6 shows that a sequence of bounded functions may converge
without being uniformly bounded; but it is trivial to see that uniform convergence of a sequence of bounded functions implies uniform boundedness.)

\begin{newexample}
    Let
    \begin{equation*}
        f_n (x) = \frac{x^2}{x^2 + (1-nx)^2}
        \quad (0 \leq x \leq 1, n = 1,2,3,...).
    \end{equation*}
    Then $\left| f_n (x) \right| \leq 1$,
    so that $\sequence{f_n}$ is uniformly bounded on $[0, 1]$.
    Also
    \begin{equation*}
        \lim_{n \to \infty} f_n (x) = 0
        \quad (0 \leq x \leq 1),
    \end{equation*}
    but
    \begin{equation*}
        f_n \left( \frac{1}{n} \right) = 1
        \quad (n = 1,2,3,...).
    \end{equation*}
    so that no subsequence can converge uniformly on $[0, 1]$.
\end{newexample}

The concept which is needed in this connection is that of equicontinuity;
it is given in the following definition.

\begin{mydef}
    \label{mydef:7.22}
    family $\mathscr{F}$ of complex functions $f$ defined on a set $E$ in a
    metric space $X$ is said to be \emph{equicontinuous} on $E$
    if for every $\varepsilon > 0$ there exists a $\delta > 0$ such that
    \begin{equation*}
        \left| f(x) - f(y) \right| < \varepsilon
    \end{equation*}
    whenever $d(x, y) < \delta$, $x \in E$, $y \in E$, and $f \in \mathscr{F}$. Here $d$ denotes the metric of $X$.

    It is clear that every member of an equicontinuous family is uniformly
    continuous.
\end{mydef}

The sequence of Example \ref{newexample:7.21} is not equicontinuous.

Theorems \ref{thm:7.24} and \ref{thm:7.25} will show that there is a very close relation between equicontinuity, on the one hand,
and unifor1n convergence of sequences of continuous functions, on the other.
But first we describe a selection process which has nothing to do with continuity.

\begin{thm}
    \label{thm:7.23}
    If $\sequence{f_n}$ is a pointwise bounded sequence of complex functions on a countable set $E$,
    then $\sequence{f_n}$ has a subsequence $\sequence{f_{n_k}}$ such that ${f_{n_k}(x)}$ converges for every $x \in E$.
\end{thm}

% todo add proof

\begin{thm}
    \label{thm:7.24}
    If $K$ is a compact metric space,
    if $f_n \in  \mathscr{C}(K)$for $n = 1, 2, 3, ...$ ,
    and if $\sequence{f_n}$ converges uniformly on $K$,
    then $\sequence{f_n}$ is equicontinuous on $K$.
\end{thm}

% todo add proof


\begin{thm}
    \label{thm:7.25}
    If $K$ is compact,
    if $f_n \in \mathscr{C}(K)$ for $n = 1, 2, 3, ...$ ,
    and if $\sequence{f_n}$ is pointwise bounded and equicontinuous on $K$, then
    \begin{enumerate}[(a)]
        \item $\sequence{f_n}$ is uniformly bounded on $K$,
        \item $\sequence{f_n}$ contains a uniformly convergent subsequence.
    \end{enumerate}
\end{thm}

% todo add proof


