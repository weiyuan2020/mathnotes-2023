% chap03sec06
\section{Series}
Consider complex-valued sequences and series

\begin{mydef}
    \label{mydef:3.21}
    Given a sequence $\sequence{a_n}$, we use the notation
    \begin{equation*}
        \sum_{n=p}^{q} a_n \quad (p \leq q)
    \end{equation*}
    to denote the sum $a_p+a_{p+1}+\dots+a_q$. With $\sequence{a_n}$ we associate a sequence $\sequence{s_n}$, where
    \begin{equation*}
        s_n = \sum_{k=1}^{n} a_k.
    \end{equation*}
    For $\sequence{s_n}$ we also use the symbolic expression
    \begin{equation*}
        a_1 + a_2 + a_3 + \dots
    \end{equation*}
    or, more concisely
    \begin{equation}
        \sum_{n=1}^{\infty} a_n.
    \end{equation}
    we call this \emph{infinite series}, or just a \emph{series}. The numbers $\sequence{s_n}$ are called the \emph{partial sums} of the series. If $\sequence{s_n}$ converges to $s$, we say that that the series \emph{converges}, and write
    \begin{equation*}
        \sum_{n=1}^{\infty} a_n = s.
    \end{equation*}
    The number $s$ is called the sum of the series; but it should be clearly understood that $s$ is the \emph{limit of a sequence of sums}, and is not obtained simply by addition.
    
    If $\sequence{s_n}$ diverges, the series is said to diverge.

    Sometimes, for convenience of notation, we shall consider series of the form
    \begin{equation}
        \sum_{n=0}^{\infty} a_n.
    \end{equation}
    And frequently, when there is no possible ambiguity, or when the distinction is immaterial, we shall simply write $\sum a_n$ , in place of (4) or (5).

    It is clear that every theorem about sequences can be stated in terms of series (putting $a_1 = s_1$, and $a_{n} = s_{n} - s_{n-1}$ for $n > 1$), and vice versa. But it is nevertheless useful to consider both concepts.
\end{mydef}
The Cauchy criterion (Theorem 3.11) can be restated in the following form:
    
\begin{thm}
    \label{thm:3.22}
    $\sum a_n$  converges if and only if for every $\varepsilon \in > 0$ there is an integer $N$ such that
    \begin{equation}
        \left|
            \sum_{k=n}^{m} a_k 
        \right| \leq \varepsilon
    \end{equation}
    if $m \geq n \geq N$. 
\end{thm}

In particular, by taking $m = n$, (6) becomes
\begin{equation*}
    |a_n| \leq \varepsilon \quad (n \geq N).
\end{equation*}

\begin{thm}
    \label{thm:3.23}
    If $\sum a_n$ converges, then $\lim_{n \rightarrow \infty} a_n = 0$. 
\end{thm}

The condition $a_n \rightarrow 0$ is not sufficient to ensure convergence of $\sum a_n$. For instance, the series
\begin{equation*}
    \sum_{n=1}^{\infty}\frac{1}{n}
\end{equation*}
diverges; for the proof we refer to Theorem 3.28.

Theorem 3.14, concerning monotonic sequences, also has an immediate
counterpart for series.
\begin{thm}
    \label{thm:3.24}
    A series of nonnegative terms converges if and only if its partial sums form a bounded sequence.
\end{thm}

\begin{thm}
    \label{thm:3.25}
    (a) If $|a_n| \leq c_n$, for $n \geq N_0$, where $N_0$ is some fixed integer, and if $\sum c_n$ converges, then $\sum a_n$ converges.

    (b) If $a_n \geq d_n \geq 0$ for $n \geq N_0$, and if $\sum d_n$, diverges, then $\sum a_n$ diverges.
\end{thm}

\mybox{比较审敛法}